\newpage
\section{Dynamic programming problems} \label{problems}
% ------------------------------------------------------------------------------------------------
\subsection{Problems classification}
Since <<dynamic programming>> defines a very general technique, we already focused on grammar and alignment problems in \ref{bg_dp}. Before exploring some particular problem instances, we want to define some characteristics that will be used through the rest of the document to describe dynamic programming problems.

% ----------------------------------------------
\subsubsection{Definitions}\ul
\item \textbf{Cost or score:} refers to the result of the dynamic programming recurrence formula.
\item \textbf{Backtrack:} the backtrack is the information related to a score that describe how it has been obtained by referring to immediately previous elements. By induction on the backtrack, the \textbf{trace} (that describe \textit{all} thee steps to obtain the result) can be obtained.
\item \textbf{Alphabets:} an alphabet is an enumeration of the possible values. Its size helps determining how many bits are required in the implementation to store all values. Alphabets are defined for input, cost, backtrack and wavefront.
\item \textbf{Dimensions:} let $n$ the size of the input and $d$ the dimension of the underlying matrix.
\item \textbf{Matrices:} we refer by \textit{matrix} or \textit{matrices} to all the memoized intermediate cost- and backtrack-related information that is necessary to solve the dynamic programming problem of interest. Matrix elements are usually denoted by $M_{(i,j)}$ ($i^{\rm th}$ line , $j^{\rm th}$ column).
\item \textbf{Computation block:} this is a part of the DP matrix (cost and backtrack) that we want to compute. A block might be either a sub-matrix (rectangular) or a parallelogram (possibly reduced by taking the intersection with its enclosing matrix).
\item \textbf{Wavefront:} the wavefront consists of the minimum data necessary to construct a computation block of the DP matrix. It might include some previous lines/columns/diagonals as well as line-/column-/diagonal-wise aggregations (min, max, sum, ...).
\item \textbf{Delay:} we call delay the maximum distance between an element and its dependencies along column and lines (ex: recurrence $M_{(i,j)}=f\big(M_{(i+1,j)}, M_{(i+2,j-1)}\big)$ has delay 3).
\ule

% ----------------------------------------------
\subsubsection{Litterature classification}
In \cite{gpu_atlp}, dynamic programming problems are classified according to two criteria:\ul
\item \textbf{Monadic/polyadic:} a problem is monadic when only one of the previously computed term appears in the right hand-side of the recurrence formula (ex: Smith-Waterman \S\ref{swat_simple}). When two or more terms appear, the problem is polyadic (ex: Fibonacci, $F_n = F_{n-1} + F_{n-2}$).
When a problem is polyadic with index $p$, it also means that its backtracking forms a $p$-ary tree (where each node has at most $p$ children).

\item \textbf{Serial/non-serial:} a problem is serial ($s=0$) when the solutions depends on a fixed number of previous solutions (ex: Fibonacci), otherwise it is said to be non-serial ($s\ge 1$), as the number of dependencies grows with the size of the subproblem. That is computing an element of the matrix would require $O(n^s)$. For example, Smith-Waterman with arbitrary gap cost (\S\ref{swat_arbitrary}) is $s=1$; we can usually infer $s$ from the number of bound variables in the recurrence formula (see recurrence formulae in \S\ref{problem_examples}).
%	\[M_{(i,j)}=\max\left\{\begin{array}{l} ... \\ M_{(i,j-1)}\\ \max\limits_{i<k<j} [ M_{(i,k)}+M_{(k+1,j)} ] \end{array}\right. \]
\ule
Note that the algorithmic complexity of a problem is exactly $O\big(n^{d+s}\big)$.

% ----------------------------------------------
\subsubsection{Recurrence formulae simplifications} \label{calc_simplifications}
In some special cases, it is possible to transform a non-serial problem into a serial problem, if we can embed the non-serial term into an additional aggregation matrix. For example:
	\[M_{(i,j)}=\max\left\{\begin{array}{l} \max\limits_{k<i} M_{(k,j)}
	\\ \sum\limits_{k<i, l<j}M_{(k,l)} \end{array}\right.
	\implies M_{(i,j)}=\max\left\{\begin{array}{l} C_{(k,j)} \\ A_{(i-1,j-1)} \end{array}\right.\]
Where the matrix $C$ stores the maximum along the column and matrix $A$ stores the sum of the array of the previous elements. Both can be easily computed with an additional recurrence:
	\[\begin{array}{rcl} C_{(i,j)}&=&\max(C_{(i-1,j)}, M_{(i,j)}) \\
	A_{(i,j)}&=&A_{(i-1,j)}+A_{(i,j-1)}-A_{(i-1,j-1)}+M_{(i,j)}\end{array}\]

Although this simplification removes some non-serial dependencies at the cost of extra storage in the wavefront, it is not sufficient to transform all non-serial monadic problems into serial problems (ex: this does not apply to Smith-Waterman with arbitrary gap cost).

% ------------------------------------------------------------------------------------------------
\subsection{Problems of interest} \label{problem_examples}
We here focus on problems that have an underlying bi-dimensional matrix ($d=2$) because they can be parallelized (as opposed to be serial if $d=1$) and can solve large problems (of size $n$). Problems of higher matrix dimensionality ($d\ge3$) require substantial memory which severely impacts their scalability. Also it seems that most problems of interest have an algorithmic complexity of at most $O(n^4)$, probably because running time would otherwise becomes a severely limiting factor for the size of the problem.

We describe problems structures: inputs, cost matrices and backtracking matrix. These all have an alphabet (that must be bounded in terms of bit-size). Unless otherwise specified, we adopt the following conventions:\ul
\item Vectors of size $n$ are indexed from 0 to $n-1$, matrices follow the same convention ($M_{(m,n)}$ is indexed from $(0,0)$ to $(m-1,n-1)$)
\item Matrices dimensions are implicitly specified by number of indices and their number of elements is usually the same as the input length (possibly with 1 extra row/column).
\item Number are all unsigned integers
\item Problem dimension is $m,n$ (or $n$) indices $i,j$ ranges are respectively $0\le i<m$, $0\le j<n$.
\item Unless otherwise specified, the recurrence applies to all non-initialized matrix elements.
\ule
We describe the problem processing in terms of both initialization and recurrences.

Although not necessary to understand the project, the description of some of the most common dynamic programming problems is relevant to capture the essence of the dynamic programming processes and be able to compare and search for similarities among problems. Would the reader be familiar with dynamic programming, he could immediately jump to the next section.

A tighter analysis on the alphabet and intermediate results size is done because FPGA was also considered as a possible execution platform.

% ------------------------------------------------------------------------------------------------
% Recurrence visualization helpers

\newcommand\Cd[3][0,-1]{\put(#2){\put(.5,.5){\circle*{.3}}\put(.5,.5){\linethickness{1.5pt}\vector(#1){#3}}}} % dependency [dx,dy]{x,y}{len}
\def\Cg#1{\put(#1){\color{lightgray}\put(0,0){\polygon*(0,0)(0,1)(1,1)(1,0)}}} % grayed cell (not to store
\def\Cz#1{\put(#1){\put(0,.35){\parbox{1\unitlength}{\centering\bf 0}}}} % zero-init cell
\def\Cm{\put(6.5,4.5){\circle*{.4}}\multiput(0,0)(1,0){9}{\line(0,1){8}}\multiput(0,0)(0,1){9}{\line(1,0){8}}} % matrix base
\def\Cfl#1{#1{0,6}#1{0,5}#1{1,5}#1{0,4}#1{1,4}#1{2,4}#1{0,3}#1{1,3}#1{2,3}#1{3,3}#1{0,2}#1{1,2}#1{2,2}#1{3,2}#1{4,2}
	#1{0,1}#1{1,1}#1{2,1}#1{3,1}#1{4,1}#1{5,1}#1{0,0}#1{1,0}#1{2,0}#1{3,0}#1{4,0}#1{5,0}#1{6,0}} % triangular lower (function)
\def\Cfd#1{#1{0,7}#1{1,6}#1{2,5}#1{3,4}#1{4,3}#1{5,2}#1{6,1}#1{7,0}} % main diagonal
\def\Cfdu#1{#1{7,1}#1{6,2}#1{5,3}#1{4,4}#1{3,5}#1{2,6}#1{1,7}} % upper diagonal

%\def\Cmlong{\put(6.5,4.5){\circle*{.4}}\multiput(0,0)(1,0){16}{\line(0,1){8}}\multiput(0,0)(0,1){9}{\line(1,0){15}}} % matrix base
%\def\Cfu#1{#1{8,7}#1{9,7}#1{10,7}#1{11,7}#1{12,7}#1{13,7}#1{14,7}#1{9,6}#1{10,6}#1{11,6}#1{12,6}#1{13,6}#1{14,6}#1{10,5}#1{11,5}#1{12,5}#1{13,5}#1{14,5}#1{11,4}#1{12,4}#1{13,4}#1{14,4}#1{12,3}#1{13,3}#1{14,3}#1{13,2}#1{14,2}#1{14,1}} % triangular upper (function)

\def\CFig#1#2{\begin{figure}[H]\begin{center}\setlength{\unitlength}{.6cm}\begin{picture}(8,9) #2 \Cm\end{picture}\end{center}\caption{#1}\end{figure}}

% ----------------------------------------------
\newpage
\subsubsection{Smith-Waterman (simple)}\label{swat_simple}
Smith-Waterman is a biological sequence alignment algorithm. It tries to find the maximum number of correspondences between two DNA sequences; variants of this algorithm include \href{http://en.wikipedia.org/wiki/Needleman–Wunsch_algorithm}{Needleman-Wunsch}, and minimum edit distance family that generalizes on strings (\href{http://en.wikipedia.org/wiki/Hamming_distance}{Hamming distance}, \href{http://en.wikipedia.org/wiki/Levenshtein_distance}{Levenshtein distance}, ...). We explore three variants of this algorithm: simple (\S\ref{swat_simple}), affine (\S\ref{swat_affine}) and arbitrary (\S\ref{swat_arbitrary}) gap cost models. We study this problem because it has the interesting properties of using multiple input sequences and being suitable for hardware generation \cite{swat_fpga}.\\[6pt]\ol
\item Problem: matching two strings $S$, $T$ with $|S|=m, |T|=n$, with constant mismatch penalty ($d$) and arbitrary matching function (${\rm cost}(\_,\_)$).
\item Matrices: $M_{(m+1) \times (n+1)}, B_{(m+1) \times (n+1)}$
\item Alphabets:\ul
	\item Input: $\Sigma(S)=\Sigma(T)=\{a,c,g,t\}$.
	\item Cost matrix: $\Sigma(M) = [0..z], z=\max({\rm cost(\_,\_)}) \cdot \min(m,n)$
	\item Backtrack matrix: $\Sigma(B)=\{stop,W,N,NW\}$
	\ule
\item Initialization:\ul
	\item Cost matrix: $M_{(i,0)}=M_{(0,j)}=0$.
	\item Backtrack matrix: $B_{(i,0)}=B_{(0,j)}=stop$.
	\ule
\item Recurrence: \[M_{(i,j)}=\max\left\{\begin{array}{l|l}
		0 & stop\\
		M_{(i-1,j-1)}+{\rm cost}(S(i-1),T(j-1)) & NW\\
		M_{(i-1,j)}-d & N\\
		M_{(i,j-1)}-d & W
	\end{array}\right\}=B_{(i,j)} \]

\item Backtracking: starts from the cell $M_{(m,n)}$ and stops at the first cell containing a $0$.
\item Visualization: by convention, we put the longest string vertically ($m\ge n$):
\CFig{Smith-Waterman (affine gap cost) dependencies (serial)}{
	\put(-.5,7.5){S}\put(-.35,7.4){\linethickness{1pt}\vector(0,-1){2}}
	\put(.2,8.2){T}\put(.8,8.4){\linethickness{1pt}\vector(1,0){2}}
	\Cz{0,0}\Cz{0,1}\Cz{0,2}\Cz{0,3}\Cz{0,4}\Cz{0,5}\Cz{0,6}\Cz{0,7}
	\Cz{1,7}\Cz{2,7}\Cz{3,7}\Cz{4,7}\Cz{5,7}\Cz{6,7}\Cz{7,7}
	\Cd{6,5}{0.8}\Cd[1,0]{5,4}{0.8}\Cd[1,-1]{5,5}{0.8}
}

\item Optimizations:\ul
	\item In serial (monadic) problems we can avoid building the matrix $M$ by maintaining only the 3 last diagonals in memory (one for the diagonal element, one for horizontal/vertical, and one being currently built). This construction extends easily to polyadic problems where we need to maintain $k+2$ diagonals in memory, where $k$ is the maximum backward lookup.
	\item We could eliminate the first line and column of the matrix as they are filled with zeroes (representing a match with empty string), however this implies more involved computations, which is cumbersome.
	\item Padding: since to fill the $i^{\rm th}$ row we refer to the $(i-1)^{\rm th}$ character of string $S$, we could prepend to both $S$ and $T$ an unused character, so that matrix and input lines are aligned. Hence valid input indices would become $S[1 \cdots m]$ and $T[1 \cdots n]$.
	\ule
\ole

% ----------------------------------------------
\subsubsection{Smith-Waterman with affine gap extension cost} \label{swat_affine}
\ol
\item Problem: matching two strings $S$, $T$ with $|S|=m, |T|=n$, where creating a gap in either sequence has an opening penalty ($\alpha$) and an extension penalty ($\beta$).
\item Matrices: $M_{(m+1) \times (n+1)}, E_{(m+1) \times (n+1)}, F_{(m+1) \times (n+1)}, B_{(m+1) \times (n+1)}$
\item Alphabets:\ul
	\item Input: $\Sigma(S)=\Sigma(T)=\{a,c,g,t\}$.
	\item Cost matrices: $\Sigma(M) = \Sigma(E) = \Sigma(F) = [0..z], z=\max({\rm cost(\_,\_)}) \cdot \min(m,n)$
	\item Backtrack matrix: $\Sigma(B)=\{stop,W,N,NW\}$
	\ule
\item Initialization:\ul
	\item No gap cost matrix: $M_{(i,0)}=M_{(0,j)}=0$.
	\item T-gap extension cost matrix: $E_{(i,0)}= 0$ \textit{<<eat S chars only>>}
	\item S-gap extension cost matrix: $F_{(0,j)}= 0$
	\item Backtrack matrix: $B_{(i,0)}=B_{(0,j)}=stop$.
	\ule
\item Recurrence for the cost matrices:
\[\begin{array}{rcl}
M_{(i,j)}&=&\max\left\{\begin{array}{l|l}
	0 & stop\\
	M_{(i-1,j-1)}+{\rm cost}(S(i-1),T(j-1)) & NW\\
	E_{(i,j)} & N\\
	F_{(i,j)} & W
\end{array}\right\}=B_{(i,j)}\\
\\
E_{(i,j)}&=&\max\left\{\begin{array}{l|l}
	M_{(i, j-1)} - \alpha & NW\\
	E_{(i,j-1)} - \beta & N\\
\end{array}\right\}=B_{(i,j)}\\
\\
F_{(i,j)}&=&\max\left\{\begin{array}{l|l}
	M_{(i-1,j)} - \alpha & NW\\
	F_{(i-1,j)} - \beta & W\\
\end{array}\right\}=B_{(i,j)}
\end{array}\]

That can be written alternatively as:
\[M_{(i,j)}=\max\left\{\begin{array}{l|l}
	0 & stop\\
	M_{(i-1,j-1)}+{\rm cost}(S(i-1),T(j-1)) & NW\\
	\max_{1 \le k \le j-1} M_{(i,k)} - \alpha - (j-1-k) \cdot \beta & N\\
	\max_{1 \le k \le i-1} M_{(k,j)} - \alpha - (i-1-k) \cdot \beta & W\\
\end{array}\right\}=B_{(i,j)} \]

Although the latter notation seems more explicit, it introduces non-serial dependencies that the former set of recurrences is free of. So we need to implement the former rules as
\[ [M;E;F]_{(i,j)} = f \big( [M;E]_{(i,j-1)}, [M;F]_{(i-1,j)}, M_{(i-1,j-1)} \big) \]

\item Backtracking and visualization are similar to \S\ref{swat_simple}
\item Optimizations: Notice that this recurrence is very similar to \S\ref{swat_simple} except that we propagate 3 values ($M,E,F$) instead of a single one ($M$). Also notice that it is possible to propagate $E$ and $F$ inside a respectively horizontal and vertical wavefront, hence removing the need of the two additional matrices.
\ole

% ----------------------------------------------
\subsubsection{Smith-Waterman with arbitrary gap cost}\label{swat_arbitrary}\ol
\item Problem: matching two strings $S$, $T$ with $|S|=m, |T|=n$ with an arbitrary gap function $g(x)\ge 0$ where $x$ is the size of the gap. For example\footnote{Intuition: long gaps should penalize less; one large gap might be better than matching with smaller gaps.}: $g(x)=\max(m,n)-x$.

\item Matrices: $M_{(m+1) \times (n+1)}, B_{(m+1) \times (n+1)}$
\item Alphabets:\ul
	\item Input: $\Sigma(S)=\Sigma(T)=\{a,c,g,t\}$.
	\item Cost matrix: $\Sigma(M) = [0..z], z=\max({\rm cost(\_,\_)}) \cdot \min(m,n)$
	\item Backtrack matrix: $\Sigma(B)=\{stop,NW,N_{\{0..m\}},W_{\{0..n\}}\}$
	\ule

\item Initialization:\ul
	\item Match cost matrix: $M_{(i,0)}=M_{(0,j)}=0$.
	\item Backtrack matrix: $B_{(i,0)}=B_{(0,j)}=stop$.
	\ule

\item Recurrence: \[M_{(i,j)}=\max\left\{\begin{array}{l|l}
	0 & stop\\
	M_{(i-1,j-1)}+{\rm cost}(S(i-1),T(j-1)) & NW\\
	\max_{1 \le k \le j-1} M_{(i,j-k)} - g(k) & N_k\\
	\max_{1 \le k \le i-1} M_{(i-k,j)} - g(k) & W_k\\
\end{array}\right\}=B_{(i,j)} \]

\item Backtracking: similar to \S\ref{swat_simple} except that you can jump of $k$ cells along the rows or along the columns.
\item Visualization:
\CFig{Smith-Waterman (arbitrary gap cost) dependencies}{
	\put(-.5,7.5){S}\put(-.35,7.4){\linethickness{1pt}\vector(0,-1){2}}
	\put(.2,8.2){T}\put(.8,8.4){\linethickness{1pt}\vector(1,0){2}}
	\Cz{0,0}\Cz{0,1}\Cz{0,2}\Cz{0,3}\Cz{0,4}\Cz{0,5}\Cz{0,6}\Cz{0,7}
	\Cz{1,7}\Cz{2,7}\Cz{3,7}\Cz{4,7}\Cz{5,7}\Cz{6,7}\Cz{7,7}
	\Cd[0,-1]{6,7}{2.8}\Cd[0,-1]{6,6}{1.8}\Cd{6,5}{0.8}
	\Cd[1,0]{0,4}{5.8}\Cd[1,0]{1,4}{4.8}\Cd[1,0]{2,4}{3.8}\Cd[1,0]{3,4}{2.8}\Cd[1,0]{4,4}{1.8}\Cd[1,0]{5,4}{0.8}
	\Cd[1,-1]{5,5}{0.8}
}

\item Optimizations: The dependencies here are non-serial, there is no optimization that we can apply out of the box here. In general, this problem has an $O(n^3)$ complexity (whereas simple and affine gap variants are $O(n^2)$).
\ole

% ----------------------------------------------
\newpage
\subsubsection{Convex polygon triangulation} \label{polygon_tri} \ol
\item Problem: triangulating a convex polygon of $n$ vertices at minimal cost. Adding an edge $[i,j]$ has a cost $S_{(i,j)}$, where $S$ is a $(n\times n)$ matrix.
\item Matrices: $M_{(n+1)\times (n+1)}, B_{(n+1)\times (n+1)}$, upper triangular matrices including main diagonal. Indices denote \textit{<<first vertex, last vertex>>}; the vertex $n$ is the same as the vertex $0$ due to the cyclic nature of the problem.
\item Alphabets:\ul
	\item Input: $\Sigma(S_{(i,j)})=\{0..m\}$ with $m=\max_{i,j} S_{(i,j)}$ determined at runtime\footnote{We need to have statistics about $S$, this is where dynamic compilation might play a role}.
	\item Cost matrix: $\Sigma(M)=\{0..z\}$ with $z = m \cdot (n-2)$ (a triangulation of a polygon of $n$ edges adds at most $n-2$ edges).
	\item Backtrack matrix: $\Sigma(B)=\{stop, 0..n\}$ (index of intermediate edge)
	\ule
\item Initialization: $M_{(i,i)}=M_{(i,i+1)}=0, B_{(i,i)}=B_{(i,i+1)}=stop \quad\forall i$
\item Recurrence: \[M_{(i,j)}=\left\{ S(i,j) + \max_{i<k<j}M_{(i,k)}+M_{(k,j)} \,\,\rule[-.75em]{.5pt}{2em}\,\,  k \right\} = B_{(i,j)} \]
	Intuition: triangulate the partial polygon $(i,..j)$ recursively. 3 cases for the last triangle:\ul
	\item Given 2 triangulations $(1..k)$ and $(k..n)$, we close the polygon with $\bigtriangleup(1,k,n)$
	\item Given a triangulation $(1..n-1)$, we close the polygon with $\bigtriangleup(1,n-1,n)$
	\item Given a triangulation $(2..n)$, we close the polygon with $\bigtriangleup(1,2,n)$
	\ule
	Since the edge to close the last triangle is already part of the polygon, its cost is 0.
\item Backtracking: Add the edges in the set given by the set ${\rm BT}(B_{(0,n)})$ where
	\[{\rm BT}(B_{(i,j)}=k) \mapsto \left\{\begin{array}{ll} \{\} & \text{if } k=stop \\
		 \{(i,j)\} \cup {\rm BT}(B_{(i,k)}) \cup {\rm BT}(B_{(k,j)}) & \text{otherwise} \end{array}\right.\]
\item Visualization:
\CFig{Convex polygon triangulation dependencies}{
	\put(-.7,6.5){\rotatebox{90}{First}}\put(-.4,6.4){\linethickness{1pt}\vector(0,-1){2}}
	\put(.2,8.2){Last}\put(1.5,8.4){\linethickness{1pt}\vector(1,0){2}}
	\Cfl{\Cg}\Cfd{\Cz}\Cfdu{\Cz}
	\Cd[0,1]{6,1}{2.8}\Cd[0,1]{6,2}{1.8}\Cd[0,1]{6,3}{0.8}
	\Cd[1,0]{3,4}{2.8}\Cd[1,0]{4,4}{1.8}\Cd[1,0]{5,4}{0.8}
	\put(3.5,4.5){\line(3,-1){3}}
	\put(4.5,4.5){\line(2,-2){2}}
	\put(5.5,4.5){\line(1,-3){1}}
}

\item Optimizations:\ul
	\item If the cost of edges between contiguous vertices is 0, we do not need to handle special cases in the DP program (i.e. existing edges cannot be added).
	\item The matrix cost $S$ is a symmetric matrix and can be stored as a triangular matrix with 0 diagonal that can be omitted), hence $|S|=\tfrac{n(n-1)}{2}=N$.
	\ule
\ole

% ----------------------------------------------
\subsubsection{Matrix chain multiplication}\label{mat_mult_plain}\ol
\item Problem: find an optimal parenthesizing of the multiplication of $n$ matrices $A_i$. Each matrix $A_i$ is of dimension $r_i \times c_i$ and $c_i=r_{i+1} \forall i$. \textit{<<r=rows, c=columns>>}
\item Matrices: $M_{n \times n}, B_{n \times n}$ \textit{(first, last matrix)}
\item Alphabets:\ul
	\item Input: matrix $A_i$ size is defined as pairs of integers $(r_i,c_i)$.
	\item Cost matrix: $\Sigma(M)= 1..z$ with $z\le n\cdot \big[ \max_i(r_i,c_i) \big]^3 $.
	\item Backtrack matrix: $\Sigma(B)=\{stop\} \cup \{0..n\}$.
	\ule
\item Initialization:\ul
	\item Cost matrix: $M_{(i,i)}=0$.
	\item Backtrack matrix: $B_{(i,i)}=stop$.
	\ule
\item Recurrence: $c_k=r_{k+1}$
	\[M_{(i,j)}=\min_{i\le k<j}\left\{\begin{array}{l|l}
		M_{(i,k)}+M_{(k+1,j)}+r_i \cdot c_k \cdot c_j & k
	\end{array}\right\}=B_{(i,j)} \]
\item Backtracking: Start at $B_{(0,n-1)}$. Use the following recursive function for parenthesizing
	\[{\rm BT}(B_{(i,j)}=k) \mapsto \left\{\begin{array}{ll} A_i & \text{if } k=stop \\
		\Big( {\rm BT}(B_{(i,k)}) \Big) \cdot \Big( {\rm BT}(B_{(k+1,j)}) \Big) & \text{otherwise} \end{array}\right.\]

\item Visualization:
\CFig{Matrix chain multiplication dependencies}{
	\put(-.7,6.5){\rotatebox{90}{First}}\put(-.4,6.4){\linethickness{1pt}\vector(0,-1){2}}
	\put(.2,8.2){Last}\put(1.5,8.4){\linethickness{1pt}\vector(1,0){2}}
	\Cfl{\Cg}\Cfd{\Cz}
	\Cd[0,1]{6,1}{2.8}\Cd[0,1]{6,2}{1.8}\Cd[0,1]{6,3}{0.8}
	\Cd[1,0]{3,4}{2.8}\Cd[1,0]{4,4}{1.8}\Cd[1,0]{5,4}{0.8}
	\put(3.5,4.5){\line(3,-1){3}}\put(4.5,4.5){\line(2,-2){2}}\put(5.5,4.5){\line(1,-3){1}}
}

\item Optimizations:\ul
	\item We could normalize the semantics of indices and use $(n+1) \times (n+1)$ matrices where the meaning of cell $(i,j)$ would be $\underset{i\le k<j}{\rm chain}(A_k)$.
	%\ul
	%\item We need to swap vertically the matrix to have a normalized progression towards bottom right. To do that, we need to map all indices $i \mapsto n-1-i$ and we obtain a new recurrence relation:
	%\[M_{(i,j)}=\min_{i\le k<j}\left\{\begin{array}{l} M_{(i,k)}+M_{(2i-1 -k,j)}+r_i \cdot c_k \cdot c_j \end{array}\right. \]
	%With the initialization at $M_{(i,n-i-1)}$
	%\ule
	\item Alternatively, we could encode the dimension of the resulting matrix within the cost matrix by using a triplet (rows,columns,cost) and taking minimum appropriately.
	\ule
\ole

% ----------------------------------------------
\newpage
\subsubsection{Nussinov algorithm} \label{nussinov} \ol
\item Problem: folding a RNA string $S$ over itself $|S|=n$.
\item Matrices: $M_{n\times n}, B_{n \times n}$
\item Alphabets:\ul
	\item Input: $\Sigma(S)=\{a,c,g,u\}$.
	\item Cost matrix: $\Sigma(M)=\{0..n\}$
	\item Backtrack matrix: $\Sigma(B)=\{stop,D,1..n\}$
	\ule
\item Initialization: \ul
	\item Cost matrix: $ M_{(i,i)}=M_{(i,i+1)}=0$
	\item Backtrack matrix: $B_{(i,i)}=B_{(i,i+1)}=stop$
	\ule
\item Recurrences:
	\[M_{(i,j)}=\max\left\{\begin{array}{l|l}
		M_{(i+1,j-1)}+\omega(i,j) & D\\
		\max_{i\le k<j}M_{(i,k)}+M_{(k+1,j)} & k
	\end{array}\right\} = B_{(i,j)} \]
	With $\omega(i,j)=1$ if $i,j$ are complementary. 0 otherwise.
\item Backtracking: Start the backtracking in $B_{(0,n-1)}$ and go backward. The backtracking is very similar to that of the matrix multiplication, except that we also introduce the diagonal matching.
\item Visualization:
\CFig{Nussinov dependencies}{
	\put(-.7,6.5){\rotatebox{90}{First}}\put(-.4,6.4){\linethickness{1pt}\vector(0,-1){2}}
	\put(.2,8.2){Last}\put(1.5,8.4){\linethickness{1pt}\vector(1,0){2}}
	\Cfl{\Cg}\Cfd{\Cz}\Cfdu{\Cz}
	\Cd[0,1]{6,1}{2.8}\Cd[0,1]{6,2}{1.8}\Cd[0,1]{6,3}{0.8}
	\Cd[1,0]{3,4}{2.8}\Cd[1,0]{4,4}{1.8}\Cd[1,0]{5,4}{0.8}
	\Cd[1,1]{5,3}{0.8}
	\put(3.5,4.5){\line(3,-1){3}}\put(4.5,4.5){\line(2,-2){2}}\put(5.5,4.5){\line(1,-3){1}}
}

\item Optimizations: note that this is very similar to the matrix multiplication except that we also need the diagonal one step backward, so the same optimization can apply.
\ole

% ----------------------------------------------
\newpage
\subsubsection{Zuker RNA folding} \label{zuker} \ol
\item Problem: folding a RNA string $S$ over itself $|S|=n$ by minimizing the free energy.
\item Matrices: $V_{n\times n}, W_{n\times n}, F_n$ (Free Energy),  $BV_{n \times n}, BW_{n \times n}, BF_n$
\item Alphabets:\ul
	\item Input: $\Sigma(S)=\{a,c,g,u\}$.
	\item Cost matrices:\ul
		\item $\Sigma(W)=\Sigma(V)=\{0..z\}$ with $z \le n \cdot b+c$
		\item $\Sigma(F)=\{0..y\}$ with $y\le \min(F_0, z\cdot n)$
		\ule
	\item Backtrack matrices: \ul
		\item $\Sigma(BW)=\{stop, L,R,V,k\}$
		\item $\Sigma(BV)=\{stop, hairpin, stack, (i,j) , k\}$ with $0\le i,j,k < n$ % \\ $HL$=HairpinLoop, $IL$=InteriorLoop, $(i,j)$=MultiLoop
		\item $\Sigma(BF)=\{stop, P, k\}$ with $0\le k < n$
		\ule
	\ule
\item Initialization:\ul
	\item Cost matrices: $W_{(i,i)}=V_{(i,i)}=0, F_{(0)}=$ energy of the unfolded RNA.
	\item Backtrack matrices: $BW_{(i,i)}=BV_{(i,i)}=BF_{(0)}=stop$.
	\ule
\item Recurrence:
\[\begin{array}{rcl}
W_{(i,j)}&=&\min\left\{\begin{array}{l|l}
	W_{(i+1,j)}+b & L\\
	W_{(i,j-1)}+b & R\\
	V_{(i,j)}+\delta(S_i,S_j) & V \\
	\min_{i<k<j}W_{(i,k)}+W_{(k+1,j)} &k
\end{array}\right\} = BW_{(i,j)}\\
\\
V_{(i,j)}&=&\min\left\{\begin{array}{l|l}
	\infty \qquad\qquad\qquad\qquad {\rm if}(S_i,S_j) \text{ is not a base pair} & stop\\\\
	eh(i,j)+b \qquad\qquad\, \text{otherwise} & hairpin\\
	V_{(i+1,j-1)}+es(i,j) & stack \\
	VBI_{(i,j)} & (i',j') \\
	\min_{i<k<j-1}\{W_{(i+1,k)}+W_{(k+1,j-1)}\} +c & k
\end{array}\right\} = BV_{(i,j)}\\
\\
VBI_{(i,j)}&=&\min\Big\{\min_{i<i'<j'<j}V_{(i',j')}+ebi(i,j,i',j')\} +c \,\,\Big|\,\, (i',j') \Big\}=BV_{(i,j)}\\
\\
F_{(j)}&=&\min\left\{\begin{array}{l|l}
	F_{(j-1)} & P \\
	\min_{1\le i< j} (F_{(i-1)} + V_{(i,j)}) & i
\end{array}\right\} = BF_{(j)}
\end{array}\]

With $\delta$ a lookup table. In practice, we don't go backward for larger values than 30, so we can replace $\min_{i<k<j}$ by $\min_{\max(i,j-30)<k<j}$ in the expressions of $VBI$.

\item Backtracking: Start at $BF_{(n)}$ using the recurrences
 \[\begin{array}{rcl}
	BF_{(j)} &=& \left\{\begin{array}{rcl} P&\implies& BF_{(j-1)} \\ i &\implies& BF_{(i-1)} + BV_{(i,j)} \end{array} \right.\\
	\\
	BV_{(i,j)} &=& \left\{\begin{array}{rcl}
		hairpin &\implies&\big< {\rm hairpin}(i,j) \big> \\
		stack &\implies& \big< {\rm stack}(i,j) \big> \oplus BV_{(i+1,j-1)} \\
		(i',j') &\implies& \big< \text{bulge from $(i,j)$ to }(i',j') \big> \oplus BV(i',j')\\
		k &\implies& BW_{(i+1,k)} \oplus BW_{(k+1,j-1)}
	\end{array}\right.\\
	\\
	BW_{(i,j)} &=& \left\{\begin{array}{rcl}
	L & \implies & \big< multi\_loop(i) \big> \oplus BW_{(i+1,j)} \\
	R & \implies & \big< multi\_loop(j) \big> \oplus BW_{(i,j+1)} \\
	V &\implies& BV_{(i,j)} \\
	k &\implies& BW_{(i+1,k)} \oplus BW_{(k+1,j-1)}
	\end{array}\right.
\end{array}\]

\item Visualization\footnote{Reproductions of the illustrations from \cite{para_dprec} pp.148,149}:
\begin{figure}[H]\begin{center}\includegraphics[width=8cm]{inc/zuker_rec.pdf}\end{center}\caption{Zuker folding dependencies}\end{figure}

The recurrence consists of two non-serial dependencies as in \S\ref{swat_arbitrary} plus a bounded 2-dimensional dependency for bulges.\\[12pt]
Since this problem is non-trivial to understand from the recurrences, we propose an additional illustration of a RNA chain folded according to the Zuker folding algorithm.
\begin{figure}[H]\begin{center}
\includegraphics[width=10cm]{inc/zuker_struct.pdf} \\
\small Types of structural features modeled by the Zuker folding algorithm include: dangling ends (1), \\ internal loop (11), stack (23), multi-loop (47), bulge (68) and hairpin loop (78).\end{center}\caption{An example of an RNA folded into a secondary structure}\end{figure}

\item Optimizations: notice that there are 3 matrices: $W$,$V$ ($VBI$ is part of $V$) that can be expressed using regular matrix, and $F$ that is of different dimension than $W$ and $V$ and requires a special construction. Also notice that the $k$ of $BV$ and $BW$ describe almost the same backtrack, but there is an additional cost $c$ in $BV$.
\ole

\textbf{Alternative:} Since the recurrence matrices described in \cite{para_dprec} are of different dimensions ($F$ matrix is $O(n)$), we might want to use another description \cite{gpu_rnafold} where all matrices are of the same dimension, such that we can have a more uniform description across DP problems:

Let $Q'_{(i,j)}$ the minimum energy of folding of a subsequence $i,j$ given that bases $i$ and $j$ form a base pair. $Q_{(i,j)}$ and $QM_{(i,j)}$ are the minimum energy of folding of the subsequence $i,j$ assuming that this subsequence is inside a multi-loop and that it contains respectively at least one and two base pairs.

A simplified model of the recursion relations can be written as:
\[\begin{array}{rcl}
Q'_{(i,j)}&=&\left\{\begin{array}{ll}
	\min\left\{\begin{array}{l}
	Eh(i,j) \\
	Es(i,j)+Q'_{i+1,j-1} \\
	\min\limits_{i<k<l<j}Ei(i,j,k,l)+Q'_{k,l} \\
	QM_{i+1,j-1}
	\end{array}\right. & \text{if } (i,j) \text{ is a basepair}\\
	\infty & \text{otherwise} \\
	\end{array}\right. \\
QM_{i,j} &=& \min\limits_{i<k<j}(Q_{i,k}+Q_{k+1,j})\\
Q_{i,j} &=& \min\{ QM_{i,j}, Q_{i+1,j},Q_{i,j-1},Q'_{i,j} \}
\end{array}\]

\newpage
The corresponding energy functions are:\ul
\item $Eh(i,j)$ energy of hairpin loop closed by the pair $i \cdot j$.
\item $Ei(i,j,k,l)$ energy of interior loop formed by two base pairs $i \cdot j$, $k \cdot l$.
\item $Es(i,j)$ energy of two stacked base pairs $i\cdot j$ and $(i+1)\cdot(j-1)$.
\ule

This latter recurrence is more amenable to be converted into a grammar as the matrix are all of the same dimension. See the example in \S\ref{ex_rnafold} for a detailed implementation of this problem.

% ------------------------------------------------------------------------------------------------
\subsection{Related problems}
The aim of this section is to demonstrate that the problems previously described are very similar or encompass a significant part of the common dynamic programming problems\footnote{There are hyperlinks on the problems name to their detailed description.}.

\begin{table}[H]\begin{center}\begin{tabular}{llcc} \toprule
\bf Serial problems & \bf Shape & \bf Matrices & \bf Wavefront \\ \midrule
Smith-Waterman \footnotesize simple (\S\ref{swat_simple}) & rectangle & 1 & -- \\
Smith-Waterman \footnotesize affine gap extension (\S\ref{swat_affine}) & rectangle & 3 & (can replace 2 matrices) \\
\href{http://en.wikipedia.org/wiki/Needleman-Wunsch_algorithm}{Needleman-Wunsch} & rectangle & 1 & -- \\
\href{http://en.wikipedia.org/wiki/Dynamic_programming#Checkerboard}{Checkerboard} & rectangle & 1 & -- \\
\href{http://en.wikipedia.org/wiki/Longest_common_subsequence_problem\#Code_for_the_dynamic_programming_solution}{Longest common subsequence} & rectangle & 1 & -- \\
\href{http://en.wikipedia.org/wiki/Longest_common_substring_problem\#Pseudocode}{Longest common substring} & triangle & 1 & -- \\
\href{http://en.wikipedia.org/wiki/Levenshtein_distance\#Computing_Levenshtein_distance}{Levenshtein distance} & rectangle & 1 & -- \\
\href{http://en.wikipedia.org/wiki/De_Boor's_algorithm}{De Boor} \footnotesize evaluating B-spline curves & rectangle & 1 & -- \vspace{.4cm} \\
\toprule
\bf Non-serial problems & \bf Shape & \bf Matrices & \bf Wavefront \\ \midrule
Smith-Waterman \footnotesize arbitrary gap cost (\S\ref{swat_arbitrary}) & rectangle & 1 & -- \\
Convex polygon triangulation \footnotesize (\S\ref{polygon_tri}) & triangle & 1 & -- \\
Matrix chain multiplication \footnotesize (\S\ref{mat_mult_plain}) & triangle & 1 & -- \\
Nussinov \footnotesize (\S\ref{nussinov}) & triangle & 1 & -- \\
\href{http://rna.tbi.univie.ac.at/cgi-bin/RNAfold.cgi}{Zuker folding} \footnotesize (\S\ref{zuker}) & triangle & 3 & -- \\
\href{http://en.wikipedia.org/wiki/CYK_algorithm}{CYK} \footnotesize Cocke-Younger-Kasami & triangle & \#rules & -- \\
\href{http://en.wikipedia.org/wiki/Knapsack_problem#Dynamic_programming}{Knapsack} \footnotesize (pseudo-polynomial) & rectangle & 1 & --\\
\end{tabular}\end{center}
\caption{Classification of related problems}
\end{table}

\subsubsection{Other problems}\ul
\item Dijkstra shortest path: can be expressed in DP and requires a $E\times V$ matrix. Informally: along $E$, forall $V$, reduce the distance. The problem is serial along the $E$ dimension and non-serial along $V$, hence its complexity is $O(|E|\cdot |V^2|)$ which is worse than both $O(|V|^2)$ (using a minimum priority queue) and $O(|E|+|V|\log |V|)$ (with Fibonacci heap).
\item Fibonacci numbers: this problem is serial 1D (in 1 dimension). $F(n)$ could be implemented with ADP using a sequence of $n$ placeholder elements, but this is inefficient.
\item \href{http://archive.ite.journal.informs.org/Vol3No1/Sniedovich/\#dpmodel}{Tower of Hanoi}: 1D non-serial
\item \href{http://www.cs.ust.hk/mjg_lib/bibs/DPSu/DPSu.Files/KnPl81.PDF}{Knuth's word wrapping}: 1D non-serial
\item \href{http://en.wikipedia.org/wiki/Longest_increasing_subsequence#Efficient_algorithms}{Longest increasing subsequence}: serial (binary search is more efficient).
\item \href{http://www.ccs.neu.edu/home/jaa/CSG713.04F/Information/Handouts/dyn_prog.pdf}{Coin Change}: 1D non-serial
\ule

These algorithms also involve dynamic programming. However, we do not thoroughly evaluate their shape and number of matrices as a detailed description is not the focus of this project.\ul
\item \href{http://en.wikipedia.org/wiki/Floyd-Warshall_algorithm}{Floyd-Warshall}
\item \href{http://en.wikipedia.org/wiki/Viterbi_algorithm}{Viterbi \footnotesize (hidden Markov models)}: $T$ non-serial iterations over a vector
\item \href{http://en.wikipedia.org/wiki/Bellman-Ford_algorithm}{Bellman-Ford} (finding the shortest distance in a graph)
\item \href{http://en.wikipedia.org/wiki/Earley_parser#Pseudocode}{Earley parser} (a type of chart parser)
\item \href{http://en.wikipedia.org/wiki/Maximum_subarray_problem}{Kadane maximum subarray} 1D serial, look at
\href{http://www.cosc.canterbury.ac.nz/tad.takaoka/cats02.pdf}{Takaoka} for 2D
\item \href{http://en.wikipedia.org/wiki/Recursive_least_squares_filter}{Recursive least squares}
\item \href{http://www.math.utep.edu/Faculty/pmdelgado2/courses/adv_algorithms/homework-08_anser.pdf}{Bitonic tour}
\item \href{http://www.cs.berkeley.edu/~vazirani/algorithms/chap6.pdf}{Shortest path, Shortest path in DAGs, All pair shortest paths, Independent sets in trees}
\item \href{http://www.algorithmist.com/index.php/Dynamic_Programming}{Subset Sum, Family Graph}
\item \href{http://www.cs.uiuc.edu/~jeffe/teaching/algorithms/notes/05-dynprog.pdf}{Optimal Binary Search Trees}
\item \href{http://www.cs.ucsb.edu/~suri/cs130b/NewDynProg.pdf}{Independent set on a tree}
\item \href{http://en.wikipedia.org/wiki/Dynamic_programming#A_type_of_balanced_0.E2.80.931_matrix}{More dynamic programming problems from Wikipedia}
\ule

\subsubsection{Conclusion} \label{problems_end}
In the rest of the report, we use a different description of the problems that is based on ADP \cite{adp}, which is more convenient but does not share much with the above description (even though ultimately the executed computations are very similar). Although not of immediate use, the description of the above problem and ad-hoc CUDA implementation of three of them (Smith-Waterman with arbitrary gap cost, Matrix chain multiplication and Convex polygon triangulation) helped us to understand:\ol
\item There is a difference between dynamic programming as seen in algorithmic schoolbooks and their concrete implementation, mainly because special care must be taken for correct indices and preventing off-by-one errors.
\item Problems can be classified in two categories: single track (input) and two-tracks (2 input sequences). Most of the interesting dynamic programming problems that could be parallelized fall in these two categories.
\item Sometimes matrices are initially padded with zeroes (or initial value), although this might be ignored at algorithm design, care must be taken for these special values and their inclusion in the matrix should be decided according to the complexity of the recurrence formula.
\item Incidentally, we proposed a cyclic variant of the convex polygon triangulation, which uses a parallelogram matrix (see \S\ref{mem_layout}). Unfortunately, this proved to be based on an erroneous recurrence relation analysis, and can only use a triangular matrix as described in \S\ref{polygon_tri}.

Although we have not found a real problem requiring a parallelogram matrix, we still present this version in \S\ref{mem_layout} and \S\ref{ad_hoc_impl}. Such matrix layout could be adapted for cyclic problems that could be broken into a linear sequence anywhere (that is for all position in the circular structure, break the cycle at this position, and solve the dynamic programming problem on the resulting flattened sequence). For example, one could be interested in finding the longest subsequence verifying some property in a cycle, such that the subsequence score changes if it is rotated.
\ole

% ------------------------------------------------------------------------------------------------
\newpage
\subsection{Related work} \label{related_work}
Work related to dynamic programming can be separated in two categories: ad-hoc implementations and grammar-based implementations. The former focus on the performance for a specific problem whereas the latter generalize and formalize the dynamic programming problem description into a parsing grammar paired with a costing algebra.

Grammar-based dynamic programming was inseminated by ADP \cite{adp_discipline} and first implemented as a Haskell DSL \cite{adp}. To overcome performance issues, multiple solutions were devised:\ul
\item Converting Haskell parsers in their C or CUDA equivalent \cite{adp_gpu}
\item Modifying Haskell execution environment to provide loop fusion to improve ADP parsers performance \cite{adp_fusion}, \cite{adp_fusion_pkg}.
\item Ultimately, the dynamic programming algebra and grammar were formalized into a specific language \cite{gapl} provided with an ad-hoc compiler \cite{gapc_thesis}, thereby allowing more advanced analysis of the grammar \cite{gapc_yield}.
\ule

The research on ad-hoc implementation has focused on three kind of problems:\ul
\item Genral problems, attempting to provide the most efficient implementation for a particular problem \cite{gpu_atlp}, \cite{robust_mapping}, \cite{gpu_from_dsl}.
\item RNA sequence folding (variants of the Zuker folding): \cite{nussinov_gpu}, \cite{gpu_rnafold}.
\item Biological sequence alignment (Smith-Waterman) for huge sequences: \cite{swat_gpu}, \cite{swat_linear} \cite{swat_mega}.
\ule

Since this project involves various domains, we also investigated in the memory management on graphic cards and existing code generation frameworks.

In an attempt to support a varying number of results per matrix cell, we considered dynamic memory allocation \cite{nvidia} (available on recent graphic cards), ad-hoc memory allocation \cite{scatter_alloc} and hash tables \cite{parallel_hashing}. However the costs associated with dynamic memory allocation makes it unattractive for this particular kind of problem, and the use of cuckoo hash tables adds a constant factor penalty to every memory access. Finally both solution introduce undesirable possibility of failure (respectively out of memory or unrecoverable collision) in the middle of the algorithm computation process.

Automated code generation and execution flow is addressed by Delite \cite{lms2}, \cite{lms3}, \cite{delite}, that leverages LMS\cite{lms_thesis} to generate from the same source code efficient implementation for heterogeneous platforms (including CUDA) at runtime. Although this shares many patterns with our project, we can not reuse this framework because the scheduling and computation is tightly interleaved in dynamic programming (see \ref{bg_lms}) whereas Delite focuses on parallelizing operations on collections (array, lists, maps, ...) of independent elements.
