\documentclass[11pt]{article}
\usepackage{amssymb,amsmath,amsthm,hyperref,verbatim,pict2e,graphicx,array,listings,appendix,color}
\usepackage{algorithm,algorithmic,booktabs,marvosym,wrapfig,xytree,multicol,multirow,arydshln,nameref}
\hypersetup{colorlinks,citecolor=black,filecolor=black,linkcolor=black,urlcolor=black
	%pdfborderstyle={/S/U/W 1},urlbordercolor=1 0 0,linkbordercolor=.5 1 1, citebordercolor=.5 1 1
}
\usepackage[usenames]{xcolor} % color names ,dvipsnames,svgnames,table
\usepackage[utf8]{inputenc}
\usepackage[T1]{fontenc}
\usepackage[english]{babel}

% margins
\pagestyle{headings}
\oddsidemargin 0.0cm
\evensidemargin 0.0cm
\topmargin 0.0cm
\headheight 0.0cm
\headsep 1.0cm
\textheight 22.0cm
\textwidth 16.0cm
\parskip 0.1cm
\parindent 0.0cm
\footskip 1.0cm

% compact titles
\usepackage[compact]{titlesec}
\titlespacing{\section}{0pt}{8pt}{0pt}
\titlespacing{\subsection}{0pt}{8pt}{0pt}
\titlespacing{\subsubsection}{0pt}{8pt}{0pt}

% compact lists
\usepackage{enumitem}
\setitemize{noitemsep,topsep=0pt,parsep=0pt,partopsep=0pt}
\setenumerate{noitemsep,topsep=0pt,parsep=0pt,partopsep=0pt}
\def\ul{\begin{itemize}}
\def\ule{\end{itemize}}
\def\ol{\begin{enumerate}}
\def\ole{\end{enumerate}}

% misc
\def\up#1{\textsuperscript{#1}}
\def\quote#1{\par\begingroup\leftskip1em\rightskip\leftskip\textit{#1}\par\endgroup}


% listings
\definecolor{dkpink}{RGB}{200,0,100}
\definecolor{gray}{RGB}{128,128,128}
\lstset{
	xleftmargin=20pt,
	numberstyle=\tiny,stepnumber=1,numbersep=5pt,
	showstringspaces=true,         % underline spaces within strings
	tabsize=2,                      % sets default tabsize to 2 spaces
	captionpos=t,                   % sets the caption-position to bottom
	breaklines=true,                % sets automatic line breaking
	breakatwhitespace=true, % sets if automatic breaks should only happen at whitespace
	title=\lstname, % show the filename of files included with \lstinputlisting; also try caption instead of title
	basicstyle=\small\tt,keywordstyle=\color{blue},commentstyle=\color{gray},stringstyle=\color{dkpink}
}
% define Scala syntax
\lstdefinelanguage{Scala}{
	morekeywords={abstract,case,catch,class,def,do,else,extends,false,final,finally,for,if,implicit,import,%
	match,mixin,new,null,object,override,package,	private,protected,requires,return,sealed,super,this,%
	throw,trait,true,try,type,val,var,while,with,yield},
	otherkeywords={=>,<-,<\%,<:,>:,\#,@},sensitive=true,
	morecomment=[l]{//},	morecomment=[n]{/*}{*/},
	morestring=[b]",morestring=[b]',morestring=[b]"""
}

% title page
\makeatletter
\gdef\@subtitle{}\def\subtitle#1{\gdef\@subtitle{#1}}
\def\my@heading{
\def\ps@headings{\let\@mkboth\markboth
	\def\@evenhead{\small \rightmark \hfill \textit{\@title}, p.~\thepage}
	\def\@oddhead{\@evenhead}}\pagestyle{headings}}
\renewcommand{\maketitle}{
	%\begin{titlepage}
	\setcounter{page}{0}\thispagestyle{empty}
	{\centering\null\vfill\includegraphics[width=5.5cm]{inc/logo_epfl.pdf} % EPFL logo
	\vspace{1.5cm}\hrule \vspace{2.5cm} {\LARGE \@title \par} {\large \emph \@subtitle \par}
	\vspace{2.75cm} {\Large \@author \par}
	\vspace{5.5cm} {\large School of Computer and Communication Sciences, EPFL \par}
	\vspace{1.0cm} {\@date \par} % date
	\vfill\null\par}\my@heading
	\newpage
	%\end{titlepage}
}
\newcommand{\shorttitle}{
	\thispagestyle{empty}
	\hfill \includegraphics[width=3cm]{inc/logo_epfl}\vspace{.1cm} % EPFL logo
	\begin{center} {\LARGE \@title} \\ \vspace{.1cm} {\large \textit{\@subtitle}} \\ \rule[1ex]{350pt}{.5pt} \\
	\@author \\ {\small School of Computer and Communication Sciences, EPFL} \vspace{.2cm} \\{\small \@date}
	\end{center} \vspace{.5cm}\my@heading
}
\makeatother

% new XeTeX title page
\usepackage[T1]{fontenc}
\usepackage{fontspec}
\newfontfamily\fonth{Helvetica}
\newfontfamily\fonthn{Helvetica Neue}
\newfontfamily\fonthc{Helvetica Neue Condensed Bold}
\newfontfamily\fonthl{Helvetica Neue UltraLight}

\makeatletter
\renewcommand{\maketitle}{
	%\begin{titlepage}
	\setcounter{page}{0}\thispagestyle{empty}
	\hfill \includegraphics[width=8.5cm]{inc/logo_epfl.pdf} \vfill
	{\fontsize{25pt}{11pt}\fonthc Master project report \vspace{0.5cm}} \\
	{\fontsize{40pt}{11pt}\fonthl \@title} \vspace{0.2cm} \\ {\fontsize{20pt}{11pt}\fonthl \@subtitle} \\
	\vspace{1.5cm} \\
	{\begin{tabular}{ll}
	Laboratory	& Programming Methods Laboratory, LAMP, EPFL \\
	Professor		& \href{mailto:martin.odersky@epfl.ch}{Martin Odersky} \\
	Supervisors	& \href{mailto:vojin.jovanovic@epfl.ch}{Vojin Jovanovic}, \href{mailto:manohar.jonnalagedda@epfl.ch}{Manohar Jonnalagedda}   \\
	Expert		& \href{mailto:mirco.dotta@typesafe.com}{Mirco Dotta}, Typesafe \\
	Student		& \href{mailto:thierry.coppey@epfl.ch}{Thierry Coppey} \\
	Semester		& Autumn 2012 \\
	\end{tabular}}
	\my@heading
	\newpage
	%\end{titlepage}
}
\makeatother

% appendix
\makeatletter
\let\origappendix\appendix
\renewcommand\appendix{\clearpage\pagenumbering{Roman}\origappendix\section*{\appendixname}\lstset{frame=tb,numbers=left}}
\makeatother

% default
\author{\href{mailto:thierry.coppey@epfl.ch}{Thierry Coppey}, \href{mailto:manohar.jonnalagedda@epfl.ch}{Manohar Jonnalagedda}} %, \href{mailto:nithin.george@epfl.ch}{Nithin George}


\title{DynaProg for Scala}
\subtitle{A Scala DSL for Dynamic Programming on CPU and GPU}
\begin{document}
\maketitle
%\shorttitle

\subsection*{Abstract}
Dynamic programming is a common pattern of Computer Science used in various domains. Yet underlying matrix recurrences might be difficult to express and error prone. Additionally, domain experts might not have the skills to make an efficient parallel implementation. In this project, we present \textit{DynaProg}, a Scala DSL for dynamic programming on heterogeneous platforms which allow to write concise programs and execute them efficiently on GPUs.

Existing work is a DSL embedded in Haskell \cite{adp} with possible conversion to CUDA code \cite{adp_gpu}, a compiler for a dynamic programming external DSL into C code \cite{gapc} or ad-hoc CUDA implementations for specific problem classes \cite{swat_mega}, \cite{gpu_atlp}.
% XXX: How we compare to other, how to evaluate
% Benchmark => prove by evaluation intro statements

Our contributions are: \ul
%\item A classification of DP problems characteristics (matrix shape, dependency graph, ...)
\item A systematic approach to process data (top-down/bottom-up) and backtracking information (focus on running time and memory efficiency)
\item A language embedded in Scala (DSL) to express DP problems concisely (based on ADP)
\item Two implementations: Scala for CPU (features) and an CUDA for GPU (efficiency)
%\item Reuse of existing compiler technology (fusion) for a specific purpose
%\item State of the art parallel implementation of these classes on GPUs
%\item Normalization of the grammar into efficient productions
%\item Code generator to transform a grammar into efficient code for CPU, GPU (and FPGA)
\ule

\vfill
This project has been achieved in collaboration with Manohar Jonnalagedda. I also would like to thank the LAMP team, including Eugene Burmako, Andro Stucki, Vojin Jovanovic and Tiark Rompf who provided insightful advices and suggestions. I hope you will enjoy your reading. \vspace{.3cm}\\
\textit{Thierry Coppey}

\newpage
\setcounter{tocdepth}{2} \tableofcontents

\subsection*{Abstract}
Dynamic programming is an algorithmic technique to solve problems that follow the Bellman's principle\cite{bellman_principle}: optimal solutions depends on optimal sub-problem solutions. The core idea behind dynamic programming is to memoize intermediate results into matrices to avoid multiple computations. Solving a dynamic programming problem consists of two phases: filling one or more matrices with intermediate solutions for sub-problems and recomposing how the final result was constructed (backtracking). In the textbooks, problems are usually described in terms of recurrence relations between matrices elements. However, matrices computation formulae and recurrences might be difficult to express, indexing element is often error prone and writing an efficient parallel implementation can take a significant amount of time.

In this project, we present \textit{DynaProg}, a language embedded in Scala (DSL) to address dynamic programming problems on heterogeneous platforms. DynaProg allows the programmer to write concise programs based on ADP \cite{adp}, using a pair of parsing grammar and algebra; these program can then be executed either on CPU or on GPU. We evaluate the performance of our implementation against existing work and own ad-hoc implementations for both the CPU and GPU versions. Experimental results show an average speedup of {\color{red} XXX} for large problems when they are run on the GPU (compared to CPU). Compared to ad-hoc GPU implementations, the generated parsers have an average slowdown of {\color{red} XXX}.
The CPU implementation has a slowdown of {\color{red} XXX} for {\color{red} PROBLEM} against {\color{red} REF,GAPC?}.
The GPU implementation has a slowdown of {\color{red} XXX} for {\color{red} PROBLEM} against {\color{red} REF,CudAlign, RNAFold}.

% Paper introduction:
% Problem to solve, what exists (related work) and how do we compare to other
% Contributions (3): 3 tensed sentences
% Benchmarks => evaluation metrics, prove introduction statements by evaluation
 
\vfill
This project has been achieved in collaboration with Manohar Jonnalagedda. I also would like to thank the LAMP team, including Eugene Burmako, Sandro Stucki, Vojin Jovanovic and Tiark Rompf who provided insightful advice and suggestions. I hope you will enjoy your reading. \vspace{.3cm}\\
\textit{Thierry Coppey}

% ------------------------------------------------------------------------------------------------
\newpage
\setcounter{tocdepth}{2} \tableofcontents
\newpage
\section{Introduction}
Dynamic programming (DP) is a technique used to solve combinatorial optimization problems that verify the Bellman's principle\cite{bellman_principle}: optimal solutions depends on optimal solutions of sub-problems. Dynamic programming often allows to find solutions in an exponential search space (number of solutions) in polynomial running time. The user is usually interested in one optimal solution of the problem, but he might also desire to have \textit{all} optimal solutions (co-optimal), a fixed number of near-optimal solutions, or some synthetic properties of the search space such as its size or the sum of all scores.

Dynamic programming problems arise in several disciplines of applied Computer Science such as biosequence analysis, natural language processing and operational research: sequence alignment, RNA sequence folding or expression parenthesisation are example of such problems which are usually described by matrix recurrence relations. Unfortunately, these problems often appear in multiple variations and with a considerable degree of sophistication such that there is a mismatch between the textbook solution and its concrete implementation. Additionally, debugging is tedious and requires a lot of time, and little changes in the formulae might imply large rewrites of the matrices and recurrences\cite{gapc_yield}.

Dynamic programming computation is usually split in two phases: first intermediate scores are memoized in a matrix (tabulation), and reused to construct scores for larger problems; then a backtrack stage retrieves the solution associated with the optimal score for the problem. This solution describes how to obtain the optimal score and is called trace or backtrack trace, and heavily depends on the matrix design. The obtained information then needs to be presented in a form that makes sense for the user (or  might drive further computations). These two phases need to be kept consistent with each other, thereby introducing a new potential source of errors.

Finally, once the implementation is correct, it is possible to turn it into an efficient implementation for specific architectures such as multi-CPU, GPU or programmable hardware (FPGA). However, a domain specialist who writes the recurrences might not be very familiar with these platforms, whereas parallelization and hardware experts might not deeply understand the domain of the dynamic programming recurrences.

% http://en.wikipedia.org/wiki/Regular_tree_grammar: language class between regular languages and the deterministic context-free languages.
To solve these concerns, Algebraic Dynamic Programming (ADP) \cite{adp} proposes a language-independent declarative approach that separate the concerns of dynamic programming algorithms into four distinct components that are tightly connected:\ol
\item The search space is described using a tree \textbf{parsing grammar} that describes how to construct intermediate candidates whose score might be inserted in the matrix.
\item Constructed candidates are then evaluated by a \textbf{scoring function} (where all these functions form an \textbf{algebra}), so that they can be compared appropriately.
\item The \textbf{objective function} (or aggregation function) operates on the scores previously obtained to retain valid candidates.
\item Finally, results are \textbf{tabulated} (memoized in an array) in corresponding matrices. Tabulation process regulates the trade-off between running time and space efficiency by memoizing appropriate results that are reused multiple times.
\ole

By using a parsing grammar, ADP makes the candidate structure explicit. A signature serves as interface between the grammar, the scoring algebra and the aggregation function making possible that different grammars share different algebra or vice versa. Tabulation indices issues are hidden from the programmer, thereby removing potential errors. Finally, since the expression of the dynamic program is formalized and abstracted into a grammar and algebra, it becomes possible to convert it to efficient recurrences for many-core platforms such as GPUs. \cite{adp_gpu}

% -----------------------------
DynaProg implements the concepts of ADP in Scala as an embedded DSL (domain-specific language) with a syntax similar to the combinators parsers of Scala library\footnote{See \url{http://www.scala-lang.org/api/current/index.html\#scala.util.parsing.combinator.Parsers}}. It extends ADP by allowing grammars for pairing two sequences (multi-track grammars) similarly as GAPC\cite{gapc_thesis}, simplifies the process of writing programs by inferring additional information (\S\ref{yield_analysis}) and can translate them into efficient CUDA\footnote{Compute Unified Device Architecture: a parallel computing platform and programming model created by NVIDIA, supported by their graphics processing units (GPUs).} program that are competitive to their handwritten counterpart (\S\ref{benchmarks}). Since the program structure is formalized in ADP framework, it can be analyzed to remove unused grammar productions (dead code elimination) and avoid some non-termination issues; since it is generated, correct scheduling is guaranteed and indices errors are avoided, thereby producing an arguably more reliable program.

DynaProg provides a generic way of backtracking the results, such that the same backtrack trace can be used with multiple algebras if they share the same grammar. This allows to construct a two step pattern for solving problems: first the DP problem is solved using the appropriate cost function; then from the backtrack of its optimal, the desired result is computed. As example, consider multiplying a chain\footnote{Assuming matrices are of appropriated dimension to be multiplied with each other} of matrices efficiently: first, optimal execution scheduling (or parenthesization) trace is found using dynamic programming and cost algebra (\S\ref{mat_mult_plain}). The backtrack trace is then used (with a multiplication algebra) to multiply the actual matrices.

Finally, offloading dynamic programming computations to CUDA devices has been made effortless for the programmer: it suffices to enable code generation to schedule dynamic compilation and execution of the GPU-optimized program, as if it was executed in plain Scala.

This project resulted is an open-source\footnote{\url{https://github.com/manojo/lamp-dp-mt}} implementation of dynamic programming parsers for Scala optimized for both on CPU and GPU and featuring several analyses (\S\ref{architecture}) to ease the writing of dynamic programs. Its contribution is an automated approach to encode and process backtracking information such that the reconstruction complexity is reduced compared to \cite{gapc_thesis} and backtrack trace can be exchanged among different algebras sharing the same grammar. 

The rest of the document consists of:\ul
\item A brief background on dynamic programming, followed by an explanation of some of the Scala programming language and LMS framework features (\S\ref{background}).
\item A classification of DP problems in terms of matrix shape and dependencies, followed by a detailed analysis of some specific problems is provided (\S\ref{problems}). Related work addressing dynamic programming challenges is presented in (\S\ref{related_work}).
\item A description of the whole parser stack (\S\ref{architecture}), going from the user facing language (\S\ref{user_lang}, \S\ref{adp_grammar}) to optimizations (\S\ref{recurrences}, \S\ref{backtracking}) and implementation constraints (\S\ref{normalization}, \S\ref{memory_constr}), describing all the architectural decisions we made.
\item The concrete implementation of these ideas (\S\ref{implementation}) in the form of a DSL for Scala (\S\ref{scala_parsers}) and in efficient CUDA code generation (\S\ref{codegen}).
\item A brief usage explanation detailing the available features for the DSL user (\S\ref{usage}).
\item An evaluation of the performance of our work by providing appropriate benchmarks against existing implementations (\S\ref{benchmarks}).
\ule

%\item Propose a systematic approach to encode backtracking information such that the backtracking process can be made linear to the size of the problem
%\item Provide an concrete implementation in the form of a language embedded DSL in Scala, leveraging the grammar and algebra concepts of ADP
%\item Describe two implementations: Scala for CPU (focusing on multiple backtracking) and an CUDA for GPU (focusing on efficiency)

%The contributions of this project are: \ul
%\item A classification of dynamic programming problems characteristics in terms of matrix shape and recurrence formulae dependencies.
%\item A systematic approach to convert a top-down recurrence description (grammar) into efficient bottom-up 
%\item A systematic approach to process backtracking information (focus on running time and memory efficiency)
%\item A language embedded in Scala (DSL) to express DP problems concisely (based on ADP)
%\item Two implementations: Scala for CPU (features) and an CUDA for GPU (efficiency)
%%\item Reuse of existing compiler technology (fusion) for a specific purpose
%%\item State of the art parallel implementation of these classes on GPUs
%%\item Normalization of the grammar into efficient productions
%%\item Code generator to transform a grammar into efficient code for CPU, GPU (and FPGA)
%\ule

% ------------------------------------------------------------------------------------------------
\newpage
\section{Background} \label{background}
\subsection{Dynamic programming} \label{bg_dp}
Dynamic programming consists of solving a problem by constructing its solution from solutions to subproblems. A famous example of dynamic programming is the Fibonacci series that is defined by the following recurrence:
\[F(n+1) = F(n)+F(n-1) \qquad \text{ with } F(0)=F(1)=1 \]
which expands to (first 21 numbers)
\[1, 1, 2, 3, 5, 8, 13, 21, 34, 55, 89, 144, 233, 377, 610, 987, 1597, 2584, 4181, 6765, 10946, ...\]

A typical characteristic is that an intermediate solution is reused multiple times to construct larger solutions (here $F(3)$ helps constructing $F(4)$ and $F(5)$). Reusing an existing solution avoids redoing expensive computations: with memoization (memorizing intermediate results), the solution for $F(n)$ would be obtained after $n$ additions whereas without memoization it requires $F(n)-1$ additions !

Formally, dynamic programming problems respect the Bellman's principle of optimality\cite{bellman_principle}: \textit{<<An optimal policy has the property that whatever the initial state and initial decision are, the remaining decisions must constitute an optimal policy with regard to the state resulting from the first decision>>}\footnote{\url{http://en.wikipedia.org/wiki/Bellman\_equation\#Bellman.27s\_Principle\_of\_Optimality}}. This means that every intermediate result needs to be computed only once, although it might be reused as a basis for multiple larger problems, hence our first observation.

There exist various categories of dynamic programming:\ul
\item Series that operate usually on a single dimension (like Fibonacci)
\item Sequences alignment (matching two sequences at best), top-down grammar analysis (parenthesizing), sequence folding, ... (see \S\ref{problems} for more examples and detailed classification)
\item Tree-related algorithms: phylogenetic, trees raking, maximum tree independent set, ... (can be viewed as a sparse version of the second category)
\ule

Since the first category operates on a single dimension, to benefit of the smaller solutions to compute larger ones, elements must be computed sequentially (one at a time), hence computations cannot be made parallel (unless duplicated, thereby hindering benefits of memoization). The third category suffers from limited parallelism \cite{philogeny} and its implementation does not share much with the previous category, hence we focus on the second type of problems.

Taking real-world examples in biology, the average input size for sequence alignment (\S\ref{swat_affine}) is around 300K whereas for problems like RNA folding (\S\ref{zuker}), input are usually around few thousands. Problems operating on multiple input sequences also require more memory: for instance matching 3 sequences is $O(n^3)$-space complex (as intermediate results needs to be stored in a position representing the progress in each of the involved sequence). Since we target a single computer with one or more attached devices (GPUs, FPGAs), and since we plan to maintain data in memory (due to the multiple reuse of intermediate solutions) the storage complexity must be relatively limited, compared to other problem that could leverage the disk storage. Hence in general, we focus on problems that have $O(n^2)$-space complexity whereas time complexity is usually $O(n^3)$ or larger. We encourage you to refer to \S\ref{problems} for further classification and examples.

% ------------------------------------------------------------------------------------------------
\newpage
\subsection{Scala} \label{bg_scala}
\textit{<<Scala is a general purpose programming language designed to express common programming patterns in a concise, elegant, and type-safe way. It smoothly integrates features of object-oriented and functional languages, enabling programmers to be more productive. Many companies depending on Java for business critical applications are turning to Scala to boost their development productivity, applications scalability and overall reliability.>>}\footnote{\url{http://www.scala-lang.org}}

As the Scala \cite{scala} programming language is developed by our laboratory (LAMP, EPFL), it seems natural to use it as host language for our project, however, we would list some of its features \cite{scala_api} that makes it an interesting development language for this project:\ul
\item The functional programming style and syntactic sugar offered by Scala allow concise writing of implementation, analysis and transformations of our DSL, allowing us to focus on \textit{what} we want to achieve instead of \textit{how}.
\item Scala is largely adopted in the industry\footnote{\url{http://www.scala-lang.org/node/1658}}, which makes both the adoption of related project easier and reduces the learning time of its potential users.
\item Since Scala programs execute in the Java Virtual Machine (JVM), they can benefit of the native interface (JNI) that offers the possibility to dynamically load libraries (usually written in C) and possibly interact with CUDA to leverage the GPU.
\item Scala is equipped with a strong typing and type inference system that reduces the syntactical constraints while putting strong guarantees on type correctness at compilation.
\item Implicit functions definition can help interfacing functions by automatically changing the type of passed arguments
\item Manifests (or TypeTags and ClassTags) allow type extraction at runtime (we use this to convert a Scala type into a C/CUDA type)
\item Macros\cite{scala_macros} and LMS (\S\ref{bg_lms}) could be used to modify the semantics of a specific part of the user program (we currently use LMS to produce the corresponding C function).
\item One Scala concept that we heavily use is \textit{traits} that can be viewed as abstract classes and mixed together, thereby allowing multiple inheritance.
\ule

% ------------------------------------------------------------------------------------------------
\subsection{Lightweight Modular Staging} \label{bg_lms}
Lightweight Modular Staging (LMS) \cite{lms}, \cite{lms_thesis} is a runtime code generation built on top of Scala virtualized \cite{scala_virtualized} that uses types to distinguish between binding time (compilation and runtime) for code compilation. This makes possible to annotate parts of the code with special types, such that their compilation is delayed until the program is executed. At run time, these parts are represented as a \textit{sea of nodes} that serve as the basis for another compilation phase where all the code executed until this point provides additional information to produce a more efficient compilation. The process of delaying the compilation is known as \textit{lifting} whereas \textit{lowering} corresponds to transforming this intermediate representation into executable code. LMS code generation is not limited to Scala, it can also target other languages like C. In short, LMS is an optimizing compiler framework that allows integration of domain-specific abstractions and optimizations into the generation process.

A discussion of the integration of LMS in our project can be found in \S\ref{lms_use}.

\newpage
\section{Dynamic programming problems}
% ------------------------------------------------------------------------------------------------
\subsection{Problems classification}
\subsubsection{Definitions}\ul
\item \textbf{Dimensions:} let $n$ the size of the input and $d$ the dimension of the underlying matrix.
\item \textbf{Matrices:} we refer indifferently by the matrix or the matrices to all the intermediate cost- and backtrack-related informations that are necessary to solve the dynamic programming problem of interest. Matrices elements are usually denoted by $M_{(i,j)}$ ($i^{\rm th}$ line , $j^{\rm th}$ column).
\item \textbf{Computation block:} this is a part of the DP matrix (cost and or backtrack) that we want to compute. A block might be either a sub-matrix (rectangular) or a parallelogram, possibly cropped at its parent matrix boundaries.
\item \textbf{Wavefront:} the wavefront consists of all the data necessary to reconstruct a computation block of the DP matrix. It might include some previous lines/columns/diagonals as well as line-/column-/diagonal-wise aggregations (min, max, sum, ...).
\item \textbf{Delay:} we call delay the maximum distance between an element and its dependencies along column and lines (ex: recurrence $M_{(i,j)}=f\big(M_{(i-1,j)}, M_{(i-2,j-1)}\big)$ has delay 3).
\ule

\subsubsection{Litterature classification}
In the literature, dynamic programming problems (DP) are classified according to two criteria:\ul
\item \textbf{Monadic/polyadic:} a problem is monadic when only one of the previously computed term appears in the right hand-side of the recurrence formula (ex: Smith-Waterman). When two or more terms appear, the problem is polyadic (ex: Fibonacci, $F_n = F_{n-1} + F_{n-2}$).
When a problem is polyadic with index $p$, it also means that its backtracking forms a $p$-ary tree (where each node has at most $p$ children).

\item \textbf{Serial/non-serial:} a problem is serial ($s=0$) when the solutions depends on a fixed number of previous solutions (ex: Fibonacci), otherwise it is said to be non-serial ($s\ge 1$), as the number of dependencies grows with the size of the subproblem. That is computing an element of the matrix would require $O(n^s)$.  (ex: Smith-Waterman with arbitrary gap is $s=1$; we can usually infer $s$ from the number of bound variables in the recurrence formula)
	\[M_{(i,j)}=\max\left\{\begin{array}{l} ... \\ M_{(i,j-1)}\\ \max\limits_{i<k<j} [ M_{(i,k)}+M_{(k+1,j)} ] \end{array}\right. \]
\ule

Note that the algorithmic complexity of a problem is exactly $O\big(n^{d+s}\big)$.

\subsubsection{Calculus simplifications}
In some special case, it is possible to transform a non-serial problem into a serial problem, if we can embed the non-serial term into an additional aggregation matrix. For example:
	\[M_{(i,j)}=\max\left\{\begin{array}{l} \max\limits_{k<i} M_{(k,j)}
	\\ \sum\limits_{k<i, l<j}M_{(k,l)} \end{array}\right.
	\implies M_{(i,j)}=\max\left\{\begin{array}{l} C_{(k,j)} \\ A_{(i-1,j-1)} \end{array}\right.\]
Where the matrix $C$ stores the maximum along the column and matrix $A$ stores the sum of the array of the previous elements. Both can be easily computed with an additional recurrence:
	\[\begin{array}{rcl} C_{(i,j)}&=&\max(C_{(i-1,j)}, M_{(i,j)}) \\
	A_{(i,j)}&=&A_{(i-1,j)}+A_{(i,j-1)}-A_{(i-1,j-1)}+M_{(i,j)}\end{array}\]

Although this simplification removes some non-serial dependencies at the cost of extra storage in the wavefront, it is not sufficient to transform all non-serial monadic problems into serial problems (ex: this does not apply to Smith-Waterman with arbitrary gap cost).

% ------------------------------------------------------------------------------------------------
\subsection{Problems of interest}
We usually focus on problem that have an underlying bi-dimensional matrix ($d=2$) because they can be parallelized (as opposed to be serial if $d=1$) and can solve large problems (of size $n$). Problems of higher matrix dimensionality ($d\ge3$) require substantial memory which severely impacts their scalability. Also we tend to limit algorithmic complexity of the problems as from $O(n^4)$ on, running time becomes a severely limiting factor.

We describe problems structures: inputs, cost matrices and backtracking matrix. These all have an alphabet (that must be bounded in terms of bit-size). Unless otherwise specified, we adopt the following conventions:\ul
\item Matrices dimensions are implicitly specified by number of indices and their number of elements is usually the same as the input length.
\item Number are all unsigned integers
\item Problem dimension is $m,n$ (or $n$) indices $i,j$ ranges are respectively $0\le i<m$, $0\le j<n$.
\item Unless otherwise specified, the recurrence applies to all non-initialized matrix elements.
\ule
We describe the problem processing in terms of both initialization and recurrences.

% ------------------------------------------------------------------------------------------------
% Recurrence visualization helpers
\newcommand\Cd[3][0,-1]{\put(#2){\put(.5,.5){\circle*{.3}}\put(.5,.5){\linethickness{1.5pt}\vector(#1){#3}}}} % dependency [dx,dy]{x,y}{len}
\def\Cg#1{\put(#1){\color{lightgray}\put(0,0){\polygon*(0,0)(0,1)(1,1)(1,0)}}} % grayed cell (not to store
\def\Cz#1{\put(#1){\put(0,.35){\parbox{1\unitlength}{\centering\bf 0}}}} % zero-init cell
\def\Cm{\put(6.5,4.5){\circle*{.4}}\multiput(0,0)(1,0){9}{\line(0,1){8}}\multiput(0,0)(0,1){9}{\line(1,0){8}}} % matrix base
\def\Cfl#1{#1{0,6}#1{0,5}#1{1,5}#1{0,4}#1{1,4}#1{2,4}#1{0,3}#1{1,3}#1{2,3}#1{3,3}#1{0,2}#1{1,2}#1{2,2}#1{3,2}#1{4,2}
	#1{0,1}#1{1,1}#1{2,1}#1{3,1}#1{4,1}#1{5,1}#1{0,0}#1{1,0}#1{2,0}#1{3,0}#1{4,0}#1{5,0}#1{6,0}} % triangular lower (function)
\def\Cfd#1{#1{0,7}#1{1,6}#1{2,5}#1{3,4}#1{4,3}#1{5,2}#1{6,1}#1{7,0}} % main diagonal

\def\Cmlong{\put(6.5,4.5){\circle*{.4}}\multiput(0,0)(1,0){16}{\line(0,1){8}}\multiput(0,0)(0,1){9}{\line(1,0){15}}} % matrix base
\def\Cfu#1{#1{8,7}#1{9,7}#1{10,7}#1{11,7}#1{12,7}#1{13,7}#1{14,7}#1{9,6}#1{10,6}#1{11,6}#1{12,6}#1{13,6}#1{14,6}#1{10,5}#1{11,5}#1{12,5}#1{13,5}#1{14,5}#1{11,4}#1{12,4}#1{13,4}#1{14,4}#1{12,3}#1{13,3}#1{14,3}#1{13,2}#1{14,2}#1{14,1}} % triangular upper (function)

% ----------------------------------------------
\newpage
\subsubsection{Smith-Waterman (simple)}\label{sswat}\ol
\item Problem: matching two strings $S$, $T$ with $|S_{\rm padded}|=m, |T_{\rm padded}|=n$.
\item Matrices: $M_{m \times n}, B_{m \times n}$
\item Alphabets:\ul
	\item Input: $\Sigma(S)=\Sigma(T)=\{a,c,g,t\}$.
	\item Cost matrix: $\Sigma(M) = [0..z], z=\max({\rm cost(\_)}) \cdot \min(m,n)$
	\item Backtrack matrix: $\Sigma(B)=\{stop,W,N,NW\}$
	\ule
\item Initialization:\ul
	\item Cost matrix: $M_{(i,0)}=M_{(0,j)}=0$.
	\item Backtrack matrix: $B_{(i,0)}=B_{(0,j)}=stop$.
	\ule
\item Recurrence: \[M_{(i,j)}=\max\left\{\begin{array}{l|l}
		0 & stop\\
		M_{(i-1,j-1)}+{\rm cost}(S(i),T(j)) & NW\\
		M_{(i-1,j)}-d & N\\
		M_{(i,j-1)}-d & W
	\end{array}\right\}=B_{(i,j)} \]

\item Backtracking: starts from the cell $\max \{M_{(m,j)} \cup M_{(i,n)}\}$, stops at
the first cell containing a $0$.
\item Visualization: by convention, we put the longest string vertically ($m\ge n$):
\begin{center}\setlength{\unitlength}{.6cm}\begin{picture}(8,9)
	\put(-.5,7.5){S}\put(-.35,7.4){\linethickness{1pt}\vector(0,-1){2}}
	\put(.2,8.2){T}\put(.8,8.4){\linethickness{1pt}\vector(1,0){2}}
	\Cz{0,0}\Cz{0,1}\Cz{0,2}\Cz{0,3}\Cz{0,4}\Cz{0,5}\Cz{0,6}\Cz{0,7}
	\Cz{1,7}\Cz{2,7}\Cz{3,7}\Cz{4,7}\Cz{5,7}\Cz{6,7}\Cz{7,7}
	\Cd{6,5}{0.8}
	\Cd[1,0]{5,4}{0.8}
	\Cd[1,-1]{5,5}{0.8}
\Cm\end{picture}\end{center}

\item Optimizations:\ul
	\item In serial (monadic) problems we can avoid building the matrix $M$ by only maintaining the 3 last diagonals in memory (one for the diagonal element, one for horizontal/vertical, and one being built). This construction extends easily to polyadic problems where we need to maintain $k+2$ diagonals in memory where $k$ is the maximum backward lookup.
	\item Padding: since first line and column of the matrix are zeroes, their initialization might be omitted, but this would implies more involved initialization and computations, which is cumbersome. Also since to fill the $i^{\rm th}$ row we refer to the $(i-1)^{\rm th}$ character of string $S$ thus we prepend to both $S$ and $T$ an unused character, so that matrix and input lines are aligned. Hence valid input indices are $S[1 \cdots m-1]$ and $T[1 \cdots n-1]$. We refer as such strings as padded strings hereafter (with $|S_{\rm padded}| = |S| + 1$).
	\ule
\ole

% ----------------------------------------------
\newpage
\subsubsection{Smith-Waterman with affine gap extension cost}\ol
\item Problem: matching two strings $S$, $T$ with $|S_{\rm padded}|=m, |T_{\rm padded}|=n$.
\item Matrices: $M_{m \times n}, E_{m \times n}, F_{m \times n}, B_{m \times n}$
\item Alphabets:\ul
	\item Input: $\Sigma(S)=\Sigma(T)=\{a,c,g,t\}$.
	\item Cost matrices: $\Sigma(M) = \Sigma(E) = \Sigma(F) = [0..z], z=\max({\rm cost(\_)}) \cdot \min(m,n)$
	\item Backtrack matrix: $\Sigma(B)=\{stop,W,N,NW\}$
	\ule
\item Initialization:\ul
	\item No gap cost matrix: $M_{(i,0)}=M_{(0,j)}=0$.
	\item T-gap extension cost matrix: $E_{(i,0)}= 0$ \textit{<<eat S chars only>>}
	\item S-gap extension cost matrix: $F_{(0,j)}= 0$
	\item Backtrack matrix: $B_{(i,0)}=B_{(0,j)}=stop$.
	\ule
\item Recurrence for the cost matrices:
\[\begin{array}{rcl}
M_{(i,j)}&=&\max\left\{\begin{array}{l|l}
	0 & stop\\
	M_{(i-1,j-1)}+{\rm cost}(S(i),T(j)) & NW\\
	E_{(i,j)} & N\\
	F_{(i,j)} & W
\end{array}\right\}=B_{(i,j)}\\
\\
E_{(i,j)}&=&\max\left\{\begin{array}{l|l}
	M_{(i, j-1)} - \alpha & NW\\
	E_{(i,j-1)} - \beta & N\\
\end{array}\right\}=B_{(i,j)}\\
\\
F_{(i,j)}&=&\max\left\{\begin{array}{l|l}
	M_{(i-1,j)} - \alpha & NW\\
	F_{(i-1,j)} - \beta & W\\
\end{array}\right\}=B_{(i,j)}
\end{array}\]

That can be written alternatively as:
\[M_{(i,j)}=\max\left\{\begin{array}{l|l}
	0 & stop\\
	M_{(i-1,j-1)}+{\rm cost}(S(i),T(j)) & NW\\
	\max_{1 \le k \le j-1} M_{(i,k)} - \alpha - (j-1-k) \cdot \beta & N\\
	\max_{1 \le k \le i-1} M_{(k,j)} - \alpha - (i-1-k) \cdot \beta & W\\
\end{array}\right\}=B_{(i,j)} \]

Although the latter notation seems more explicit, it introduces non-serial dependencies that the former set of recurrences is free of. So we need to implement the former rules whose kernel is 
\[ [M;E;F]_{(i,j)} = f_{\rm kernel} ( [M;E]_{(i,j-1)}, [M;F]_{(i-1,j)}, M_{(i-1,j-1)} ) \]
Notice that this recurrence is very similar to \nameref{sswat} except that we propagate 3 values ($M,E,F$) instead of a single one ($M$). Also notice that it is possible to propagate $E$ and $F$ inside a resp. horizontal and vertical wavefront.

\item Backtracking: same as \nameref{sswat}
\item Visualization: same as \nameref{sswat}
\item Optimizations: same as \nameref{sswat}
\ole

% ----------------------------------------------
\newpage
\subsubsection{Smith-Waterman with arbitrary gap cost}\label{aswat}\ol
\item Problem: matching two strings $S$, $T$ with $|S_{\rm padded}|=m, |T_{\rm padded}|=n$ with an arbitrary gap function $g(x)\ge 0$ where $x$ is the size of the gap. Without loss of generality, let $m\ge n$\footnote{Otherwise if $|T|>|N|$ we only need to swap both the inputs and backtracking pairs.}. Example penalty function could be\footnote{Intuition: long gaps penalize less, at some point, one large gap is better than matching and smaller gaps.} $g(x)=m-x$.
\item Matrices: $M_{m \times n}, B_{m \times n \times m}$
\item Alphabets:\ul
	\item Input: $\Sigma(S)=\Sigma(T)=\{a,c,g,t\}$.
	\item Cost matrix: $\Sigma(M) = [0..z], z=\max({\rm cost(\_)}) \cdot \min(m,n)$
	\item Backtrack matrix: $\Sigma(B)=\{stop,NW,N_{\{0..m\}},W_{\{0..n\}}\}$
	\ule

\item Initialization:\ul
	\item Match cost matrix: $M_{(i,0)}=M_{(0,j)}=0$.
	\item Backtrack matrix: $B_{(i,0)}=B_{(0,j)}=stop$.
	\ule

\item Recurrence: \[M_{(i,j)}=\max\left\{\begin{array}{l|l}
	0 & stop\\
	M_{(i-1,j-1)}+{\rm cost}(S(i),T(j)) & NW\\
	\max_{1 \le k \le j-1} M_{(i,j-k)} - g(k) & N_k\\
	\max_{1 \le k \le i-1} M_{(i-k,j)} - g(k) & W_k\\
\end{array}\right\}=B_{(i,j)} \]

\item Backtracking: similar to \nameref{sswat} except that you can jump of $k$ cells.
\item Visualization:
	\begin{center}\setlength{\unitlength}{.6cm}\begin{picture}(8,9)
		\put(-.5,7.5){S}\put(-.35,7.4){\linethickness{1pt}\vector(0,-1){2}}
		\put(.2,8.2){T}\put(.8,8.4){\linethickness{1pt}\vector(1,0){2}}
	\Cz{0,0}\Cz{0,1}\Cz{0,2}\Cz{0,3}\Cz{0,4}\Cz{0,5}\Cz{0,6}\Cz{0,7}
	\Cz{1,7}\Cz{2,7}\Cz{3,7}\Cz{4,7}\Cz{5,7}\Cz{6,7}\Cz{7,7}
		\Cd[0,-1]{6,7}{2.8}\Cd[0,-1]{6,6}{1.8}\Cd{6,5}{0.8}
		\Cd[1,0]{0,4}{5.8}\Cd[1,0]{1,4}{4.8}\Cd[1,0]{2,4}{3.8}\Cd[1,0]{3,4}{2.8}\Cd[1,0]{4,4}{1.8}\Cd[1,0]{5,4}{0.8}
		\Cd[1,-1]{5,5}{0.8}
	\Cm\end{picture}\end{center}

\item Optimizations: The dependencies here are non-serial, there is no optimization that we can 
apply out of the box here.
\ole


% ----------------------------------------------
\newpage
\subsubsection{Convex polygon triangulation}\ol
\item Problem: triangulating a polygon of $n$ vertices with least total cost for added edges. We denote the cost of adding an edge between the  pair of edges $i,j$ by $S(i,j)$, Where $S_{n \times n}$ is a lower triangular matrix compacted in memory (rows are contiguous) with a 0 diagonal that is omitted \footnote{Arbitrary convention for both architectural implementation and code generator. Rationale: in lower triangular matrix, element address is independent of the matrix size.}, hence $|S|=\tfrac{n^2}{2}=N$.

\item Matrices: $M_{n\times 2n}, B_{n \times 2n}$ \textit{<<first edge, last edge>>} upper triangular including main diagonal 
\item Alphabets:\ul
	\item Input: $\Sigma(S_{(i,j)})=\{0..m\}$ with $m=\max_S(i,j) \forall i,j$ determined at runtime\footnote{We need to scan/have stats about $S$ and that's where LMS plays a role}.
	\item Cost matrix: $\Sigma(M)=\{0..z\}$ with $z = m \cdot (n-2)$ (we add at most $n-2$ edges).
	\item Backtrack matrix: $\Sigma(B)=\{stop, 0..n\}$ (the index of the edge we add)
	\ule
\item Initialization: $M_{(i,i)}=0, M_{(i,i+1)}=0, B_{(i,i)}=stop \quad\forall i$

\item Recurrence: \[M_{(i,j)}=\left\{ S(i,j) + \max_{i<k<j}M_{(i,k)}+M_{(k,j)} \,\,\rule[-.75em]{.5pt}{2em}\,\,  k \right\} = B_{(i,j)} \]
	It is interesting to note that even in the sequential world, this problem is solved 
	by filling the diagonals, ie. computing sub-solutions for all polygons of size $k$ before
	those of size $k+1$.

\item Backtracking: Start at $B_{(1,n)}$. Use the following recursive function for the smaller polygons:
	\[{\rm BT}(B_{(i,j)}=k) \mapsto \left\{\begin{array}{ll} A_i & \text{if } k=0 \lor k=j \\
		\Big( {\rm BT}(B_{(i,k)}) \Big) \cdot \Big( {\rm BT}(B_{(k+1,j)}) \Big) & \text{otherwise} \end{array}\right.\]

\item Visualization: the layout is the a matrix of size $n \times (2n-2)$, because of polygons being 
"cyclical" in nature.
\begin{center}\setlength{\unitlength}{.6cm}\begin{picture}(16,9)
	\put(-.7,6.5){\rotatebox{90}{First}}\put(-.4,6.4){\linethickness{1pt}\vector(0,-1){2}}
	\put(.2,8.2){Last}\put(1.5,8.4){\linethickness{1pt}\vector(1,0){2}}
	\Cfl{\Cg}\Cfd{\Cz}\Cfu{\Cg}
	\Cd[0,1]{6,1}{2.8}\Cd[0,1]{6,2}{1.8}\Cd[0,1]{6,3}{0.8}
	\Cd[1,0]{3,4}{2.8}\Cd[1,0]{4,4}{1.8}\Cd[1,0]{5,4}{0.8}
	\put(3.5,4.5){\line(3,-1){3}}
	\put(4.5,4.5){\line(2,-2){2}}
	\put(5.5,4.5){\line(1,-3){1}}
	%\Cd[1,-1]{5,2}{0.8}
\Cmlong\end{picture}\end{center}

\item Optimizations: we need to rotate that matrix to progress in the same direction as usual, that is towards bottom right.
{\color{red}}
\ole

% ----------------------------------------------
\newpage
\subsubsection{Matrix chain multiplication}\ol
\item Problem: find an optimal parenthesizing of the multiplication of $n$ matrices $A_i$. Each matrix $A_i$ is of dimension $r_i \times c_i$ and $c_i=r_{i+1} \forall i$. \textit{<<r=rows, c=columns>>}
\item Matrices: $M_{n \times n}, B_{n \times n}$ \textit{(first, last matrix)}
\item Alphabets:\ul
	\item Input: matrix $A_i$ size is defined as pairs of integers $(r_i,c_i)$.
	\item Cost matrix: $\Sigma(M)= 1..z$ with $z\le n\cdot \big[ \max_i(r_i,c_i) \big]^3 $.
	\item Backtrack matrix: $\Sigma(B)=\{stop\} \cup \{0..n\}$.
	\ule
\item Initialization:\ul
	\item Cost matrix: $M_{(i,i)}=0$.
	\item Backtrack matrix: $B_{(i,i)}=stop$.
	\ule
\item Recurrence: $c_k=r_{k+1}$
	\[M_{(i,j)}=\min_{i\le k<j}\left\{\begin{array}{l|l}
		M_{(i,k)}+M_{(k+1,j)}+r_i \cdot c_k \cdot c_j & k
	\end{array}\right\}=B_{(i,j)} \]
\item Backtracking: Start at $B_{(1,n)}$. Use the following recursive function for parenthesizing
	\[{\rm BT}(B_{(i,j)}=k) \mapsto \left\{\begin{array}{ll} A_i & \text{if } k=0 \lor k=j \\
		\Big( {\rm BT}(B_{(i,k)}) \Big) \cdot \Big( {\rm BT}(B_{(k+1,j)}) \Big) & \text{otherwise} \end{array}\right.\]

\item Visualization:
	\begin{center}\setlength{\unitlength}{.6cm}\begin{picture}(8,9)
		\put(-.7,6.5){\rotatebox{90}{First}}\put(-.4,6.4){\linethickness{1pt}\vector(0,-1){2}}
		\put(.2,8.2){Last}\put(1.5,8.4){\linethickness{1pt}\vector(1,0){2}}
		\Cfl{\Cg}\Cfd{\Cz}
		\Cd[0,1]{6,1}{2.8}\Cd[0,1]{6,2}{1.8}\Cd[0,1]{6,3}{0.8}
		\Cd[1,0]{3,4}{2.8}\Cd[1,0]{4,4}{1.8}\Cd[1,0]{5,4}{0.8}
		\put(3.5,4.5){\line(3,-1){3}}\put(4.5,4.5){\line(2,-2){2}}\put(5.5,4.5){\line(1,-3){1}}
	\Cm\end{picture}\end{center}

\item Optimizations:\ul
	\item We need to swap vertically the matrix to have a normalized progression towards bottom right. To do that, we need to map all indices $i \mapsto n-1-i$ and we obtain a new recurrence relation:
	\[M_{(i,j)}=\min_{i\le k<j}\left\{\begin{array}{l} M_{(i,k)}+M_{(2i-1 -k,j)}+r_i \cdot c_k \cdot c_j \end{array}\right. \]
	With the initialization at $M_{(i,n-i-1)}$
	\ule
\ole

% ----------------------------------------------
\newpage
\subsubsection{Nussinov algorithm}\ol
\item Problem: folding a RNA string $S$ over itself $\left\lfloor |S| / 2 \right\rfloor = n$.
\item Matrices: $M_{n\times n}, B_{n \times n}$
\item Alphabets:\ul
	\item Input: $\Sigma(S)=\{A,C,G,U\}$.
	\item Cost matrix: $\Sigma(M)=\{0..n\}$
	\item Backtrack matrix: $\Sigma(B)=\{stop,D,1..n\}$
	\ule
\item Initialization: \ul
	\item Cost matrix: $ M_{(i,i)}=M_{(i,i-1)}=0$
	\item Backtrack matrix: $B_{(i,i)}=B_{(i,i-1)}=stop$
	\ule
\item Recurrences:
	\[M_{(i,j)}=\max\left\{\begin{array}{l|l}
		M_{(i+1,j-1)}+\omega(i,j) & D\\
		\max_{i\le k<j}M_{(i,k)}+M_{(k+1,j)} & k
	\end{array}\right\} = B_{(i,j)} \]
	With $\omega(i,j)=1$ if $i,j$ are complementary. 0 otherwise.
\item Backtracking: Start the backtracking in $B_{(1,n)}$ and go backward. The backtracking is very similar to that of the matrix multiplication, except that we also introduce the diagonal matching.
\item Visualization:
	\begin{center}\setlength{\unitlength}{.6cm}\begin{picture}(8,9)
		\put(-.7,6.5){\rotatebox{90}{First}}\put(-.4,6.4){\linethickness{1pt}\vector(0,-1){2}}
		\put(.2,8.2){Last}\put(1.5,8.4){\linethickness{1pt}\vector(1,0){2}}
		\Cfl{\Cg}\Cfd{\Cz}
		\Cd[0,1]{6,1}{2.8}\Cd[0,1]{6,2}{1.8}\Cd[0,1]{6,3}{0.8}
		\Cd[1,0]{3,4}{2.8}\Cd[1,0]{4,4}{1.8}\Cd[1,0]{5,4}{0.8}
		\Cd[1,1]{5,3}{0.8}
		\put(3.5,4.5){\line(3,-1){3}}\put(4.5,4.5){\line(2,-2){2}}\put(5.5,4.5){\line(1,-3){1}}
	\Cm\end{picture}\end{center}

\item Optimizations: note that this is very similar to the matrix multiplication except that we also need the diagonal one step backward, so the same optimization can apply.
\ole

% ----------------------------------------------
\newpage
\subsubsection{Zuker folding}\ol
\item Problem: folding a RNA string $S$ over itself $\left\lfloor |S| / 2 \right\rfloor = n$.
\item Matrices: $V_{n\times n}, W_{n\times n}, F_n$ (Free Energy),  $BV_{n \times n}, BW_{n \times n}, BF_n$
\item Alphabets:\ul
	\item Input: $\Sigma(S)=\{A,C,G,U\}$.
	\item Cost matrices:\ul
		\item $\Sigma(W)=\Sigma(V)=\{0..z\}$ with $z \le n*b+c$
		\item $\Sigma(F)=\{0..y\}$ with $y\le \min(F_0, z\cdot n)$
		\ule
	\item Backtrack matrices: \ul
		\item $\Sigma(BW)=\{stop, S,W,V,k\}$
		\item $\Sigma(BV)=\{stop, HL, IL, SW, (i,j) , k\}$ with $0\le i,j,k < n$ \\
		$HL$=HairpinLoop, $IL$=InteriorLoop, $(i,j)$=MultiLoop
		
		
		\item $\Sigma(BF)=\{stop, L, k\}$ with $0\le k < n$
		\ule
	\ule
\item Initialization:\ul
	\item Cost matrices: $W_{(i,i)}=V_{(i,i)}=0, F_{(0)}=$ energy of the unfolded RNA.
	\item Backtrack matrices: $BW_{(i,i)}=BV_{(i,i)}=BF_{(0)}=stop$.
	\ule
\item Recurrence:
\[\begin{array}{rcl}
W_{(i,j)}&=&\min\left\{\begin{array}{l|l}
	W_{(i+1,j)}+b & S\\
	W_{(i,j-1)}+b & W\\
	V_{(i,j)}+\delta(S_i,S_j) & V \\
	\min_{i<k<j}W_{(i,k)}+W_{(k+1,j)} &k
\end{array}\right\} = BW_{(i,j)}\\
\\
V_{(i,j)}&=&\min\left\{\begin{array}{l|l}
	\infty \qquad\qquad\qquad\qquad {\rm if}(S_i,S_j) \text{ is not a base pair} & stop\\\\
	eh(i,j)+b \qquad\qquad\, \text{otherwise} & HL\\
	V_{(i+1,j-1)}+es(i,j) & IL \\
	VBI_{(i,j)} & (i',j') \\
	\min_{i<k<j-1}\{W_{(i+1,k)}+W_{(k+1,j-1)}\} +c & k
\end{array}\right\} = BV_{(i,j)}\\
\\
VBI_{(i,j)}&=&\min\Big\{\min_{i<i'<j'<j}V_{(i',j')}+ebi(i,j,i',j')\} +c \,\,\Big|\,\, (i',j') \Big\}=BV_{(i,j)}\\
\\
F_{(j)}&=&\min\left\{\begin{array}{l|l}
	F_{(j-1)} & L \\ 
	\min_{1\le i< j} (F_{(i-1)} + V_{(i,j)}) & i
\end{array}\right\} = BF_{(j)}
\end{array}\]

In practice, we don't go backward for larger values than 30, so we can replace $\min_{i<k<j}$ by $\min_{\max(i,j-30)<k<j}$ in the expressions of $VBI$.

\item Backtracking: Start at $BF_{(n)}$ using the recurrences
 \[\begin{array}{rcl}
	BF_{(j)} &=& \left\{\begin{array}{rcl} L&\implies& BF_{(j-1)} \\ i &\implies& BF_{(i-1)} + BV_{(i,j)} \end{array} \right.\\
	\\
	BV_{(i,j)} &=& \color{red} \left\{\begin{array}{rcl}
		HL &\implies&\big< {\rm hairpin}(i,j) \big> \\
		IL &\implies& \big< {\rm stack}(i,j) \big> BV_{(i+1,j-1)} \\
		(i',j') &\implies& \big< \text{multi-loop from $(i,j)$ to }(i',j') \big> BV(i',j')\\
		k &\implies& BW_{(i+1,k)} BW_{(k+1,j-1)}
	\end{array}\right.\\
	\\
	BW_{(i,j)} &=& \color{red} \left\{\begin{array}{rcl}
	S & \implies & \big< bulge(i) \big> BW_{(i+1,j)} \\
	W & \implies & \big< bulge(j) \big> BW_{(i,j+1)} \\
	V &\implies& BV_{(i,j)} \\
	k &\implies& BW_{(i+1,k)} BW_{(k+1,j-1)}
	\end{array}\right.
\end{array}\]

\item Visualization: \begin{center}\includegraphics[width=8cm]{inc/zucker.pdf}\end{center}
	% source: <<Parallization of dynamic programming recurrences in computational biology>> paper
\item Optimizations: {\color{red} XXX: notice that there are 3 matrices: $W$,$V$ ($VBI$ is part of $V$) that can be expressed using regular matrix, and $F$ that is of different dimension than $W$ and $V$ and requires a special construction (in the wavefront?). We need to find a nice way to encode both its construction and backtrack into the existing framework (implement 1D DP recursively?)}
\ole

% ------------------------------------------------------------------------------------------------
\newpage
\subsection{Related problems}
The goal of this section is to demonstrate that our framework can accommodate with many {\color{red} (20-50)} problems that we have not considered at the design time.

\subsubsection{Serial problems}
\begin{tabular}{llcc} \toprule
\bf Problem & \bf Shape & \bf Matrices & \bf Wavefront \\ \midrule
Smith-Waterman \footnotesize simple & rectangle & 1 & -- \\
Smith-Waterman \footnotesize affine gap extension & rectangle & 3 & -- \\
\href{http://en.wikipedia.org/wiki/Needleman-Wunsch_algorithm}{Needleman-Wunsch} & rectangle & 1 & -- \\

\href{http://en.wikipedia.org/wiki/Dynamic_programming#Checkerboard}{Checkerboard} & rectangle & 1 & -- \\
\href{http://en.wikipedia.org/wiki/Longest_common_subsequence_problem\#Code_for_the_dynamic_programming_solution}{Longest common subsequence} & rectangle & 1 & -- \\
\href{http://en.wikipedia.org/wiki/Longest_common_substring_problem\#Pseudocode}{Longest common substring} & triangle & 1 & -- \\
\href{http://en.wikipedia.org/wiki/Levenshtein_distance\#Computing_Levenshtein_distance}{Levenshtein distance} & rectangle & 1 & -- \\
\href{http://en.wikipedia.org/wiki/De_Boor's_algorithm}{De Boor} \footnotesize evaluating B-spline curves & rectangle & 1 & -- \\
\end{tabular}

\subsubsection{Non-serial problems}
\begin{tabular}{llcc} \toprule
\bf Problem & \bf Shape & \bf Matrices & \bf Wavefront \\ \midrule
Smith-Waterman \footnotesize arbitrary gap cost & rectangle & 1 & -- \\
Convex polygon triangulation & parallelogram & 1 & -- \\
Matrix chain multiplication & triangle & 1 & -- \\
Nussinov & triangle & 1 & -- \\
Zuker folding & triangle & \color{red} 3? & \color{red} 0? \\
\href{http://en.wikipedia.org/wiki/CYK_algorithm}{CYK} \footnotesize Cocke-Younger-Kasami & triangle & \#rules & -- \\
\href{http://en.wikipedia.org/wiki/Knapsack_problem#Dynamic_programming}{Unbounded Knapsack} \footnotesize (input sensitive) & rectangle & 1 & --\\
\end{tabular}

\subsubsection{Other problems}\ul
\item Dijkstra shortest path: we need a $E\times V$ matrix, along $E$ forall $V$ reduce its distance, problem is serial along $E$, non-serial along $V$ hence of complexity $O(|E|\cdot |V^2|)$ which is far worse than both $O(|V|^2)$ (min-priority queue) and $O(|E|+|V|\log |V|)$ (Fibonacci heap).
\item Fibonacci: this problem is serial 1D. Could be implemented using a placeholder element in one of the matrix dimension.
\item \href{http://archive.ite.journal.informs.org/Vol3No1/Sniedovich/\#dpmodel}{Tower of Hanoi}: 1D non-serial
\item \href{http://www.cs.ust.hk/mjg_lib/bibs/DPSu/DPSu.Files/KnPl81.PDF}{Knuth's word wrapping}: 1D non-serial
\item \href{http://en.wikipedia.org/wiki/Longest_increasing_subsequence#Efficient_algorithms}{Longest increasing subsequence}: serial but binary search algorithm more efficient: $O(n \log n)$.
\item \href{http://www.ccs.neu.edu/home/jaa/CSG713.04F/Information/Handouts/dyn_prog.pdf}{Coin Change}: 1D non-serial


{\color{red}
\item \href{http://en.wikipedia.org/wiki/Floyd-Warshall_algorithm}{Floyd-Warshall}: <it possible to move the $k$ external loop inside?> Serial with $n$ iterations
\item \href{http://en.wikipedia.org/wiki/Viterbi_algorithm}{Viterbi \footnotesize (hidden Markov models)}: $T$ non-serial iterations over a vector
\item \href{http://en.wikipedia.org/wiki/Bellman-Ford_algorithm}{Bellman-Ford} (finding the shortest distance in a graph)
\item \href{http://en.wikipedia.org/wiki/Earley_parser#Pseudocode}{Earley parser} (a type of chart parser)
\item \href{http://en.wikipedia.org/wiki/Maximum_subarray_problem}{Kadane maximum subarray} 1D serial, look at 
\href{http://www.cosc.canterbury.ac.nz/tad.takaoka/cats02.pdf}{Takaoka} for 2D
\item Structural alignment (MAMMOTH, SSAP), \href{http://rna.tbi.univie.ac.at/cgi-bin/RNAfold.cgi}{RNA structure prediction}.
\item \href{http://en.wikipedia.org/wiki/Recursive_least_squares_filter}{Recursive least squares}
\item \href{http://www.math.utep.edu/Faculty/pmdelgado2/courses/adv_algorithms/homework-08_anser.pdf}{Bitonic tour}
}
\ule

\begin{verbatim}
- Balanced 0-1 matrix
- Recurrent solutions to lattice models for protein-DNA binding
- Backward induction as a solution method for finite-horizon discrete-time dynamic optimization problems
- Method of undetermined coefficients can be used to solve the Bellman equation in infinite-horizon, discrete-time, discounted, time-invariant dynamic optimization problems
- Many algorithmic problems on graphs can be solved efficiently for graphs of bounded treewidth or bounded clique-width by using dynamic programming on a tree decomposition of the graph.
- Transposition tables and refutation tables in computer chess
- Pseudo-polynomial time algorithms for the subset sum and knapsack and partition problems
- The dynamic time warping algorithm for computing the global distance between two time series
- The Selinger (a.k.a. System R) algorithm for relational database query optimization
- Duckworth-Lewis method for resolving the problem when games of cricket are interrupted
- The Value Iteration method for solving Markov decision processes
- Some graphic image edge following selection methods such as the "magnet" selection tool in Photoshop
- Some methods for solving interval scheduling problems
- Some methods for solving word wrap problems
- Some methods for solving the traveling salesman problem, either exactly (in exponential time) or approximately (e.g. via the bitonic tour)
- Beat tracking in music information retrieval.
- Adaptive-critic training strategy for artificial neural networks
- Stereo algorithms for solving the correspondence problem used in stereo vision.
- Seam carving (content aware image resizing)
- Some approximate solution methods for the linear search problem.
=====> http://en.wikipedia.org/wiki/Dynamic_programming#A_type_of_balanced_0.E2.80.931_matrix

- Shortest path in DAGs
- Shortest path
- All pair shortest paths
- Independent sets in trees
=> also see exercises for more problems
=====> http://www.cs.berkeley.edu/~vazirani/algorithms/chap6.pdf
- 
- Subset Sum, Coin Change, Family Graph
=====> http://www.algorithmist.com/index.php/Dynamic_Programming

- Optimal Binary Search Trees
=====> http://www.cs.uiuc.edu/~jeffe/teaching/algorithms/notes/05-dynprog.pdf

- Independent set on a tree
- 0-1 Knapsack
=====> http://www.cs.ucsb.edu/~suri/cs130b/NewDynProg.pdf
\end{verbatim}

\newpage
\section{Architecture design and technical decisions}
% ------------------------------------------------------------------------------------------------
\subsection{User facing language requirements}
\quote{The Algebraic Dynamic Programming approach (ADP) introduces a conceptual splitting of a DP algorithm into a recognition and an evaluation phase. The evaluation phase is specified by an evaluation algebra, the recognition phase by a yield grammar. Each grammar can be combined with a variety of algebras to solve different but related problems, for which heretofore DP recurrences had to be developed independently. Grammar and algebra together describe a DP algorithm on a high level of abstraction, supporting the development of ideas and the comparison of algorithms. A notation for yield grammars is provided that serves as a domain-specific language.}

Given such formalization \cite{adp} of dynamic programming on sequences, it seems natural to borrow from it and extend it to other types of DP problems. In short, this framework allow the user to define a grammar using parsers, which are then run over an input string and produce intermediate results that are memoized into a table, when multiple solutions are possible, the user can define an aggregation function ($h$) to retain only some candidates for further combination.

The benefits of ADP framework is that it does not constraint the result of the evaluation to be a single value, but could extend parsers to backtracking parsers or pretty-printers. Such functionality is easy to support in plain Scala, however, it could hampers the performance of the GPU implementation if we leave to the user the duty of representing and processing all the information within the parsers. Its major shortcoming in expressivity is that it only operates on strings, making only triangular-matrix problem easily representable.

The additional features we want to support are:\ol
\item \textbf{Cyclic problems:} are inherently very similar to string problems, except that the input is cyclic. To support such problem efficiently, we only need to mark the grammar cyclic such that it would apply on any unfolding of the cyclic input string.
\item \textbf{Input pair algebra:} the original ADP framework only support single input, we want to support a pair of inputs such that we can treat problem such as Smith Waterman or Needleman-Wunsch. However, it does not make sense to treat more than two sequences because of the $O(n^3)$ storage requirements that limits the problems size more dramatically.
\item \textbf{Windowing}: this can be easily encoded by passing the windowing parameter that limits the computation, then it could be possible to collect either the best or $k$-best results.
\item \textbf{Device restrictions:} since CUDA (and FPGA) cannot operate on an arbitrary Scala classes, we need to restrict the language to primary types (int, float, ... and structures of them). However, we want to preserve the expressiveness available for Scala and impose restrictions on the input and answers available to CUDA. A typical restriction we want to make is that processed structures are of fixed size so that we avoid memory management issues and thread divergence.

The general ADP framework supports multiple solutions for intermediate results. Similarly, we want to impose the restriction that only the best result would be stored on device while preserving the support for multiple partial results in Scala.

\item \textbf{Automatic backtracking:} Since efficient code has to be devised we impose restrictions on the output that could be generated by the parsers on devices. However, on the other side, the backtracking information would be of primary interest for the DSL user, hence we need to automate the backtracking to both fulfill the goals of efficiency and usefulness of the device specific implementation.

Enforcing automatic backtracking presents the advantage to ensure constant size for intermediate results, hence ensuring an $O(n^2)$ storage requirement while memoizing the backtracking with the results would have the drawback of growing towards final result and duplicated unnecessary information, hence requiring both $O(n^3)$ space and memory management features on devices. Collecting the backtracking list can be easily done in $O(n)$ and then inverted whether we prefer bottom-up or top-down construction (the backtrack is usually a lattice of nodes that constitute a tree whose leaves are input elements).
\ole
Since we construct the dependency tree bottom up, and since we need to work with a fixed-size memory, we need to do the additional steps in our pre-processing:\ul
\item \textbf{Normalization:} in order to automate the backtracking, we need the rule to present a certain shape (in particular we want to remove alternatives) so that we can define uniquely the backtracking information. Another step we can make in the process is to break down complex rules into simpler ones if they are expressible so, hereby reducing the complexity of the overall algorithm, would the user production grammar not be optimal. Also we can strip away unnecessary rules.
\item \textbf{Dependency analysis:} a precise order between the rules application must be given. This is required as we use a bottom-up approach where all previous results must be computed versus a top-down approach with a table where a fallback to computation always exist if the desired element is not present.
\ule

{\color{red}
Size analysis to know what storage size we require

We define the aggregate function h as $List[T]\to List[T]$ and a special string emitter for C code that produces the initial element and a comparator between two (cost) elements.
}

% ------------------------------------------------------------------------------------------------
\subsection{Parsing grammar (ADP)}
{\color{red}
Borrowing from ADP XXX

XXX: explain ADP concepts

XXX: normalization, transforms applied

XXX: Think how to implement {\tt cyclic} problems and {\tt windowing} properties within the language

\begin{verbatim}
restrict to only 1 max/min (more restrictive than ADP)
for optimization we want to split grammar into binary productions, to do that we need
to have equality on the following formula. We can solve it more easily if functions are linear
\end{verbatim}
\[\min\limits_{i<k_1<k_2<j}\big[ f(i,k_1,k_2,j) \big] \le  \min\limits_{i<k_2<j} \big[ g_2(i,\min\limits_{i<k_1<k_2} \big[ (g_1(i,k_1,k_2  ) \big],k_2,j) \big]  \]

}

\subsection{Backtracking}
In order to do a clean and efficient transformation from ADP-like language to plain C recurrences, we need to construct bottom-up recurrences from top-down parser rules. To do that, we slightly need to modify the ADP language in order to separate the backtracking and the scoring, because we want to obtain an efficient algorithm: backtrack writes are in $O(n^2)$ whereas score reads are proportional to the algorithmic complexity ($O(n^3)$ or more for non-serial). To deal with this problem, we are facing two options:\ul
\item \textbf{Explicit backtracking:} requires clear syntactical separation between the score and the backtrack and is not implemented in vanilla ADP (unless the whole backtrack is made part of the scoring: big performance impact and non-constant memory requirement issues make its GPU implementation hard and not desirable). Since the backtracking is user-defined, there is no way to generate the backtracking algorithm automatically, hence the user also needs to provide it. 

\item \textbf{Implicit backtracking:} implies that every rule needs to be normalized, and transformed such that given a rule identifier and a set of indices (subproblems breaking), it is possible to retrieve the subproblems combination that build the problem. To do that we need to apply the following transformations\ol
	\item Normalize rules and identify them uniquely by \ul
		\item Removing all the <<or>> operators by breaking a parser into multiple subrules
		\item Forwarding names: replace a rule that is a single invocation by its content (i.e. name a terminal parser with its own name instead of its containing parser's)
		\item Assigning each subrule an unique identifier $r$ and create a mapping table $T: r \to$ user-defined name of the parser.
	\ule
	\item The data elements corresponding to a rule are (scoring, backtracking) and named after the tabulation. The scoring is a user-defined primary type (int, double, ...) and the backtracking is a tuple $({\rm id}_{\rm subrule}, k_1, k_2, \ldots , k_m)$ where $m$ is the maximal number of splits that can occur in the subrules.
	\item During the matrix computation of cell $(i,j)$, if the subrule $r$ applies, the backtrack will be set as $(r,k_1,k_2,\ldots,k_{m_r})$ (obviously with $i\le k_1\le k_2\le \ldots k_{m_r}\le j$ and $m_r\le m$).
	\item During backtracking, when reading the cell $(i,j)$ with backtrack $(r,k_1,k_2,\ldots,k_m)$, given $r$, we recover the subrule hence the user-defined name of the rule that applies and $k_{m_r}$, which allows us to enqueue the subwords $(i,k_1), (k_1,k_2), ..., (k_{m_r})$ for further backtracking. If $r$ refers to a terminal, we stop the backtracking.
	\item The backtracking can be returned to the user as a mapping table $T$ and a list of triplets $(r,i,j)$ where $r$ is the index of the sub-rule that can be easily mapped into the user name (such that $T: r_u \to$ user-defined rule name) and $(i,j)$ is the subword on which the rules applies.
\ole
In short, we break parsers into normalized rules, the backtracking information is the rule id (which rule to unfold) and a list of indices (how to unfold it).
\ule
Assuming that the backtracking information is meant to guide further processing, we provide this information into a list constructed bottom-up: it can be simply processed by the user program in-order, applying for each user-defined rule the underlying transformation, and storing intermediate results (i.e. in a hash map) until they are processed by another rule. Since consumed sub-result can be dropped and since the backtracking contains only relevant sub-problems, ultimately, only the result will be stored.

\subsubsection{Backtracking with multiple backtrack elements}
The backtracking technique described above work fine when there is a single element stored per matrix cell (which is usually the case with min/max problems). However, in the generalization introduced by ADP, it is possible that a matrix cell stores multiple results. In such case, we need to select a correct result to avoid backtracking inconsistencies.

Additionally, we need to keep track of the multiplicities of the solutions, that is if we want to obtain the $k$ best solutions, we need to make sure that we return $k$ different backtracks. To do that, we maintain a multiplicity counter at every step: \ul
\item While there is a single solution for all possible incoming paths, we continue in this direction with the same multiplicity (we have no choice).
\item When there is $r$ different solutions available, and the path multiplicity at this point is $k$ we have the following cases: \ol
	\item If $k\ge r$: we explore all paths with multiplicity $k-r+1$. This is because each branch may produce only one solution and we don't know ahead of time which path will provide multiple solutions. Finally, we retain only the $k$ best solutions.
	\item If $k<r$ (there is more paths than needed): we can only explore the $k$ first paths with multiplicity 1 and safely ignore the other (as we only need $k$ results).
	\ole
\ule

Now it remains the problem to generate all possible results and check whether they are correct correct. To do that we can simply re-apply the parsers while maintaining the source elements of all production and then retain only those with desired score and backtrack. Since we know the backtrack for one element, we can do the following optimization at backtrack parser computation:\ol
\item Defuse or's: since we know exactly (by the subrule id, maintained in the backtrack) which alternative has been taken to obtain the result, we can skip undesired branches of or parsers.
\item Feed concatenation indices: since the backtrack stores the concatenation indices, we can reuse in the concatenation parsers. This removes the $f(n)$ factor in the backtrack complexity (as concatenation backtrack parsers <<know>> where to split).
\item Skip filters: since filter are applied before their inner solution is computed, their are only position-dependent. Hence if a backtrack involves a filter, since its position is fixed, the filter must have been passed at matrix construction time.
\ole

\subsubsection{Backtracking complexity}
We want to have an estimation of the complexity of the two operations to see what is the overhead of multiple backtracking. For single-element backtrack, we only need to revert the parser to find involved subwords, which is linear in the parser size (because the backtracking identifies uniquely the alternative in or's and indices in concatenations). At every backtrack step, either:\ul
\item The word is removed at least one character, which leads to maximal backtrack length of $n$.
\item The word is split in $s$ subwords, with recurrence $f(n)=2f(n/2)+1$ by solving this recurrence we see that there can be at most $n$ final nodes and $n$ intermediate nodes (when $s=2$). Hence the backtrack length is at most $2n$.
\ule

Hence for single backtrack we would have at most $O(2n) \cdot O({\rm parser})$ complexity. For the $k$-elements backtrack since we regenerate all possible solutions, that is $O(k^c)$ candidates (with $c$ the maximal number of concatenation in the parser), the overall complexity is $O(2n \cdot k^c) \cdot O({\rm parser})$. Hence there is a $k^c$ factor to pay if we want to backtrack the $k$ best solutions\footnote{Note that the same factor lies in the matrix computation complexity.}.




%Algo: Read matrix, reapply the parser but by maintaining list of origins. Filter results to match multiplicities
% à priori ça a une complexité en k^3*2n pour les k meilleurs backtracks



% ------------------------------------------------------------------------------------------------
\subsection{Memory constraints}
We denote by \textit{device} the computational device on which the processing of the DP matrix (or of a computational block) is done and $M_D$ its memory. This can be the GPU or the FPGA internal memory. Usually the main memory is larger than device memory and can ultimately be extended by either disk or network storage.

We propose to evaluate the device memory requirements to solve the above problem classes. We need first to define additional problem properties related to implementation:\ul
\item \textbf{Number of matrices:} multiple matrices can be encoded as 1 matrix with multiple values per cell. Hence the implementation differentiates only between cost and backtrack matrices with respective element sizes $S_C$ and $S_B$.
\item \textbf{Delay of dependencies:} In case the problem does not fit into memory, partial matrix content needs to be transferred across sub-problems. Such data amount is usually proportional to the delay of dependencies. If this delay is small, it might be worth to duplicate matrix data in the wavefront, otherwise it might be more efficient to allow access to the previous computational blocks of the matrix.
\item \textbf{Wavefront size:} Finally some aggregation that is made along some dimension of the matrix does not need to be written at every cell but can be propagated and aggregated along with computation (ex: maximum along one row or column). Hence such information can be maintained in a single place (in the wavefront) and progress together with the computation. We denote by $S_W$ the size of wavefront elements.
\item \textbf{Input size:} the size of an input letter (from input alphabet) is denoted by $S_I$.
\ule

% ----------------------------------------------
\subsubsection{Small problems (in-memory)}
Problem that can fit in memory can be solved in a single pass on the device. Such problem must satisfy the equation:
	\[(S_I+S_W) \cdot (m+n) + (S_C+S_B) \cdot (m\cdot n) \le M_D\]

For instance, assuming that $m=n$, $M_D=1024{\rm Mb}$, that backtrack is 2b (<16384, 3 directions) and that the cost can be represented on 4 b (int or float), that input is 1b (char) and that there is no wavefront, we can treat problems of size $n$ such that $2n+5n^2 \le 2^{30} \implies  n\le 14650$. We might also possibly need to take into account extra padding memory used for coalesced accesses. But it is reasonable to estimate that problems up to 14K fit in memory.

% ----------------------------------------------
\subsubsection{Large problems}
To handle large problems, we need to split the matrix into blocks of size $B_H \times B_W$. For simplification, we assume a square matrix made of square blocks with $b$ blocks per row/column.

% ----------------------------------------------
\subsubsection{Non-serial problems}
Non-serial problems need to potentially access all elements that have been previously computed. We restrict\footnote{As we have not encountered a problem with non-serial dependencies along the diagonal.} ourselves to the following dependencies: \ul
\item Non-serial dependencies along row and column
\item Serial dependencies along diagonal, with delay smaller or equal to one block size
\ule
Such restriction implies that all the block of the line and the row, and one additional block to cover diagonal dependencies must be held in memory (independently of the matrix shape).

For simplification, let $m=n$ and assume that we have $b$ square blocks per row and per column. Hence we have the following memory restriction:
\[ 2\frac{n}{b}(S_I+S_W) + 2 \cdot \frac{n^2}{b}S_C + \frac{n^2}{b^2} S_B \le M_D\]

We also need to take into account the transfer between main memory (or disk) and device memory. Dependency blocks only need to be read, computed blocks need to be written back. Ignoring the backtrack and focusing only on the cost blocks, the transfers (in blocks) are:
\[\begin{array}{rclll}
		b^2 +% computed line writeback (once)
		(b-1)^2 + % diagonal block loads (1 per block)
		\sum\limits_{i=0}^{b-1} i \cdot b % column dependencies loads (for line i)
		&=&\tfrac{1}{2}b^3+\tfrac{3}{2}b^2-2b+1 &\qquad& \rm (Rectangle)
	\\
		\sum\limits_{i=1}^{b} \Big(1+2\cdot(i-1)\Big) \cdot (b+1-i) % on i_th diagonal
		&=& \tfrac{1}{3} b^3 + \tfrac{1}{2}b^2 + \tfrac{1}{6}b &\qquad& \rm (Triangle)
	\\
		\sum\limits_{i=1}^{b} \Big(1+2\cdot(i-1)\Big) \cdot b % on i_th diagonal
		&=& b^3 &\qquad& \rm (Parallelogram)
\end{array}\]

Putting these two formula together, and using most of the device memory available, we obtain the following results with $S_C=4, S_B\footnote{To deal with larger matrices, backtrack data need to be extended.}=4, S_I=1, S_W=0$ and $M_D=2^{30}$:
\begin{center}\includegraphics[width=14cm]{inc/ns_large.pdf}\end{center}

Given an experimental bandwidth of 5.3743 Gb/s between CPU and GPU, processing matrices one order of magnitude larger (128K) would result in respectively 13\up{(R)}, 8.5\up{(T)} and 25.4\up{(P)} minutes of transfer delay. Extrapolating the preliminary results of small problems, a computation on input of size 128K would require respectively 7 days 13h\up{(R)}, 2 days 22h\up{(T)} and 6 days 10h\up{(P)}, assuming there is no other scalability issues. Although this overhead seems appealing compared to the computation time, the total time blows up (because of the  $O(n^3)$ complexity) and make the processing of such large problem less relevant. Given that real problems -- like RNA folding -- operate at input sizes up to 4096, it would not be of much relevancy to implement a version for larger cases, although perfectly feasible.

% ----------------------------------------------
\subsubsection{Serial problems}
The serial problem have the interesting property to access to a fixed number of previous elements. These elements can be either stored either explicitly in a scoring matrix or implicitly as moving aggregation into a wavefront. Since the dependencies are fixed, the computation direction gains an additional degree of freedom: matrix can be solved in diagonal (as non-serial problems) or line-wise or column-wiser. This allows to store the whole necessary state to make progress into a limited number of lines (or columns), and sweep vertically (resp. horizontally) along the matrix.

Since serial problems are of complexity $O(n^2)$ (due to the matrix dimension and the finite number of dependencies), it is possible to tackle much larger problem than non-serial during the same running time. Hence, it seems obvious to let serial problems grow larger than the memory.

Mixing the dependency property and size requirements, we can split the matrix into sub-matrices, store special lines (and/or columns) into memory (or hard disk), and repeat computations to solve the backtrack (similarly as in \cite{swat_gpu},\cite{swat_mega}, but this implementation use problem-specific knowledge that might not generalize).

To store intermediate lines and columns, we are facing two different strategies to explore:\ul
\item \textbf{Fixed subproblem size:} we decompose the algorithm as follows\ol
\item Define a grid of <<major column and rows>>, where each cell's data (input, output, cost and backtrack matrices) fits into the device memory.
\item Compute the values of the grid's major columns and rows in one pass.
\item Second (on-demand) computation to process backtracking inside relevant cells.
\ole
Let $b$ the number of cells that we have on each row/column, the total computation running time would be $(b^2 + 2b) \cdot t_b$ where $t_b$ is the time to compute one cell's matrix. This division has the advantage of providing the minimal computation time at the expense of external storage proportional to $O(n)$ (if we store only lines or columns) or $O(n^2)$ (if we store both).

\item \textbf{Myers and Miller’s algorithm:} (divide and conquer)
This algorithm break the DP problem into 2 (or 4) subproblems such that once the middle line/column is computed, the problem can be solved for 1 submatrix and backtracking among up to 2 of the 3 other. This breaking is applied recursively until the submatrix data fits into memory. The storage requirements are $4 \cdot O(n)$ (we store along both dimension $1+\tfrac{1}{2}+\tfrac{1}{4}+...$ lines/columns).

The algorithm proceeds as follows: first it solves the problem to obtain the first backtracking element, then it breaks the matrix in 4 submatrices, and refine it until backtrack is tractable. Since there is at most $\log n/b$ refinements and since every part of the matrix may be involved in backtrack, running time is $O(n^2 \log_2 n)$.
\item \textbf{Hybrid approach:} a hybrid approach might be created to take advantage of additional available memory, however, the running time decreases logarithmically to the space used, this means that using twice more storage space would only result in a $2\times$ speedup (measuring only the computation time). Hence an hybrid approach would be to decide a $k$ such that at each step we partition the desired submatix into a intermediate grid of $k$ rows/columns. The space usage would be in $2 k \log_k (n/b)$ and the running time complexity would be $O(n^2 \cdot \log_k n)$. Then the user would be able to fix a storage space $S \ge 4 \log_2 (n/b)$ and obtain the corresponding $k$ for a given $n$.
\ule

{\color{red}
try to setup wavefront size (if needed) => just enlarge the matrix by 1 so that we go wavefront-to-wavefront

XXX: what's the maximal size of the wavefront ??

XXX: can we avoid to store some matrices and put them in the wavefront ??

Split into blocks:\ul
\item Decide the shape of the blocks
\item Decide the size of the blocks
\item Decide of a strategy to store intermediate lines/columns: space/time tradeoff.
\ule

XXX: make it work up to 14K for all 3 problems using multiple kernels
XXX: example of non-symmetric serial problem

The three last elements, combined with the above one, provide a precise estimation of the memory consumption, and the implementation difficulty 

all the problem are subject to the input dimensions $n$.

the two latter one gives an estimation of the constant factor.

The delay of dependencies might also have an impact: if the matrix is too large to fit in the memory (device or main memory), it becomes necessary to maintain partial matrix content (all the intermediate elements) within the wavefront. Also the number of cost matrices might affect the performance, simply because maintaining them requires computations and memory accesses.
}

% ------------------------------------------------------------------------------------------------
\subsection{Memory layout}
{\color{red} XXX}


% ------------------------------------------------------------------------------------------------
\newpage
{\color{red} 
\subsection{LMS compiler stack}
\textbf{User-language}: define additional parameters for the recurrence\ul
\item Windowing (to convert non-serial into serial problems)
\item Input sizes, and alphabets (backtrack, input, cost)
\item Backtrack (implicitly by backtrack alphabet size) and cost matrices bit-sizes (cost maximum may be inferred using <<Yield size/grammar analysis>>)
\item Recurrence functions, devices available
\item What to keep in memory (cost, backtrack or both).
\ule
$\Downarrow$ Conversion (using an existing technique)

\textbf{Intermediate representation}

$\Downarrow$ Optimizations\ul
\item Transform non-serial into serial \ul
	\item Use aggregation functions/transformations
	\item Use windowing from user (if no other technique succeed)
	\ule
\item Define the wavefront depth
\item Avoiding the cost matrix by moving it into the wavefront
\ule

\textbf{Code specification}\ul
\item Kernel function (1-element function), inputs, outputs, wave front, dependencies, bit sizes
\item Device-level interface => setup the block sizes(w/h), input and memory sizes
\item Define the device-specific implementation of the block (CPU/FPGA/CUDA)
\item Define the co-processor memory aggregation function
\item Define the scheduling of the blocks and aggregation (software pipelining)
\item Define the data movement back and forth to disk
\ule

$\Downarrow$ Generation\ul
\item Generate the kernel for specific device
\item Generate the scheduling and barriers
\ule

\textbf{Binary program}
}
%\ol
%\item Make sure we encompass all the most common patterns of DP: check if we have higher dimensions or more complex formula.
%\item Let the user tune the window size if he wants to reduce non-serial to serial.
%\item Concerns separation: common architecture enables flexibility (exchange components) \ul
%	\item Block processor: CPU, GPU, FPGA, must allow variation of width and height
%	\item Memory stats computation (min, sum, ... column/line combinations): CPU, GPU
%	\item Scheduler (CPU): interleave block computation and memory statistics
%	\ule
%\item Common description \ul
%	\item Block kernel processor
%	\item Full block
%	\ule
%\item Discussion: wavefront, design similarities, polyhedral theory
%\ole

%%\newcolumntype{C}[1]{>{\centering\let\newline\\\arraybackslash}m{#1}}
%\newcolumntype{C}{@{\hspace{7pt}}c@{\hspace{7pt}}}
%\def\mnl{\rule{0pt}{2.6ex}\rule[-1.2ex]{0pt}{0pt} \\ \hline}
%$\begin{array}{|C|C|C|C|C|C|} \hline
%0 & 0 & 0 & 0 & 0 & 0 \mnl
%0 &  &  &  &  & \mnl
%0 & M_{23}  &  &  &  & \mnl
%0 &  & \sum  &  &  & \mnl
%0 &  &  &  &  & \mnl
%0 &  &  &  &  & \mnl
%\end{array}$
%\end{document}

\newpage
\section{Implementation} \label{implementation}
% ------------------------------------------------------------------------------------------------
\subsection{CUDA baseline} \label{baseline_impl}
In the project planning, an baseline implementation phase immediately followed the problem analysis (we also present the parallelogram matrix case). The goal of this phase is threefold:\ol
\item Better understand the challenges in CUDA implementation of dynamic programming problems and get on par with state-of-art implementations.
\item Have a baseline implementation that is independent of the hardware and that could be benchmarked. We also tried to contact the authors of \cite{swat_mega} and \cite{gpu_atlp} to obtain their implementation. The former provided us with their implementation, which turned out to address large serial problems whereas our focus was on smaller non-serial problems, the latter did not respond to our solicitations.
\item Have an optimal implementation that can serve as a   to be imitated and generalized by the code generation.
\ole

Leveraging the insights provided by \cite{gpu_atlp} and \cite{gpu_barrier}, we started with a basic implementation (where each CUDA thread processes one matrix line) with three additional optimizations:\ul
\item Memory accesses must be coalesced (memory accesses account for a significant part of the total running time, according to both manufacturer documentation and experiments \cite{perfeval_gpu})
\item Synchronization between threads can be done according to \cite{gpu_barrier}, additionally, we can slightly loosen the synchronization restrictions, as the paper describes a thread barrier whereas we only require a condition on previous thread progress (except for the parallelogram case, where we still require a barrier).
\item Computation progresses element-wise along the diagonal (maximizes the parallelism level)
\item Thread block size = warp size (32) to benefit from implicit synchronization within warps
\ule

% ----------------------------------------------
\subsubsection{Related work}
Since \cite{swat_mega} focuses on a different class of problem, we compare our implementation against \cite{gpu_atlp}, which provides an efficient matrix multiplication implementation. However, since we have neither the source code (or binary) nor the same evaluation hardware, we need to normalize the results. To do that, we present hardware differences and their result:

\def\unt#1{& \footnotesize #1}
\begin{table}[H]\begin{center}\begin{tabular}{lrrr} \toprule
\bf Graphic card &				& \bf Our 		& \bf  ATLP\cite{gpu_atlp} \\ \midrule
Model &						& GeForce GT 650M	& Tesla C1060 \\
Architecture, capability  &			& Kepler (3.0)	& GT200 (1.3) \\
Memory \unt{Mb}				& 1024		& 4096 \\
CUDA cores &					& 384		& 240 \\
Clock (core, memory) \unt{MHz}	& 756, 1953	& 1300, 1600 \\
Memory bus \unt{bit}				& 128		& 512 \\
Memory bandwidth \unt{GB/s}		& 28.8		& 102.4 \\
Processing power \unt{GFLOPS}	&564.5		& 622.08 \\ \midrule
\bf Processing speedup & 		& 1			& 1.07 \\
\bf Memory speedup & 			& 1			& 3.55 \\ \bottomrule
\end{tabular}\end{center}\caption{Graphic cards technical specifications (source: \href{http://en.wikipedia.org/wiki/Comparison_of_Nvidia_graphics_processing_units}{Wikipedia})}\end{table}

\begin{table}[H]\begin{center}\begin{tabular}{lrrrrrrrrrr} \toprule
\bf Matrix size & 128 & 256 & 512 & 1024 & 1536 & 2048 & 2560 & 3072 & 3584 & 4096 \\ \midrule
\bf No split & 0.07 & 0.09 & 0.19 & 0.59 & 1.27 & 2.25 & 3.51 & 5.07 & 6.92 & 9.06 \\
\bf Split at 1 & 0.06 & 0.07 & 0.08 & 0.14 & 0.26 & 0.47 & 0.77 & 1.21 & 1.80 & 2.57 \\ \bottomrule
\end{tabular}\end{center}\caption{ATLP\cite{gpu_atlp} results: matrix chain multiplication, execution time (in seconds)}\end{table}

% ----------------------------------------------
\subsubsection{Results}
We present here the timings of our baseline implementation. For correctness, we first implemented a CPU single thread version (in C) that we used to compare CUDA results against. Input data is made of random numbers. The implemented dynamic programming problems are:\ul
\item Rectangle: Smith-Waterman with arbitrary cost (\S\ref{swat_arbitrary})
\item Triangle: matrix chain multiplication (\S\ref{mat_mult_plain})
\item Parallelogram: polygon triangulation (\S\ref{polygon_tri}) using a matrix larger than necessary (\S\ref{problems_end}). Note that this implementation uses at most 32 blocks to prevent dead locks on our hardware (restriction due to the number of concurrent threads on the device).
\ule

\begin{table}[H]
\begin{center}\begin{tabular}{rlrrr} \toprule
\bf Matrix size & \bf Comment & \bf R & \bf T & \bf P \\ \midrule
1024 & CPU					& 1.965		& 1.191		& 6.069 \\
2048 & CPU					& 27.229		& 15.296		& 57.323 \\
4096 & CPU					& 			& 177.608	&  \\
1024 & GPU baseline			& 0.838		& 0.500		& 0.516 \\
1024 & GPU sync improved		& 0.642		& 0.316		& 0.343 \\
2048 & GPU P $\le32$ blocks		& 2.864		& 1.427		& 2.096 \\
4096 & GPU 8 splits				& 21.902		& 8.841		& 16.767 \\
8192 & GPU 64 splits			& 159.058	& 62.064		& 135.793 \\
12288 & GPU 256 splits			& 419.030	& 196.971	& 460.912 \\
\bottomrule \end{tabular}\end{center}
\caption{Execution time (in seconds) for R=rectangle, T=triangle, P=parallelogram}
\end{table}

% ----------------------------------------------
\subsubsection{Results discussion} \label{results_discussion} \ul
\item \textbf{User interface:} It has been put in evidence in \cite{perfeval_gpu} that using the GPU exclusively for CUDA or in combination with UI display (Mac OS) affects the performance (GeForce 330M). With the newer architecture, this difference has been reduced to less than 3.5\%, decoupled UI and CUDA   performing best. So we can safely ignore this issue.
\item \textbf{Blocks synchronization:}\ul
	\item Removing {\tt \_\_threadfence()} before the synchronization is not syntactically correct but results still remains valid, this confirms the observation made by \cite{gpu_barrier}. Speedup for matrix size of 1024 are 67ms (parallelogram) 100ms (triangle) 180ms (rectangle).
	\item In the parallelogram case, using all threads to monitor other blocks status instead of the first one only results in a 6.4x speedup (22.72$\to$3.52ms) for the parallelogram.
	\ule
\item \textbf{Multiple threads per matrix cell:} in the case of a triangular matrix, at each step, the number of cells to be computed (on the diagonal) decrease while the computation complexity increases (there is one more dependency). According to \cite{gpu_atlp}, the solution lies in adaptive thread mapping, using more than one thread to compute one matrix cell, depending on the complexity. However, in our setup (memory layout+algorithm+hardware), we did not find any improvement by doing so. We want to explore the reason for that: we pose as hypothesis that the bandwidth is the bottleneck of our setup and test it.\ul
\item

First we need to prove that we use almost all the available memory bandwidth: for matrix multiplication, in a triangular matrix, we have
\[\text{Total transfer}=\frac{n(n+1)}{2} \text{ writes} + \sum_{i=0}^{n-1} 2 i \cdot (n-i) \text{ reads}\]
where each write is 10 bytes (long+short), and each read is 8 bytes (long). For $n=4096$ we transfer
% n*(n + 1)/2*12 + Sum[2*i*(n - i), {i, 0, n - 1}]*8
183'352'614'912 bytes which corresponds to 183.35GB. In 8.841 seconds, we can transfer theoretically at most $8.841\cdot 28.8 = 254.6 \rm GB$. Hence  72\% of the algorithm running time is spent into memory accesses.

\item On a 4096 matrix, if we assume that the \cite{gpu_atlp} card would have the same bandwidth as our card, their running time would be
\[2.57 \cdot (1-.72) + 2.57 \cdot 0.72 \cdot \tfrac{102.4_{GB/s}}{28.8_{GB/s}} = 7.30\rm s_{\text{ ATLP}} < 8.84\rm s_{\text{ our}}\]
This shows that our algorithm is comparable to theirs. However, we must avoid a close comparison because the fundamental hardware differences would make a tight computation almost intractable (additionally, we do not have \cite{gpu_atlp} source code).
\ule
As a conclusion, (1) it seems that the technique used in \cite{gpu_atlp} brings more performance improvement with legacy hardware, however this remains a supposition (as we can not compare) and (2) we are slightly worse than one of the best current implementations.

\item \textbf{Number of threads:} reducing the number of threads launched at different splits of the algorithm (especially in latest splits in rectangular and triangular shapes) does not bring any speedup. Even worse, it slows down slightly the computation. We might attribute this to a better constant transformation by the compiler. Hence, having many idle threads does not impede performance.

\item \textbf{Unrolling:} unrolling the inner loops (non-serial dependencies) a small number of times provide some speedup, for a 2048-matrix respectively 10.9\% (rectangle, $2.765s\to 2.464s$), 14.1\% (triangle, $1.427s\to 1.225s$) and 9.7\% (parallelogram $1.539s\to 1.389s$). The best experimental number of unrolling is 5.
\ule

% ------------------------------------------------------------------------------------------------
\subsection{Scala parsers} \label{scala_parsers}
The Scala parsers consist in 4 traits that are used to construct a DSL program:\ul
\item \textbf{Signature:} abstraction to define input ({\tt Alphabet}) and output ({\tt Answer}) types, and the aggregation function. The signature is implemented by all other traits (in particular algebras and grammars).
\item \textbf{BaseParsers:} serves as basis for the two other traits and defines common features. It implements the {\tt Parser} abstraction and all its inheriting classes: {\tt Tabulate}, (abstract) {\tt Terminal}, {\tt Aggregate}, {\tt Filter}, {\tt Map}, {\tt Or}, {\tt Concat}. Terminals are further specialized in the two other traits (ADPParsers and TTParsers). The parser abstraction specifies 3 methods:\ul
	\item {\tt apply(subword)} computes the parser result; it is used to obtain the corresponding results.
	\item {\tt unapply(subword,backtrack)} computes the previous step of the backtrack by returning subsequences at the origin of the result; it is invoked recursively to obtain the full backtrack trace.
	\item {\tt reapply(subword,backtrack)} is very similar to apply, except that it  computes only the results matching the backtrack. It is used to construct the result corresponding to a backtrack trace (possibly in a different domain, pretty printing, ...).
	\ule
	To support analysis, the parsers carry additional values:\ul
	\item Minimum and maximum yield size: functions evaluated recursively except for tabulations where value is attributed in the yield analysis phase.
	\item Number of inner alternatives: helps counting alternatives, thereby guaranteeing an unique number for each (provided that parsers obtain non-overlapping ranges).
	\item Number of inner moving concatenations: helps determining required storage for the backtrack as well as retrieving the appropriate index in the backtrack phase
	\ule
	Additionally, the BaseParser implements the analysis that is shared by both the Scala and the CUDA version: dead rules elimination, yield analysis and dependencies ordering. Finally, it provides some implicit functions to flatten nested tuples (that are constructed by multiple concatenations).
\item \textbf{ADPParsers:} used as basis for a single track DP grammar (using one input sequence). It defines the concatenation operator $\sim$ ({\tt Concat} wrapper), and the terminals (empty, element and sequence). Additionally, it defines the interface functions {\tt parse(input)}, {\tt backtrack(input)} and {\tt build(in,backtrack)} that respectively compute the result, the backtrack and the result corresponding to a trace.
\item \textbf{TTParsers:} used to define two-track DP grammar (using a pair of sequences as input). Similarly, this class defines concatenations $-\!\!\sim$ and $\sim\!\!-$, terminals (for each track) and the {\tt parse(in1,in2)}, {\tt backtrack(in1,in2)} and {\tt build(in1,in2,backtrack)} functions.
\ule

\begin{figure}[H]\begin{center}\setlength{\unitlength}{.6cm}\begin{picture}(14,11)
\put(3,8){\tbox{8}{2.5}{{\bf Signature} \footnotesize\\ Types: Alphabet, Answer \\ $h$ (aggregation function)}}
\put(0,4){\tbox{14}{2.5}{{\bf BaseParsers} \footnotesize\\ Tabulate, Terminal, Aggregate, Filter, Map, Or, Concat \\ Analysis: dead rules, yield analysis, dependencies}}
\put(0,0){\tbox{6}{2.5}{{\bf ADPParsers} \footnotesize\\ $\sim$, $\sim(a,b,c,d)\sim$ \\ Single track terminals}}
\put(8,0){\tbox{6}{2.5}{{\bf TTParsers} \footnotesize\\ $-\!\!\sim$, $\sim\!\!-$ \\ Two-tracks terminals}}
{\linethickness{1.5pt}\put(3,2.5){\vector(1,1){1.5}}\put(11,2.5){\vector(-1,1){1.5}}\put(7,6.5){\vector(0,1){1.5}}}
\end{picture}\end{center}\caption{Scala parsers class diagram (simplified)}\end{figure}

% ------------------------------------------------------------------------------------------------
\newpage
\subsection{Code generation} \label{codegen}
The code generation step produces multiple outputs that are tightly bound to each other. Besides the Scala wrapper (a simple JNI interface), in the C/CUDA code generated we distinguish:\ol
\item JNI input and output conversion functions
\item Host helpers for memory management and scheduling of CUDA kernels
\item CUDA matrix computation, which can be further decomposed into matrix scheduling (loops) and (matrix cell) computation.
\item CUDA backtrack collection kernel
\ole

\begin{figure}[H]\begin{center}\setlength{\unitlength}{.6cm}\begin{picture}(16,8)
\put(0,-.5){
{\color{lightgray}\put(0,7){\tbox{4}{1}{\footnotesize\bf BaseParsers}}
	\put(0,2){\rotatebox{90}{\tbox{4}{1}{\footnotesize\bf ADPParsers}}}
	\put(1.5,2){\rotatebox{90}{\tbox{4}{1}{\footnotesize\bf TTParsers}}}
	{\linethickness{1.5pt}\put(.5,6){\vector(1,1){1}}\put(2,6){\vector(0,1){1}}}}
{\linethickness{1.5pt}\put(6,6){\vector(-7,2){3.5}}\put(8,1.33){\vector(3,1){2}}\put(8,1.33){\vector(3,-1){2}}}
\put(3,0){\tbox{6}{6}{{\bf CodeGen} \footnotesize\\ code generation: parsers, backtrack, helpers, JNI \\[12pt] CodeCompiler}}
\put(10,3.5){\tbox{6}{2.5}{{\bf CodeHeader} \footnotesize\\ Types conversion, headers management}}
\put(10,1.5){\tbox{6}{1}{\bf ScalaCompiler}}\put(10,0){\tbox{6}{1}{\bf CCompiler}}
\moveto(2,2)\lineto(2,1.5)\lineto(2.5,1.5)\strokepath\put(2.5,1){\vector(1,0){.5}}
\moveto(.5,2)\lineto(.5,1)\lineto(2.5,1)\strokepath\put(2.5,1.5){\vector(1,0){.5}}
\put(-1.25,-.1){\tbox[0l]{4.5}{1}{\footnotesize automatic call \tiny\\ if CodeGen mixed-in}}
}
\end{picture}\end{center}\caption{Code generation and runtime engine class diagram (simplified)}\end{figure}

% ----------------------------------------------
\subsubsection{Scala structures conversion (JNI)}
Since general Scala types can be extremely complex and might depend of the JVM context (file stream, closures, ...), we want to restrict the supported types; additionally types should be of fixed size for more efficient processing and easier memory allocation. We support the following types:\ul
\item \textbf{Primitive types:} natively supported in both Java and C. Since there is some little semantics difference between these two languages types, we used C (signed) types as reference. Supported types are: boolean, byte (unsigned char), char, short, int (32bit), long (64bit), float and double.
\item \textbf{Empty case classes:} user-defined types might be more complex, so we allow users to define case classes that serve as data container and would be translated into C {\tt struct}s.
\item \textbf{Tuples:} if the user-defined type is fairly simple, a named case class might be cumbersome. Tuples are a syntactical lightweight alternative to case classes, although they translate very similarly. Since Tuple classes are generic and can carry different member types; need to name tuple types uniquely, according to their arity and inner types.
\ule

Currently we use {\tt Manifest}s and reflection to extract types, and convert their string representation into our restricted subset. Manifests expands tuple inner types and reflection can be used to find class member's types. This imposes the additional restriction that we can not nest tuples into case classes, because generic types are then erased. However, the same effect could be achieved with Scala 2.10 {\tt TypeTag}s, converting immediately to concrete type tree representation using macros expansion\footnote{Hint provided by Eugene Burmako, \url{https://gist.github.com/4407488}}.

The JNI functions are involved at input to decode sequences arrays and at output, to encode the result and possibly its corresponding trace. Input method is constructed in two steps:\ul
\item Recursively obtain the classes and accessor methods of the composite input type. A subtle variation is that case classes primitive types are immediately converted into native types whereas tuple members are boxed in their respective class (i.e. {\tt java.lang.Integer}, ...).
\item For each element of the input array, retrieve the objects recursively and write their primitive values in the corresponding {\tt struct} array.
\ule
The output method consist of two different steps:\ul
\item Converting the result into its JVM counterpart by using the opposite rule as for decoding input (but with JNI types specified in the constructor lookup instead of accessors).
\item Optionally encoding the backtrack: this is pretty straightforward as the structure is more regular (and make uses of Lists); additional care should be taken to avoid bloating concatenation indices lists with unnecessary elements (as C uses fixed memory whereas Scala lists length might vary).
\ule

% ----------------------------------------------
\subsubsection{Host wrappers}
Host wrappers are functions bridging between JNI and CUDA; their duties are:\ul
\item Exposing JNI parsing and backtracking functions
\item Calling appropriate conversion methods
\item Allocating host and CUDA memory (and managing transfers between them)
\item Launching CUDA kernels: matrix computation, backtrack, and possibly aggregation within window (additional aggregation among window results, would this option be set)
\ule

One peculiarity of our execution environment, is that the kernel execution duration is bound to approximately 10 seconds\footnote{Hard limit imposed by the operating system. Although workarounds exist for Linux and Windows (requiring a second graphic card to display the UI), none of them is compatible with Mac OS. Eventually, a hack has been devised to force the UI on CPU while keeping the dedicated CUDA card powered; unfortunately this does not alleviate the kernel execution timeout.}. To solve this issue, we estimate the overall complexity of matrix computation, which allows us to estimate running time, then break computation into multiple kernels sufficiently small to fit in the time limit.

Since computations are made diagonal-by-diagonal (see \ref{matrix_scheduling}), we can easily decompose the matrix computation by adapting the number of diagonals computed per kernel. The global complexity being the product of the number of elements and the complexity per element, the latter being equal to the number of unbounded concatenations (where maximal size is infinite).

\textbf{Problems larger than device memory}\\
Problems larger than the device memory can actually be processed on recent CUDA devices (with CUDA architecture $\ge 2.0$) as these are able to address the main memory from the device. However, since the distance between CUDA processors and memory is increased, there is an approximate $5\times$ slowdown penalty to be paid in this configuration (experimentally, on a $1024\times 1024$ triangular matrix). Nevertheless, this workaround implementation has 2 benefits:\ul
\item It allows larger problem to be solved, with very little implementation effort, would the user be patient enough for the computation to terminate
\item It provides a good estimation of the main memory usage penalty, and thereby a strong argument in favor of the implementation described in \ref{ns_mem_transfer} (with lest than 1\% overhead due to transfers). However, since we have not found concrete applications with such matrix size, the benefit of supporting large matrices is unclear, hence we leave the optimal implementation for future work.
\ule

% ----------------------------------------------
\subsubsection{Matrix computation scheduling} \label{matrix_scheduling}
Similarly as in the baseline implementation, progress is made along the diagonal (see \ref{mem_layout}) and each thread is responsible of one line. That is, the matrix is swept horizontally by a <<diagonal of threads>>, that are enabled only if they are within a valid matrix cell.

\begin{figure}[H]\begin{center}\setlength{\unitlength}{.6cm}\begin{picture}(6,6)
	\def\Cfl2#1{#1{0,4}#1{0,3}#1{1,3}#1{0,2}#1{1,2}#1{2,2}#1{0,1}#1{1,1}#1{2,1}#1{3,1}#1{0,0}#1{1,0}#1{2,0}#1{3,0}#1{4,0}}
	\Cfl2{\Cg}
	{\color{cyan}\Cd[0,1]{4,1}{2.8}\Cd[0,1]{4,2}{1.8}\Cd[1,0]{1,4}{2.8}\Cd[1,0]{2,4}{1.8}\Cd[1,1]{3,3}{0.8}\Cd[2,1]{2,3}{1.8}\Cd[1,2]{3,2}{.9}}
	\Cd[0,1]{4,3}{0.8}\Cd[1,0]{3,4}{0.8}
	\multiput(3.5,5.5)(1,-1){6}{\circle{.4}}
	\multiput(0,0)(1,0){7}{\line(0,1){6}}\multiput(0,0)(0,1){7}{\line(1,0){6}} % matrix
	\put(3.5,5.5){\color{lightgray}\line(1,-1){5}}
	\multiput(3.7,5.5)(1,-1){6}{\color{red}\linethickness{1.5pt}\vector(1,0){2}}
	\put(8.6,0.05){\tiny thread 0}
	\put(3.6,5.75){\tiny thread 5}
\end{picture}\end{center}\caption{<<Diagonal of threads>> and maximal dependencies}\label{fig:diag_deps}\end{figure}

Special care must be taken to handle computation dependencies: within a warp, all threads are executed at the same time, hence no synchronization is necessary. To benefit from this implicit synchronization, we set block size being equal to wrap size. It remains to provide inter-block synchronization: dependencies are along line, column and possibly intermediate elements. By induction on rows and columns, it suffice to have the last column and row element valid. Since line is computed by the current thread (thereby valid), it only remains to guarantee that the column element of the previous line is valid (in figure \ref{fig:diag_deps}, previous refers to the line immediately below). To do that, each block writes last valid diagonal in a <<lock>> array, and next block need only to wait (polling) until desired element is marked valid. Notice that {\tt \_\_threadfence} is not mandatory (thereby slightly improving performance), verifying the observation of \cite{gpu_barrier}.

\begin{lstlisting}[language=C,caption=Synchronization with previous thread block (active waiting)]
__global__ void gpu_solve(/*...*/ volatile unsigned* lock, // = {0}
		unsigned d_start, unsigned d_stop) {
	const unsigned tB = blockIdx.x;
	unsigned tP=d_start; // block progress

	for (unsigned diag=d_start; diag<d_stop; ++diag) {
		/* ... compute diagonal values ... */

		// __threadfence();
		if (threadIdx.x == 0) {
			lock[tB] = ++tP;
			if (tB > 0) while(lock[tB-1]<tP) {}
		}
		__syncthreads();
	}
}
\end{lstlisting}

% ----------------------------------------------
\subsubsection{Parsers code generation}
Parsers generation is independent of user-defined function generation (see \ref{user_fun_gen}). Tabulation inner parsers are first wrapped in additional aggregation (by $h$, thereby ensuring they produce at most one result) and normalized (according to \ref{normalization}); code generation then occurs recursively, producing a list of loops and conditions, and body (possibly with a hoisted part). Additionally, position variables are maintained and subrule index and concatenation indices are propagated. We give an overview of each parser transformation:\ul
\item \textbf{Terminal:} provides its own C code, which correspond usually to the input element value, its position or the position of the matching range.
\item \textbf{Tabulate:} is a simple value load, possibly wrapped into a validity check. Useless validity verification can be removed by marking the tabulation as <<always valid>>.
\item \textbf{Aggregate:} corresponds to an intermediate (value,backtrack) pair where inner parsers write their result; outermost aggregation is written back to corresponding (cost, backtrack) matrices. Validity information, and concatenation indices are propagated within backtrack. To preserve a correct semantic, inner aggregations body is hoisted outside loops and condition checks of the enclosing parser.
\item \textbf{Or:} since parsers are normalized and operate on a single aggregation result, it suffice to emit sequentially code of alternatives.
\item \textbf{Map:} wraps its argument into a the user-defined function call
\item \textbf{Filter:} wraps its body into user-defined condition check
\item \textbf{Concat:} fixed size concatenation are wrapped in simple conditions; moving concatenations are wrapped in a {\tt for} loop. The loops and conditions are further simplified to reduce range and remove useless conditions before actual code is emitted.
\ule

Intermediate types must be correctly declared. To do that each user-defined function provides its input and output types. Aggregation temporary values declaration is ensured by a \textit{exists-or-declare} header policy that is called for every type declaration.

% ----------------------------------------------
\subsubsection{Backtracking on the GPU}
The backtracking is processed similarly to the Scala parser, the major difference being that since we are generating C code, we can provide an immediate mapping from the subrule index to the backtrack elements to add to the trace. The backtrack is done in 3 steps:\ul
\item If a window is set, the windowing aggregation kernel is run to determine the position of the best result within the matrix. Otherwise the best position can be found at the last computed element of the matrix.
\item For a $m\times n$ matrix, allocate a $m+n$ vector with two heads (reading, writing, initialized at the same position). Write the best element in the vector.
\item While there is a vector element that has been written but not read\ul
	\item From the parser id and its position retrieve the corresponding (subrule, concatenation indices) pair by reading in the corresponding matrix cell
	\item Using this, write new backtrack items that are at the origin of the current element.
	\ule
\ule

Since code is generated, it is possible to write the last step using a switch case, thereby flattening the writes in the vector (compared to recursive calls in Scala). Finally, since the trace has to be reversed, we can obtain this transformation for free by constructing the trace list from the end in the JNI conversion. Reversing the list presents the advantage that the trace is immediately usable to construct the desired element. It might be possible that Scala and CUDA parsers provide different traces to construct the same result, because the trace verifies the dependency order, which is only a partial order.

% ----------------------------------------------
\subsubsection{User functions generation} \label{user_fun_gen}
The user generation function needs to be tightly integrated with the rest of the code generation. To do that, we need to establish a relation between the Scala function and its C counterpart. This is done by modifying the Scala function such that it embeds its C code and related types (input, output and possibly internal structures). To do that, LMS is used to generate both Scala and C code (as the user would want to write only once his function, using the corresponding LMS {\tt Rep} types). The two implementations are then mixed to provide the augmented Scala function that can then be used at appropriate places by either the Scala parsers or the code generator.

Actually, the idea of mixing the two implementations into a single function emerged from experiments with the Scala macros \cite{scala_macros}, where it is possible to modify the AST of the Scala program before actually compiling it. Macros could also be an alternative to LMS in the sense that they have the same power in this particular case (because the code is just converted from Scala to C and does not benefit of additional run-time information); however, relying only on macros would imply rewriting significant portions of code conversion, which might end up being a duplicated effort with LMS. The most interesting use of the macros would actually be to stage plain Scala to its LMS representation in the <<context>> of user functions, thereby unleashing the power of LMS without forcing the DSL user to explicitly specify {\tt Rep} types\footnote{Since this is an ongoing project at LAMP with different schedule as this project, we do not want to duplicate effort currently but might integrate it at a later stage.}.

Another advantage of using LMS only for user-specific function, is that it does not impose any restriction on the types manipulated by Scala, thereby providing the opportunity to solve the DP problem (possibly on CUDA using restricted types) and apply the solution (in Scala) on complex types that would have no representation in LMS.

% ------------------------------------------------------------------------------------------------
\subsection{Runtime execution engine}
The runtime execution engine is made of two instrumented compilers:\ul
\item A wrapper for {\tt g++} and {\tt nvcc} that can combine different file types ({\tt .h}, {\tt .c}, {\tt .cu}) into a JNI library which is then loaded into the current JVM instance. If necessary, paths can be customized to fit the user environment.
\item A wrapper for the Scala compiler, which allow the creation of Scala interface to the freshly compiled JNI libraries. It should be noted there that using {\tt VirtualDirectory} as compilation target prevents the interaction with JNI, hence physical path has to be used.
\ule

These two compilers interfaces are then mixed in another class that transform the previously (see \ref{codegen}) generated code, fixing input sizes and splits (number of kernels to launch to respect the time limit) constants, and execute it.

% ------------------------------------------------------------------------------------------------
\newpage
\subsection{LibRNA}
Since the energy computation for RNA secondary structure prediction (folding the sequence in two dimensions) involve complex coefficients and computations (seemingly standardized in coefficient files), we might want to provide the user with a simple interface to benefit from it. To do that, we based our library on the work of GAPC\cite{gapc_thesis} which itself is based on \href{http://www.tbi.univie.ac.at/~ivo/RNA/}{ViennaRNA}\cite{vienna_rna}. Since the library is provided in C, we rely on JNI to reuse the code without modifying it; this allows Scala to immediately benefit from it, but also makes possible to write a GPU version, provided that the related functions are simple enough to be expressible in CUDA. Our work in this direction is mainly focused on integration, we do not want to discuss the implementation details here but simply give a overview of what we transformed and adapted to suit our needs.

First, we adapted the library embedded in GAPC to obtain coefficients. Since GAPC is written in C, we had to write a JNI interface to let the Scala code communicate with the libraries. Once this step has been achieved, we focused on obtaining correct results for RNA folding. This has been achieved by a thorough analysis of the GAPC related code, and took quite a long time due to bugs that were hard to find.

In parallel with this work, we focused on making the code compatible with CUDA. We managed to significantly reduce its size by removing unused functions (actually, all the programs of the Vienna package reuse the library, each introducing its own functions). This also enabled us to have a better understanding of the involved computations, thereby helping us to clearly separate the coefficient file processing and the energies computations. We also provided small optimizations towards parallelization and simplified the sequence management (because it needs to be converted in a particular format for the library to efficiently process it).

Once the library was ready to fit on CUDA devices, we integrated it into DynaProg. Unfortunately, since the library adds a significant volume of code (because some coefficients are embedded in C files), the compilation process length was increased (from approx. 2 to 7 seconds). To reduce this penalty (towards benchmarking), we introduced memoization of the compilation results in our code generator, thereby avoiding duplicated compilations of the program for same length of sequences (generated programs are tailored for particular sequences lengths).

Finally, we made two small nevertheless important enhancements to the CUDA version: because the energy functions heavily rely on the sequence and the coefficients, we would like to have fast access to them. Since we address small sequences ($\le 16$KB), these can easily fit in the shared memory. We also would like to benefit from the constant memory to store the coefficients. Unfortunately, since this memory is too small to contain all of them, we need to distinguish two cases: the most frequently used coefficients are stored in the constant memory whereas the other have to be put in the global memory. These two modifications provided substantial speed improvement by moving the data frequently used closer to the processing units.

% ------------------------------------------------------------------------------------------------
\newpage
\section{Usage} \label{usage}
\subsection{Program examples}
In this section, we explain how to use the DynaProg DSL using an example based approach. We focus on three additional examples: Smith-Waterman (\S\ref{swat_affine}) and Needleman-Wunsch to present two-tracks grammars and multiple algebras, RNAfold\cite{gpu_rnafold} (\S\ref{zuker}, alternative) to describe RNA library usage and reconstruction from backtrack trace, and finally we extend matrix chain multiplication (\S\ref{adp_practice}) with CUDA code generation.

% ----------------------------------------------
\subsubsection{Smith-Waterman and Needleman-Wunsch} \label{ex_swat}
First define a signature that can fit both algebrae, then specify for each algebra the related functions. In this example, both algebra operate on the same output domain and share the same optimization function (although this is not true in general).
\begin{lstlisting}[language=Scala,captionpos=none]
trait SeqAlignSignature extends Signature {
  type Alphabet = Char
  def start(x:Unit):Answer
  def gap1(g:(Int,Int),a:Answer):Answer
  def gap2(a:Answer,g:(Int,Int)):Answer
  def pair(c1:Alphabet,a:Answer,c2:Alphabet):Answer
}

trait SmithWatermanAlgebra extends SeqAlignSignature {
  type Answer = Int
  override val h = max[Int] _
  private val open = -3
  private val extend = -1
  def start(x:Unit) = 0
  def gap1(g:(Int,Int),a:Int) = gap2(a,g) // by symmetry
  def gap2(a:Int,g:(Int,Int)) =
    { val size=g._2-g._1; Math.max(0, a + ( open + (size-1)*extend )) }
  def pair(c1:Char,a:Int,c2:Char) = a + (if (c1==c2) 10 else -3)
}

trait NeedlemanWunschAlgebra extends SeqAlignSignature {
  type Answer = Int
  override val h = max[Int] _
  private val open = -15
  private val extend = -1
  def start(x:Unit) = 0
  def gap1(g:(Int,Int),a:Int) = gap2(a,g) // by symmetry
  def gap2(a:Int,g:(Int,Int)) =
    { val size=g._2-g._1; a + ( open + (size-1)*extend ) }
  def pair(c1:Char,a:Int,c2:Char) = a + (if (c1==c2) 4 else -3)
}
\end{lstlisting}

To obtain a visual representation of the alignment, a naive idea would be to construct the two aligned strings immediately in the forward phase (in the {\tt Answer}). However, this approach must be avoided as it is extremely inefficient, both in terms of running time and space complexity because intermediate strings are created (and stored in memory) for every intermediate result. The correct way to solve this issue is to use backtracking and forward construct these strings with a pretty printing algebra:
\begin{lstlisting}[language=Scala,captionpos=none]
trait SeqPrettyPrint extends SeqAlignSignature {
  type Answer = (String,String)
  def in1(k:Int):Alphabet; def in2(k:Int):Alphabet // make it visible
  private def gap(sw:(Int,Int),in:Function1[Int,Char]) = {
    val g=(sw._1 until sw._2).toList
    (g.map{x=>in(x)}.mkString,g.map{x=>"-"}.mkString)
  }
  def start(x:Unit) = (".",".")
  def gap1(g:(Int,Int),a:Answer) =
    { val (g1,g2)=gap(g,in1); (a._1+g1,a._2+g2) }
  def gap2(a:Answer,g:(Int,Int)) =
    { val (g2,g1)=gap(g,in2); (a._1+g1,a._2+g2) }
  def pair(c1:Char,a:Answer,c2:Char) = (a._1+c1,a._2+c2)
}
\end{lstlisting}

Finally, we describe the associated grammar and the programs that mixes the algebrae and the grammar. Note that we need one instance of each pair of grammar and algebra. Once we have done that, we can request scores and backtracks associated with an evaluation algebra (Smith-Waterman or Needleman-Wunsch) and reuse the obtained backtrack to construct the matching aligned sequences:
\begin{lstlisting}[language=Scala,captionpos=none]
trait SeqAlignGrammar extends TTParsers with SeqAlignSignature {
  val axiom:Tabulate = tabulate("M",(
    empty                     ^^ start
  | seq1() -~ axiom           ^^ gap1
  |           axiom ~- seq2() ^^ gap2
  | el1    -~ axiom ~- el2    ^^ pair
  ) aggregate h)
}

object SeqAlign extends App {
  object SWat extends SeqAlignGrammar with SmithWatermanAlgebra
  object NWun extends SeqAlignGrammar with NeedlemanWunschAlgebra
  object pretty extends SeqAlignGrammar with SeqPrettyPrint
  val seq1 = "CGATTACA"
  val seq2 = "CCCATTAGAG"

  def align(name:String,s1:String,s2:String,g:SeqAlignGrammar) = {
    val (score,bt) = g.backtrack(s1.toArray,s2.toArray).head
    val (a1,a2) = pretty.build(s1.toArray,s2.toArray,bt)
    println(name+" alignment\n- Score: "+score)
    println("- Seq1: "+a1+"\n- Seq2: "+a2+"\n")
  }
  align("Smith-Waterman",seq1,seq2,SWat)
  align("Needleman-Wunsch",seq1,seq2,SWat)
}
\end{lstlisting}

% ----------------------------------------------
\newpage
\subsubsection{RNA folding} \label{ex_rnafold}
We define a signature with two evaluation algebras: {\tt RNAFoldAlgebra} actually computes the folding whereas {\tt RNAFoldPrettyPrint} describes the folding in a string. The energy functions are provided by an external library ({\tt LibRNA}). This library encodes substring as (first character, last character) whereas our framework encodes them as (first character, first character + length), which explains the off-by-one corrections. {\tt energies} variable is set to false in {\tt RNAFoldPrettyPrint} because this algebra does not involve the LibRNA energies functions (that require encoding the input RNA sequence in a special format; this option is enabled by default in the RNASignature trait).

\begin{lstlisting}[language=Scala,captionpos=none]
trait RNAFoldSig extends RNASignature {
  def hairpin(ij:(Int,Int)):Answer
  def stack(i:Int,s:Answer,j:Int):Answer
  def iloop(ik:(Int,Int),s:Answer,lj:(Int,Int)):Answer
  def mloop(i:Int,s:Answer,j:Int):Answer
  def left(l:Answer,r:Int):Answer
  def right(l:Int,r:Answer):Answer
  def join(l:Answer,r:Answer):Answer
}

trait RNAFoldAlgebra extends RNAFoldSig {
  type Answer = Int
  import librna.LibRNA._ // indexing convention: first base,last base
  def hairpin(ij:(Int,Int)) = hl_energy(ij._1,ij._2-1) // Eh
  def stack(i:Int,s:Int,j:Int) = sr_energy(i,j) + s // Es
  def iloop(ik:(Int,Int),s:Int,lj:(Int,Int)) =
      il_energy(ik._1,ik._2,lj._1-1,lj._2-1) + s // Ei
  def mloop(i:Int,s:Int,j:Int) = s
  def left(l:Int,r:Int) = l
  def right(l:Int,r:Int) = r
  def join(l:Int,r:Int) = l+r
  override val h = min[Answer] _
}

trait RNAFoldPrettyPrint extends RNAFoldSig {
  type Answer = String
  override val energies=false
  private def dots(n:Int,c:Char='.') = (0 until n).map{_=>c}.mkString
  def hairpin(ij:(Int,Int)) = "("+dots(ij._2-ij._1-2)+")"
  def stack(i:Int,s:String,j:Int) = "("+s+")"
  def iloop(ik:(Int,Int),s:String,lj:(Int,Int)) =
      "("+dots(ik._2-1-ik._1)+s+dots(lj._2-1-lj._1)+")"
  def mloop(i:Int,s:String,j:Int) = "("+s+")"
  def left(l:String,r:Int) = l+"."
  def right(l:Int,r:String) = "."+r
  def join(l:String,r:String) = l+r
}
\end{lstlisting}

\newpage
We can then define the associated grammar
\begin{lstlisting}[language=Scala,captionpos=none]
trait RNAFoldGrammar extends ADPParsers with RNAFoldSig {
  lazy val Qp:Tabulate = tabulate("Qp",(
    seq(3,maxN)        ^^ hairpin
  | eli   ~ Qp ~ eli   ^^ stack
  | seq() ~ Qp ~ seq() ^^ iloop
  | eli   ~ QM ~ eli   ^^ mloop
  ) filter basepairing aggregate h)

  lazy val QM:Tabulate = tabulate("QM",(Q ~ Q ^^ join) filter((i:Int,j:Int)=>i<=j+4) aggregate h)

  lazy val Q:Tabulate = tabulate("Q",(
    QM
  | Q ~ eli ^^ left
  | eli ~ Q ^^ right
  | Qp
  ) filter((i:Int,j:Int)=>i<=j+2) aggregate h)

  override val axiom = Q
}
\end{lstlisting}

In the application, we create two objects, each combining the grammar with a particular algebra. We can optionally specify a coefficient parameter file with {\tt setParams(file:String)}, otherwise the \textit{Turner2004} coefficients are used. The library is automatically loaded and fed with the sequence to produce correct energy coefficients. We request both the score and the backtrack trace (in {\tt bt}) so that we can reconstruct the folding using the pretty printing grammar.
\begin{lstlisting}[language=Scala,captionpos=none]
object RNAFold extends App {
  object fold extends RNAFoldGrammar with RNAFoldAlgebra
  object pretty extends RNAFoldGrammar with RNAFoldPrettyPrint

  val seq="aaaaaagggaaaagaacaaaggagacucuucuccuuuuucaaaggaagagg"

  val (score,bt) = fold.backtrack(seq.toArray).head
  val res = pretty.build(seq.toArray,bt)
  println("Folding : "+res+" (%5.2f)".format(score/100.0));
}
\end{lstlisting}

% ----------------------------------------------
\newpage
\subsubsection{Matrix multiplication with CUDA code generation} \label{ex_matmult_cuda_lms}
Leveraging the existing definitions of the signature and grammar (repeated here for convenience)
\begin{lstlisting}[language=Scala,captionpos=none]
trait MatrixSig extends Signature {
  type Alphabet = (Int,Int) // Matrix(rows, columns)
  val single:Alphabet=>Answer
  val mult:(Answer,Answer)=>Answer
}

trait MatrixGrammar extends ADPParsers with MatrixSig {
  val axiom:Tabulate = tabulate("M",
    (el ^^ single | axiom ~ axiom ^^ mult) aggregate h)
}
\end{lstlisting}
We need describe the algebra functions in the LMS syntax ({\tt RepWorld}) that we can later compile to use as regular functions, augmented with C code description (necessary for code generation). Finally, we need to mix the {\tt CodeGen} trait to enable code generation and provide the manifest for input and ouput types ({\tt Alphabet} and {\tt Answer}).

\begin{lstlisting}[language=Scala]
trait RepWorld extends NumericOps with TupleOps {
  type Alphabet = (Int, Int)
  type Answer = (Int, Int, Int)

  def hf(a: Rep[Answer]) :Rep[Int] = a._2
  def repSingle(a: Rep[Alphabet]): Rep[Answer] = (a._1, unit(0), a._2)
  def repMult(l: Rep[Answer], r: Rep[Answer]): Rep[Answer] =
    (l._1, l._2 + r._2 + l._1 * l._3 * r._3, r._3)
}

object MatrixMultLMS extends MatrixSig with MatrixGrammar
    with CodeGen with App {
  val tps=(manifest[Alphabet],manifest[Answer])
  override val benchmark = true // display timing measurements

  // Algebra is defined immediately in the concrete program
  type Answer = (Int, Int, Int)
  val concreteProg = new RepWorld with RepPackage
  override val h = minBy(concreteProg.gen(concreteProg.hf))
  val single = concreteProg.gen(concreteProg.repSingle)
  val mult = concreteProg.gen2(concreteProg.repMult)

  val input = List((1,2),(2,20),(20,2),(2,4),(4,2),(2,1),(1,7),(7,3)).toArray
  println(parse(input).head) // -> 1x3 matrix, 122 multiplications
}
\end{lstlisting}

The complete source file of the presented problems can be found in the {\tt report/} folder. For further examples and variants, we encourage you to have a look in the {\tt examples/} folder.

% ------------------------------------------------------------------------------------------------
\newpage
\subsection{Other usage options}
We here provide a list of relevant variables and traits that the programmer might be interested to use. This list only serves the purpose of documenting features that might otherwise be difficult to find within the code.

Although the whole program can be defined in a single trait, it is preferable to cleanly separate the signature from the grammar and the algebra, this good practice would help adding new algebrae easily. The signature needs to inherit either from {\tt Signature} or {\tt RNASignature}, would the RNA folding energies be needed. The grammar can be either single track or two-tracks by inheriting respectively from  {\tt ADPParsers} and {\tt TTParsers}. Note that RNA folding only works for single track grammars and library setup is enabled by the usage of the trait {\tt RNASignature} (this could be changed by disabling the flag {\tt energies}).

The code generator is used by simply mixing in the {\tt CodeGen} trait, using the following idiom
	\[\text{\tt val tps=(manifest[Alphabet],manifest[Answer])}\]
anywhere at the intersection of the CodeGen inheritance and definition of these types (usually in the final program). Further configuration of the execution environment can be tuned by overriding the following variables: {\tt compiler} (for system paths and flags), {\tt cudaSplit} and {\tt cudaDevice}. The {\tt benchmark} flag can be set to enable timing measurements. Also it is possible to use bottom-up parsers with Scala to reduce the stack size by enabling the {\tt bottomUp} flag. If special concatenations are needed, it is possible to replace the $\sim$ concatenation by $\sim\.(l_{\rm min},l_{\rm max},r_{\rm min},r_{\rm max})\.\sim$ where $l,r$ design respectively the yield size of left and right operands.

Finally, the user can look in the files {\tt ADPParsers.scala} and {\tt TTParsers.scala} for a list of the available terminals, and possibly create new ones.

\section{Benchmarks}
{\color{red} XXX}

\section{Future work} \ul
\item Implementation for serial problems larger than memory, using hybrid (Myers and Miller's algorithm with $\log_k$ with $k$ depending on available memory)
\item Transforming non-serial into serial using aggregation functions/transformations
\item Avoid to maintain the cost matrix in memory by moving data in the wavefront
\item Non-serial scheduling for large problems (with memory loads for non-serializable)
\item Add FPGA target platform (George Nithin)
\item Pack the data => less memory transfer (i.e. GATC=>4 letters in 1 char)
\item Operate on larger words (ex 64 bits) to increase thread locality and reduce memory accesses
\ule

\begin{verbatim}
How to Encode multi-dimensional matrices efficiently
1. assume they have the same type put one after another => different dimensions ok
2. assume of same size => put into a struct
=> but using different pointers seems more reliable => completely different matrices => fixed list of matrices by dimensionality (O(1), O(n), O(n^2), ...) of structs (determined by number of indices to access object)

Core function F:
- in: s,t strings, neighbor costs: top, left, top+left, neighbor stats: top, left, top+left
- out: backtrack information, cost(i,j), stats
         stats (x)
          ||
          vv
stats -> [  ] -> new_stats (y')
(y)       || \
          vv  --+ backtrack info (Bxy)
      new_stats (x')

For stats, we need to maintain an horizontal and vertical front
For backtracking we need to maintain the whole table
... | . |
..A | B | min/max: Min(D)=Min(Min(C),Min(B))
----+---+ sum    : Sum(D)=Sum(C)+Sum(B)-Sum(A)
..C | D | avg    : sum both #cells and values then divide at appropriate cell
----+---+
we may not need to store what's the previous cell, however, the previous
cell information (usually limited range) is much more compact than the score(32-64 bits)
\end{verbatim}

\section{Conclusion}
{\color{red} XXX}

% ------------------------------------------------------------------------------------------------
\section*{Planning}
\textbf{Paper introduction}\ul
\item Problem to solve, what exists (related work), how we compare to other, how to evaluate
\item Contributions (3): 3 tensed sentences
\item benchmark => prove by evaluation intro statements
\ule

\subsubsection*{Roadmap}
\begin{tabular}{rl}
16.11 & Rules normalization and automatic backtracking \\
	& GenScala on LMS + GenCuda + LMS CudaCompiler \\
23.11 & Problem generalization: "cyclic keyword", Zucker problem / CudaLoop optimization \\
30.11 &--- Gap due to LMS missing knowledge \\
7.12 & Benchmarking, grammar analysis \\
14.12 & First thoughts for larger than device memory \\
21.12 & Writing report \\
28.12 & --- holiday --- \\
 4.01 & --- holiday --- \\
11.01 & Writing report: implementation description and plan for future work \\
18.01 & Writing report
\end{tabular}

\subsubsection*{Todo @TCK}\ul
\item Test/proof parsers are correct -- make sure implementation is correct
\item Automate test to compare against implementation
\item Benchmarks -- use CUDA profiler(?)
\item Write report
\item Port LibRNA for CUDA?
\ule

\subsubsection*{Todo @Manohar}\ul
\item Integrate LMS code generation into v4.
\item Fix Zuker coefficients
\ule

%Some GPU algorithms: http://hgpu.org/?cat=11
%Translation into C++: http://bibiserv.cebitec.uni-bielefeld.de/macports/resources/download/
%CUDPP libraries (but awfully big resulting binary): http://code.google.com/p/cudpp/
%13 dwarfs: http://developer.amd.com/afds/assets/presentations/2155_final.pdf
%http://tutorials.jenkov.com/java-reflection/fields.html
%http://lampwww.epfl.ch/~michelou/scala/scala-reflection.html
%
%Hint: use TypeClass to put a predicate on types
%  def fun[T: CanTranslateToC](...)
%  def fun[T](implicit ev:CanTranslateToC[T])
%  class CanTranslateToC[T] { def translate:String }
%  implicit def canTranslateInt = new CanTranslateToC[Int] = { def translate = "Int" }

% ------------------------------------------------------------------------------------------------
\newpage
%\usepackage{multicol}
%\usepackage{etoolbox}
%\patchcmd{\thebibliography}{\section*{\refname}}{\begin{multicols}{2}[\section*{\refname}]}{}{}
%\patchcmd{\endthebibliography}{\endlist}{\endlist\end{multicols}}{}{}
%\bibliographystyle{acm}
\bibliographystyle{plain}
\bibliography{bibliography.bib}
\end{document}
