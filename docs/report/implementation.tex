\newpage
\section{Implementation}
% ------------------------------------------------------------------------------------------------
\subsection{CUDA ad-hoc}
In the project planning, an ad-hoc implementation phase immediately followed the problem analysis. The goals of this phase is threefold:\ol
\item Better understand the challenges in CUDA implementation of dynamic programming problems and get on par with state-of-art implementations.
\item Have a baseline implementation that is independent of the hardware and that could be benchmarked. We also tried to contact the authors of \cite{swat_mega} and \cite{gpu_atlp} to obtain their implementation. The former provided us with their implementation, which turned out to address large serial problems whereas our focus was on smaller non-serial problems, the latter did not respond to our solicitations.
\item Have an optimal implementation that can serve as target to be imitated and generalized by the code generation.
\ole

{\center\color{red} XXXXXXXXXXXXXXXXXXXXXXXXXXXXXXXXXXXXXXXXXXXXXXXXXXXXXXXXX\\ CONTINUE HERE\\ XXXXXXXXXXXXXXXXXXXXXXXXXXXXXXXXXXXXXXXXXXXXXXXXXXXXXXXXX}

The project development was scheduled in \ul
\item Problems analysis
\item Ad-hoc implementation (CUDA)
\item Porting ADP parsers
\item Improving parsers / adding functionalities
\item Integrating CUDA generation into parsers
\ule

During the first phase of project, and after the decision to focus on non-serial problems implementation on GPU, we developed an ad-hoc version of 3 non-serial problems. Since we erroneously complicated the matrix of the polygon triangulation problem, we devised 3 different matrix shape, that is upper triangular for single-track problems, rectangular for two-track problems and parallelogram for the polygon triangulation.




\subsection{LibRNA (?)}

\begin{verbatim}
1 week on hash maps
2 weeks to define and analyze the problems (also with FPGA in mind)
3 weeks restriction to non-serial and ad-hoc implementation on GPU (rectangle, triangle, parallelogram); translation of ADP parsers in Scala (Manohar)
1 week run-time engine for Scala/JNI/CUDA
05.11 - rework of the ADP parser to aim at generating C-like code
12.11 - rework of ADP parsers to extend usages (cyclic, two-track) and aim at automatic backtracking
19.11 - explorations in LMS / macros
26.11 - full backtracking: apply/unapply/reapply, rework of the classes
03.12 - Zuker/JNI
10.12 - Yield analysis, code generation
17.12 - code generation: detupling, generic backtrack (vs. ad-hoc), nested aggregates, empty results support
24.12 - code generation, sick
31.12 - report
\end{verbatim}


\subsection{ADP parsers}
\subsection{Transformations and simplifications}

\begin{verbatim}
 * Parser construction:
 * 1. Assign a "OR_id" (sub_rule_id) and "CONCAT_id" to all parsers so that we know which rule applies
 *    How to skip some indices ?
 *
 * Scala running:
 * 2. Provide meaningful backtrack: rule_id and list of concat indices
 *
 * Code generation:
 * 1. Normalize rules (at hash-map insertion (?))
 * 2. Compute dependency analysis between rules => order them into a list/queue
 *    - If rule R contains another rule S unconcatenated (or concatenated with empty)
 *      then we have S -> T (S before T)
 * 3. Compute the maximal number of concatenation among each rule (field in Treeable(?))
 * 4. Break rules into subrules (at each Or, which must be at top of the rule)
\end{verbatim}

\subsection{LMS integration}
Why cannot use LMS for everything
Where and how is LMS useful

\subsection{Code generation}

\subsection{Runtime execution engine}

\subsection{CUDA implementation}

