\usepackage{amssymb,amsmath,amsthm,hyperref,verbatim,pict2e,graphicx,array,listings,appendix,color}
\usepackage{algorithm,algorithmic,booktabs,marvosym,wrapfig,xytree,multicol,multirow,arydshln,nameref}
\hypersetup{colorlinks,citecolor=black,filecolor=black,linkcolor=black,urlcolor=black
	%pdfborderstyle={/S/U/W 1},urlbordercolor=1 0 0,linkbordercolor=.5 1 1, citebordercolor=.5 1 1
}
\usepackage[usenames]{xcolor} % color names ,dvipsnames,svgnames,table
\usepackage[utf8]{inputenc}
\usepackage[T1]{fontenc}
\usepackage[english]{babel}

% margins
\pagestyle{headings}
\oddsidemargin 0.0cm
\evensidemargin 0.0cm
\topmargin 0.0cm
\headheight 0.0cm
\headsep 1.0cm
\textheight 22.0cm
\textwidth 16.0cm
\parskip 0.1cm
\parindent 0.0cm
\footskip 1.0cm

% compact titles
\usepackage[compact]{titlesec}
\titlespacing{\section}{0pt}{8pt}{0pt}
\titlespacing{\subsection}{0pt}{8pt}{0pt}
\titlespacing{\subsubsection}{0pt}{8pt}{0pt}

% compact lists
\usepackage{enumitem}
\setitemize{noitemsep,topsep=0pt,parsep=0pt,partopsep=0pt}
\setenumerate{noitemsep,topsep=0pt,parsep=0pt,partopsep=0pt}
\def\ul{\begin{itemize}}
\def\ule{\end{itemize}}
\def\ol{\begin{enumerate}}
\def\ole{\end{enumerate}}

% misc
\def\up#1{\textsuperscript{#1}}
\def\quote#1{\par\begingroup\leftskip1em\rightskip\leftskip\textit{#1}\par\endgroup}


% listings
\definecolor{dkpink}{RGB}{200,0,100}
\definecolor{gray}{RGB}{128,128,128}
\lstset{
	xleftmargin=20pt,
	numberstyle=\tiny,stepnumber=1,numbersep=5pt,
	showstringspaces=true,         % underline spaces within strings
	tabsize=2,                      % sets default tabsize to 2 spaces
	captionpos=t,                   % sets the caption-position to bottom
	breaklines=true,                % sets automatic line breaking
	breakatwhitespace=true, % sets if automatic breaks should only happen at whitespace
	title=\lstname, % show the filename of files included with \lstinputlisting; also try caption instead of title
	basicstyle=\small\tt,keywordstyle=\color{blue},commentstyle=\color{gray},stringstyle=\color{dkpink}
}
% define Scala syntax
\lstdefinelanguage{Scala}{
	morekeywords={abstract,case,catch,class,def,do,else,extends,false,final,finally,for,if,implicit,import,%
	match,mixin,new,null,object,override,package,	private,protected,requires,return,sealed,super,this,%
	throw,trait,true,try,type,val,var,while,with,yield},
	otherkeywords={=>,<-,<\%,<:,>:,\#,@},sensitive=true,
	morecomment=[l]{//},	morecomment=[n]{/*}{*/},
	morestring=[b]",morestring=[b]',morestring=[b]"""
}

% title page
\makeatletter
\gdef\@subtitle{}\def\subtitle#1{\gdef\@subtitle{#1}}
\def\my@heading{
\def\ps@headings{\let\@mkboth\markboth
	\def\@evenhead{\small \rightmark \hfill \textit{\@title}, p.~\thepage}
	\def\@oddhead{\@evenhead}}\pagestyle{headings}}
\renewcommand{\maketitle}{
	%\begin{titlepage}
	\setcounter{page}{0}\thispagestyle{empty}
	{\centering\null\vfill\includegraphics[width=5.5cm]{inc/logo_epfl.pdf} % EPFL logo
	\vspace{1.5cm}\hrule \vspace{2.5cm} {\LARGE \@title \par} {\large \emph \@subtitle \par}
	\vspace{2.75cm} {\Large \@author \par}
	\vspace{5.5cm} {\large School of Computer and Communication Sciences, EPFL \par}
	\vspace{1.0cm} {\@date \par} % date
	\vfill\null\par}\my@heading
	\newpage
	%\end{titlepage}
}
\newcommand{\shorttitle}{
	\thispagestyle{empty}
	\hfill \includegraphics[width=3cm]{inc/logo_epfl}\vspace{.1cm} % EPFL logo
	\begin{center} {\LARGE \@title} \\ \vspace{.1cm} {\large \textit{\@subtitle}} \\ \rule[1ex]{350pt}{.5pt} \\
	\@author \\ {\small School of Computer and Communication Sciences, EPFL} \vspace{.2cm} \\{\small \@date}
	\end{center} \vspace{.5cm}\my@heading
}
\makeatother

% new XeTeX title page
\usepackage[T1]{fontenc}
\usepackage{fontspec}
\newfontfamily\fonth{Helvetica}
\newfontfamily\fonthn{Helvetica Neue}
\newfontfamily\fonthc{Helvetica Neue Condensed Bold}
\newfontfamily\fonthl{Helvetica Neue UltraLight}

\makeatletter
\renewcommand{\maketitle}{
	%\begin{titlepage}
	\setcounter{page}{0}\thispagestyle{empty}
	\hfill \includegraphics[width=8.5cm]{inc/logo_epfl.pdf} \vfill
	{\fontsize{25pt}{11pt}\fonthc Master project report \vspace{0.5cm}} \\
	{\fontsize{40pt}{11pt}\fonthl \@title} \vspace{0.2cm} \\ {\fontsize{20pt}{11pt}\fonthl \@subtitle} \\
	\vspace{1.5cm} \\
	{\begin{tabular}{ll}
	Laboratory	& Programming Methods Laboratory, LAMP, EPFL \\
	Professor		& \href{mailto:martin.odersky@epfl.ch}{Martin Odersky} \\
	Supervisors	& \href{mailto:vojin.jovanovic@epfl.ch}{Vojin Jovanovic}, \href{mailto:manohar.jonnalagedda@epfl.ch}{Manohar Jonnalagedda}   \\
	Expert		& \href{mailto:mirco.dotta@typesafe.com}{Mirco Dotta}, Typesafe \\
	Student		& \href{mailto:thierry.coppey@epfl.ch}{Thierry Coppey} \\
	Semester		& Autumn 2012 \\
	\end{tabular}}
	\my@heading
	\newpage
	%\end{titlepage}
}
\makeatother

% appendix
\makeatletter
\let\origappendix\appendix
\renewcommand\appendix{\clearpage\pagenumbering{Roman}\origappendix\section*{\appendixname}\lstset{frame=tb,numbers=left}}
\makeatother

% default
\author{\href{mailto:thierry.coppey@epfl.ch}{Thierry Coppey}, \href{mailto:manohar.jonnalagedda@epfl.ch}{Manohar Jonnalagedda}} %, \href{mailto:nithin.george@epfl.ch}{Nithin George}


\title{Draft}
\begin{document}
\maketitle
\pagestyle{headings}
\setcounter{tocdepth}{2} \tableofcontents

% ------------------------------------------------------------------------------------------------
\newpage
\section{Planning/draft}
\subsubsection*{Contributions}\ul
\item Identify DP classes and give a generalization of their shape/dependency graph
\item Provide state of the art/on par parallel implementations for all these classes
\item Generalize/simplify expression by using a DSL to implement these problems
\item Provide multi-platform support (CPU/GPU/FPGA)
\ule

\subsubsection*{Todo @TCK}
\begin{verbatim}
1. check the vanilla implementation
2. read ADP legacy parsers
3. presentation for Thursday
4. Propose an IR and code generators related to
Note. We only need to generate code directly for CUDA, no interaction. Look at string templating .
"""xxx $var xxx"""
- Fork LMS on GitHub and cleanup all the CUDA generator

% class problem into categories
% think of hashmap implementation. is it useful ?
- CPU+GPU implementation for serial (NS?) problems larger than device memory
- optimizations for in-memory computations

1. restriction vs ADP: we give only the best answer, do we care about semi-optimal ones?
2. separate initialization (base cases, atoms) and processing (rules, recurrences) => less cases to handle (no if)
3. use sub-cell parallelism (seen in 2 papers already)

Plan:
We need to extend the language to support all 3 cases
- Parallelogram is triangular + cyclic => add a cyclic keyword that applies to the whole problem
- Square: what is the transform being done in ADP extension?

1. optimize: terminals of bounded yield + binary decomposition if possible
2. transform into: axioms (init/fixed) + rules (iterations)
3. transform to code
     axiom(i := k(i)) -> M[i,i]=k(i)
     rule(x~y := h(x,y)) -> M[i,j]=forall_i<k<j h(M[i,k],M[k,j]) for both unbounded yield
     rule(x~y_l := h(x,y)) -> M[i,j]=forall_j-l<k<j h(M[i,k],M[k,j]) for y bounded by l
     rule(x~y_l := h(x,y)) -> M[i,j]=h(M[i,j-l],M[j-l,j]) for y of exactly l

XXX: how to encode multi-dimensional matrices
1. assume they have the same type put one after another => different dimensions ok
2. assume of same size => put into a struct
=> but using different pointers seems more reliable => completely different matrices => fixed list of matrices by dimensionality (O(1), O(n), O(n^2), ...) of structs (determined by number of indices to access object)
\end{verbatim}

\subsubsection*{Todo @Manohar}
\begin{verbatim}
- generate chain matrix recurrences to generate implementation
- more theory behind what we want to support, how we encompass all cases, ...
\end{verbatim}

\subsection*{Plan} \ol
\item {\color{gray}
\textbf{Problems description:} parallel tree-raking, does not share much with other algorithms (sparse version of computations, might not scale efficiently). Most common patterns are already enclosed by the above problems. Real input size is around 300K. We might want to also look at an $O(n^3)$-space-complex problem (like matching 3 strings $S,T,U$).}

All the problems we consider use 2D storage matrix, their dependencies are an union of: \ul
\item Serial dependencies
\item Non-serial horizontal or vertical dependencies (1D non-serial)
\item Non-serial horizontal+vertical dependencies in the form $M_{(i,j)} = {\rm op}_k f ( M_{(i,k)}, M_{(k,j)} ) $
\item We have not found other type of dependencies in the literature
\ule

\item \textbf{User facing language:} goals are flexibility and compactness.\ul
	\item User-facing language should be similar to related paper \cite{adp_gpu} or \href{http://hackage.haskell.org/package/ADPfusion}{\it ADP fusion}. We want to reuse the transformation mapping (problem description) $\mapsto$ (kernel implementation) for a single element.
	\item We also may want to try to make implicit transformation for code like \\
		{\tt @DP def Fib(n:Int) = if (n<=2) return 1 else Fib(n-1)+Fib(n-2)}.
	\item Windowing: the user should be able to force a windowing (i.e. force a non-serial problem to be a $k$-polyadic serial problem).	
	\item 3 different cases: we care about backtrack cost or both.
	\item Backtracking: create an operator that produces the whole backtrack sequence indices.
	\ule

$\implies$ \emph{end of October}.

\item \textbf{Prototyping:} get a prototype to understand difficulties and share common base. \\ Implement a working prototype of \nameref{aswat} on CPU (for correctness), and specific platform (CUDA/FPGA). This will give us an idea of how to implement the general case. We also need to benchmark and compare both implementations to see how we compare to existing implementations and see the direction to take (which decide is faster and by how much). Here we aim to do as good an implementation for the specific platform (CPU/GPA) as possible.\\
$\implies$ \emph{end of October}.

\item \textbf{Baseline:} Also use benchmarks provided by existing implementations as baselines.\\
$\implies$ \emph{end of October}.

\item \textbf{Formalize IR:} describe the intermediate representation, formalize the framework provided to the code generators (i.e. memory management, ...).
\item \textbf{Full compiler stack:} enrich the compiler stack from both top-down (translate best user-facing language parsers) and bottom-up (parametric code generators), core of the work.
\item \textbf{Benchmark:} make sure implementations are correct, compare them other papers.
\item \textbf{Optimizations:} improve as much as possible / as long as time permits
\ole

% --------------------------------------------------------------------------------
\newpage
\section{Dynamic programming problems}
% ------------------------------------------------------------------------------------------------
\subsection{Problems classification}
\subsubsection{Definitions}\ul
\item \textbf{Dimensions:} let $n$ the size of the input and $d$ the dimension of the underlying matrix.
\item \textbf{Matrices:} we refer indifferently by the matrix or the matrices to all the intermediate cost- and backtrack-related informations that are necessary to solve the dynamic programming problem of interest. Matrices elements are usually denoted by $M_{(i,j)}$ ($i^{\rm th}$ line , $j^{\rm th}$ column).
\item \textbf{Computation block:} this is a part of the DP matrix (cost and or backtrack) that we want to compute. A block might be either a sub-matrix (rectangular) or a parallelogram, possibly cropped at its parent matrix boundaries.
\item \textbf{Wavefront:} the wavefront consists of all the data necessary to reconstruct a computation block of the DP matrix. It might include some previous lines/columns/diagonals as well as line-/column-/diagonal-wise aggregations (min, max, sum, ...).
\item \textbf{Delay:} we call delay the maximum distance between an element and its dependencies along column and lines (ex: recurrence $M_{(i,j)}=f\big(M_{(i-1,j)}, M_{(i-2,j-1)}\big)$ has delay 3).
\ule

\subsubsection{Litterature classification}
In the literature, dynamic programming problems (DP) are classified according to two criteria:\ul
\item \textbf{Monadic/polyadic:} a problem is monadic when only one of the previously computed term appears in the right hand-side of the recurrence formula (ex: Smith-Waterman). When two or more terms appear, the problem is polyadic (ex: Fibonacci, $F_n = F_{n-1} + F_{n-2}$).
When a problem is polyadic with index $p$, it also means that its backtracking forms a $p$-ary tree (where each node has at most $p$ children).

\item \textbf{Serial/non-serial:} a problem is serial ($s=0$) when the solutions depends on a fixed number of previous solutions (ex: Fibonacci), otherwise it is said to be non-serial ($s\ge 1$), as the number of dependencies grows with the size of the subproblem. That is computing an element of the matrix would require $O(n^s)$.  (ex: Smith-Waterman with arbitrary gap is $s=1$; we can usually infer $s$ from the number of bound variables in the recurrence formula)
	\[M_{(i,j)}=\max\left\{\begin{array}{l} ... \\ M_{(i,j-1)}\\ \max\limits_{i<k<j} [ M_{(i,k)}+M_{(k+1,j)} ] \end{array}\right. \]
\ule

Note that the algorithmic complexity of a problem is exactly $O\big(n^{d+s}\big)$.

\subsubsection{Calculus simplifications}
In some special case, it is possible to transform a non-serial problem into a serial problem, if we can embed the non-serial term into an additional aggregation matrix. For example:
	\[M_{(i,j)}=\max\left\{\begin{array}{l} \max\limits_{k<i} M_{(k,j)}
	\\ \sum\limits_{k<i, l<j}M_{(k,l)} \end{array}\right.
	\implies M_{(i,j)}=\max\left\{\begin{array}{l} C_{(k,j)} \\ A_{(i-1,j-1)} \end{array}\right.\]
Where the matrix $C$ stores the maximum along the column and matrix $A$ stores the sum of the array of the previous elements. Both can be easily computed with an additional recurrence:
	\[\begin{array}{rcl} C_{(i,j)}&=&\max(C_{(i-1,j)}, M_{(i,j)}) \\
	A_{(i,j)}&=&A_{(i-1,j)}+A_{(i,j-1)}-A_{(i-1,j-1)}+M_{(i,j)}\end{array}\]

Although this simplification removes some non-serial dependencies at the cost of extra storage in the wavefront, it is not sufficient to transform all non-serial monadic problems into serial problems (ex: this does not apply to Smith-Waterman with arbitrary gap cost).

% ------------------------------------------------------------------------------------------------
\subsection{Problems of interest}
We usually focus on problem that have an underlying bi-dimensional matrix ($d=2$) because they can be parallelized (as opposed to be serial if $d=1$) and can solve large problems (of size $n$). Problems of higher matrix dimensionality ($d\ge3$) require substantial memory which severely impacts their scalability. Also we tend to limit algorithmic complexity of the problems as from $O(n^4)$ on, running time becomes a severely limiting factor.

We describe problems structures: inputs, cost matrices and backtracking matrix. These all have an alphabet (that must be bounded in terms of bit-size). Unless otherwise specified, we adopt the following conventions:\ul
\item Matrices dimensions are implicitly specified by number of indices and their number of elements is usually the same as the input length.
\item Number are all unsigned integers
\item Problem dimension is $m,n$ (or $n$) indices $i,j$ ranges are respectively $0\le i<m$, $0\le j<n$.
\item Unless otherwise specified, the recurrence applies to all non-initialized matrix elements.
\ule
We describe the problem processing in terms of both initialization and recurrences.

% ------------------------------------------------------------------------------------------------
% Recurrence visualization helpers
\newcommand\Cd[3][0,-1]{\put(#2){\put(.5,.5){\circle*{.3}}\put(.5,.5){\linethickness{1.5pt}\vector(#1){#3}}}} % dependency [dx,dy]{x,y}{len}
\def\Cg#1{\put(#1){\color{lightgray}\put(0,0){\polygon*(0,0)(0,1)(1,1)(1,0)}}} % grayed cell (not to store
\def\Cz#1{\put(#1){\put(0,.35){\parbox{1\unitlength}{\centering\bf 0}}}} % zero-init cell
\def\Cm{\put(6.5,4.5){\circle*{.4}}\multiput(0,0)(1,0){9}{\line(0,1){8}}\multiput(0,0)(0,1){9}{\line(1,0){8}}} % matrix base
\def\Cfl#1{#1{0,6}#1{0,5}#1{1,5}#1{0,4}#1{1,4}#1{2,4}#1{0,3}#1{1,3}#1{2,3}#1{3,3}#1{0,2}#1{1,2}#1{2,2}#1{3,2}#1{4,2}
	#1{0,1}#1{1,1}#1{2,1}#1{3,1}#1{4,1}#1{5,1}#1{0,0}#1{1,0}#1{2,0}#1{3,0}#1{4,0}#1{5,0}#1{6,0}} % triangular lower (function)
\def\Cfd#1{#1{0,7}#1{1,6}#1{2,5}#1{3,4}#1{4,3}#1{5,2}#1{6,1}#1{7,0}} % main diagonal

\def\Cmlong{\put(6.5,4.5){\circle*{.4}}\multiput(0,0)(1,0){16}{\line(0,1){8}}\multiput(0,0)(0,1){9}{\line(1,0){15}}} % matrix base
\def\Cfu#1{#1{8,7}#1{9,7}#1{10,7}#1{11,7}#1{12,7}#1{13,7}#1{14,7}#1{9,6}#1{10,6}#1{11,6}#1{12,6}#1{13,6}#1{14,6}#1{10,5}#1{11,5}#1{12,5}#1{13,5}#1{14,5}#1{11,4}#1{12,4}#1{13,4}#1{14,4}#1{12,3}#1{13,3}#1{14,3}#1{13,2}#1{14,2}#1{14,1}} % triangular upper (function)

% ----------------------------------------------
\newpage
\subsubsection{Smith-Waterman (simple)}\label{sswat}\ol
\item Problem: matching two strings $S$, $T$ with $|S_{\rm padded}|=m, |T_{\rm padded}|=n$.
\item Matrices: $M_{m \times n}, B_{m \times n}$
\item Alphabets:\ul
	\item Input: $\Sigma(S)=\Sigma(T)=\{a,c,g,t\}$.
	\item Cost matrix: $\Sigma(M) = [0..z], z=\max({\rm cost(\_)}) \cdot \min(m,n)$
	\item Backtrack matrix: $\Sigma(B)=\{stop,W,N,NW\}$
	\ule
\item Initialization:\ul
	\item Cost matrix: $M_{(i,0)}=M_{(0,j)}=0$.
	\item Backtrack matrix: $B_{(i,0)}=B_{(0,j)}=stop$.
	\ule
\item Recurrence: \[M_{(i,j)}=\max\left\{\begin{array}{l|l}
		0 & stop\\
		M_{(i-1,j-1)}+{\rm cost}(S(i),T(j)) & NW\\
		M_{(i-1,j)}-d & N\\
		M_{(i,j-1)}-d & W
	\end{array}\right\}=B_{(i,j)} \]

\item Backtracking: starts from the cell $\max \{M_{(m,j)} \cup M_{(i,n)}\}$, stops at
the first cell containing a $0$.
\item Visualization: by convention, we put the longest string vertically ($m\ge n$):
\begin{center}\setlength{\unitlength}{.6cm}\begin{picture}(8,9)
	\put(-.5,7.5){S}\put(-.35,7.4){\linethickness{1pt}\vector(0,-1){2}}
	\put(.2,8.2){T}\put(.8,8.4){\linethickness{1pt}\vector(1,0){2}}
	\Cz{0,0}\Cz{0,1}\Cz{0,2}\Cz{0,3}\Cz{0,4}\Cz{0,5}\Cz{0,6}\Cz{0,7}
	\Cz{1,7}\Cz{2,7}\Cz{3,7}\Cz{4,7}\Cz{5,7}\Cz{6,7}\Cz{7,7}
	\Cd{6,5}{0.8}
	\Cd[1,0]{5,4}{0.8}
	\Cd[1,-1]{5,5}{0.8}
\Cm\end{picture}\end{center}

\item Optimizations:\ul
	\item In serial (monadic) problems we can avoid building the matrix $M$ by only maintaining the 3 last diagonals in memory (one for the diagonal element, one for horizontal/vertical, and one being built). This construction extends easily to polyadic problems where we need to maintain $k+2$ diagonals in memory where $k$ is the maximum backward lookup.
	\item Padding: since first line and column of the matrix are zeroes, their initialization might be omitted, but this would implies more involved initialization and computations, which is cumbersome. Also since to fill the $i^{\rm th}$ row we refer to the $(i-1)^{\rm th}$ character of string $S$ thus we prepend to both $S$ and $T$ an unused character, so that matrix and input lines are aligned. Hence valid input indices are $S[1 \cdots m-1]$ and $T[1 \cdots n-1]$. We refer as such strings as padded strings hereafter (with $|S_{\rm padded}| = |S| + 1$).
	\ule
\ole

% ----------------------------------------------
\newpage
\subsubsection{Smith-Waterman with affine gap extension cost}\ol
\item Problem: matching two strings $S$, $T$ with $|S_{\rm padded}|=m, |T_{\rm padded}|=n$.
\item Matrices: $M_{m \times n}, E_{m \times n}, F_{m \times n}, B_{m \times n}$
\item Alphabets:\ul
	\item Input: $\Sigma(S)=\Sigma(T)=\{a,c,g,t\}$.
	\item Cost matrices: $\Sigma(M) = \Sigma(E) = \Sigma(F) = [0..z], z=\max({\rm cost(\_)}) \cdot \min(m,n)$
	\item Backtrack matrix: $\Sigma(B)=\{stop,W,N,NW\}$
	\ule
\item Initialization:\ul
	\item No gap cost matrix: $M_{(i,0)}=M_{(0,j)}=0$.
	\item T-gap extension cost matrix: $E_{(i,0)}= 0$ \textit{<<eat S chars only>>}
	\item S-gap extension cost matrix: $F_{(0,j)}= 0$
	\item Backtrack matrix: $B_{(i,0)}=B_{(0,j)}=stop$.
	\ule
\item Recurrence for the cost matrices:
\[\begin{array}{rcl}
M_{(i,j)}&=&\max\left\{\begin{array}{l|l}
	0 & stop\\
	M_{(i-1,j-1)}+{\rm cost}(S(i),T(j)) & NW\\
	E_{(i,j)} & N\\
	F_{(i,j)} & W
\end{array}\right\}=B_{(i,j)}\\
\\
E_{(i,j)}&=&\max\left\{\begin{array}{l|l}
	M_{(i, j-1)} - \alpha & NW\\
	E_{(i,j-1)} - \beta & N\\
\end{array}\right\}=B_{(i,j)}\\
\\
F_{(i,j)}&=&\max\left\{\begin{array}{l|l}
	M_{(i-1,j)} - \alpha & NW\\
	F_{(i-1,j)} - \beta & W\\
\end{array}\right\}=B_{(i,j)}
\end{array}\]

That can be written alternatively as:
\[M_{(i,j)}=\max\left\{\begin{array}{l|l}
	0 & stop\\
	M_{(i-1,j-1)}+{\rm cost}(S(i),T(j)) & NW\\
	\max_{1 \le k \le j-1} M_{(i,k)} - \alpha - (j-1-k) \cdot \beta & N\\
	\max_{1 \le k \le i-1} M_{(k,j)} - \alpha - (i-1-k) \cdot \beta & W\\
\end{array}\right\}=B_{(i,j)} \]

Although the latter notation seems more explicit, it introduces non-serial dependencies that the former set of recurrences is free of. So we need to implement the former rules whose kernel is 
\[ [M;E;F]_{(i,j)} = f_{\rm kernel} ( [M;E]_{(i,j-1)}, [M;F]_{(i-1,j)}, M_{(i-1,j-1)} ) \]
Notice that this recurrence is very similar to \nameref{sswat} except that we propagate 3 values ($M,E,F$) instead of a single one ($M$). Also notice that it is possible to propagate $E$ and $F$ inside a resp. horizontal and vertical wavefront.

\item Backtracking: same as \nameref{sswat}
\item Visualization: same as \nameref{sswat}
\item Optimizations: same as \nameref{sswat}
\ole

% ----------------------------------------------
\newpage
\subsubsection{Smith-Waterman with arbitrary gap cost}\label{aswat}\ol
\item Problem: matching two strings $S$, $T$ with $|S_{\rm padded}|=m, |T_{\rm padded}|=n$ with an arbitrary gap function $g(x)\ge 0$ where $x$ is the size of the gap. Without loss of generality, let $m\ge n$\footnote{Otherwise if $|T|>|N|$ we only need to swap both the inputs and backtracking pairs.}. Example penalty function could be\footnote{Intuition: long gaps penalize less, at some point, one large gap is better than matching and smaller gaps.} $g(x)=m-x$.
\item Matrices: $M_{m \times n}, B_{m \times n \times m}$
\item Alphabets:\ul
	\item Input: $\Sigma(S)=\Sigma(T)=\{a,c,g,t\}$.
	\item Cost matrix: $\Sigma(M) = [0..z], z=\max({\rm cost(\_)}) \cdot \min(m,n)$
	\item Backtrack matrix: $\Sigma(B)=\{stop,NW,N_{\{0..m\}},W_{\{0..n\}}\}$
	\ule

\item Initialization:\ul
	\item Match cost matrix: $M_{(i,0)}=M_{(0,j)}=0$.
	\item Backtrack matrix: $B_{(i,0)}=B_{(0,j)}=stop$.
	\ule

\item Recurrence: \[M_{(i,j)}=\max\left\{\begin{array}{l|l}
	0 & stop\\
	M_{(i-1,j-1)}+{\rm cost}(S(i),T(j)) & NW\\
	\max_{1 \le k \le j-1} M_{(i,j-k)} - g(k) & N_k\\
	\max_{1 \le k \le i-1} M_{(i-k,j)} - g(k) & W_k\\
\end{array}\right\}=B_{(i,j)} \]

\item Backtracking: similar to \nameref{sswat} except that you can jump of $k$ cells.
\item Visualization:
	\begin{center}\setlength{\unitlength}{.6cm}\begin{picture}(8,9)
		\put(-.5,7.5){S}\put(-.35,7.4){\linethickness{1pt}\vector(0,-1){2}}
		\put(.2,8.2){T}\put(.8,8.4){\linethickness{1pt}\vector(1,0){2}}
	\Cz{0,0}\Cz{0,1}\Cz{0,2}\Cz{0,3}\Cz{0,4}\Cz{0,5}\Cz{0,6}\Cz{0,7}
	\Cz{1,7}\Cz{2,7}\Cz{3,7}\Cz{4,7}\Cz{5,7}\Cz{6,7}\Cz{7,7}
		\Cd[0,-1]{6,7}{2.8}\Cd[0,-1]{6,6}{1.8}\Cd{6,5}{0.8}
		\Cd[1,0]{0,4}{5.8}\Cd[1,0]{1,4}{4.8}\Cd[1,0]{2,4}{3.8}\Cd[1,0]{3,4}{2.8}\Cd[1,0]{4,4}{1.8}\Cd[1,0]{5,4}{0.8}
		\Cd[1,-1]{5,5}{0.8}
	\Cm\end{picture}\end{center}

\item Optimizations: The dependencies here are non-serial, there is no optimization that we can 
apply out of the box here.
\ole


% ----------------------------------------------
\newpage
\subsubsection{Convex polygon triangulation}\ol
\item Problem: triangulating a polygon of $n$ vertices with least total cost for added edges. We denote the cost of adding an edge between the  pair of edges $i,j$ by $S(i,j)$, Where $S_{n \times n}$ is a lower triangular matrix compacted in memory (rows are contiguous) with a 0 diagonal that is omitted \footnote{Arbitrary convention for both architectural implementation and code generator. Rationale: in lower triangular matrix, element address is independent of the matrix size.}, hence $|S|=\tfrac{n^2}{2}=N$.

\item Matrices: $M_{n\times 2n}, B_{n \times 2n}$ \textit{<<first edge, last edge>>} upper triangular including main diagonal 
\item Alphabets:\ul
	\item Input: $\Sigma(S_{(i,j)})=\{0..m\}$ with $m=\max_S(i,j) \forall i,j$ determined at runtime\footnote{We need to scan/have stats about $S$ and that's where LMS plays a role}.
	\item Cost matrix: $\Sigma(M)=\{0..z\}$ with $z = m \cdot (n-2)$ (we add at most $n-2$ edges).
	\item Backtrack matrix: $\Sigma(B)=\{stop, 0..n\}$ (the index of the edge we add)
	\ule
\item Initialization: $M_{(i,i)}=0, M_{(i,i+1)}=0, B_{(i,i)}=stop \quad\forall i$

\item Recurrence: \[M_{(i,j)}=\left\{ S(i,j) + \max_{i<k<j}M_{(i,k)}+M_{(k,j)} \,\,\rule[-.75em]{.5pt}{2em}\,\,  k \right\} = B_{(i,j)} \]
	It is interesting to note that even in the sequential world, this problem is solved 
	by filling the diagonals, ie. computing sub-solutions for all polygons of size $k$ before
	those of size $k+1$.

\item Backtracking: Start at $B_{(1,n)}$. Use the following recursive function for the smaller polygons:
	\[{\rm BT}(B_{(i,j)}=k) \mapsto \left\{\begin{array}{ll} A_i & \text{if } k=0 \lor k=j \\
		\Big( {\rm BT}(B_{(i,k)}) \Big) \cdot \Big( {\rm BT}(B_{(k+1,j)}) \Big) & \text{otherwise} \end{array}\right.\]

\item Visualization: the layout is the a matrix of size $n \times (2n-2)$, because of polygons being 
"cyclical" in nature.
\begin{center}\setlength{\unitlength}{.6cm}\begin{picture}(16,9)
	\put(-.7,6.5){\rotatebox{90}{First}}\put(-.4,6.4){\linethickness{1pt}\vector(0,-1){2}}
	\put(.2,8.2){Last}\put(1.5,8.4){\linethickness{1pt}\vector(1,0){2}}
	\Cfl{\Cg}\Cfd{\Cz}\Cfu{\Cg}
	\Cd[0,1]{6,1}{2.8}\Cd[0,1]{6,2}{1.8}\Cd[0,1]{6,3}{0.8}
	\Cd[1,0]{3,4}{2.8}\Cd[1,0]{4,4}{1.8}\Cd[1,0]{5,4}{0.8}
	\put(3.5,4.5){\line(3,-1){3}}
	\put(4.5,4.5){\line(2,-2){2}}
	\put(5.5,4.5){\line(1,-3){1}}
	%\Cd[1,-1]{5,2}{0.8}
\Cmlong\end{picture}\end{center}

\item Optimizations: we need to rotate that matrix to progress in the same direction as usual, that is towards bottom right.
{\color{red}}
\ole

% ----------------------------------------------
\newpage
\subsubsection{Matrix chain multiplication}\ol
\item Problem: find an optimal parenthesizing of the multiplication of $n$ matrices $A_i$. Each matrix $A_i$ is of dimension $r_i \times c_i$ and $c_i=r_{i+1} \forall i$. \textit{<<r=rows, c=columns>>}
\item Matrices: $M_{n \times n}, B_{n \times n}$ \textit{(first, last matrix)}
\item Alphabets:\ul
	\item Input: matrix $A_i$ size is defined as pairs of integers $(r_i,c_i)$.
	\item Cost matrix: $\Sigma(M)= 1..z$ with $z\le n\cdot \big[ \max_i(r_i,c_i) \big]^3 $.
	\item Backtrack matrix: $\Sigma(B)=\{stop\} \cup \{0..n\}$.
	\ule
\item Initialization:\ul
	\item Cost matrix: $M_{(i,i)}=0$.
	\item Backtrack matrix: $B_{(i,i)}=stop$.
	\ule
\item Recurrence: $c_k=r_{k+1}$
	\[M_{(i,j)}=\min_{i\le k<j}\left\{\begin{array}{l|l}
		M_{(i,k)}+M_{(k+1,j)}+r_i \cdot c_k \cdot c_j & k
	\end{array}\right\}=B_{(i,j)} \]
\item Backtracking: Start at $B_{(1,n)}$. Use the following recursive function for parenthesizing
	\[{\rm BT}(B_{(i,j)}=k) \mapsto \left\{\begin{array}{ll} A_i & \text{if } k=0 \lor k=j \\
		\Big( {\rm BT}(B_{(i,k)}) \Big) \cdot \Big( {\rm BT}(B_{(k+1,j)}) \Big) & \text{otherwise} \end{array}\right.\]

\item Visualization:
	\begin{center}\setlength{\unitlength}{.6cm}\begin{picture}(8,9)
		\put(-.7,6.5){\rotatebox{90}{First}}\put(-.4,6.4){\linethickness{1pt}\vector(0,-1){2}}
		\put(.2,8.2){Last}\put(1.5,8.4){\linethickness{1pt}\vector(1,0){2}}
		\Cfl{\Cg}\Cfd{\Cz}
		\Cd[0,1]{6,1}{2.8}\Cd[0,1]{6,2}{1.8}\Cd[0,1]{6,3}{0.8}
		\Cd[1,0]{3,4}{2.8}\Cd[1,0]{4,4}{1.8}\Cd[1,0]{5,4}{0.8}
		\put(3.5,4.5){\line(3,-1){3}}\put(4.5,4.5){\line(2,-2){2}}\put(5.5,4.5){\line(1,-3){1}}
	\Cm\end{picture}\end{center}

\item Optimizations:\ul
	\item We need to swap vertically the matrix to have a normalized progression towards bottom right. To do that, we need to map all indices $i \mapsto n-1-i$ and we obtain a new recurrence relation:
	\[M_{(i,j)}=\min_{i\le k<j}\left\{\begin{array}{l} M_{(i,k)}+M_{(2i-1 -k,j)}+r_i \cdot c_k \cdot c_j \end{array}\right. \]
	With the initialization at $M_{(i,n-i-1)}$
	\ule
\ole

% ----------------------------------------------
\newpage
\subsubsection{Nussinov algorithm}\ol
\item Problem: folding a RNA string $S$ over itself $\left\lfloor |S| / 2 \right\rfloor = n$.
\item Matrices: $M_{n\times n}, B_{n \times n}$
\item Alphabets:\ul
	\item Input: $\Sigma(S)=\{A,C,G,U\}$.
	\item Cost matrix: $\Sigma(M)=\{0..n\}$
	\item Backtrack matrix: $\Sigma(B)=\{stop,D,1..n\}$
	\ule
\item Initialization: \ul
	\item Cost matrix: $ M_{(i,i)}=M_{(i,i-1)}=0$
	\item Backtrack matrix: $B_{(i,i)}=B_{(i,i-1)}=stop$
	\ule
\item Recurrences:
	\[M_{(i,j)}=\max\left\{\begin{array}{l|l}
		M_{(i+1,j-1)}+\omega(i,j) & D\\
		\max_{i\le k<j}M_{(i,k)}+M_{(k+1,j)} & k
	\end{array}\right\} = B_{(i,j)} \]
	With $\omega(i,j)=1$ if $i,j$ are complementary. 0 otherwise.
\item Backtracking: Start the backtracking in $B_{(1,n)}$ and go backward. The backtracking is very similar to that of the matrix multiplication, except that we also introduce the diagonal matching.
\item Visualization:
	\begin{center}\setlength{\unitlength}{.6cm}\begin{picture}(8,9)
		\put(-.7,6.5){\rotatebox{90}{First}}\put(-.4,6.4){\linethickness{1pt}\vector(0,-1){2}}
		\put(.2,8.2){Last}\put(1.5,8.4){\linethickness{1pt}\vector(1,0){2}}
		\Cfl{\Cg}\Cfd{\Cz}
		\Cd[0,1]{6,1}{2.8}\Cd[0,1]{6,2}{1.8}\Cd[0,1]{6,3}{0.8}
		\Cd[1,0]{3,4}{2.8}\Cd[1,0]{4,4}{1.8}\Cd[1,0]{5,4}{0.8}
		\Cd[1,1]{5,3}{0.8}
		\put(3.5,4.5){\line(3,-1){3}}\put(4.5,4.5){\line(2,-2){2}}\put(5.5,4.5){\line(1,-3){1}}
	\Cm\end{picture}\end{center}

\item Optimizations: note that this is very similar to the matrix multiplication except that we also need the diagonal one step backward, so the same optimization can apply.
\ole

% ----------------------------------------------
\newpage
\subsubsection{Zuker folding}\ol
\item Problem: folding a RNA string $S$ over itself $\left\lfloor |S| / 2 \right\rfloor = n$.
\item Matrices: $V_{n\times n}, W_{n\times n}, F_n$ (Free Energy),  $BV_{n \times n}, BW_{n \times n}, BF_n$
\item Alphabets:\ul
	\item Input: $\Sigma(S)=\{A,C,G,U\}$.
	\item Cost matrices:\ul
		\item $\Sigma(W)=\Sigma(V)=\{0..z\}$ with $z \le n*b+c$
		\item $\Sigma(F)=\{0..y\}$ with $y\le \min(F_0, z\cdot n)$
		\ule
	\item Backtrack matrices: \ul
		\item $\Sigma(BW)=\{stop, S,W,V,k\}$
		\item $\Sigma(BV)=\{stop, HL, IL, SW, (i,j) , k\}$ with $0\le i,j,k < n$ \\
		$HL$=HairpinLoop, $IL$=InteriorLoop, $(i,j)$=MultiLoop
		
		
		\item $\Sigma(BF)=\{stop, L, k\}$ with $0\le k < n$
		\ule
	\ule
\item Initialization:\ul
	\item Cost matrices: $W_{(i,i)}=V_{(i,i)}=0, F_{(0)}=$ energy of the unfolded RNA.
	\item Backtrack matrices: $BW_{(i,i)}=BV_{(i,i)}=BF_{(0)}=stop$.
	\ule
\item Recurrence:
\[\begin{array}{rcl}
W_{(i,j)}&=&\min\left\{\begin{array}{l|l}
	W_{(i+1,j)}+b & S\\
	W_{(i,j-1)}+b & W\\
	V_{(i,j)}+\delta(S_i,S_j) & V \\
	\min_{i<k<j}W_{(i,k)}+W_{(k+1,j)} &k
\end{array}\right\} = BW_{(i,j)}\\
\\
V_{(i,j)}&=&\min\left\{\begin{array}{l|l}
	\infty \qquad\qquad\qquad\qquad {\rm if}(S_i,S_j) \text{ is not a base pair} & stop\\\\
	eh(i,j)+b \qquad\qquad\, \text{otherwise} & HL\\
	V_{(i+1,j-1)}+es(i,j) & IL \\
	VBI_{(i,j)} & (i',j') \\
	\min_{i<k<j-1}\{W_{(i+1,k)}+W_{(k+1,j-1)}\} +c & k
\end{array}\right\} = BV_{(i,j)}\\
\\
VBI_{(i,j)}&=&\min\Big\{\min_{i<i'<j'<j}V_{(i',j')}+ebi(i,j,i',j')\} +c \,\,\Big|\,\, (i',j') \Big\}=BV_{(i,j)}\\
\\
F_{(j)}&=&\min\left\{\begin{array}{l|l}
	F_{(j-1)} & L \\ 
	\min_{1\le i< j} (F_{(i-1)} + V_{(i,j)}) & i
\end{array}\right\} = BF_{(j)}
\end{array}\]

In practice, we don't go backward for larger values than 30, so we can replace $\min_{i<k<j}$ by $\min_{\max(i,j-30)<k<j}$ in the expressions of $VBI$.

\item Backtracking: Start at $BF_{(n)}$ using the recurrences
 \[\begin{array}{rcl}
	BF_{(j)} &=& \left\{\begin{array}{rcl} L&\implies& BF_{(j-1)} \\ i &\implies& BF_{(i-1)} + BV_{(i,j)} \end{array} \right.\\
	\\
	BV_{(i,j)} &=& \color{red} \left\{\begin{array}{rcl}
		HL &\implies&\big< {\rm hairpin}(i,j) \big> \\
		IL &\implies& \big< {\rm stack}(i,j) \big> BV_{(i+1,j-1)} \\
		(i',j') &\implies& \big< \text{multi-loop from $(i,j)$ to }(i',j') \big> BV(i',j')\\
		k &\implies& BW_{(i+1,k)} BW_{(k+1,j-1)}
	\end{array}\right.\\
	\\
	BW_{(i,j)} &=& \color{red} \left\{\begin{array}{rcl}
	S & \implies & \big< bulge(i) \big> BW_{(i+1,j)} \\
	W & \implies & \big< bulge(j) \big> BW_{(i,j+1)} \\
	V &\implies& BV_{(i,j)} \\
	k &\implies& BW_{(i+1,k)} BW_{(k+1,j-1)}
	\end{array}\right.
\end{array}\]

\item Visualization: \begin{center}\includegraphics[width=8cm]{inc/zucker.pdf}\end{center}
	% source: <<Parallization of dynamic programming recurrences in computational biology>> paper
\item Optimizations: {\color{red} XXX: notice that there are 3 matrices: $W$,$V$ ($VBI$ is part of $V$) that can be expressed using regular matrix, and $F$ that is of different dimension than $W$ and $V$ and requires a special construction (in the wavefront?). We need to find a nice way to encode both its construction and backtrack into the existing framework (implement 1D DP recursively?)}
\ole

% ------------------------------------------------------------------------------------------------
\newpage
\subsection{Related problems}
The goal of this section is to demonstrate that our framework can accommodate with many {\color{red} (20-50)} problems that we have not considered at the design time.

\subsubsection{Serial problems}
\begin{tabular}{llcc} \toprule
\bf Problem & \bf Shape & \bf Matrices & \bf Wavefront \\ \midrule
Smith-Waterman \footnotesize simple & rectangle & 1 & -- \\
Smith-Waterman \footnotesize affine gap extension & rectangle & 3 & -- \\
\href{http://en.wikipedia.org/wiki/Needleman-Wunsch_algorithm}{Needleman-Wunsch} & rectangle & 1 & -- \\

\href{http://en.wikipedia.org/wiki/Dynamic_programming#Checkerboard}{Checkerboard} & rectangle & 1 & -- \\
\href{http://en.wikipedia.org/wiki/Longest_common_subsequence_problem\#Code_for_the_dynamic_programming_solution}{Longest common subsequence} & rectangle & 1 & -- \\
\href{http://en.wikipedia.org/wiki/Longest_common_substring_problem\#Pseudocode}{Longest common substring} & triangle & 1 & -- \\
\href{http://en.wikipedia.org/wiki/Levenshtein_distance\#Computing_Levenshtein_distance}{Levenshtein distance} & rectangle & 1 & -- \\
\href{http://en.wikipedia.org/wiki/De_Boor's_algorithm}{De Boor} \footnotesize evaluating B-spline curves & rectangle & 1 & -- \\
\end{tabular}

\subsubsection{Non-serial problems}
\begin{tabular}{llcc} \toprule
\bf Problem & \bf Shape & \bf Matrices & \bf Wavefront \\ \midrule
Smith-Waterman \footnotesize arbitrary gap cost & rectangle & 1 & -- \\
Convex polygon triangulation & parallelogram & 1 & -- \\
Matrix chain multiplication & triangle & 1 & -- \\
Nussinov & triangle & 1 & -- \\
Zuker folding & triangle & \color{red} 3? & \color{red} 0? \\
\href{http://en.wikipedia.org/wiki/CYK_algorithm}{CYK} \footnotesize Cocke-Younger-Kasami & triangle & \#rules & -- \\
\href{http://en.wikipedia.org/wiki/Knapsack_problem#Dynamic_programming}{Unbounded Knapsack} \footnotesize (input sensitive) & rectangle & 1 & --\\
\end{tabular}

\subsubsection{Other problems}\ul
\item Dijkstra shortest path: we need a $E\times V$ matrix, along $E$ forall $V$ reduce its distance, problem is serial along $E$, non-serial along $V$ hence of complexity $O(|E|\cdot |V^2|)$ which is far worse than both $O(|V|^2)$ (min-priority queue) and $O(|E|+|V|\log |V|)$ (Fibonacci heap).
\item Fibonacci: this problem is serial 1D. Could be implemented using a placeholder element in one of the matrix dimension.
\item \href{http://archive.ite.journal.informs.org/Vol3No1/Sniedovich/\#dpmodel}{Tower of Hanoi}: 1D non-serial
\item \href{http://www.cs.ust.hk/mjg_lib/bibs/DPSu/DPSu.Files/KnPl81.PDF}{Knuth's word wrapping}: 1D non-serial
\item \href{http://en.wikipedia.org/wiki/Longest_increasing_subsequence#Efficient_algorithms}{Longest increasing subsequence}: serial but binary search algorithm more efficient: $O(n \log n)$.
\item \href{http://www.ccs.neu.edu/home/jaa/CSG713.04F/Information/Handouts/dyn_prog.pdf}{Coin Change}: 1D non-serial


{\color{red}
\item \href{http://en.wikipedia.org/wiki/Floyd-Warshall_algorithm}{Floyd-Warshall}: <it possible to move the $k$ external loop inside?> Serial with $n$ iterations
\item \href{http://en.wikipedia.org/wiki/Viterbi_algorithm}{Viterbi \footnotesize (hidden Markov models)}: $T$ non-serial iterations over a vector
\item \href{http://en.wikipedia.org/wiki/Bellman-Ford_algorithm}{Bellman-Ford} (finding the shortest distance in a graph)
\item \href{http://en.wikipedia.org/wiki/Earley_parser#Pseudocode}{Earley parser} (a type of chart parser)
\item \href{http://en.wikipedia.org/wiki/Maximum_subarray_problem}{Kadane maximum subarray} 1D serial, look at 
\href{http://www.cosc.canterbury.ac.nz/tad.takaoka/cats02.pdf}{Takaoka} for 2D
\item Structural alignment (MAMMOTH, SSAP), \href{http://rna.tbi.univie.ac.at/cgi-bin/RNAfold.cgi}{RNA structure prediction}.
\item \href{http://en.wikipedia.org/wiki/Recursive_least_squares_filter}{Recursive least squares}
\item \href{http://www.math.utep.edu/Faculty/pmdelgado2/courses/adv_algorithms/homework-08_anser.pdf}{Bitonic tour}
}
\ule

\begin{verbatim}
- Balanced 0-1 matrix
- Recurrent solutions to lattice models for protein-DNA binding
- Backward induction as a solution method for finite-horizon discrete-time dynamic optimization problems
- Method of undetermined coefficients can be used to solve the Bellman equation in infinite-horizon, discrete-time, discounted, time-invariant dynamic optimization problems
- Many algorithmic problems on graphs can be solved efficiently for graphs of bounded treewidth or bounded clique-width by using dynamic programming on a tree decomposition of the graph.
- Transposition tables and refutation tables in computer chess
- Pseudo-polynomial time algorithms for the subset sum and knapsack and partition problems
- The dynamic time warping algorithm for computing the global distance between two time series
- The Selinger (a.k.a. System R) algorithm for relational database query optimization
- Duckworth-Lewis method for resolving the problem when games of cricket are interrupted
- The Value Iteration method for solving Markov decision processes
- Some graphic image edge following selection methods such as the "magnet" selection tool in Photoshop
- Some methods for solving interval scheduling problems
- Some methods for solving word wrap problems
- Some methods for solving the traveling salesman problem, either exactly (in exponential time) or approximately (e.g. via the bitonic tour)
- Beat tracking in music information retrieval.
- Adaptive-critic training strategy for artificial neural networks
- Stereo algorithms for solving the correspondence problem used in stereo vision.
- Seam carving (content aware image resizing)
- Some approximate solution methods for the linear search problem.
=====> http://en.wikipedia.org/wiki/Dynamic_programming#A_type_of_balanced_0.E2.80.931_matrix

- Shortest path in DAGs
- Shortest path
- All pair shortest paths
- Independent sets in trees
=> also see exercises for more problems
=====> http://www.cs.berkeley.edu/~vazirani/algorithms/chap6.pdf
- 
- Subset Sum, Coin Change, Family Graph
=====> http://www.algorithmist.com/index.php/Dynamic_Programming

- Optimal Binary Search Trees
=====> http://www.cs.uiuc.edu/~jeffe/teaching/algorithms/notes/05-dynprog.pdf

- Independent set on a tree
- 0-1 Knapsack
=====> http://www.cs.ucsb.edu/~suri/cs130b/NewDynProg.pdf
\end{verbatim}

%\documentclass[11pt]{article}\usepackage{amssymb,amsmath,amsthm,hyperref,verbatim,pict2e,graphicx,array,listings,appendix,color}
\usepackage{algorithm,algorithmic,booktabs,marvosym,wrapfig,xytree,multicol,multirow,arydshln,nameref}
\hypersetup{colorlinks,citecolor=black,filecolor=black,linkcolor=black,urlcolor=black
	%pdfborderstyle={/S/U/W 1},urlbordercolor=1 0 0,linkbordercolor=.5 1 1, citebordercolor=.5 1 1
}
\usepackage[usenames]{xcolor} % color names ,dvipsnames,svgnames,table
\usepackage[utf8]{inputenc}
\usepackage[T1]{fontenc}
\usepackage[english]{babel}

% margins
\pagestyle{headings}
\oddsidemargin 0.0cm
\evensidemargin 0.0cm
\topmargin 0.0cm
\headheight 0.0cm
\headsep 1.0cm
\textheight 22.0cm
\textwidth 16.0cm
\parskip 0.1cm
\parindent 0.0cm
\footskip 1.0cm

% compact titles
\usepackage[compact]{titlesec}
\titlespacing{\section}{0pt}{8pt}{0pt}
\titlespacing{\subsection}{0pt}{8pt}{0pt}
\titlespacing{\subsubsection}{0pt}{8pt}{0pt}

% compact lists
\usepackage{enumitem}
\setitemize{noitemsep,topsep=0pt,parsep=0pt,partopsep=0pt}
\setenumerate{noitemsep,topsep=0pt,parsep=0pt,partopsep=0pt}
\def\ul{\begin{itemize}}
\def\ule{\end{itemize}}
\def\ol{\begin{enumerate}}
\def\ole{\end{enumerate}}

% misc
\def\up#1{\textsuperscript{#1}}
\def\quote#1{\par\begingroup\leftskip1em\rightskip\leftskip\textit{#1}\par\endgroup}


% listings
\definecolor{dkpink}{RGB}{200,0,100}
\definecolor{gray}{RGB}{128,128,128}
\lstset{
	xleftmargin=20pt,
	numberstyle=\tiny,stepnumber=1,numbersep=5pt,
	showstringspaces=true,         % underline spaces within strings
	tabsize=2,                      % sets default tabsize to 2 spaces
	captionpos=t,                   % sets the caption-position to bottom
	breaklines=true,                % sets automatic line breaking
	breakatwhitespace=true, % sets if automatic breaks should only happen at whitespace
	title=\lstname, % show the filename of files included with \lstinputlisting; also try caption instead of title
	basicstyle=\small\tt,keywordstyle=\color{blue},commentstyle=\color{gray},stringstyle=\color{dkpink}
}
% define Scala syntax
\lstdefinelanguage{Scala}{
	morekeywords={abstract,case,catch,class,def,do,else,extends,false,final,finally,for,if,implicit,import,%
	match,mixin,new,null,object,override,package,	private,protected,requires,return,sealed,super,this,%
	throw,trait,true,try,type,val,var,while,with,yield},
	otherkeywords={=>,<-,<\%,<:,>:,\#,@},sensitive=true,
	morecomment=[l]{//},	morecomment=[n]{/*}{*/},
	morestring=[b]",morestring=[b]',morestring=[b]"""
}

% title page
\makeatletter
\gdef\@subtitle{}\def\subtitle#1{\gdef\@subtitle{#1}}
\def\my@heading{
\def\ps@headings{\let\@mkboth\markboth
	\def\@evenhead{\small \rightmark \hfill \textit{\@title}, p.~\thepage}
	\def\@oddhead{\@evenhead}}\pagestyle{headings}}
\renewcommand{\maketitle}{
	%\begin{titlepage}
	\setcounter{page}{0}\thispagestyle{empty}
	{\centering\null\vfill\includegraphics[width=5.5cm]{inc/logo_epfl.pdf} % EPFL logo
	\vspace{1.5cm}\hrule \vspace{2.5cm} {\LARGE \@title \par} {\large \emph \@subtitle \par}
	\vspace{2.75cm} {\Large \@author \par}
	\vspace{5.5cm} {\large School of Computer and Communication Sciences, EPFL \par}
	\vspace{1.0cm} {\@date \par} % date
	\vfill\null\par}\my@heading
	\newpage
	%\end{titlepage}
}
\newcommand{\shorttitle}{
	\thispagestyle{empty}
	\hfill \includegraphics[width=3cm]{inc/logo_epfl}\vspace{.1cm} % EPFL logo
	\begin{center} {\LARGE \@title} \\ \vspace{.1cm} {\large \textit{\@subtitle}} \\ \rule[1ex]{350pt}{.5pt} \\
	\@author \\ {\small School of Computer and Communication Sciences, EPFL} \vspace{.2cm} \\{\small \@date}
	\end{center} \vspace{.5cm}\my@heading
}
\makeatother

% new XeTeX title page
\usepackage[T1]{fontenc}
\usepackage{fontspec}
\newfontfamily\fonth{Helvetica}
\newfontfamily\fonthn{Helvetica Neue}
\newfontfamily\fonthc{Helvetica Neue Condensed Bold}
\newfontfamily\fonthl{Helvetica Neue UltraLight}

\makeatletter
\renewcommand{\maketitle}{
	%\begin{titlepage}
	\setcounter{page}{0}\thispagestyle{empty}
	\hfill \includegraphics[width=8.5cm]{inc/logo_epfl.pdf} \vfill
	{\fontsize{25pt}{11pt}\fonthc Master project report \vspace{0.5cm}} \\
	{\fontsize{40pt}{11pt}\fonthl \@title} \vspace{0.2cm} \\ {\fontsize{20pt}{11pt}\fonthl \@subtitle} \\
	\vspace{1.5cm} \\
	{\begin{tabular}{ll}
	Laboratory	& Programming Methods Laboratory, LAMP, EPFL \\
	Professor		& \href{mailto:martin.odersky@epfl.ch}{Martin Odersky} \\
	Supervisors	& \href{mailto:vojin.jovanovic@epfl.ch}{Vojin Jovanovic}, \href{mailto:manohar.jonnalagedda@epfl.ch}{Manohar Jonnalagedda}   \\
	Expert		& \href{mailto:mirco.dotta@typesafe.com}{Mirco Dotta}, Typesafe \\
	Student		& \href{mailto:thierry.coppey@epfl.ch}{Thierry Coppey} \\
	Semester		& Autumn 2012 \\
	\end{tabular}}
	\my@heading
	\newpage
	%\end{titlepage}
}
\makeatother

% appendix
\makeatletter
\let\origappendix\appendix
\renewcommand\appendix{\clearpage\pagenumbering{Roman}\origappendix\section*{\appendixname}\lstset{frame=tb,numbers=left}}
\makeatother

% default
\author{\href{mailto:thierry.coppey@epfl.ch}{Thierry Coppey}, \href{mailto:manohar.jonnalagedda@epfl.ch}{Manohar Jonnalagedda}} %, \href{mailto:nithin.george@epfl.ch}{Nithin George}
\begin{document}

\section{Benchmarks} \label{benchmarks}
In an attempt to provide realistic benchmarks, we tried to gather related implementations. The authors of \cite{gpu_atlp} did not respond to our multiple solicitations. The authors of \cite{swat_mega} were very friendly and provided us their source code. Unfortunately, since they address a different category of problem (they focus on huge serial problems whereas we focus on smaller non-serial problems) their implementation might be biased towards large sequences that our implementation cannot address. Finally, we asked lately the authors of \cite{gpu_rnafold} who did not respond either to our solicitations.

We organize the benchmarks as follow: if we have at our disposal a working implementation that could be run on our evaluation platform, we use it, otherwise, we refer to the related paper and rescale the result according to the memory throughput and computation throughput of the related device so that we can have a good approximation of how they could compare.

\subsection{Metrics} \label{metrics}
The main metrics of interest is the running time. In an attempt to reduce the variance, we would like to run multiple consecutive test and take the median running time, since the median is less sensitive to outlier than the average\cite{perfeval}. Unfortunately, several factors hampers these ideal conditions. First the variance in the running time of CUDA kernels might be significant, in particular for short running time. This is due to the fact that the GPU needs to be 'warmed-up' before actual computation can happen. Similarly, the JVM is also subject to running time variance that is mainly due to the garbage collection\footnote{\url{http://www.oracle.com/technetwork/java/javase/gc-tuning-6-140523.html\#cms.overhead}} and JIT optimizations \cite{java_jit}.

Also the input and problem might introduce variance. As example, we can consider two extreme cases: matrix chain multiplication and Zuker RNA folding, with a test environment of 100 random inputs (of length respectively 512 and 80) and a GPU warmup of 10 computations. In this settings, matrix chain multiplication computations are executed in a perfectly constant time\footnote{With respect to truncation and measurment accurcy, has less than 1\% of variation (not observable).} (0.127 seconds), which mean that we sufficiently reduced the noise. Oppositely, the Zuker RNA folding running times appear much more scattered as presented below:

\begin{figure}[H]\begin{center}\includegraphics[width=16cm]{inc/var_zuker.pdf}\end{center}
\caption{Zuker folding running time (seconds). Quartiles: 0.152, 0.183 (median), 0.214}\label{fig:var_zuker}\end{figure}

Using the QQplot\footnote{Quantile-to-quantile plot, used to compare two distributions against each other.}, the distribution is heavily tailed (has more results towards the ends of the range) than a Gaussian distribution (fig.~\ref{fig:var_zuker} center) but fits better an uniform distribution (fig.~\ref{fig:var_zuker} right). If we run multiple time the program over the same input, we obtain the same behavior as with the matrix multiplication (strictly identical time); hence we can conclude that Zuker is an input sensitive problem whereas matrix chain multiplication is not. It follows that we need to be careful to test with exactly identical input set different implementations.

As the device memory is quite limited, it seems interesting to also take into account the space usage. The space requirement limits the maximal size of addressable problems  on a particular hardware. This might be a concern for large problems, because they would require special adaptation to handle such cases both correctly and efficiently. However, this metric heavily depends on the problem and simple solutions like using a device with larger memory or using main memory (if a $5\times$ slowdown is still acceptable) could solve this issue, hence we do not consider this metric hereafter (except as an upper bound on the dimension of the input).

\subsection{Benchmarking platform}
Our benchmarking platform is an Apple notebook with a Core i7-3720QM with 16Gb of main memory and an NVIDIA GeForce GT 650M running under MacOS X 10.8 and Oracle JDK 1.7.0-10. A workaround (see listing~\ref{cpu_workaround}) allows us to use the CPU to render the user interface while leaving the graphic card available to execute CUDA kernels. Unfortunately, due to impossibility to disable the watchdog timer in MacOS, CUDA kernels are limited to few seconds of running time before they are automatically aborted.

% ----------------------------------------------
\def\hh{\normalsize\bf}
\def\hl#1#2{\begin{minipage}{3.5cm} {\bf #1} \\[-2pt] \footnotesize #2 \vspace{6pt} \end{minipage}}
\def\hdps{\hl{DynaProg}{Scala parsers}}
\def\hdpc{\hl{DynaProg}{CUDA parsers}}
\def\hhoc{\hl{Optimized}{C, single thread}}
\def\hhog{\hl{Optimized}{CUDA, 64-bit}}
\def\hgapc{\hl{GAPC}{\cite{gapc_thesis}, C, single thread}}
\def\hatlp{\hl{ATLP}{\cite{gpu_atlp}, rescaled results$^{(1)}$}}
\def\hcua{\hl{CUDAlign}{\cite{swat_linear}, version 2.0}}

\def\hvien{\hl{ViennaRNA}{\cite{vienna_rna}}}
\def\hrna{\hl{RNAFold}{\cite{gpu_rnafold}}}

% ----------------------------------------------
% MatrixMult-512, Mac+JDK7
% -> Original: 24.658 sec
% -> Optimized concatenation: 18.924 sec
% -> Tabulation as arrays+inline: 14.849 sec
% -> Re-optimized concatenation: 10.76 sec
\newpage
\subsection{Matrix chain multiplication}
Since we have seen that this problem is not input sensitive in (\S\ref{metrics}), we can safely use different random number generators among different implementations without compromising the validity of the results.  Also note that the hand-optimized results are slightly worse than those presented in (\S\ref{baseline_impl}), this is caused by enabling the 64-bit mode. Since external libraries linked with the Java virtual machine must be in 64 bit, we also enabled this mode in hand-optimized version to maintain a fair comparison, thereby slightly reducing the performance of CUDA operations.
% CPU: DynaProg(Scala), hand optimized C, GAPC and ADPFusion
% GPU: GPU: DynaProg(CUDA), hand optimized CUDA, ATLP and GAPC-OpenCL?

\begin{table}[H]\begin{center}{\small\begin{tabular}{llrrrrrrr}\toprule
&\hh  Matrix dimension &\hh 64 &\hh 128 &\hh 192 &\hh 256 &\hh 384 &\hh 512 &\hh 768 \\
\midrule \multirow{4}{*}{\rotatebox{90}{\normalsize\bf CPU $\quad$}}
& \hdps	& 0.05	& 0.20	& 0.80	& 2.03	& 6.65	& 15.10	& 47.40	\\
& \hhoc	& <0.01	& <0.01	& <0.01	& 0.01	& 0.03	& 0.08	& 0.28	\\
& \hgapc	& 0.01	& 0.01	& 0.03	& 0.05	& 0.15	& 0.35	& 1.16	\\[-2pt]
\midrule \multirow{4}{*}{\rotatebox{90}{\normalsize\bf GPU $\quad$}}
& \hdpc	& 0.03	& 0.04	& 0.05	& 0.07	& 0.13	& 0.13	& 0.21	\\
& \hhog	& <0.01	& 0.01	& 0.01	& 0.02	& 0.04	& 0.08	& 0.17	\\
& \hatlp	& 0.17	&	 	&		& 0.20	& 		& 0.23	& 		\\
\midrule
&\hh Matrix dimension &\hh 1024 &\hh 1536 &\hh 2048 &\hh 3072 &\hh 4096 &\hh 6144 &\hh 8192 \\
\midrule \multirow{4}{*}{\rotatebox{90}{\normalsize\bf CPU $\quad$}}
& \hdps	& 109.77	& 368.21	& 877.30	& 3059.42 & 		& 		& 		\\
& \hhoc	& 1.18	& 7.06	& 19.81	& 78.90	& 206.56	& 799.53	& 2010.49 \\
& \hgapc	& 2.82	& 10.02	& 25.16	& 91.69	& 224.70	& 		&  		\\[-2pt]
\midrule \multirow{4}{*}{\rotatebox{90}{\normalsize\bf GPU $\quad$}}
& \hdpc	& 0.35	& 0.85	& 1.69	& 4.79	& 10.32	& 31.60	& 71.22	\\
& \hhog	& 0.32	& 0.82	& 1.65	& 4.74	& 10.35	& 31.94	& 72.38	\\
& \hatlp	& 0.40	& 0.74	& 1.33	& 3.43	& 7.29	& 		& 		\\
\bottomrule\end{tabular}}\end{center}\caption{Running time of matrix chain multiplication (in seconds)}\end{table}

$^{(1)}$ Assuming that 72\% of the running time is due to memory accesses, and considering a $3.55\times$ memory throughput slowdown of the original results (see \S\ref{results_discussion}). % 0.06 & 0.07 & 0.08 & 0.14 & 0.47 & 2.57 => f[x_] := (x*(1 - .72)) + (x*0.72)*(102.4/28.8)
% 0.1704 & 0.1988 & 0.2272 & .3976 & 1.3348 & 7.2988

The running time of DynaProg/CUDA includes the overhead of back and forth JNI conversion (scales linearly between 0.018 and 0.057 seconds) but does not include the overhead due to the code generation which decomposes in 0.068 seconds for analysis and code synthesis (once per algebra/grammar pair) and 0.086 + 1.753 seconds for respectively Scala and CUDA compilation (constant time, once per problem dimension). The results of the following problems are presented similarly.

For DynaProg/Scala we use a variant of the problem description: the original version only stores the matrix multiplication score whereas the modified version also stores the matrix dimension. This allows a speedup of $2.9\times$ probably due to the additional lookups overhead. Also with the default JVM parameters, the program cannot address sequences longer than $\sim 420$ elements due to a stack overflow, for these benchmarks, we increased this limit.

\subsection{Smith-Watermann}
% CPU: DynaProg(Scala), hand optimized C, GAPC and ADPFusion
% GPU: DynaProg(CUDA), hand optimized CUDA, CUDAlign
To maintain a fair comparison with other implementations, our algorithm needs to have the same complexity. Hence we use a variant of the Smith-Waterman algorithm: we encode our grammar in 3 tabulations as described in \S\ref{swat_affine}, whereas application like CUDAlign can leverage the domain-specific information that information of two of these matrices can be stored in the computation wavefront (see \S\ref{calc_simplifications}).

\begin{table}[H]\begin{center}{\small\begin{tabular}{llrrrrrrr}\toprule
&\hh  Matrix dimension &\hh 64 &\hh 128 &\hh 192 &\hh 256 &\hh 384 &\hh 512 &\hh 768 \\
\midrule \multirow{4}{*}{\rotatebox{90}{\normalsize\bf CPU $\quad$}}
& \hdps	& 0.04	& 0.13	& 0.27	& 0.48	& 1.07	& 1.92	& 4.33	\\
& \hhoc	& <0.01	& <0.01	& <0.01	& <0.01	& <0.01	& <0.01	& 0.01	\\
& \hgapc	& 0.01	& 0.01	& 0.01	& 0.01	& 0.02	& 0.03	& 0.06	\\[-2pt]
\midrule \multirow{4}{*}{\rotatebox{90}{\normalsize\bf GPU $\quad$}}
& \hdpc	& 0.03	& 0.03	& 0.03	& 0.04	& 0.05	& 0.05	& 0.06	\\
& \hhog	& 0.00	& 0.00	& 0.00	& 0.00	& 0.00	& 0.00	& 0.00	\\
& \hcua	& 0.11	& 0.12	& 0.07	& 0.07	& 0.07	& 0.07	& 0.13	\\
\midrule
&\hh Matrix dimension &\hh 1024 &\hh 1536 &\hh 2048 &\hh 3072 &\hh 4096 &\hh 6144 &\hh 8192 \\
\midrule \multirow{4}{*}{\rotatebox{90}{\normalsize\bf CPU $\quad$}}
& \hdps	& 7.84	& 18.95	& 33.63	& 70.86	& 		& 		& $^{(1)}\infty$ \\
& \hhoc	& 0.01	& 0.02	& 0.04	& 0.10	& 0.17	& 0.40	& 0.71	\\
& \hgapc	& 0.10	& 0.22	& 0.39	& 0.91	& 1.62	& 4.41	& 11.20 	\\[-2pt]
\midrule \multirow{4}{*}{\rotatebox{90}{\normalsize\bf GPU $\quad$}}
& \hdpc	& 0.07	& 0.11	& 0.15	& 0.13	& 0.20	& 0.32	& $^{(2)}$3.21 	\\
& \hhog	& 0.01	& 0.02	& 0.02	& 0.04	& 0.07	& 0.14	& 0.27	\\
& \hcua	& 0.13	& 0.15	& 0.14	& 0.14	& 0.15	& 0.17	& 0.20 	\\
\bottomrule\end{tabular}}\end{center}\caption{Running time of Smith-Waterman (in seconds)}\end{table}

$^{(1)}$ Extremely little progress due to intensive JVM garbage collection after some delay, even by tuning the JVM parameters ({\tt -Xss512m -Xmx12G -Xms12G}), and independently of whether top-down or bottom-up parsing approaches are taken. After a moment, most of the time is spent in (full) garbage collection. The aggregation function application contributes to approximately 30\% of the total running time.

$^{(2)}$ Since the memory requirements are larger than the device capacity, the backtrack matrix overflows in the main memory, thereby significantly degrading the performance. This extra memory requirement is due to the use of 3 matrices to avoid the non-serial dependencies (hence requiring at least $3 \cdot 2$ bytes of memory per matrix element for backtrack).

\subsection{Zuker RNA folding}
% CPU: DynaProg(Scala), GAPC, ADPFusion and ViennaRNA
% GPU: DynaProg(CUDA), RNAFold?/Lavenier VS ViennaRNA-OpenCL?, GAPC-OpenCL?

\begin{table}[H]\begin{center}{\small\begin{tabular}{llrrrrrrr}\toprule
&\hh  Matrix dimension &\hh 64 &\hh 128 &\hh 192 &\hh 256 &\hh 384 &\hh 512 &\hh 768 \\
\midrule \multirow{4}{*}{\rotatebox{90}{\normalsize\bf CPU $\quad$}}
& \hdps	& 0.07	& 0.67	& 2.26	& 4.98	& 17.61	& 39.44	& 130.71	\\
& \hvien	& 0.01	& 0.01	& 0.02	& 0.03	& 0.07	& 0.12	& 0.29	\\
& \hgapc	& 0.01	& 0.03	& 0.07	& 0.13	& 0.41	& 0.93	& 2.89	\\[-2pt]
\midrule \multirow{4}{*}{\vspace{12pt}\rotatebox{90}{\normalsize\bf GPU}}
& \hdpc	& 0.15	& 0.53	& 0.98	& 1.55	& 3.13	& 5.04	& 9.43	\\
& \hrna	& 0.09	& 0.14	& 0.20	& 0.44	& 0.80	& 1.89	& 3.52	\\
\midrule
&\hh Matrix dimension &\hh 1024 &\hh 1536 &\hh 2048 &\hh 3072 &\hh 4096 &\hh 6144 &\hh 8192 \\
\midrule \multirow{4}{*}{\rotatebox{90}{\normalsize\bf CPU $\quad$}}
& \hdps	& 306.12	& 1036.78 &		& 		& 		& 		& 	 	\\
& \hvien	& 0.57	& 1.53	& 3.19	& 9.37	& 20.18	& 59.65	& 133.65 	\\
& \hgapc	& 6.66	& 22.91	& 56.97	& 208.33	& 		& 		&  		\\[-2pt]
\midrule \multirow{4}{*}{\vspace{12pt}\rotatebox{90}{\normalsize\bf GPU}}
& \hdpc	& 15.10	& 33.62	& 61.76	& 166.84	& 393.76	& 		&  		\\
& \hrna	& 9.08	& 19.59	& 67.32	& 163.14	& 		& 		& 		\\
\bottomrule\end{tabular}}\end{center}\caption{Running time of Zuker RNA folding (in seconds)}\end{table}

{\color{red} XXX: comment, explain energy coefficients, complexity, limit search space for bulges, algorithm variants and constraints RNAfold/Zuker}

% ------------------------------------------------------------------------------------------------
{\center\color{red} \noindent\rule{16cm}{0.4pt} \\ XXX: CONTINUE HERE :XXX \\}
% ------------------------------------------------------------------------------------------------

{\color{red}\ol
\item Write some conclusion ideas
\item Run benchmark and add results in report (automate, generate raw Matlab/TeX output)
\item Stabilize times, statistics, use CUDA profiler(?)
\ole}

\subsection{Synthetical results}
XXX

{\color{red} We want to see the benefits of moving to CUDA, also compare to how far Scala is from C.}

RUN 5-10x until running time stabilizes on Scala

 {\color{red} Size analysis to know what storage size we require: ex: Zucker requires $O(n^2)+O(n)$ storage...}

XXX: future work: proper support of two-track grammars

XXX: future work, resize tables appropriately as in GAPL.

XXX: future work: infer automatically "always non-empty" property

{\color{red} XXX: say that Haskell package did not install and also time limited to provide largest result (but that are also less relevant as users do not want to wait)}

%\end{document}


% --------------------------------------------------------------------------------
\bibliographystyle{plain}
\bibliography{bibliography.bib}
\end{document}