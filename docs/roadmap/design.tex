\newpage
\section{Architecture design and technical decisions}
% ------------------------------------------------------------------------------------------------
\subsection{User facing language requirements}
\quote{The Algebraic Dynamic Programming approach (ADP) introduces a conceptual splitting of a DP algorithm into a recognition and an evaluation phase. The evaluation phase is specified by an evaluation algebra, the recognition phase by a yield grammar. Each grammar can be combined with a variety of algebras to solve different but related problems, for which heretofore DP recurrences had to be developed independently. Grammar and algebra together describe a DP algorithm on a high level of abstraction, supporting the development of ideas and the comparison of algorithms. A notation for yield grammars is provided that serves as a domain-specific language.}

Given such formalization \cite{adp} of dynamic programming on sequences, it seems natural to borrow from it and extend it to other types of DP problems. In short, this framework allow the user to define a grammar using parsers, which are then run over an input string and produce intermediate results that are memoized into a table, when multiple solutions are possible, the user can define an aggregation function ($h$) to retain only some candidates for further combination.

The benefits of ADP framework is that it does not constraint the result of the evaluation to be a single value, but could extend parsers to backtracking parsers or pretty-printers. Such functionality is easy to support in plain Scala, however, it could hampers the performance of the GPU implementation if we leave to the user the duty of representing and processing all the information within the parsers. Its major shortcoming in expressivity is that it only operates on strings, making only triangular-matrix problem easily representable.

The additional features we want to support are:\ol
\item \textbf{Cyclic problems:} are inherently very similar to string problems, except that the input is cyclic. To support such problem efficiently, we only need to mark the grammar cyclic such that it would apply on any unfolding of the cyclic input string.
\item \textbf{Input pair algebra:} the original ADP framework only support single input, we want to support a pair of inputs such that we can treat problem such as Smith Waterman or Needleman-Wunsch. However, it does not make sense to treat more than two sequences because of the $O(n^3)$ storage requirements that limits the problems size more dramatically.
\item \textbf{Windowing}: this can be easily encoded by passing the windowing parameter that limits the computation, then it could be possible to collect either the best or $k$-best results.
\item \textbf{Device restrictions:} since CUDA (and FPGA) cannot operate on an arbitrary Scala classes, we need to restrict the language to primary types (int, float, ... and structures of them). However, we want to preserve the expressiveness available for Scala and impose restrictions on the input and answers available to CUDA. A typical restriction we want to make is that processed structures are of fixed size so that we avoid memory management issues and thread divergence.

The general ADP framework supports multiple solutions for intermediate results. Similarly, we want to impose the restriction that only the best result would be stored on device while preserving the support for multiple partial results in Scala.

\item \textbf{Automatic backtracking:} Since efficient code has to be devised we impose restrictions on the output that could be generated by the parsers on devices. However, on the other side, the backtracking information would be of primary interest for the DSL user, hence we need to automate the backtracking to both fulfill the goals of efficiency and usefulness of the device specific implementation.

Enforcing automatic backtracking presents the advantage to ensure constant size for intermediate results, hence ensuring an $O(n^2)$ storage requirement while memoizing the backtracking with the results would have the drawback of growing towards final result and duplicated unnecessary information, hence requiring both $O(n^3)$ space and memory management features on devices. Collecting the backtracking list can be easily done in $O(n)$ and then inverted whether we prefer bottom-up or top-down construction (the backtrack is usually a lattice of nodes that constitute a tree whose leaves are input elements).
\ole
Since we construct the dependency tree bottom up, and since we need to work with a fixed-size memory, we need to do the additional steps in our pre-processing:\ul
\item \textbf{Normalization:} in order to automate the backtracking, we need the rule to present a certain shape (in particular we want to remove alternatives) so that we can define uniquely the backtracking information. Another step we can make in the process is to break down complex rules into simpler ones if they are expressible so, hereby reducing the complexity of the overall algorithm, would the user production grammar not be optimal. Also we can strip away unnecessary rules.
\item \textbf{Dependency analysis:} a precise order between the rules application must be given. This is required as we use a bottom-up approach where all previous results must be computed versus a top-down approach with a table where a fallback to computation always exist if the desired element is not present.
\ule

{\color{red}
Size analysis to know what storage size we require

We define the aggregate function h as $List[T]\to List[T]$ and a special string emitter for C code that produces the initial element and a comparator between two (cost) elements.
}

% ------------------------------------------------------------------------------------------------
\subsection{Parsing grammar (ADP)}
{\color{red}
Borrowing from ADP XXX

XXX: explain ADP concepts

XXX: normalization, transforms applied

XXX: Think how to implement {\tt cyclic} problems and {\tt windowing} properties within the language

\begin{verbatim}
restrict to only 1 max/min (more restrictive than ADP)
for optimization we want to split grammar into binary productions, to do that we need
to have equality on the following formula. We can solve it more easily if functions are linear
\end{verbatim}
\[\min\limits_{i<k_1<k_2<j}\big[ f(i,k_1,k_2,j) \big] \le  \min\limits_{i<k_2<j} \big[ g_2(i,\min\limits_{i<k_1<k_2} \big[ (g_1(i,k_1,k_2  ) \big],k_2,j) \big]  \]

}

\subsection{Backtracking}
In order to do a clean and efficient transformation from ADP-like language to plain C recurrences, we need to construct bottom-up recurrences from top-down parser rules. To do that, we slightly need to modify the ADP language in order to separate the backtracking and the scoring, because we want to obtain an efficient algorithm: backtrack writes are in $O(n^2)$ whereas score reads are proportional to the algorithmic complexity ($O(n^3)$ or more for non-serial). To deal with this problem, we are facing two options:\ul
\item \textbf{Explicit backtracking:} requires clear syntactical separation between the score and the backtrack and is not implemented in vanilla ADP (unless the whole backtrack is made part of the scoring: big performance impact and non-constant memory requirement issues make its GPU implementation hard and not desirable). Since the backtracking is user-defined, there is no way to generate the backtracking algorithm automatically, hence the user also needs to provide it. 

\item \textbf{Implicit backtracking:} implies that every rule needs to be normalized, and transformed such that given a rule identifier and a set of indices (subproblems breaking), it is possible to retrieve the subproblems combination that build the problem. To do that we need to apply the following transformations\ol
	\item Normalize rules and identify them uniquely by \ul
		\item Removing all the <<or>> operators by breaking a parser into multiple subrules
		\item Forwarding names: replace a rule that is a single invocation by its content (i.e. name a terminal parser with its own name instead of its containing parser's)
		\item Assigning each subrule an unique identifier $r$ and create a mapping table $T: r \to$ user-defined name of the parser.
	\ule
	\item The data elements corresponding to a rule are (scoring, backtracking) and named after the tabulation. The scoring is a user-defined primary type (int, double, ...) and the backtracking is a tuple $({\rm id}_{\rm subrule}, k_1, k_2, \ldots , k_m)$ where $m$ is the maximal number of splits that can occur in the subrules.
	\item During the matrix computation of cell $(i,j)$, if the subrule $r$ applies, the backtrack will be set as $(r,k_1,k_2,\ldots,k_{m_r})$ (obviously with $i\le k_1\le k_2\le \ldots k_{m_r}\le j$ and $m_r\le m$).
	\item During backtracking, when reading the cell $(i,j)$ with backtrack $(r,k_1,k_2,\ldots,k_m)$, given $r$, we recover the subrule hence the user-defined name of the rule that applies and $k_{m_r}$, which allows us to enqueue the subwords $(i,k_1), (k_1,k_2), ..., (k_{m_r})$ for further backtracking. If $r$ refers to a terminal, we stop the backtracking.
	\item The backtracking can be returned to the user as a mapping table $T$ and a list of triplets $(r,i,j)$ where $r$ is the index of the sub-rule that can be easily mapped into the user name (such that $T: r_u \to$ user-defined rule name) and $(i,j)$ is the subword on which the rules applies.
\ole
In short, we break parsers into normalized rules, the backtracking information is the rule id (which rule to unfold) and a list of indices (how to unfold it).
\ule
Assuming that the backtracking information is meant to guide further processing, we provide this information into a list constructed bottom-up: it can be simply processed by the user program in-order, applying for each user-defined rule the underlying transformation, and storing intermediate results (i.e. in a hash map) until they are processed by another rule. Since consumed sub-result can be dropped and since the backtracking contains only relevant sub-problems, ultimately, only the result will be stored.

\subsubsection{Backtracking with multiple backtrack elements}
The backtracking technique described above work fine when there is a single element stored per matrix cell (which is usually the case with min/max problems). However, in the generalization introduced by ADP, it is possible that a matrix cell stores multiple results. In such case, we need to select a correct result to avoid backtracking inconsistencies.

Additionally, we need to keep track of the multiplicities of the solutions, that is if we want to obtain the $k$ best solutions, we need to make sure that we return $k$ different backtracks. To do that, we maintain a multiplicity counter at every step: \ul
\item While there is a single solution for all possible incoming paths, we continue in this direction with the same multiplicity (we have no choice).
\item When there is $r$ different solutions available, and the path multiplicity at this point is $k$ we have the following cases: \ol
	\item If $k\ge r$: we explore all paths with multiplicity $k-r+1$. This is because each branch may produce only one solution and we don't know ahead of time which path will provide multiple solutions. Finally, we retain only the $k$ best solutions.
	\item If $k<r$ (there is more paths than needed): we can only explore the $k$ first paths with multiplicity 1 and safely ignore the other (as we only need $k$ results).
	\ole
\ule

Now it remains the problem to generate all possible results and check whether they are correct correct. To do that we can simply re-apply the parsers while maintaining the source elements of all production and then retain only those with desired score and backtrack. Since we know the backtrack for one element, we can do the following optimization at backtrack parser computation:\ol
\item Defuse or's: since we know exactly (by the subrule id, maintained in the backtrack) which alternative has been taken to obtain the result, we can skip undesired branches of or parsers.
\item Feed concatenation indices: since the backtrack stores the concatenation indices, we can reuse in the concatenation parsers. This removes the $f(n)$ factor in the backtrack complexity (as concatenation backtrack parsers <<know>> where to split).
\item Skip filters: since filter are applied before their inner solution is computed, their are only position-dependent. Hence if a backtrack involves a filter, since its position is fixed, the filter must have been passed at matrix construction time.
\ole

\subsubsection{Backtracking complexity}
We want to have an estimation of the complexity of the two operations to see what is the overhead of multiple backtracking. For single-element backtrack, we only need to revert the parser to find involved subwords, which is linear in the parser size (because the backtracking identifies uniquely the alternative in or's and indices in concatenations). At every backtrack step, either:\ul
\item The word is removed at least one character, which leads to maximal backtrack length of $n$.
\item The word is split in $s$ subwords, with recurrence $f(n)=2f(n/2)+1$ by solving this recurrence we see that there can be at most $n$ final nodes and $n$ intermediate nodes (when $s=2$). Hence the backtrack length is at most $2n$.
\ule

Hence for single backtrack we would have at most $O(2n) \cdot O({\rm parser})$ complexity. For the $k$-elements backtrack since we regenerate all possible solutions, that is $O(k^c)$ candidates (with $c$ the maximal number of concatenation in the parser), the overall complexity is $O(2n \cdot k^c) \cdot O({\rm parser})$. Hence there is a $k^c$ factor to pay if we want to backtrack the $k$ best solutions\footnote{Note that the same factor lies in the matrix computation complexity.}.




%Algo: Read matrix, reapply the parser but by maintaining list of origins. Filter results to match multiplicities
% à priori ça a une complexité en k^3*2n pour les k meilleurs backtracks



% ------------------------------------------------------------------------------------------------
\subsection{Memory constraints}
We denote by \textit{device} the computational device on which the processing of the DP matrix (or of a computational block) is done and $M_D$ its memory. This can be the GPU or the FPGA internal memory. Usually the main memory is larger than device memory and can ultimately be extended by either disk or network storage.

We propose to evaluate the device memory requirements to solve the above problem classes. We need first to define additional problem properties related to implementation:\ul
\item \textbf{Number of matrices:} multiple matrices can be encoded as 1 matrix with multiple values per cell. Hence the implementation differentiates only between cost and backtrack matrices with respective element sizes $S_C$ and $S_B$.
\item \textbf{Delay of dependencies:} In case the problem does not fit into memory, partial matrix content needs to be transferred across sub-problems. Such data amount is usually proportional to the delay of dependencies. If this delay is small, it might be worth to duplicate matrix data in the wavefront, otherwise it might be more efficient to allow access to the previous computational blocks of the matrix.
\item \textbf{Wavefront size:} Finally some aggregation that is made along some dimension of the matrix does not need to be written at every cell but can be propagated and aggregated along with computation (ex: maximum along one row or column). Hence such information can be maintained in a single place (in the wavefront) and progress together with the computation. We denote by $S_W$ the size of wavefront elements.
\item \textbf{Input size:} the size of an input letter (from input alphabet) is denoted by $S_I$.
\ule

% ----------------------------------------------
\subsubsection{Small problems (in-memory)}
Problem that can fit in memory can be solved in a single pass on the device. Such problem must satisfy the equation:
	\[(S_I+S_W) \cdot (m+n) + (S_C+S_B) \cdot (m\cdot n) \le M_D\]

For instance, assuming that $m=n$, $M_D=1024{\rm Mb}$, that backtrack is 2b (<16384, 3 directions) and that the cost can be represented on 4 b (int or float), that input is 1b (char) and that there is no wavefront, we can treat problems of size $n$ such that $2n+5n^2 \le 2^{30} \implies  n\le 14650$. We might also possibly need to take into account extra padding memory used for coalesced accesses. But it is reasonable to estimate that problems up to 14K fit in memory.

% ----------------------------------------------
\subsubsection{Large problems}
To handle large problems, we need to split the matrix into blocks of size $B_H \times B_W$. For simplification, we assume a square matrix made of square blocks with $b$ blocks per row/column.

% ----------------------------------------------
\subsubsection{Non-serial problems}
Non-serial problems need to potentially access all elements that have been previously computed. We restrict\footnote{As we have not encountered a problem with non-serial dependencies along the diagonal.} ourselves to the following dependencies: \ul
\item Non-serial dependencies along row and column
\item Serial dependencies along diagonal, with delay smaller or equal to one block size
\ule
Such restriction implies that all the block of the line and the row, and one additional block to cover diagonal dependencies must be held in memory (independently of the matrix shape).

For simplification, let $m=n$ and assume that we have $b$ square blocks per row and per column. Hence we have the following memory restriction:
\[ 2\frac{n}{b}(S_I+S_W) + 2 \cdot \frac{n^2}{b}S_C + \frac{n^2}{b^2} S_B \le M_D\]

We also need to take into account the transfer between main memory (or disk) and device memory. Dependency blocks only need to be read, computed blocks need to be written back. Ignoring the backtrack and focusing only on the cost blocks, the transfers (in blocks) are:
\[\begin{array}{rclll}
		b^2 +% computed line writeback (once)
		(b-1)^2 + % diagonal block loads (1 per block)
		\sum\limits_{i=0}^{b-1} i \cdot b % column dependencies loads (for line i)
		&=&\tfrac{1}{2}b^3+\tfrac{3}{2}b^2-2b+1 &\qquad& \rm (Rectangle)
	\\
		\sum\limits_{i=1}^{b} \Big(1+2\cdot(i-1)\Big) \cdot (b+1-i) % on i_th diagonal
		&=& \tfrac{1}{3} b^3 + \tfrac{1}{2}b^2 + \tfrac{1}{6}b &\qquad& \rm (Triangle)
	\\
		\sum\limits_{i=1}^{b} \Big(1+2\cdot(i-1)\Big) \cdot b % on i_th diagonal
		&=& b^3 &\qquad& \rm (Parallelogram)
\end{array}\]

Putting these two formula together, and using most of the device memory available, we obtain the following results with $S_C=4, S_B\footnote{To deal with larger matrices, backtrack data need to be extended.}=4, S_I=1, S_W=0$ and $M_D=2^{30}$:
\begin{center}\includegraphics[width=14cm]{inc/ns_large.pdf}\end{center}

Given an experimental bandwidth of 5.3743 Gb/s between CPU and GPU, processing matrices one order of magnitude larger (128K) would result in respectively 13\up{(R)}, 8.5\up{(T)} and 25.4\up{(P)} minutes of transfer delay. Extrapolating the preliminary results of small problems, a computation on input of size 128K would require respectively 7 days 13h\up{(R)}, 2 days 22h\up{(T)} and 6 days 10h\up{(P)}, assuming there is no other scalability issues. Although this overhead seems appealing compared to the computation time, the total time blows up (because of the  $O(n^3)$ complexity) and make the processing of such large problem less relevant. Given that real problems -- like RNA folding -- operate at input sizes up to 4096, it would not be of much relevancy to implement a version for larger cases, although perfectly feasible.

% ----------------------------------------------
\subsubsection{Serial problems}
The serial problem have the interesting property to access to a fixed number of previous elements. These elements can be either stored either explicitly in a scoring matrix or implicitly as moving aggregation into a wavefront. Since the dependencies are fixed, the computation direction gains an additional degree of freedom: matrix can be solved in diagonal (as non-serial problems) or line-wise or column-wiser. This allows to store the whole necessary state to make progress into a limited number of lines (or columns), and sweep vertically (resp. horizontally) along the matrix.

Since serial problems are of complexity $O(n^2)$ (due to the matrix dimension and the finite number of dependencies), it is possible to tackle much larger problem than non-serial during the same running time. Hence, it seems obvious to let serial problems grow larger than the memory.

Mixing the dependency property and size requirements, we can split the matrix into sub-matrices, store special lines (and/or columns) into memory (or hard disk), and repeat computations to solve the backtrack (similarly as in \cite{swat_gpu},\cite{swat_mega}, but this implementation use problem-specific knowledge that might not generalize).

To store intermediate lines and columns, we are facing two different strategies to explore:\ul
\item \textbf{Fixed subproblem size:} we decompose the algorithm as follows\ol
\item Define a grid of <<major column and rows>>, where each cell's data (input, output, cost and backtrack matrices) fits into the device memory.
\item Compute the values of the grid's major columns and rows in one pass.
\item Second (on-demand) computation to process backtracking inside relevant cells.
\ole
Let $b$ the number of cells that we have on each row/column, the total computation running time would be $(b^2 + 2b) \cdot t_b$ where $t_b$ is the time to compute one cell's matrix. This division has the advantage of providing the minimal computation time at the expense of external storage proportional to $O(n)$ (if we store only lines or columns) or $O(n^2)$ (if we store both).

\item \textbf{Myers and Miller’s algorithm:} (divide and conquer)
This algorithm break the DP problem into 2 (or 4) subproblems such that once the middle line/column is computed, the problem can be solved for 1 submatrix and backtracking among up to 2 of the 3 other. This breaking is applied recursively until the submatrix data fits into memory. The storage requirements are $4 \cdot O(n)$ (we store along both dimension $1+\tfrac{1}{2}+\tfrac{1}{4}+...$ lines/columns).

The algorithm proceeds as follows: first it solves the problem to obtain the first backtracking element, then it breaks the matrix in 4 submatrices, and refine it until backtrack is tractable. Since there is at most $\log n/b$ refinements and since every part of the matrix may be involved in backtrack, running time is $O(n^2 \log_2 n)$.
\item \textbf{Hybrid approach:} a hybrid approach might be created to take advantage of additional available memory, however, the running time decreases logarithmically to the space used, this means that using twice more storage space would only result in a $2\times$ speedup (measuring only the computation time). Hence an hybrid approach would be to decide a $k$ such that at each step we partition the desired submatix into a intermediate grid of $k$ rows/columns. The space usage would be in $2 k \log_k (n/b)$ and the running time complexity would be $O(n^2 \cdot \log_k n)$. Then the user would be able to fix a storage space $S \ge 4 \log_2 (n/b)$ and obtain the corresponding $k$ for a given $n$.
\ule

{\color{red}
try to setup wavefront size (if needed) => just enlarge the matrix by 1 so that we go wavefront-to-wavefront

XXX: what's the maximal size of the wavefront ??

XXX: can we avoid to store some matrices and put them in the wavefront ??

Split into blocks:\ul
\item Decide the shape of the blocks
\item Decide the size of the blocks
\item Decide of a strategy to store intermediate lines/columns: space/time tradeoff.
\ule

XXX: make it work up to 14K for all 3 problems using multiple kernels
XXX: example of non-symmetric serial problem

The three last elements, combined with the above one, provide a precise estimation of the memory consumption, and the implementation difficulty 

all the problem are subject to the input dimensions $n$.

the two latter one gives an estimation of the constant factor.

The delay of dependencies might also have an impact: if the matrix is too large to fit in the memory (device or main memory), it becomes necessary to maintain partial matrix content (all the intermediate elements) within the wavefront. Also the number of cost matrices might affect the performance, simply because maintaining them requires computations and memory accesses.
}

% ------------------------------------------------------------------------------------------------
\subsection{Memory layout}
{\color{red} XXX}


% ------------------------------------------------------------------------------------------------
\newpage
{\color{red} 
\subsection{LMS compiler stack}
\textbf{User-language}: define additional parameters for the recurrence\ul
\item Windowing (to convert non-serial into serial problems)
\item Input sizes, and alphabets (backtrack, input, cost)
\item Backtrack (implicitly by backtrack alphabet size) and cost matrices bit-sizes (cost maximum may be inferred using <<Yield size/grammar analysis>>)
\item Recurrence functions, devices available
\item What to keep in memory (cost, backtrack or both).
\ule
$\Downarrow$ Conversion (using an existing technique)

\textbf{Intermediate representation}

$\Downarrow$ Optimizations\ul
\item Transform non-serial into serial \ul
	\item Use aggregation functions/transformations
	\item Use windowing from user (if no other technique succeed)
	\ule
\item Define the wavefront depth
\item Avoiding the cost matrix by moving it into the wavefront
\ule

\textbf{Code specification}\ul
\item Kernel function (1-element function), inputs, outputs, wave front, dependencies, bit sizes
\item Device-level interface => setup the block sizes(w/h), input and memory sizes
\item Define the device-specific implementation of the block (CPU/FPGA/CUDA)
\item Define the co-processor memory aggregation function
\item Define the scheduling of the blocks and aggregation (software pipelining)
\item Define the data movement back and forth to disk
\ule

$\Downarrow$ Generation\ul
\item Generate the kernel for specific device
\item Generate the scheduling and barriers
\ule

\textbf{Binary program}
}
%\ol
%\item Make sure we encompass all the most common patterns of DP: check if we have higher dimensions or more complex formula.
%\item Let the user tune the window size if he wants to reduce non-serial to serial.
%\item Concerns separation: common architecture enables flexibility (exchange components) \ul
%	\item Block processor: CPU, GPU, FPGA, must allow variation of width and height
%	\item Memory stats computation (min, sum, ... column/line combinations): CPU, GPU
%	\item Scheduler (CPU): interleave block computation and memory statistics
%	\ule
%\item Common description \ul
%	\item Block kernel processor
%	\item Full block
%	\ule
%\item Discussion: wavefront, design similarities, polyhedral theory
%\ole

%%\newcolumntype{C}[1]{>{\centering\let\newline\\\arraybackslash}m{#1}}
%\newcolumntype{C}{@{\hspace{7pt}}c@{\hspace{7pt}}}
%\def\mnl{\rule{0pt}{2.6ex}\rule[-1.2ex]{0pt}{0pt} \\ \hline}
%$\begin{array}{|C|C|C|C|C|C|} \hline
%0 & 0 & 0 & 0 & 0 & 0 \mnl
%0 &  &  &  &  & \mnl
%0 & M_{23}  &  &  &  & \mnl
%0 &  & \sum  &  &  & \mnl
%0 &  &  &  &  & \mnl
%0 &  &  &  &  & \mnl
%\end{array}$
%\end{document}
