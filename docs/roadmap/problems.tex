\newpage
\section{Dynamic programming problems}
% ------------------------------------------------------------------------------------------------
\subsection{Problems classification}
\subsubsection{Definitions}\ul
\item \textbf{Dimensions:} let $n$ the size of the input and $d$ the dimension of the underlying matrix.
\item \textbf{Matrices:} we refer indifferently by the matrix or the matrices to all the intermediate cost- and backtrack-related informations that are necessary to solve the dynamic programming problem of interest. Matrices elements are usually denoted by $M_{(i,j)}$ ($i^{\rm th}$ line , $j^{\rm th}$ column).
\item \textbf{Computation block:} this is a part of the DP matrix (cost and or backtrack) that we want to compute. A block might be either a sub-matrix (rectangular) or a parallelogram, possibly cropped at its parent matrix boundaries.
\item \textbf{Wavefront:} the wavefront consists of all the data necessary to reconstruct a computation block of the DP matrix. It might include some previous lines/columns/diagonals as well as line-/column-/diagonal-wise aggregations (min, max, sum, ...).
\item \textbf{Delay:} we call delay the maximum distance between an element and its dependencies along column and lines (ex: recurrence $M_{(i,j)}=f\big(M_{(i-1,j)}, M_{(i-2,j-1)}\big)$ has delay 3).
\ule

\subsubsection{Litterature classification}
In the literature, dynamic programming problems (DP) are classified according to two criteria:\ul
\item \textbf{Monadic/polyadic:} a problem is monadic when only one of the previously computed term appears in the right hand-side of the recurrence formula (ex: Smith-Waterman). When two or more terms appear, the problem is polyadic (ex: Fibonacci, $F_n = F_{n-1} + F_{n-2}$).
When a problem is polyadic with index $p$, it also means that its backtracking forms a $p$-ary tree (where each node has at most $p$ children).

\item \textbf{Serial/non-serial:} a problem is serial ($s=0$) when the solutions depends on a fixed number of previous solutions (ex: Fibonacci), otherwise it is said to be non-serial ($s\ge 1$), as the number of dependencies grows with the size of the subproblem. That is computing an element of the matrix would require $O(n^s)$.  (ex: Smith-Waterman with arbitrary gap is $s=1$; we can usually infer $s$ from the number of bound variables in the recurrence formula)
	\[M_{(i,j)}=\max\left\{\begin{array}{l} ... \\ M_{(i,j-1)}\\ \max\limits_{i<k<j} [ M_{(i,k)}+M_{(k+1,j)} ] \end{array}\right. \]
\ule

Note that the algorithmic complexity of a problem is exactly $O\big(n^{d+s}\big)$.

\subsubsection{Calculus simplifications}
In some special case, it is possible to transform a non-serial problem into a serial problem, if we can embed the non-serial term into an additional aggregation matrix. For example:
	\[M_{(i,j)}=\max\left\{\begin{array}{l} \max\limits_{k<i} M_{(k,j)}
	\\ \sum\limits_{k<i, l<j}M_{(k,l)} \end{array}\right.
	\implies M_{(i,j)}=\max\left\{\begin{array}{l} C_{(k,j)} \\ A_{(i-1,j-1)} \end{array}\right.\]
Where the matrix $C$ stores the maximum along the column and matrix $A$ stores the sum of the array of the previous elements. Both can be easily computed with an additional recurrence:
	\[\begin{array}{rcl} C_{(i,j)}&=&\max(C_{(i-1,j)}, M_{(i,j)}) \\
	A_{(i,j)}&=&A_{(i-1,j)}+A_{(i,j-1)}-A_{(i-1,j-1)}+M_{(i,j)}\end{array}\]

Although this simplification removes some non-serial dependencies at the cost of extra storage in the wavefront, it is not sufficient to transform all non-serial monadic problems into serial problems (ex: this does not apply to Smith-Waterman with arbitrary gap cost).

% ------------------------------------------------------------------------------------------------
\subsection{Problems of interest}
We usually focus on problem that have an underlying bi-dimensional matrix ($d=2$) because they can be parallelized (as opposed to be serial if $d=1$) and can solve large problems (of size $n$). Problems of higher matrix dimensionality ($d\ge3$) require substantial memory which severely impacts their scalability. Also we tend to limit algorithmic complexity of the problems as from $O(n^4)$ on, running time becomes a severely limiting factor.

We describe problems structures: inputs, cost matrices and backtracking matrix. These all have an alphabet (that must be bounded in terms of bit-size). Unless otherwise specified, we adopt the following conventions:\ul
\item Matrices dimensions are implicitly specified by number of indices and their number of elements is usually the same as the input length.
\item Number are all unsigned integers
\item Problem dimension is $m,n$ (or $n$) indices $i,j$ ranges are respectively $0\le i<m$, $0\le j<n$.
\item Unless otherwise specified, the recurrence applies to all non-initialized matrix elements.
\ule
We describe the problem processing in terms of both initialization and recurrences.

% ------------------------------------------------------------------------------------------------
% Recurrence visualization helpers
\newcommand\Cd[3][0,-1]{\put(#2){\put(.5,.5){\circle*{.3}}\put(.5,.5){\linethickness{1.5pt}\vector(#1){#3}}}} % dependency [dx,dy]{x,y}{len}
\def\Cg#1{\put(#1){\color{lightgray}\put(0,0){\polygon*(0,0)(0,1)(1,1)(1,0)}}} % grayed cell (not to store
\def\Cz#1{\put(#1){\put(0,.35){\parbox{1\unitlength}{\centering\bf 0}}}} % zero-init cell
\def\Cm{\put(6.5,4.5){\circle*{.4}}\multiput(0,0)(1,0){9}{\line(0,1){8}}\multiput(0,0)(0,1){9}{\line(1,0){8}}} % matrix base
\def\Cfl#1{#1{0,6}#1{0,5}#1{1,5}#1{0,4}#1{1,4}#1{2,4}#1{0,3}#1{1,3}#1{2,3}#1{3,3}#1{0,2}#1{1,2}#1{2,2}#1{3,2}#1{4,2}
	#1{0,1}#1{1,1}#1{2,1}#1{3,1}#1{4,1}#1{5,1}#1{0,0}#1{1,0}#1{2,0}#1{3,0}#1{4,0}#1{5,0}#1{6,0}} % triangular lower (function)
\def\Cfd#1{#1{0,7}#1{1,6}#1{2,5}#1{3,4}#1{4,3}#1{5,2}#1{6,1}#1{7,0}} % main diagonal

\def\Cmlong{\put(6.5,4.5){\circle*{.4}}\multiput(0,0)(1,0){16}{\line(0,1){8}}\multiput(0,0)(0,1){9}{\line(1,0){15}}} % matrix base
\def\Cfu#1{#1{8,7}#1{9,7}#1{10,7}#1{11,7}#1{12,7}#1{13,7}#1{14,7}#1{9,6}#1{10,6}#1{11,6}#1{12,6}#1{13,6}#1{14,6}#1{10,5}#1{11,5}#1{12,5}#1{13,5}#1{14,5}#1{11,4}#1{12,4}#1{13,4}#1{14,4}#1{12,3}#1{13,3}#1{14,3}#1{13,2}#1{14,2}#1{14,1}} % triangular upper (function)

% ----------------------------------------------
\newpage
\subsubsection{Smith-Waterman (simple)}\label{sswat}\ol
\item Problem: matching two strings $S$, $T$ with $|S_{\rm padded}|=m, |T_{\rm padded}|=n$.
\item Matrices: $M_{m \times n}, B_{m \times n}$
\item Alphabets:\ul
	\item Input: $\Sigma(S)=\Sigma(T)=\{a,c,g,t\}$.
	\item Cost matrix: $\Sigma(M) = [0..z], z=\max({\rm cost(\_)}) \cdot \min(m,n)$
	\item Backtrack matrix: $\Sigma(B)=\{stop,W,N,NW\}$
	\ule
\item Initialization:\ul
	\item Cost matrix: $M_{(i,0)}=M_{(0,j)}=0$.
	\item Backtrack matrix: $B_{(i,0)}=B_{(0,j)}=stop$.
	\ule
\item Recurrence: \[M_{(i,j)}=\max\left\{\begin{array}{l|l}
		0 & stop\\
		M_{(i-1,j-1)}+{\rm cost}(S(i),T(j)) & NW\\
		M_{(i-1,j)}-d & N\\
		M_{(i,j-1)}-d & W
	\end{array}\right\}=B_{(i,j)} \]

\item Backtracking: starts from the cell $\max \{M_{(m,j)} \cup M_{(i,n)}\}$, stops at
the first cell containing a $0$.
\item Visualization: by convention, we put the longest string vertically ($m\ge n$):
\begin{center}\setlength{\unitlength}{.6cm}\begin{picture}(8,9)
	\put(-.5,7.5){S}\put(-.35,7.4){\linethickness{1pt}\vector(0,-1){2}}
	\put(.2,8.2){T}\put(.8,8.4){\linethickness{1pt}\vector(1,0){2}}
	\Cz{0,0}\Cz{0,1}\Cz{0,2}\Cz{0,3}\Cz{0,4}\Cz{0,5}\Cz{0,6}\Cz{0,7}
	\Cz{1,7}\Cz{2,7}\Cz{3,7}\Cz{4,7}\Cz{5,7}\Cz{6,7}\Cz{7,7}
	\Cd{6,5}{0.8}
	\Cd[1,0]{5,4}{0.8}
	\Cd[1,-1]{5,5}{0.8}
\Cm\end{picture}\end{center}

\item Optimizations:\ul
	\item In serial (monadic) problems we can avoid building the matrix $M$ by only maintaining the 3 last diagonals in memory (one for the diagonal element, one for horizontal/vertical, and one being built). This construction extends easily to polyadic problems where we need to maintain $k+2$ diagonals in memory where $k$ is the maximum backward lookup.
	\item Padding: since first line and column of the matrix are zeroes, their initialization might be omitted, but this would implies more involved initialization and computations, which is cumbersome. Also since to fill the $i^{\rm th}$ row we refer to the $(i-1)^{\rm th}$ character of string $S$ thus we prepend to both $S$ and $T$ an unused character, so that matrix and input lines are aligned. Hence valid input indices are $S[1 \cdots m-1]$ and $T[1 \cdots n-1]$. We refer as such strings as padded strings hereafter (with $|S_{\rm padded}| = |S| + 1$).
	\ule
\ole

% ----------------------------------------------
\newpage
\subsubsection{Smith-Waterman with affine gap extension cost}\ol
\item Problem: matching two strings $S$, $T$ with $|S_{\rm padded}|=m, |T_{\rm padded}|=n$.
\item Matrices: $M_{m \times n}, E_{m \times n}, F_{m \times n}, B_{m \times n}$
\item Alphabets:\ul
	\item Input: $\Sigma(S)=\Sigma(T)=\{a,c,g,t\}$.
	\item Cost matrices: $\Sigma(M) = \Sigma(E) = \Sigma(F) = [0..z], z=\max({\rm cost(\_)}) \cdot \min(m,n)$
	\item Backtrack matrix: $\Sigma(B)=\{stop,W,N,NW\}$
	\ule
\item Initialization:\ul
	\item No gap cost matrix: $M_{(i,0)}=M_{(0,j)}=0$.
	\item T-gap extension cost matrix: $E_{(i,0)}= 0$ \textit{<<eat S chars only>>}
	\item S-gap extension cost matrix: $F_{(0,j)}= 0$
	\item Backtrack matrix: $B_{(i,0)}=B_{(0,j)}=stop$.
	\ule
\item Recurrence for the cost matrices:
\[\begin{array}{rcl}
M_{(i,j)}&=&\max\left\{\begin{array}{l|l}
	0 & stop\\
	M_{(i-1,j-1)}+{\rm cost}(S(i),T(j)) & NW\\
	E_{(i,j)} & N\\
	F_{(i,j)} & W
\end{array}\right\}=B_{(i,j)}\\
\\
E_{(i,j)}&=&\max\left\{\begin{array}{l|l}
	M_{(i, j-1)} - \alpha & NW\\
	E_{(i,j-1)} - \beta & N\\
\end{array}\right\}=B_{(i,j)}\\
\\
F_{(i,j)}&=&\max\left\{\begin{array}{l|l}
	M_{(i-1,j)} - \alpha & NW\\
	F_{(i-1,j)} - \beta & W\\
\end{array}\right\}=B_{(i,j)}
\end{array}\]

That can be written alternatively as:
\[M_{(i,j)}=\max\left\{\begin{array}{l|l}
	0 & stop\\
	M_{(i-1,j-1)}+{\rm cost}(S(i),T(j)) & NW\\
	\max_{1 \le k \le j-1} M_{(i,k)} - \alpha - (j-1-k) \cdot \beta & N\\
	\max_{1 \le k \le i-1} M_{(k,j)} - \alpha - (i-1-k) \cdot \beta & W\\
\end{array}\right\}=B_{(i,j)} \]

Although the latter notation seems more explicit, it introduces non-serial dependencies that the former set of recurrences is free of. So we need to implement the former rules whose kernel is 
\[ [M;E;F]_{(i,j)} = f_{\rm kernel} ( [M;E]_{(i,j-1)}, [M;F]_{(i-1,j)}, M_{(i-1,j-1)} ) \]
Notice that this recurrence is very similar to \nameref{sswat} except that we propagate 3 values ($M,E,F$) instead of a single one ($M$). Also notice that it is possible to propagate $E$ and $F$ inside a resp. horizontal and vertical wavefront.

\item Backtracking: same as \nameref{sswat}
\item Visualization: same as \nameref{sswat}
\item Optimizations: same as \nameref{sswat}
\ole

% ----------------------------------------------
\newpage
\subsubsection{Smith-Waterman with arbitrary gap cost}\label{aswat}\ol
\item Problem: matching two strings $S$, $T$ with $|S_{\rm padded}|=m, |T_{\rm padded}|=n$ with an arbitrary gap function $g(x)\ge 0$ where $x$ is the size of the gap. Without loss of generality, let $m\ge n$\footnote{Otherwise if $|T|>|N|$ we only need to swap both the inputs and backtracking pairs.}. Example penalty function could be\footnote{Intuition: long gaps penalize less, at some point, one large gap is better than matching and smaller gaps.} $g(x)=m-x$.
\item Matrices: $M_{m \times n}, B_{m \times n \times m}$
\item Alphabets:\ul
	\item Input: $\Sigma(S)=\Sigma(T)=\{a,c,g,t\}$.
	\item Cost matrix: $\Sigma(M) = [0..z], z=\max({\rm cost(\_)}) \cdot \min(m,n)$
	\item Backtrack matrix: $\Sigma(B)=\{stop,NW,N_{\{0..m\}},W_{\{0..n\}}\}$
	\ule

\item Initialization:\ul
	\item Match cost matrix: $M_{(i,0)}=M_{(0,j)}=0$.
	\item Backtrack matrix: $B_{(i,0)}=B_{(0,j)}=stop$.
	\ule

\item Recurrence: \[M_{(i,j)}=\max\left\{\begin{array}{l|l}
	0 & stop\\
	M_{(i-1,j-1)}+{\rm cost}(S(i),T(j)) & NW\\
	\max_{1 \le k \le j-1} M_{(i,j-k)} - g(k) & N_k\\
	\max_{1 \le k \le i-1} M_{(i-k,j)} - g(k) & W_k\\
\end{array}\right\}=B_{(i,j)} \]

\item Backtracking: similar to \nameref{sswat} except that you can jump of $k$ cells.
\item Visualization:
	\begin{center}\setlength{\unitlength}{.6cm}\begin{picture}(8,9)
		\put(-.5,7.5){S}\put(-.35,7.4){\linethickness{1pt}\vector(0,-1){2}}
		\put(.2,8.2){T}\put(.8,8.4){\linethickness{1pt}\vector(1,0){2}}
	\Cz{0,0}\Cz{0,1}\Cz{0,2}\Cz{0,3}\Cz{0,4}\Cz{0,5}\Cz{0,6}\Cz{0,7}
	\Cz{1,7}\Cz{2,7}\Cz{3,7}\Cz{4,7}\Cz{5,7}\Cz{6,7}\Cz{7,7}
		\Cd[0,-1]{6,7}{2.8}\Cd[0,-1]{6,6}{1.8}\Cd{6,5}{0.8}
		\Cd[1,0]{0,4}{5.8}\Cd[1,0]{1,4}{4.8}\Cd[1,0]{2,4}{3.8}\Cd[1,0]{3,4}{2.8}\Cd[1,0]{4,4}{1.8}\Cd[1,0]{5,4}{0.8}
		\Cd[1,-1]{5,5}{0.8}
	\Cm\end{picture}\end{center}

\item Optimizations: The dependencies here are non-serial, there is no optimization that we can 
apply out of the box here.
\ole


% ----------------------------------------------
\newpage
\subsubsection{Convex polygon triangulation}\ol
\item Problem: triangulating a polygon of $n$ vertices with least total cost for added edges. We denote the cost of adding an edge between the  pair of edges $i,j$ by $S(i,j)$, Where $S_{n \times n}$ is a lower triangular matrix compacted in memory (rows are contiguous) with a 0 diagonal that is omitted \footnote{Arbitrary convention for both architectural implementation and code generator. Rationale: in lower triangular matrix, element address is independent of the matrix size.}, hence $|S|=\tfrac{n^2}{2}=N$.

\item Matrices: $M_{n\times 2n}, B_{n \times 2n}$ \textit{<<first edge, last edge>>} upper triangular including main diagonal 
\item Alphabets:\ul
	\item Input: $\Sigma(S_{(i,j)})=\{0..m\}$ with $m=\max_S(i,j) \forall i,j$ determined at runtime\footnote{We need to scan/have stats about $S$ and that's where LMS plays a role}.
	\item Cost matrix: $\Sigma(M)=\{0..z\}$ with $z = m \cdot (n-2)$ (we add at most $n-2$ edges).
	\item Backtrack matrix: $\Sigma(B)=\{stop, 0..n\}$ (the index of the edge we add)
	\ule
\item Initialization: $M_{(i,i)}=0, M_{(i,i+1)}=0, B_{(i,i)}=stop \quad\forall i$

\item Recurrence: \[M_{(i,j)}=\left\{ S(i,j) + \max_{i<k<j}M_{(i,k)}+M_{(k,j)} \,\,\rule[-.75em]{.5pt}{2em}\,\,  k \right\} = B_{(i,j)} \]
	It is interesting to note that even in the sequential world, this problem is solved 
	by filling the diagonals, ie. computing sub-solutions for all polygons of size $k$ before
	those of size $k+1$.

\item Backtracking: Start at $B_{(1,n)}$. Use the following recursive function for the smaller polygons:
	\[{\rm BT}(B_{(i,j)}=k) \mapsto \left\{\begin{array}{ll} A_i & \text{if } k=0 \lor k=j \\
		\Big( {\rm BT}(B_{(i,k)}) \Big) \cdot \Big( {\rm BT}(B_{(k+1,j)}) \Big) & \text{otherwise} \end{array}\right.\]

\item Visualization: the layout is the a matrix of size $n \times (2n-2)$, because of polygons being 
"cyclical" in nature.
\begin{center}\setlength{\unitlength}{.6cm}\begin{picture}(16,9)
	\put(-.7,6.5){\rotatebox{90}{First}}\put(-.4,6.4){\linethickness{1pt}\vector(0,-1){2}}
	\put(.2,8.2){Last}\put(1.5,8.4){\linethickness{1pt}\vector(1,0){2}}
	\Cfl{\Cg}\Cfd{\Cz}\Cfu{\Cg}
	\Cd[0,1]{6,1}{2.8}\Cd[0,1]{6,2}{1.8}\Cd[0,1]{6,3}{0.8}
	\Cd[1,0]{3,4}{2.8}\Cd[1,0]{4,4}{1.8}\Cd[1,0]{5,4}{0.8}
	\put(3.5,4.5){\line(3,-1){3}}
	\put(4.5,4.5){\line(2,-2){2}}
	\put(5.5,4.5){\line(1,-3){1}}
	%\Cd[1,-1]{5,2}{0.8}
\Cmlong\end{picture}\end{center}

\item Optimizations: we need to rotate that matrix to progress in the same direction as usual, that is towards bottom right.
{\color{red}}
\ole

% ----------------------------------------------
\newpage
\subsubsection{Matrix chain multiplication}\ol
\item Problem: find an optimal parenthesizing of the multiplication of $n$ matrices $A_i$. Each matrix $A_i$ is of dimension $r_i \times c_i$ and $c_i=r_{i+1} \forall i$. \textit{<<r=rows, c=columns>>}
\item Matrices: $M_{n \times n}, B_{n \times n}$ \textit{(first, last matrix)}
\item Alphabets:\ul
	\item Input: matrix $A_i$ size is defined as pairs of integers $(r_i,c_i)$.
	\item Cost matrix: $\Sigma(M)= 1..z$ with $z\le n\cdot \big[ \max_i(r_i,c_i) \big]^3 $.
	\item Backtrack matrix: $\Sigma(B)=\{stop\} \cup \{0..n\}$.
	\ule
\item Initialization:\ul
	\item Cost matrix: $M_{(i,i)}=0$.
	\item Backtrack matrix: $B_{(i,i)}=stop$.
	\ule
\item Recurrence: $c_k=r_{k+1}$
	\[M_{(i,j)}=\min_{i\le k<j}\left\{\begin{array}{l|l}
		M_{(i,k)}+M_{(k+1,j)}+r_i \cdot c_k \cdot c_j & k
	\end{array}\right\}=B_{(i,j)} \]
\item Backtracking: Start at $B_{(1,n)}$. Use the following recursive function for parenthesizing
	\[{\rm BT}(B_{(i,j)}=k) \mapsto \left\{\begin{array}{ll} A_i & \text{if } k=0 \lor k=j \\
		\Big( {\rm BT}(B_{(i,k)}) \Big) \cdot \Big( {\rm BT}(B_{(k+1,j)}) \Big) & \text{otherwise} \end{array}\right.\]

\item Visualization:
	\begin{center}\setlength{\unitlength}{.6cm}\begin{picture}(8,9)
		\put(-.7,6.5){\rotatebox{90}{First}}\put(-.4,6.4){\linethickness{1pt}\vector(0,-1){2}}
		\put(.2,8.2){Last}\put(1.5,8.4){\linethickness{1pt}\vector(1,0){2}}
		\Cfl{\Cg}\Cfd{\Cz}
		\Cd[0,1]{6,1}{2.8}\Cd[0,1]{6,2}{1.8}\Cd[0,1]{6,3}{0.8}
		\Cd[1,0]{3,4}{2.8}\Cd[1,0]{4,4}{1.8}\Cd[1,0]{5,4}{0.8}
		\put(3.5,4.5){\line(3,-1){3}}\put(4.5,4.5){\line(2,-2){2}}\put(5.5,4.5){\line(1,-3){1}}
	\Cm\end{picture}\end{center}

\item Optimizations:\ul
	\item We need to swap vertically the matrix to have a normalized progression towards bottom right. To do that, we need to map all indices $i \mapsto n-1-i$ and we obtain a new recurrence relation:
	\[M_{(i,j)}=\min_{i\le k<j}\left\{\begin{array}{l} M_{(i,k)}+M_{(2i-1 -k,j)}+r_i \cdot c_k \cdot c_j \end{array}\right. \]
	With the initialization at $M_{(i,n-i-1)}$
	\ule
\ole

% ----------------------------------------------
\newpage
\subsubsection{Nussinov algorithm}\ol
\item Problem: folding a RNA string $S$ over itself $\left\lfloor |S| / 2 \right\rfloor = n$.
\item Matrices: $M_{n\times n}, B_{n \times n}$
\item Alphabets:\ul
	\item Input: $\Sigma(S)=\{A,C,G,U\}$.
	\item Cost matrix: $\Sigma(M)=\{0..n\}$
	\item Backtrack matrix: $\Sigma(B)=\{stop,D,1..n\}$
	\ule
\item Initialization: \ul
	\item Cost matrix: $ M_{(i,i)}=M_{(i,i-1)}=0$
	\item Backtrack matrix: $B_{(i,i)}=B_{(i,i-1)}=stop$
	\ule
\item Recurrences:
	\[M_{(i,j)}=\max\left\{\begin{array}{l|l}
		M_{(i+1,j-1)}+\omega(i,j) & D\\
		\max_{i\le k<j}M_{(i,k)}+M_{(k+1,j)} & k
	\end{array}\right\} = B_{(i,j)} \]
	With $\omega(i,j)=1$ if $i,j$ are complementary. 0 otherwise.
\item Backtracking: Start the backtracking in $B_{(1,n)}$ and go backward. The backtracking is very similar to that of the matrix multiplication, except that we also introduce the diagonal matching.
\item Visualization:
	\begin{center}\setlength{\unitlength}{.6cm}\begin{picture}(8,9)
		\put(-.7,6.5){\rotatebox{90}{First}}\put(-.4,6.4){\linethickness{1pt}\vector(0,-1){2}}
		\put(.2,8.2){Last}\put(1.5,8.4){\linethickness{1pt}\vector(1,0){2}}
		\Cfl{\Cg}\Cfd{\Cz}
		\Cd[0,1]{6,1}{2.8}\Cd[0,1]{6,2}{1.8}\Cd[0,1]{6,3}{0.8}
		\Cd[1,0]{3,4}{2.8}\Cd[1,0]{4,4}{1.8}\Cd[1,0]{5,4}{0.8}
		\Cd[1,1]{5,3}{0.8}
		\put(3.5,4.5){\line(3,-1){3}}\put(4.5,4.5){\line(2,-2){2}}\put(5.5,4.5){\line(1,-3){1}}
	\Cm\end{picture}\end{center}

\item Optimizations: note that this is very similar to the matrix multiplication except that we also need the diagonal one step backward, so the same optimization can apply.
\ole

% ----------------------------------------------
\newpage
\subsubsection{Zuker folding}\ol
\item Problem: folding a RNA string $S$ over itself $\left\lfloor |S| / 2 \right\rfloor = n$.
\item Matrices: $V_{n\times n}, W_{n\times n}, F_n$ (Free Energy),  $BV_{n \times n}, BW_{n \times n}, BF_n$
\item Alphabets:\ul
	\item Input: $\Sigma(S)=\{A,C,G,U\}$.
	\item Cost matrices:\ul
		\item $\Sigma(W)=\Sigma(V)=\{0..z\}$ with $z \le n*b+c$
		\item $\Sigma(F)=\{0..y\}$ with $y\le \min(F_0, z\cdot n)$
		\ule
	\item Backtrack matrices: \ul
		\item $\Sigma(BW)=\{stop, S,W,V,k\}$
		\item $\Sigma(BV)=\{stop, HL, IL, SW, (i,j) , k\}$ with $0\le i,j,k < n$ \\
		$HL$=HairpinLoop, $IL$=InteriorLoop, $(i,j)$=MultiLoop
		
		
		\item $\Sigma(BF)=\{stop, L, k\}$ with $0\le k < n$
		\ule
	\ule
\item Initialization:\ul
	\item Cost matrices: $W_{(i,i)}=V_{(i,i)}=0, F_{(0)}=$ energy of the unfolded RNA.
	\item Backtrack matrices: $BW_{(i,i)}=BV_{(i,i)}=BF_{(0)}=stop$.
	\ule
\item Recurrence:
\[\begin{array}{rcl}
W_{(i,j)}&=&\min\left\{\begin{array}{l|l}
	W_{(i+1,j)}+b & S\\
	W_{(i,j-1)}+b & W\\
	V_{(i,j)}+\delta(S_i,S_j) & V \\
	\min_{i<k<j}W_{(i,k)}+W_{(k+1,j)} &k
\end{array}\right\} = BW_{(i,j)}\\
\\
V_{(i,j)}&=&\min\left\{\begin{array}{l|l}
	\infty \qquad\qquad\qquad\qquad {\rm if}(S_i,S_j) \text{ is not a base pair} & stop\\\\
	eh(i,j)+b \qquad\qquad\, \text{otherwise} & HL\\
	V_{(i+1,j-1)}+es(i,j) & IL \\
	VBI_{(i,j)} & (i',j') \\
	\min_{i<k<j-1}\{W_{(i+1,k)}+W_{(k+1,j-1)}\} +c & k
\end{array}\right\} = BV_{(i,j)}\\
\\
VBI_{(i,j)}&=&\min\Big\{\min_{i<i'<j'<j}V_{(i',j')}+ebi(i,j,i',j')\} +c \,\,\Big|\,\, (i',j') \Big\}=BV_{(i,j)}\\
\\
F_{(j)}&=&\min\left\{\begin{array}{l|l}
	F_{(j-1)} & L \\ 
	\min_{1\le i< j} (F_{(i-1)} + V_{(i,j)}) & i
\end{array}\right\} = BF_{(j)}
\end{array}\]

In practice, we don't go backward for larger values than 30, so we can replace $\min_{i<k<j}$ by $\min_{\max(i,j-30)<k<j}$ in the expressions of $VBI$.

\item Backtracking: Start at $BF_{(n)}$ using the recurrences
 \[\begin{array}{rcl}
	BF_{(j)} &=& \left\{\begin{array}{rcl} L&\implies& BF_{(j-1)} \\ i &\implies& BF_{(i-1)} + BV_{(i,j)} \end{array} \right.\\
	\\
	BV_{(i,j)} &=& \color{red} \left\{\begin{array}{rcl}
		HL &\implies&\big< {\rm hairpin}(i,j) \big> \\
		IL &\implies& \big< {\rm stack}(i,j) \big> BV_{(i+1,j-1)} \\
		(i',j') &\implies& \big< \text{multi-loop from $(i,j)$ to }(i',j') \big> BV(i',j')\\
		k &\implies& BW_{(i+1,k)} BW_{(k+1,j-1)}
	\end{array}\right.\\
	\\
	BW_{(i,j)} &=& \color{red} \left\{\begin{array}{rcl}
	S & \implies & \big< bulge(i) \big> BW_{(i+1,j)} \\
	W & \implies & \big< bulge(j) \big> BW_{(i,j+1)} \\
	V &\implies& BV_{(i,j)} \\
	k &\implies& BW_{(i+1,k)} BW_{(k+1,j-1)}
	\end{array}\right.
\end{array}\]

\item Visualization: \begin{center}\includegraphics[width=8cm]{inc/zucker.pdf}\end{center}
	% source: <<Parallization of dynamic programming recurrences in computational biology>> paper
\item Optimizations: {\color{red} XXX: notice that there are 3 matrices: $W$,$V$ ($VBI$ is part of $V$) that can be expressed using regular matrix, and $F$ that is of different dimension than $W$ and $V$ and requires a special construction (in the wavefront?). We need to find a nice way to encode both its construction and backtrack into the existing framework (implement 1D DP recursively?)}
\ole

% ------------------------------------------------------------------------------------------------
\newpage
\subsection{Related problems}
The goal of this section is to demonstrate that our framework can accommodate with many {\color{red} (20-50)} problems that we have not considered at the design time.

\subsubsection{Serial problems}
\begin{tabular}{llcc} \toprule
\bf Problem & \bf Shape & \bf Matrices & \bf Wavefront \\ \midrule
Smith-Waterman \footnotesize simple & rectangle & 1 & -- \\
Smith-Waterman \footnotesize affine gap extension & rectangle & 3 & -- \\
\href{http://en.wikipedia.org/wiki/Needleman-Wunsch_algorithm}{Needleman-Wunsch} & rectangle & 1 & -- \\

\href{http://en.wikipedia.org/wiki/Dynamic_programming#Checkerboard}{Checkerboard} & rectangle & 1 & -- \\
\href{http://en.wikipedia.org/wiki/Longest_common_subsequence_problem\#Code_for_the_dynamic_programming_solution}{Longest common subsequence} & rectangle & 1 & -- \\
\href{http://en.wikipedia.org/wiki/Longest_common_substring_problem\#Pseudocode}{Longest common substring} & triangle & 1 & -- \\
\href{http://en.wikipedia.org/wiki/Levenshtein_distance\#Computing_Levenshtein_distance}{Levenshtein distance} & rectangle & 1 & -- \\
\href{http://en.wikipedia.org/wiki/De_Boor's_algorithm}{De Boor} \footnotesize evaluating B-spline curves & rectangle & 1 & -- \\
\end{tabular}

\subsubsection{Non-serial problems}
\begin{tabular}{llcc} \toprule
\bf Problem & \bf Shape & \bf Matrices & \bf Wavefront \\ \midrule
Smith-Waterman \footnotesize arbitrary gap cost & rectangle & 1 & -- \\
Convex polygon triangulation & parallelogram & 1 & -- \\
Matrix chain multiplication & triangle & 1 & -- \\
Nussinov & triangle & 1 & -- \\
Zuker folding & triangle & \color{red} 3? & \color{red} 0? \\
\href{http://en.wikipedia.org/wiki/CYK_algorithm}{CYK} \footnotesize Cocke-Younger-Kasami & triangle & \#rules & -- \\
\href{http://en.wikipedia.org/wiki/Knapsack_problem#Dynamic_programming}{Unbounded Knapsack} \footnotesize (input sensitive) & rectangle & 1 & --\\
\end{tabular}

\subsubsection{Other problems}\ul
\item Dijkstra shortest path: we need a $E\times V$ matrix, along $E$ forall $V$ reduce its distance, problem is serial along $E$, non-serial along $V$ hence of complexity $O(|E|\cdot |V^2|)$ which is far worse than both $O(|V|^2)$ (min-priority queue) and $O(|E|+|V|\log |V|)$ (Fibonacci heap).
\item Fibonacci: this problem is serial 1D. Could be implemented using a placeholder element in one of the matrix dimension.
\item \href{http://archive.ite.journal.informs.org/Vol3No1/Sniedovich/\#dpmodel}{Tower of Hanoi}: 1D non-serial
\item \href{http://www.cs.ust.hk/mjg_lib/bibs/DPSu/DPSu.Files/KnPl81.PDF}{Knuth's word wrapping}: 1D non-serial
\item \href{http://en.wikipedia.org/wiki/Longest_increasing_subsequence#Efficient_algorithms}{Longest increasing subsequence}: serial but binary search algorithm more efficient: $O(n \log n)$.
\item \href{http://www.ccs.neu.edu/home/jaa/CSG713.04F/Information/Handouts/dyn_prog.pdf}{Coin Change}: 1D non-serial


{\color{red}
\item \href{http://en.wikipedia.org/wiki/Floyd-Warshall_algorithm}{Floyd-Warshall}: <it possible to move the $k$ external loop inside?> Serial with $n$ iterations
\item \href{http://en.wikipedia.org/wiki/Viterbi_algorithm}{Viterbi \footnotesize (hidden Markov models)}: $T$ non-serial iterations over a vector
\item \href{http://en.wikipedia.org/wiki/Bellman-Ford_algorithm}{Bellman-Ford} (finding the shortest distance in a graph)
\item \href{http://en.wikipedia.org/wiki/Earley_parser#Pseudocode}{Earley parser} (a type of chart parser)
\item \href{http://en.wikipedia.org/wiki/Maximum_subarray_problem}{Kadane maximum subarray} 1D serial, look at 
\href{http://www.cosc.canterbury.ac.nz/tad.takaoka/cats02.pdf}{Takaoka} for 2D
\item Structural alignment (MAMMOTH, SSAP), \href{http://rna.tbi.univie.ac.at/cgi-bin/RNAfold.cgi}{RNA structure prediction}.
\item \href{http://en.wikipedia.org/wiki/Recursive_least_squares_filter}{Recursive least squares}
\item \href{http://www.math.utep.edu/Faculty/pmdelgado2/courses/adv_algorithms/homework-08_anser.pdf}{Bitonic tour}
}
\ule

\begin{verbatim}
- Balanced 0-1 matrix
- Recurrent solutions to lattice models for protein-DNA binding
- Backward induction as a solution method for finite-horizon discrete-time dynamic optimization problems
- Method of undetermined coefficients can be used to solve the Bellman equation in infinite-horizon, discrete-time, discounted, time-invariant dynamic optimization problems
- Many algorithmic problems on graphs can be solved efficiently for graphs of bounded treewidth or bounded clique-width by using dynamic programming on a tree decomposition of the graph.
- Transposition tables and refutation tables in computer chess
- Pseudo-polynomial time algorithms for the subset sum and knapsack and partition problems
- The dynamic time warping algorithm for computing the global distance between two time series
- The Selinger (a.k.a. System R) algorithm for relational database query optimization
- Duckworth-Lewis method for resolving the problem when games of cricket are interrupted
- The Value Iteration method for solving Markov decision processes
- Some graphic image edge following selection methods such as the "magnet" selection tool in Photoshop
- Some methods for solving interval scheduling problems
- Some methods for solving word wrap problems
- Some methods for solving the traveling salesman problem, either exactly (in exponential time) or approximately (e.g. via the bitonic tour)
- Beat tracking in music information retrieval.
- Adaptive-critic training strategy for artificial neural networks
- Stereo algorithms for solving the correspondence problem used in stereo vision.
- Seam carving (content aware image resizing)
- Some approximate solution methods for the linear search problem.
=====> http://en.wikipedia.org/wiki/Dynamic_programming#A_type_of_balanced_0.E2.80.931_matrix

- Shortest path in DAGs
- Shortest path
- All pair shortest paths
- Independent sets in trees
=> also see exercises for more problems
=====> http://www.cs.berkeley.edu/~vazirani/algorithms/chap6.pdf
- 
- Subset Sum, Coin Change, Family Graph
=====> http://www.algorithmist.com/index.php/Dynamic_Programming

- Optimal Binary Search Trees
=====> http://www.cs.uiuc.edu/~jeffe/teaching/algorithms/notes/05-dynprog.pdf

- Independent set on a tree
- 0-1 Knapsack
=====> http://www.cs.ucsb.edu/~suri/cs130b/NewDynProg.pdf
\end{verbatim}
