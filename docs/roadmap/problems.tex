\documentclass[11pt]{article}
\usepackage{amssymb,amsmath,amsthm,hyperref,verbatim,pict2e,graphicx,marvosym,array,booktabs,nameref}
\hypersetup{colorlinks,citecolor=black,filecolor=black,linkcolor=black,urlcolor=black}
\usepackage[usenames]{xcolor} % color names ,dvipsnames,svgnames,table
\usepackage[utf8]{inputenc}
\usepackage[T1]{fontenc}
\usepackage[english]{babel}
\pagestyle{empty} % remove page numbers

\oddsidemargin 0.0cm
\evensidemargin 0.0cm
\topmargin 0.0cm
\headheight 0.0cm
\headsep 1.0cm
\textheight 21.0cm
\textwidth 16.0cm
\parskip 0.1cm
\parindent 0.0cm
\footskip 1.0cm

\usepackage[compact]{titlesec}
\titlespacing{\section}{0pt}{8pt}{0pt}
\titlespacing{\subsection}{0pt}{8pt}{0pt}
\titlespacing{\subsubsection}{0pt}{8pt}{0pt}
\usepackage{enumitem}
\setitemize{noitemsep,topsep=0pt,parsep=0pt,partopsep=0pt}
\setenumerate{noitemsep,topsep=0pt,parsep=0pt,partopsep=0pt}
\def\ul{\begin{itemize}}
\def\ule{\end{itemize}}
\def\ol{\begin{enumerate}}
\def\ole{\end{enumerate}}

% Recurrence visualisation helpers
\newcommand\Cd[3][0,-1]{\put(#2){\put(.5,.5){\circle*{.3}}\put(.5,.5){\linethickness{1.5pt}\vector(#1){#3}}}} % dependency [dx,dy]{x,y}{len}
\def\Cg#1{\put(#1){\color{lightgray}\put(0,0){\polygon*(0,0)(0,1)(1,1)(1,0)}}} % grayed cell (not to store
\def\Cz#1{\put(#1){\put(0,.35){\parbox{1\unitlength}{\centering\bf 0}}}} % zero-init cell
\def\Cm{\put(6.5,4.5){\circle*{.4}}\multiput(0,0)(1,0){9}{\line(0,1){8}}\multiput(0,0)(0,1){9}{\line(1,0){8}}} % matrix base
\def\Cfl#1{#1{0,6}#1{0,5}#1{1,5}#1{0,4}#1{1,4}#1{2,4}#1{0,3}#1{1,3}#1{2,3}#1{3,3}#1{0,2}#1{1,2}#1{2,2}#1{3,2}#1{4,2}
	#1{0,1}#1{1,1}#1{2,1}#1{3,1}#1{4,1}#1{5,1}#1{0,0}#1{1,0}#1{2,0}#1{3,0}#1{4,0}#1{5,0}#1{6,0}} % triangular lower (function)
\def\Cfd#1{#1{0,7}#1{1,6}#1{2,5}#1{3,4}#1{4,3}#1{5,2}#1{6,1}#1{7,0}} % main diagonal

\title{DP Problems of Interest}
\author{Manohar Jonnalagedda, Thierry Coppey}
\date{}
\begin{document}
\maketitle
\pagestyle{headings}

% ----------------------------------------------------------------------------------------------------------------------------------------
\section{Introduction}
\subsection{Definitions}\ul
\item Block of computation: a block is simply a part of the DP matrix that we want to compute.
\item Wavefront: this is the place around which computation happen, typically. There should be some memory to store intermediate information between block of computations
\ule

\subsection{Problems classification}
In the literature, dynamic programming problems (DP) are classified according to two criteria:

\textbf{Monadic/polyadic}\ul
\item \textbf{Monadic:} on the right hand-side of the recurrence formula, only one term appears. For instance, Smith-Waterman with constant penalty is monadic
	\[M_{(i,j)}=\max\left\{\begin{array}{l} 0 \\ M_{(i-1,j-1)}+{\rm cost}(S(i),T(j))\\ M_{(i-1,j)}-d\\ M_{(i,j-1)}-d \end{array}\right. \]
\item \textbf{Polyadic:} when multiple terms of the recurrence occur in the right and-side of the recurrence formula. For instance Fibonacci is polyadic: \[F(n) = F(n-1) + F(n-2)\]
\ule

\textbf{Serial/non-serial} \ul
\item \textbf{Serial:} when the solution depends only of a fixed number of immediately previous solutions (i.e. neighbor cells). For instance Fibonacci is serial (it accesses only 2 cells backward).
\item \textbf{Non-serial:} when the solution depends of an arbitrary number of previous solutions. Typically Smith-Waterman with arbitrary gap penalty and Nussinov are non-serial:
	\[M_{(i,j)}=\max\left\{\begin{array}{l} ... \\ M_{(i,j-1)}\\ \max\limits_{i<k<j} [ M_{(i,k)}+M_{(k+1,j)} ] \end{array}\right. \]
\ule

\subsection{Simplifications}
\subsubsection{Calculus}
In some special case, it is possible to transform a non-serial problem into a serial problem, if we can embed the non-serial term into an additional aggregation matrix. For example:
	\[M_{(i,j)}=\max\left\{\begin{array}{l} \max\limits_{k<i} M_{(k,j)}
	\\ \sum\limits_{k<i, l<j}M_{(k,l)} \end{array}\right.
	\implies M_{(i,j)}=\max\left\{\begin{array}{l} C_{(k,j)} \\ A_{(i-1,j-1)} \end{array}\right.\]
Where the matrix $C$ stores the maximum along the column and matrix $A$ stores the sum of the array of the previous elements. Both can be easily computed with an additional recurrence:
	\[\begin{array}{rcl} C_{(i,j)}&=&\max(C_{(i-1,j)}, M_{(i,j)}) \\
	A_{(i,j)}&=&A_{(i-1,j)}+A_{(i,j-1)}-A_{(i-1,j-1)}+M_{(i,j)}\end{array}\]

This simplification avoids the non-serial dependencies at the cost of extra storage in the wavefront, unfortunately, it might be applicable only for some special cases.

\subsubsection{Precomputations}
When a calculus transformation is impossible, it might be worth to interleave a computation phase that will aggregate some of the results that are necessary to the computation block. For instance, for Nussinov term $\max_{i<k<j} [ M_{(i,k)}+M_{(k+1,j)} ]$, we can precompute it over all rows and columns of the block, and for all elements that are not part of the block, then pass these partial results together at the block launch to finish the computation.

On GPU, this could be done by interleaving a new kernel for this specific purpose, on FPGA, this could be done by preparing the maximums in another memory area whose pointer will later be passed to the co-processor. We may notice that since this phase is at the same time necessary for both architecture, and can be run independently, we can both execute them concurrently and mix between architectures (use CUDA for pre-computation and FPGA for actual tile computation for instance).

% ----------------------------------------------------------------------------------------------------------------------------------------
\section{Problems of interest}
We describe problems structures: inputs, cost matrices and backtracking matrix. These all have an alphabet (that must be bounded in terms of bit-size). Unless otherwise specified, we adopt the following conventions:\ul
\item Matrices dimensions are implicitly specified by number of indices and their number of elements is usually the same as the input length.
\item Number are all unsigned integers
\item Problem dimension is $m,n$ (or $n$) indices $i,j$ ranges are respectively $0\le i<m$, $0\le j<n$.
\item Unless otherwise specified, the recurrence applies to all non-initialized matrix elements.
\ule
We describe the problem processing in terms of both initialization and recurrences.

% ----------------------------------------------
\newpage
\subsection{Smith-Waterman (simple)}\label{sswat}\ol
\item Problem: matching two strings $S$, $T$ with $|S|=m, |T|=n$.
\item Matrices: $M_{m \times n}, B_{m \times n}$
\item Alphabets:\ul
	\item Input: $\Sigma(S)=\Sigma(T)=\{a,c,g,t\}$.
	\item Cost matrix: $\Sigma(M) = [0..z], z=\max({\rm cost(\_)}) \cdot \min(m,n)$
	\item Backtrack matrix: $\Sigma(B)=\{stop,W,N,NW\}$
	\ule
\item Initialization:\ul
	\item Cost matrix: $M_{(i,0)}=M_{(0,j)}=0$.
	\item Backtrack matrix: $B_{(i,0)}=B_{(0,j)}=stop$.
	\ule
\item Recurrence: \[M_{(i,j)}=\max\left\{\begin{array}{l|l}
		0 & stop\\
		M_{(i-1,j-1)}+{\rm cost}(S(i),T(j)) & NW\\
		M_{(i-1,j)}-d & N\\
		M_{(i,j-1)}-d & W
	\end{array}\right\}=B_{(i,j)} \]

\item Backtracking: starts from the cell $M_{(m,j)} \cup M_{(i,n)} $
\item Visualisation: by convention, we put the longest string vertically ($m\ge n$):
\begin{center}\setlength{\unitlength}{.6cm}\begin{picture}(8,9)
	\put(-.5,7.5){S}\put(-.35,7.4){\linethickness{1pt}\vector(0,-1){2}}
	\put(.2,8.2){T}\put(.8,8.4){\linethickness{1pt}\vector(1,0){2}}
	\Cz{0,0}\Cz{0,1}\Cz{0,2}\Cz{0,3}\Cz{0,4}\Cz{0,5}\Cz{0,6}\Cz{0,7}
	\Cz{1,7}\Cz{2,7}\Cz{3,7}\Cz{4,7}\Cz{5,7}\Cz{6,7}\Cz{7,7}
	\Cd{6,5}{0.8}
	\Cd[1,0]{5,4}{0.8}
	\Cd[1,-1]{5,5}{0.8}
\Cm\end{picture}\end{center}

\item Optimizations:\ul
	\item In serial (monadic) problems we can avoid building the matrix $M$ by only maintaining the 3 last diagonals in memory (one for the diagonal element, one for horizontal/vertical, and one being built). This construction extends easily to polyadic problems where we need to maintain $k+2$ diagonals in memory where $k$ is the maximum backward lookup.
	\ule
\ole

% ----------------------------------------------
\newpage
\subsection{Smith-Waterman (with gap extension at different cost)}\ol
\item Problem: matching two strings $S$, $T$ with $|S|=m, |T|=n$.
\item Matrices: $M_{m \times n}, E_{m \times n}, F_{m \times n}, B_{m \times n}$
\item Alphabets:\ul
	\item Input: $\Sigma(S)=\Sigma(T)=\{a,c,g,t\}$.
	\item Cost matrices: $\Sigma(M) = \Sigma(E) = \Sigma(F) = [0..z], z=\max({\rm cost(\_)}) \cdot \min(m,n)$
	\item Backtrack matrix: $\Sigma(B)=\{stop,W,N,NW\}$
	\ule
\item Initialization:\ul
	\item No gap cost matrix: $M_{(i,0)}=M_{(0,j)}=0$.
	\item T-gap extension cost matrix: $E_{(i,0)}= 0$ \textit{<<eat S chars only>>}
	\item S-gap extension cost matrix: $F_{(0,j)}= 0$
	\item Backtrack matrix: $B_{(i,0)}=B_{(0,j)}=stop$.
	\ule
\item Recurrence for the cost matrices:
\[\begin{array}{rcl}
M_{(i,j)}&=&\max\left\{\begin{array}{l|l}
	0 & stop\\
	M_{(i-1,j-1)}+{\rm cost}(S(i),T(j)) & NW\\
	E_{(i,j)} & N\\
	F_{(i,j)} & W
\end{array}\right\}=B_{(i,j)}\\
\\
E_{(i,j)}&=&\max\left\{\begin{array}{l|l}
	M_{(i, j-1)} - \alpha & NW\\
	E_{(i,j-1)} - \beta & N\\
\end{array}\right\}=B_{(i,j)}\\
\\
F_{(i,j)}&=&\max\left\{\begin{array}{l|l}
	M_{(i-1,j)} - \alpha & NW\\
	F_{(i-1,j)} - \beta & W\\
\end{array}\right\}=B_{(i,j)}
\end{array}\]

That can be written alternatively as:
\[M_{(i,j)}=\max\left\{\begin{array}{l|l}
	0 & stop\\
	M_{(i-1,j-1)}+{\rm cost}(S(i),T(j)) & NW\\
	\max_{1 \le k \le j-1} M_{(i,k)} - \alpha - (j-1-k) \cdot \beta & N\\
	\max_{1 \le k \le i-1} M_{(k,j)} - \alpha - (i-1-k) \cdot \beta & W\\
\end{array}\right\}=B_{(i,j)} \]

Although the latter notation seems more explicit, it introduces non-serial dependencies that the former set of recurrences is free of. So we need to implement the former rules whose kernel is 
\[ [M;E;F]_{(i,j)} = f_{\rm kernel} ( [M;E]_{(i,j-1)}, [M;F]_{(i-1,j)}, M_{(i-1,j-1)} ) \]
Notice that this recurrence is very similar to \nameref{sswat} except that we propagate 3 values ($M,E,F$) instead of a single one ($M$).

\item Backtracking: same as \nameref{sswat}
\item Visualisation: same as \nameref{sswat}
\item Optimizations: same as \nameref{sswat}
\ole

% ----------------------------------------------
\newpage
\subsection{Smith-Waterman with arbitrary gap cost}\ol
\item Problem: matching two strings $S$, $T$ with $|S|=m, |T|=n$ with an arbitrary gap function $g(x)\ge 0$ where $x$ is the size of the gap. Without loss of generality, let $m\ge n$\footnote{Otherwise if $|T|>|N|$ we only need to swap both the inputs and backtracking pairs.}. Example penalty function could be\footnote{Intuition: long gaps penalize less, at some point, one large gap is better than matching and smaller gaps.} $g(x)=m-x$.
\item Matrices: $M_{m \times n}, B_{m \times n \times m}$
\item Alphabets:\ul
	\item Input: $\Sigma(S)=\Sigma(T)=\{a,c,g,t\}$.
	\item Cost matrix: $\Sigma(M) = [0..z], z=\max({\rm cost(\_)}) \cdot \min(m,n)$
	\item Backtrack matrix: $\Sigma(B)=\{stop,NW,N_{\{0..m\}},W_{\{0..n\}}\}$
	\ule

\item Initialization:\ul
	\item No gap cost matrix: $M_{(i,0)}=M_{(0,j)}=0$.
	\item Backtrack matrix: $B_{(i,0)}=B_{(0,j)}=stop$.
	\ule

{\color{red}
\item Initialization: 
\item Recurrence:

\[M_{(i,j)}=\max\left\{\begin{array}{l|l}
	0 & stop\\
	M_{(i-1,j-1)}+{\rm cost}(S(i),T(j)) & NW\\
	\max_{1 \le k \le j-1} M_{(i,j-k)} - g(k) & N_k\\
	\max_{1 \le k \le i-1} M_{(i-k,j)} - g(k) & W_k\\
\end{array}\right\}=B_{(i,j)} \]



\item Backtracking:
\item Visualisation:
\item Optimizations:
}
\ole


% ----------------------------------------------
\newpage
\subsection{Convex polygon triangulation}\ol
\item Problem: triangulating a polygon of $n$ vertices with least total cost for added edges. We denote the cost of adding an edge between the  pair of edges $i,j$ by $S(i,j)$, Where $S_{n \times n}$ is a lower triangular matrix compacted in memory (rows are contiguous) with a 0 diagonal that is omitted \footnote{Arbitrary convention for both architectural implementation and code generator. Rationale: in lower triangular matrix, element address is independent of the matrix size.}, hence $|S|=\tfrac{n^2}{2}=N$.

{\color{red}
\item Matrices: $M_{n\times n}, B_{n \times n}$ \textit{<<first edge, last edge>>} upper triangular including main diagonal 
\item Alphabets:\ul
	\item Input: $\Sigma(S_{(i,j)})=\{0..m\}$ with $m=\max_S(i,j) \forall i,j$ determined at runtime\footnote{We need to scan/have stats about $S$ and that's where LMS plays a role}.
	\item Cost matrix: $\Sigma(M)=\{0..z\}$ with $z = m \cdot (n-2)$ (we add at most $n-2$ edges).
	\item Backtrack matrix: $\Sigma(B)=\{stop, 0..n\}$ (the index of the edge we add)
	\ule
\item Initialization: $M_{(i,i)}=0, B_{(i,i)}=stop \quad\forall i$
\item Recurrence: \[M_{(i,j)}=\left\{ \max_{i<k<j}M_{(i,k)}+M_{(k+1,j)} + S(i,k) \,\,\rule[-.75em]{.5pt}{2em}\,\,  k \right\} = B_{(i,j)} \]
	It is interesting to note that even in the sequential world, this problem is best solved 
	by filling the diagonals, ie. computing sub-solutions for all polygons on size $k$ before
	those of size $k+1$.
\item Backtracking: Starting at $B_{(0,n)}$ and let current point you backtrack drawing an edge from the current position to the backtracking position ??

\item Visualisation: in its traditional form the matrix is upper triangular with diagonal excluded.
\begin{center}\setlength{\unitlength}{.6cm}\begin{picture}(8,9)
	\put(-.7,6.5){\rotatebox{90}{First}}\put(-.4,6.4){\linethickness{1pt}\vector(0,-1){2}}
	\put(.2,8.2){Last}\put(1.5,8.4){\linethickness{1pt}\vector(1,0){2}}
	\Cfl{\Cg}\Cfd{\Cz}
	\Cd[0,1]{6,1}{2.8}\Cd[0,1]{6,2}{1.8}\Cd[0,1]{6,3}{0.8}
	\Cd[1,0]{3,4}{2.8}\Cd[1,0]{4,4}{1.8}\Cd[1,0]{5,4}{0.8}
	\put(3.5,4.5){\line(3,-1){3}}
	\put(4.5,4.5){\line(2,-2){2}}
	\put(5.5,4.5){\line(1,-3){1}}
	%\Cd[1,-1]{5,2}{0.8}
\Cm\end{picture}\end{center}

\item Optimizations: XXX we need to rotate that matrix to progress in the same direction as usual, that is towards bottom right.
}
\ole


% ----------------------------------------------
\newpage
\subsection{Matrix chain multiplication}\ol
\item Problem: find an optimal parenthesizing of the multiplication of $n$ matrices $A_i$. Each matrix $A_i$ is of dimension $r_i \times c_i$ and $c_i=r_{i+1} \forall i$. \textit{<<r=rows, c=columns>>}
\item Matrices: $M_{n \times n}, B_{n \times n}$ \textit{(first, last matrix)}
\item Alphabets:\ul
	\item Input: matrix $A_i$ size is defined as pairs of integers $(r_i,c_i)$.
	\item Cost matrix: $\Sigma(M):=$ huge integer\footnote{Integer multiplication might blow up, use float or doubles and addition of logarithms instead.}.
	\item Backtrack matrix: $\Sigma(B)=\{stop\} \cup \{0..n\}$.
	\ule
\item Initialization:\ul
	\item Cost matrix: $M_{(i,i)}=0$.
	\item Backtrack matrix: $B_{(i,i)}=stop$.
	\ule
\item Recurrence: $c_k=r_{k+1}$
	\[M_{(i,j)}=\min_{i\le k<j}\left\{\begin{array}{l|l}
		M_{(i,k)}+M_{(k+1,j)}+r_i \cdot c_k \cdot c_j & k
	\end{array}\right\}=B_{(i,j)} \]
\item Backtracking: Start at $B_{(1,n)}$. Use the following recursive function for parenthesizing
	\[{\rm BT}(B_{(i,j)}=k) \mapsto \left\{\begin{array}{ll} A_i & \text{if } k=0 \lor k=j \\
		\Big( {\rm BT}(B_{(i,k)}) \Big) \cdot \Big( {\rm BT}(B_{(k+1,j)}) \Big) & \text{otherwise} \end{array}\right.\]

\item Visualisation:
	\begin{center}\setlength{\unitlength}{.6cm}\begin{picture}(8,9)
		\put(-.7,6.5){\rotatebox{90}{First}}\put(-.4,6.4){\linethickness{1pt}\vector(0,-1){2}}
		\put(.2,8.2){Last}\put(1.5,8.4){\linethickness{1pt}\vector(1,0){2}}
		\Cfl{\Cg}\Cfd{\Cz}
		\Cd[0,1]{6,1}{2.8}\Cd[0,1]{6,2}{1.8}\Cd[0,1]{6,3}{0.8}
		\Cd[1,0]{3,4}{2.8}\Cd[1,0]{4,4}{1.8}\Cd[1,0]{5,4}{0.8}
		\put(3.5,4.5){\line(3,-1){3}}\put(4.5,4.5){\line(2,-2){2}}\put(5.5,4.5){\line(1,-3){1}}
	\Cm\end{picture}\end{center}

\item Optimizations:\ul
	\item {\color{red}We need to swap vertically the matrix to have a normalized progression towards bottom right. To do that, we need to map all indices $i \mapsto n-1-i$, but since we want to store the matrix sparsely, we might want to transform it into a lower triangular matrix as we can provide a size-independent mapping of element indices.
	XXX: what's the best trade off ? progress towards bottom right VS map indices more efficiently ?}
	\ule
\ole

% ----------------------------------------------
\newpage
\subsection{Nussinov algorithm}\ol
\item Problem: folding a RNA string $S$ over itself $\left\lfloor |S| / 2 \right\rfloor = n$.
\item Matrices: $M_{n\times n}, B_{n \times n}$
\item Alphabets:\ul
	\item Input: $\Sigma(S)=\{A,C,G,U\}$.
	\item Cost matrix: $\Sigma(M)=\{0..n\}$
	\item Backtrack matrix: $\Sigma(B)=\{stop,W,S,SW, 1..n\}$
	\ule
\item Initialization: \ul
	\item Cost matrix: $ M_{(i,i)}=M_{(i,i-1)}=0$
	\item Backtrack matrix: $B_{(i,i)}=B_{(i,i-1)}=stop$
	\ule

{\color{red}
\item Recurrences (XXX: isn't it $\max_{i\le k < j}$ ?):
	\[M_{(i,j)}=\max\left\{\begin{array}{l|l}
		M_{(i+1,j-1)}+\omega(i,j) & SW\\
		M_{(i+1,j)} & S\\
		M_{(i,j-1)} & W\\
		\max_{i<k<j}M_{(i,k)}+M_{(k+1,j)} & k
	\end{array}\right\} = B_{(i,j)} \]
	With $\omega(i,j)=1$ if $i,j$ are complementary. 0 otherwise.
\item Backtracking: XXX
\item Visualisation: XXX
\item Optimizations: XXX
}
\ole

% ------------------------------------------------------------------------------------------------
\newpage
\subsection{Zuker folding}\ol
\item Problem: folding a RNA string $S$ over itself $\left\lfloor |S| / 2 \right\rfloor = n$.
\item Matrices: $M_{n\times n}, V_{n\times n}, B_{n \times n}, {\color{red} F???}$
\item Alphabets:\ul
	\item Input: $\Sigma(S)=\{A,C,G,U\}$.
	\item Cost matrix: $\Sigma(W)=\Sigma(V)=\Sigma(F)=\{0..n\}$
	\item Backtrack matrix: $\Sigma(B)=\{stop, {\color{red} S,W, XXX here} \}$
	\ule
{\color{red}
\item Initialization:\ul
	\item Cost matrices:\ul
		\item XXX: $W$,$V$
		\item $F_{(0)}=0$
	\ule
	\item Backtrack matrix: $B$
	\ule

\item Recurrence:
\[\begin{array}{rcl}
W_{(i,j)}&=&\min\left\{\begin{array}{l|l}
	W_{(i+1,j)}+b & S\\
	W_{(i,j-1)}+b & W\\
	V_{(i,j)}+\delta(S_i,S_j) & ?V? \\
	\min_{i<k<j}W_{(i,k)}+W_{(k+1,j)} & ?k?
\end{array}\right.\\
\\
V_{(i,j)}&=&\min\left\{\begin{array}{ll}
	\infty \qquad\qquad\qquad\qquad {\rm if}(S_i,S_j) \text{ is not a base pair}\\
	eh(i,j)+b \qquad\qquad\, \text{otherwise} \\
	V_{(i+1,j-1)}+es(i,j) \\
	VBI(i,j) \\
	\min_{i<k<j-1}\{W_{(i+,k)}+W_{(k+1,j-1)}\} +c
\end{array}\right.\\
\\
 F_{(j)}&=&\min\left\{\begin{array}{ll} F_{(j-1)} \\ \min_{1\le i< j} (V_{(i,j)} + F_{(j-1)})
%
\end{array}\right. \text{(Free Energy)}\\\\
VBI(i,j)&=&\min_{i<i'<j'<j}\{V_{(i',j')}+ebi(i,j,i',j')\} +c
\end{array}\]

\item Backtracking: XXX
\item Visualisation: XXX
\item Optimizations: XXX
}
\ole

%%\newcolumntype{C}[1]{>{\centering\let\newline\\\arraybackslash}m{#1}}
%\newcolumntype{C}{@{\hspace{7pt}}c@{\hspace{7pt}}}
%\def\mnl{\rule{0pt}{2.6ex}\rule[-1.2ex]{0pt}{0pt} \\ \hline}
%$\begin{array}{|C|C|C|C|C|C|} \hline
%0 & 0 & 0 & 0 & 0 & 0 \mnl
%0 &  &  &  &  & \mnl
%0 & M_{23}  &  &  &  & \mnl
%0 &  & \sum  &  &  & \mnl
%0 &  &  &  &  & \mnl
%0 &  &  &  &  & \mnl
%\end{array}$
\end{document}
