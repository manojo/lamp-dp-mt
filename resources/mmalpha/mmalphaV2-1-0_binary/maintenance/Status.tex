\documentclass[12pt]{article}
\begin{document}

\newcommand{\mmalfa}{{\sc MMAlpha}}
\newcommand{\mma}{{\sc Mathematica}}
\title{Current status of \mmalfa\\
Last change: \today}
\author{Patrice Quinton}
\date{}
\maketitle

\section{Introduction}
This document is where I put things I have to remember about the current status of 
\mmalfa{}. The document is available directly in the main directory of the
distribution.




\section{A day to day memo}
A * before a date means that the correction was committed. A \texttt{?} means that I
am not sure that it was committed. A \texttt{-} means that it has to be done. A \texttt{!} 
says that I should commit the correction. 



\begin{description}
\item[18/1/2009:] 
\item[14/1/2009:] corrections � faire.
\begin{itemize}
\item tailles des donn�es: 2 bits en entr�e, idem pour les rom, r�sultats sur 4 bits � la sortie des mult, 
6 bits pour ovsf, une ligne pour compl�ter � 16 bits
\item regarder si on met des signed ou des std vectors
\item emetteur : ds le composant moduloaddress, erreur sur le type de port (std logic vectors)
\item taille des compteurts de periodic est � revoir (periodic enable4)
pour les stdlogic vector, les valeurs son entre ""
\item enlever size dans la liste de sensibilit� de periodicenable
\item trop de ports dans les rom? est-ce qu'on fait un registre
\item dans la rom, changer la limite de la decl de addr
\item enlever size dans moduloaddress de la liste de sensibilit�
\item remplacer values
\end{itemize}
\item[24/05/2010:] creating a new version MMALPHA-2-1-0. To be checked:
\begin{itemize}
\item In demos, check that there are not identifiers with identical names. Change analyze accordingly.
\item Check the init.m file in the config file.
\end{itemize}
\item[10/1/2009:] change fixParameter, so that the parameters are changed, depending on their
use... Has to be modified. Actually, it would be better to generate Vhdl using rules
for the parameters, and to generate on the fly as many elements than there are 
different calls.
\item[9/1/2009:] Made a few modifications in the Vertex Scheduler in order to be adapted
to the new version where coefficients are MMA variables. Made a correction to a bug in 
Simplex: if a variable of a constraint does not appear in the list of the variables of Minimize, then
the MMA function never stops.
\item[*6/1/2009:] New version released and installed. Waiting for an answer of Arpit. 
\item[4/1/2009:] Modification of the Web site. New version almost ready. Cleaning of some 
notebooks before installation. Created a test \texttt{TestSynthesis} and added it to the
distribution.
\item[2/1/2009:] Trying to check a new version. Tested it on Linux (machine sabre at IRISA). Succeeded
in compliling Polylib, but not domlib, as the MMA version is too old. Has to recover a previous 
version of the Domlib to make the test. 
\item[1/1/2009:] I made again changes in the tests, adding the defiition of a package
for each one of these, otherwise, global symbols are created when reading the file.
Has to change the documentation accordingly. In \texttt{TestMatrix.m}, there are
some tests (by inclusion of files) which were left as they were, and have to be 
included.
\item[31/12/2008:] I made major modifications and committed almost everything (except 
allocate.m and PipeControl-new.m which are not read. See 19/10/2008). I prepared a new version 
\texttt{mmalphaV2-0-1}. New tests were built (test1, 2, 3 and 4), but they are to 
be changed in order to introduce symbols in packages, and not in 
Global. I also made a correction to \texttt{dom2mma} in order to create the 
symbols that are the translation in \mma{} of those of the domain, in a work space
called \texttt{Alpha`Work`}. This had an effect on many packages. I also had to 
replace \texttt{Cut.m} by \texttt{CutMMA.m}, as there is a conflict with a function
in the \texttt{Combinatorica} package.
\item[*12/12/2008:] modification to Options.m (vhdlDir option usage was made more
precise).
\item[*12/12/2008:] modifications done in Vhdl2.m. Added an option vhdlPatterns to 
a2v, when true, Rom and a bunch of components are not generated. Instead, 
patterns are inserted in the file, and they are filled afterwards. Modifications were done
also to autoload.m, to VhdlCell.sem, and Synthesis.m.
\item[*4/12/2008:] started to implement the vhdl generator for periodic systems. In Alpha.m, 
there is a special version of load, which reads templates (suffixes are vhdx). In Vhdl2.m, 
created a function genVhdl that allows vhdl files to be created from templates. The templates
are currently in \$MMALPHA/VHDL. There are functions for ROM, for Fsm and for Periodic
enable signals. A documentation has to be provided. A function synPeriodic was included
in the \texttt{Synthesis.m} package. 
\item[*30/11/2008:] controlled a few traces in Alphard.m and ToAlpha0v2.m. Not commited yet.
\item[*30/11/2008:] made a small modification in vhdlCell.sem and vhdlModule.m, as there
was an extra message when the mute option was set. Not committed yet.
\item[*30/11/2008:] In Control.m, there was a lot of uncontrolled traces, set when I wanted
to make modifications to MakeSTCase. I controlled the traces. 
\item[*22/10/2008:] remove the definitions and work in the input of the files. Add a 
return.
\item[*21/10/2008:] tried the palindrome. Made modifications to ToAlpha0v2, but not sure
that it is necessary. 
\item[19/10/2008:] I made modifications to PipeControl in order to add the automatic
detection of pipeIO in pipeIO. I also rewrote part of the generation of pipeall commands, in 
pipeInfo. I added a few new functions to get information on polyhedra, regarding space
and time matter. The modification seems to work (on all 1D examples). Still a couple of 
extra prints to remove. I did not yet tested the 2D cases (matrix multiplication). I did not 
commit all this stuff. These modifs are currently in PipeControl-new.m and the previous
stuff is in PipeControl.m. Has to be checked.

\item[08/10/2008:] it would be good to separate the type declarations (package des types) 
in order to get the testbench separately.
\item[*07/10/2008:] there was a mistake in the controller generator, as the initial time
was shifted by the mintime, but not all instants of time. I made the correction (now, things start at 
mintime, until truetime, then we change to false time). Has to commit stuff. Added the
printing of the fsm in a2v when verbose is true. Only package to commit : vhdlCont.sem.
\item[*04/10/2008:] Tried to recompile \texttt{Write\_Alpha} on my laptop, and did not succeed, 
as the gcc loader has something wrong... I thought that it 
was probably related to the new version of MacOS. Was right, 
updated XCode, and it worked, except for read\_Alpha. Actually, there was 3 problems. First, 
there are some warnings in the compilation of yacc.y in some directories, because functions
strcat, strcopy and strlen seem to be used in a wrong way, but I believe that it does nothing more
than giving a warning message. There was a syntax error in the yacc.y file for Write\_C : a
semicolon missing. Seems that previously, there was no such problem... There was another
problem in yacc.y for read\_Alpha. It issues an error message "? bad char: (octal 0)" in a
systematic way. This message comes from a fprintf at the very end of the file. I just removed
the error message, and everything works fine. Why??? no idea.
\item[*04/10/2008:] Made a modification in AlpHard.m, to the function alpha0ToAlphard. Now, 
at the end, all systems are cleaned (option simplify $\rightarrow$ True added) before saving the library.
Note that there is an error in the synthesis of the fir, as the allocation in incorrect if one takes
the simplest schedule. One has to make sure that the first coefficient of the schedule
is 1. Also, possible to synthesize the fir without pipeing variables. Should add a noPipe 
option to the function syn, and see what happens. 
\item[16/09/2008:] started a new package, called Allocate.m, not yet automatically
downloaded. A corresponding example is in myNotebooks/Allocate. There is a function
allocTable, that creates an allocation table, with the name of the system,
its inputs, outputs, its processor space, its operators and their timing space. 
It is displayed with showAT. Its purpuse is to prepare something to merge
differente cells into a single one. 
\item[*13/09/2008:] added a new option to function scd. Option durationsNonZero gives the positions
of the dependences in the dep table which are non zero (duration assumed to be one). 
\item[*6/09/2008:] made a new version, seems to be running properly on Linux. 
Modified the Web site. On my laptop, the new Web site is in directory Copy-MMALPHA-Site/ALPHA.
It is accessible through 
\begin{verbatim}
http://webdav.irisa.fr/www.irisa.fr/htdocs/cosi
\end{verbatim}. 
I have to upgrade this... In the new version, called mmalphaV2-0-1.zip, I have
to modify the diagnose.m file, as it does not work properly for Mathematica version 4.0.
Also, add the information that unzip works... 
\item[3/09/2008:] tried to get a running Linux version. I zipped mmalpha, made a 
copy of it in my directory at irisa. Then I did a make checkvars in sources. It did not
work. I had to replace the makefile.linux in Domlib by a copy of the makefile.darwin. 
I also had to create a Makefile.linux.4.2 in MakeIncludes. Then make checkvars
worked. Then I made make all. Did not work. Seems that I have to provide an old
version of Domlib... because of the incompatibility of the Mathlink library. 
I made a copy of Domlib, and now, I put the old version of domlib.c. But I have
to compile PolyLib... Actually, everything works with the bin.linux version... seems so.
I forgot to set the \$path variable.
\item[*28/07/2008:] corrected the inlining in synthesis. In fact, it is needed to remove identical
equations, normalize, then simplify the system. In the normalization, there is something
strange: the parameter domain is not projected over the domains... Moreover, in a case
statement, if a domain is not empty, then the branch is not removed... 
\item[*26/07/2008:] now using eclipse to manage mmalpha... Wrote a first documentation 
for synthesis. With eclipse, I had a few problems. 
\item[*26/07/2008:] cleaned and committed some demos (Fir, Samba, etc.)
\item[*23/07/2008:] installed in my directory software MinGW, with examples. Contains
\texttt{gcc} to compile for Windows and Linux architectures. Not tried yet...
\item[*7/05/2008:] continued to add the mute option in AlphaToAlpha0v2, Alphard. 
In Subsystem, this is more difficult... since no options in Fixparameters...
\item[*5/05/2008:] succeeded in compiling Domlib. See Domlib documentation for
explanations... Should commit it soon. 
\item[*2/05/2008:] tried to recompile \texttt{writeC}. First, the use of this function is not clear, 
as if no file is given, we have the same input as if a parameter string was given... Second, 
there are missing libraries in the Write\_C program, and I added them. Third, the use of 
strcat seems not to be accepted by gcc. I replaced it by strcopy, and afterwards, I got
a linkedit error. Actually, I had to put the new libraries in yacc.y (which produces yacc.c), 
and make sure it was produced in one single text command, otherwise too many arguments
in Vsep. 
\item[*25/04/2008:] added \texttt{mute} option to \texttt{schedule.m}, \texttt{farkasSchedule.m},
\texttt{analyze}, etc. Committed.
\item[*21/04/2008:] added a \texttt{mute} option to \texttt{Alpha.m} package. \texttt{Alpha.m}, 
\texttt{Synthesis.m}, \texttt{Options.m} and \texttt{Vhdl2.m} were committed. 
Also the demo
directory for \texttt{Synthesis}. Committed.
\item[*20/04/2008:] created a new demo notebooks called \texttt{Synthesis}. Demo
of \texttt{Samba} does not work. Add a parameter for
\texttt{a2v} with the directory where we create \texttt{vhdl} files. Also check all options of
\texttt{a2v} in \texttt{Options.m} and develop them.
\item[19/04/2008:] created a Synthesis.m package, containing a \texttt[syn] function. Documentation
has to be done. Also commit \texttt{Alpha.m} and \texttt{autoload.m}. Next improvements : 
modify \texttt{syn} to handle structured designs, first by simple inlining, then using
real structured design. Add the possibility of designing the vhdl and the schedules, by
reading complex IPs. This could be done by modifying load in such a way that it
reads files with a different suffix using MMA, and then find out the structure of the 
program. 
\item[*18/04/2008:] Committed everything in lib, except \texttt{Normalization.m}. Corrected
a bug in Domlib (in DomEmptyQ). I removed the call to solveEqualities in noSolutions, 
and replaced it by a call to \texttt{solveDiophantine} which works. Should add a note in 
the documentation to explain that \texttt{DomEmptyQ} does not find really if a domain is
empty. Notice also that one can use the function Reduce to check if a domain is empty.
\texttt{Reduce[{i>1,i<2},{i,j},Integer]}. 
\item[*1/04/2008:] Dans Substitution.m, removed a \texttt{;;}. This created pbs in version 6.0. I.
\item[*1/04/2008:] Change in Alpha.m to include Visual3D. I think that it was committed.
\item[*1/04/2008:] Tried to clean up tests. Also, in packages, whenever there are accents, 
there is a problem in version 6.0. Guess it was committed.
\item[*1/04/2008:] In Domlib, I changed Global`scdc into scdc, which may be a problem?
\item[*1/04/2008:] TestsDomlib.m was modified and committed.
\item[?8/01/2008:] Dans pipeVars[], s'il y a un pb lors de l'application d'un pipeline, le
programme boucle...Voir myNotebook firlspg. 
\item[5/01/2008:] The test for the Static package does not work. Also, check the Static doc package.
Check also the doc package of StructuredScheduler. There is an error in the test. 
In analyze, there is a pb when the output of a system is produced by a subsystem (see
in doc package for StructuredScheduler). The test for Substitution is wrong. 
Le package de Uniformization ne marche pas... Il n'�tait pas install�, il faut tout revoir. 
\item[*31/12/2007:] I checked the Schedule package. I made corrections to scd so that it
can work on multidimensional cases... There is a bug in schematics, that you can 
see in the matmult example (after change of basis). See the nice example of test bench 
generation in matvect.
\item[29/12/2007:] the Shedule doc package has to be checked... Was not commited yet. 
\item[*29/12/2007:] the PipeControl doc package has been copied. Some options of report
are interesting and have to be documented. Committed.
\item[29/12/2007:] the Meta documentation (in doc/Meta and in doc/Packages/Meta) was
checked, updated and committed. The alpha2mma translator still works, but has to 
be completely debugged... The lyx file was translated and may now be forgotten... 
(I did not remember this lyx file, and I therefore was not able to find out this
documentation which I wrote a while ago...)
\item[*26/12/2007:] checking the repository. Committed various modifications in doc. 
Adding the Doc/Package directory. In the CheckAlpHard dir, there are some .simul files
which result, I think, from an attempt to write a simulator of Alpha in MMA. But I do not
know where the corresponding meta files are. In the Meta package doc, there is a 
lyx file which gives a precious documentation on the Meta package... Has to be translated...
\item[15/11/2007:] added functions isOutputRegular, areAllOutputsRegular, mkOutputRegular
and mkAllOutputsRegular to Substitution.m. Also changed getNewName and newName (in
schematics.m). Goal is to detect (in analyze) that outputs are not of the form o = v
where v is a variable and provide a function to change this. Has to commit Analyze, 
Static.m and Substitution.m. Also added MakeDoc.m in packages, and has to 
commit this. There are strange bugs, but documentation is OK. Has to commit TestsSubstitione.
There are bugs in tests of Static.
Commit change to VertexSchedule.m
\item[4/11/2007:] trying the Neural Network of Sanjay in myNotebooks, NN-Sanjay
BackPropNN (main.nb). Everything goes well until the ToAlpha0V2, where I added
some traces. Could not solve the problem. See notebook. 
\item[23/10/2007:] trying to install mmalpha at CSU. Did not work, because Polylib
was not adapted to PPC Mac... Still a problem. I recompiled Polylib (notice that the documentation
regarding Polylib is not up-to-date). Then, I had the problem that the Mathematica version
was wrong. Then, I had to recompile Domlib together with the other programs to get
a new library for Polylib... Eventually, I succeded. Morality : first, get the
result of compilation... Second, try to adapt the compiler to the PPC case... 
The right way to do this would be to generate different bin: bin.macosXppc and bin.macosXintel...
\item[7/9/2007 -- Pip:]�found the bug, it was in FarkasSchedule, the calls to 
sed. A corrected version of this package is in the desktop.
\item[29/8/2007 -- Pip:] There are problems regarding Pip on Windows, and I cannot 
find out what these problems are. Actually, there are also some problems with Domlib.
In the old version of Anne-Marie, Pip works, but tests["Domlib"] leads to a bug...
In the current version of Anne-Marie, Pip does not work: it returns an error message
saying that there is no solution (the same version works on the Mac). If I run the old
version with the new directory of Pip and Domlib, I also get an error in Pip, but
this time, the error is different as there is a call to DomMatrixSimplify which is
wrong... So there are some interferences between pip and Domlib. Actually, 
DomMatrixSimplify is used in Domlib, so I do not know where FarkasSchedule.m
calls it. Also, if
I run the new mmalpha with the old binaries, it does not work either... So I
put some Print statements in the FarkasSchedule.m of my Mac version and I 
will try to explore this with Anne-Marie... 
\item[24/7/2007 -- Domlib:] To try to correct the error in Polylib, I worked on 
Domlib in order to be able to recompile it. See in doc-install-domlib my notes
regarding Xcode etc. and all I could not fix. 

In the makefiles of Domlib, I made a few changes in order to recompile it. In 
particular, I set the place of Mathlink to the MacOSX-x86 version. 

Polylib works correctly, but not domlib. readDom works also nicely. So the 
problem seems to come from Mathlink, or maybe, domlib.c is not uptodate 
anymore... 
\item[20/7/2007 -- Polylib:] I again tried to look at Polylib and recompiling all stuff on 
Intel machine. 

First, I downloaded polylib from strasburg (see the note in doc/PolyLib). 
This was successful. 

Actually, it seems that to put a new version of PolyLib in MMAlpha, 
one just download it, then install it following the procedure described in the documentation
(see again the note), and then one recompile domlib etc. 

I did this in mmalpha-2 (after changing MMALPHA variable). 

Then, I made a copy of mmalpha into mmalpha-2. I changed the setenv of MMALPHA (otherwise
the makefile in the sources dir may not work). 

I edited the makefile in the sources dir. I changed the DIR variable to only Pretty. 
Then, I made make clean, which removes Obj.dir in Pretty, and I made a make all.
There are some errors, but it produces in Obj.darwin in Pretty a pretty binary file.

Then I tried the same for read\_Alpha. This does not work. It compiles properly read\_alpha.c, 
node.c, matrix.c, and vector.c but not polyhedron.c. 

Then, I replaced the Polylib directory in the sources by the new Polylib, and... it worked !

Then, I tried the same for Write\_Alpha, and it did not work... 

In the meantime, when putting in DIR of makefile Pretty + read\_ALpha + Write\_ALpha, 
I got a loop... 

Then I tried the same only for writeAlpha, and it worked ! Seems that make clean does not
work properly.

I tried again clean with all files in Dir and it worked... 

I tried the same for Write\_C and gen.o is missing... I tried to add gen.o from Codegen in 
Obj.darwin of Write\_C, but it did not work.

OK. I went back to Domlib. I had to modify several things. First, make sure that the new Polylib
was recompiled, and installed in directory MMALPHA/sources/Poly. (By the way, in mmalpha, 
I made a copy of bin.darwin renamed bin.darwin-ppc, because I know that this one works
on any mac. I also made a copy of Poly into Poly-darwin, just in case. With the indroduction
of intel mac, all this stuff has to made cleaner... 

In case of backup, there is always the solution of the dest-mmalpha-o, and also
the lip forge repository.

For Domlib to compile, changes have to be made in Makefile.darwin of Domlib (where is Polylib), 
Makefile of MakeIncludes (where the mprep is ). 



\item[20/7/2007 -- Domlib.m:] there is a bug (I guess) in \texttt{DomSimplify}. The
example is shown here:
\begin{verbatim}
d1 = readDom[  "{ i | i<=0}" ];
d2 = readDom[  "{ i | 0<=i}" ];
domSimplify[ d1, d2 ]
\end{verbatim}
will return domain $\{ i | 1>=0 \}$, i.e. the full space. 
\begin{verbatim}
d1 = readDom[  "{ i | i=0}" ];
d2 = readDom[  "{ i | 0<=i}" ];
domSimplify[ d1, d2 ]
\end{verbatim}
will return domain $\{ i | i=0 \}$, which is correct. When looking in Polylib, I discover
that there is another version of 

\item[*15/4/2007 -- All:] checked committements of lib.cygwin, lib.darwin, bin.cygwin, and
doc, until Schedule.
\item[*11/4/2007 -- All:] committed the config directory
\item[*10/4/2007 -- All:] committed all modifications to the distribution in lib. Added \texttt{Visual3D}
and \texttt{Schematics} in the distribution. 
\item[*9/4/2007 -- All:] checked that bin. are OK. 
\item[*8/4/2007 -- Alpha.m:] corrected to print use dependencies. This was not possible in the
MacOSX version.
\item[*7/4/2007 -- VertexSchedule.m:] Edition of some usages in VertexSchedule was done (parameterValues
for example).
\item[*6/4/2007 -- VertexSchedule:] was modified in order to comply with old version
of Mathematica.
\item[*11/12/2006 -- VertexSchedule:] has been modified for the new version of 
MMAlpha. Added options. A new function, changeIndexes, was added to 
changeOfBasis.
\item[*15/9/2006 -- appSched:] has been modified in order to change easily.
\item[9/9/2006: ] The 2D filter works, almost. There is a problem with appSched. In this example, 
when it is applied, it results in an infinite array. The solution is then to apply changes of bases
individually. I also put a check in Vhdl2 and vhdlModule.sem when a use has an 
infinite extension domain. This problem has been fixed. 
\item[5/9/2006:] Tried to work out a 2D filter given by Anne-Marie. See myNotebooks/filter2D. 
The problem was related to structured schedule. Remember that using schedule (Farkas 
scheduler) one needs to use structSched (or schedule[ subSystems->True ]. This is explained
in the schedule documentation. 
\item[25/5/2006:] Received a notebook of Anne-Marie (see myNotebooks/Chana) where she
tries to obtain vhdl code from a simple program. This does not work for several reasons, and
it could be possible to fix these problems. In particular, creating Vhdl code could be possible
by extension of the method to 0 dimensional variables. 
\item[*25/3/2006 -- PipeInfo:] Tanguy sent me a notebook, and PipeInfo does not work. Actually, 
there is a new version corrected in PipeControl-new.m. Removed in it the 
latex command and also, checked that no pipeline is done on a union of a domain.
Has to be commited.
\item[*1/3/2006 -- Schedule, FarkasSchedule, ScheduleTools:] I made a few modifications to \texttt{Schedule.m}, 
\texttt{FarkasSchedule.m}, and maybe also to \texttt{VertexSchedule.m} and
\texttt{ScheduleTools.m}. The modification was to unify the way schedules are
added to \texttt{\$scheduleLibrary}. I added to \texttt{Schedule.m} an internal
function \texttt{storeSchedule} of \texttt{VertexSchedule.m} which adds a
schedule to \texttt{\$scheduleLibrary} only by keeping the input and output
schedules. Thus, one may now schedule a subsystem using both \texttt{scd} or
\texttt{schedule} and still be able to use one or the other scheduler for the 
structured scheduline. 
\item[*15/2/2005 -- Schematics: ] while trying to use \texttt{schematics}, I noticed that the \texttt{Schematic.m}
package is not in the current distribution. OK, committed.
\item[2/12/2005 -- Scheduler] corrections done to the scheduler documentation in \verb!/doc/Schedule!
Added in the doc notebooks the Schedule directory. Not commited yet.
\item[1/12/2005:] created an example called SignAlpha.nb, for Anne-Marie
\item[30/11/2005 -- StructuredScheduled:] added StructuredScheduler in the demos. Is not totally checked yet, and
was not committed.
\item[30/11/2005 -- Demos: ] added Fifo in the demos, but it does not work completely... Not committed yet.
\end{description}

\section{The to do list}
\begin{description}
\item[10/1/2009:] make the corrections seen by the installation of Anne-Marie
\item[9/1/2009:] check that the scheduler is adapted to the new version with string additionnal constraints
and update documentation
\item[*4/1/2009:] update web site
\item[4/1/2009:] check that testSynthesis work, create and test new distribution
\item[2/1/2009:] remove documentation DOMLIB-Install
\item[2/1/2009:] there is a bus error in readAlpha in test4[]
\item[2/1/2009:] look in FarkasSchedule.m to see what happens with \$Runtime aso
\item[* 2/1/2009:] update installation documentation.
\item[2/1/2009:] filter more carefully the arguments of ashow and show. When called with 22 for example,
ashow tries show before falling, which causes a message to be issued in the shell
\item[2/1/2009:] repair \texttt{showLib} (prints out the library directly in the shell)
\item[* 2/1/2009:] clean \texttt{TestAlpha.m} from parasite shell error messages.
\item[* 2/1/2009:] remove ugly messages issued in the shell window when trying to remove files from 
PiP. Probably have to redirect stderr to another file?
\item poursuivre l'analyse de top\_fir pour enlever la fifo et aussi regarder l'affaire de
l'initialisation
\item comprendre pourquoi YOutType n'appara�t nulle part
\item demander � Steven si le fonctionnement des m�moires est correcte
\item peut-on mettre un type en g�n�ric. Voir tutorial...
\item If you apply \texttt{toAlpha0v2} to a program that has not been appscheded, it 
fails...
\item There is a message in the Write\_Alpha program, "Could not find the declaration..." that should be removed, as it pollutes the main window...
\item In the config directory, clean red files of the LipForge repository
\item In the lib directory, look at the files which are not in the distribution and add those
which are useful. They are shown in red in the \texttt{Description.xls} file in 
the \texttt{maintenance} directory.
\item By the way, I noticed that \texttt{structSched} does not
work for multi-dimensional schedules, and that would be great to have it. 
\item Add a variable \$pipeInfo that contains the result of the PipeInfo command.
\item Commit \texttt{Schedule.m} etc. (see above).
\item Add \texttt{Schematics.m} to the distribution.
\item Commit new document packages
\item Complete the description of appendices in this document using the README
file in this directory
\item Check demo FIFO
\item Add bibliography to scheduler documentation and add documentation for the
scd command
\item Check the doc package directory for the vertex scheduler
\item Create a doc package directory for the farkas scheduler
\item Extend the vhdl generator to the case of non parallel programs (see myNotebooks/Chana)
\end{description}

\section{The done list}
\begin{description}
\item[2/1/2009:] clean \texttt{TestSubstitution.m} which creates parasite symbols.
\item[-2/1/2009:] Remove \texttt{\$NewName} in Alpha.m. Done.
\item[-1/1/2009:] SystemProgramming is missing in CopyMMA. Some tests are missing in CopyMMA. Could
not understand why CopyDirectory does not work uniquely for TestSchedule ! Done anyway.
\end{description}

\appendix
\section{How to use LipForge}
This version of \mmalfa{} comes from the 
LipForge repository. To use if, the following operations are needed:
\begin{itemize}
\item Open a terminal window.
\item Export the environment variable \verb\CVS\\_RSH\ using
the \texttt{bash} command:
\begin{verbatim}
export CVS_RSH=ssh
\end{verbatim}
or the \texttt{csh} command:
\begin{verbatim}
setenv CVS_RSH ssh
\end{verbatim}
\item Each time you want to apply a \texttt{cvs} command, say 
\texttt{ccc}, use
\begin{verbatim}
cvs -z3 -d:ext:quinton@lipforge.ens-lyon.fr:/local/chroot\
/cvsroot/irisa ccc
\end{verbatim}
Alternatively, save the first part of the command (before \texttt{ccc}) as
an alias
\begin{verbatim}
alias cvs='cvs -z3 -d:ext:quinton@lipforge.ens-lyon.fr:/local\
/chroot/cvsroot/irisa'
\end{verbatim}
and then use the command
\begin{verbatim}
cvs ccc
\end{verbatim}
\end{itemize}
\newcommand{\macosx}{{\sc MacOSX}}

\section{How to connect to my files at Irisa}
In the Finder, select a server. 
\begin{verbatim}
smb://nas1a.irisa.fr/quinton
\end{verbatim}

\section{How to run \mmalfa{} on another server}
On a Linux server
\begin{verbatim}
ssh sabre.irisa.fr
\end{verbatim}

\section{How to access to the web server of MMAlpha}
In the Finder, select a server. 
\begin{verbatim}
http://webdav.irisa/www.irisa.fr/htdocs/cosi
\end{verbatim}
On my laptop, the old site is in
\begin{verbatim}
Copy-MMALPHA-Site
\end{verbatim}

\section{How to run MMA on Mac OS X}
For unknown reasons, it is not possible to start \mmalfa{} directly
from the normal interface of \macosx{}. The solution:
\begin{enumerate}
\item Open a terminal window.
\item Make sure that an environment variable \texttt{MMALPHA}�is 
set to the directory where the \mma{} distribution is placed. For example
\begin{verbatim}
export MMALPHA=/Users/quinton/MMAlpha
\end{verbatim}
or under \texttt{csh}:
\begin{verbatim}
setenv MMALPHA /Users/quinton/MMAlpha
\end{verbatim}
\item Then run \mma{}. The \mma{} software is placed 
in directory 
\begin{verbatim}
/Applications/Mathematica 5.1.app/Contents/MacOS
\end{verbatim}
(this may differ depending on the version number.)
\item Normally, \mma{} when started should call a
\texttt{init.m} file, but for unknown reasons, this is not so...
\end{enumerate}
Additional comments:
\begin{itemize}
\item Add a command to \texttt{.alias} in order to be able
to call \mma{} directly, such as for example:
\begin{verbatim}
alias mma /Applications/Mathematica 5.1.app/Contents/MacOS/Mathematica
\end{verbatim}
\item Add a command in the .cshrc file, in order to 
\end{itemize}
\begin{verbatim}
setenv MMALPHA /Users/quinton/mmalpha
echo 'This is my MMA environment: ' ${MMALPHA}
#   For Mac OS X
alias mma '/Applications/Mathematica\ 5.1.app/Contents/MacOS/Mathematica'
alias cvsmma 'cvs -z3 -d:ext:quinton@lipforge.ens-lyon.fr:/local/chroot/cvsroot/irisa'
set path=($MMALPHA/bin.$OSTYPE $path)
set path=(/usr/local/bin $path)
\end{verbatim}
\end{document}
