\documentclass[11pt]{article}
\usepackage{fullpage,makeidx,fancyvrb,graphicx}
\usepackage{moreverb}

\begin{document}

\newcommand{\mmalfa}{{\sc MMAlpha}}
\newcommand{\mma}{{\sc mathematica}}
\newcommand{\alfa}{{\sc Alpha}}
\newcommand{\vhdl}{{\sc vhdl}}
\newcommand{\alfazero}{{\sc alpha0}}
\title{The \mmalfa{} Test Bench\thanks{Version printed on \today}}
\author{Patrice Quinton}
\date{February 24, 2010}
\maketitle
\begin{abstract}
We describe a program to generate test bench automatically for 
\mmalfa{} programs. We also describe how to generate stimuli files, and
how to generate a bit true simulator. 
\end{abstract}

\section{Introduction}
In Section~\ref{tb}, we present how to generate a test bench.
In Section~\ref{stimuli}, we present how to generate stimuli files. 
In Section~\ref{bittrue}, we explain how to simulate a program using 
a bit true simulation.

\section{Test Bench Generation}
\label{tb}
The \texttt{VhdlTestBench} package contains a function that allows a \vhdl{} test bench
to be automatically generated for an \alfa{} program. This program must be in \textsc{Alpha0}
format, such that obtained after a successful utilization of the \texttt{a2v} command. This
means that the current program (contained in the library variable, \texttt{\$library}), has been translated
into \vhdl{}. 

The function is called \texttt{vhdlTestBenchGen}.
The test bench is produced in a file called \texttt{fileTB.vhd}, where \texttt{file} is
the name of the system currently contained in the \texttt{sys} parameter of the
\texttt{vhdlTestBenchGen} function\footnote{See the usage of this function.}.

The test bench file is written in the current directory of \textsc{Mathematica}. If such
a file already exists, it is copied in a temporary file named
\texttt{file\_.TB.vhd} (a previous version of such a file is deleted, if any exists).

The test-bench that is produced must be used with stimuli files for each
of the input variables of the \alfa{} program; it will produce one output file for
each output variable of the \alfa{}�program. How to produce these stimuli files
is explained in Section~\ref{stimuli} of this documentation. 

The \texttt{vhdlTestBenchGen} function allows test benches to be generated for each one of the 
elements of the library. Recall that, after generating an \alfazero{}
program, all modules produced are contained in an \mmalfa{} 
variable called \texttt{\$library}. The current program is in
variable \texttt{\$result}, and it is usually the main program. Say this 
program is \texttt{fir} (corresponding to an initial \texttt{fir.alpha} \alfa{}
program), then executing
\begin{verbatim}
vhdlTestBenchGen[ ]
\end{verbatim}
will produced a test-bench named \texttt{firTB.vhd}. It is essential that 
the content of \texttt{\$library} be consistent with the program for which the
test bench is generated. 

The \texttt{vhdlTestBenchGen} function checks various possible errors:
\begin{itemize}
\item The given program must be recognized as a cell, a module or a controller
by \mmalfa{} (using the \texttt{checkCell},  \texttt{checkModule}, or
 \texttt{checkController} functions.
 \item The name of the given program must appear in the library contained 
 in \texttt{\$library}.
 \item 
\end{itemize}

More can be found on this topic


\section{Technicalities}
\label{technicalities}
\begin{itemize}
\item
There is a provision for generating a \vhdl{} file with the 1987 \vhdl{}
syntax (see the \texttt{s87} flag, directly in the code). 
\item
The clock rate is 20\,ns, and the clock initialization time is 30\,ns. 
\item
If there is already a \texttt{component} file for the program contained
in \texttt{\$result} (say for example, \texttt{fir.component}) in the directory, 
this component is used in the test bench, otherwise, a component is built
on the fly (using the \texttt{buildComponent} function). Therefore, if there 
exists an inconsistency between the program and its component, an 
error may happen. 
\end{itemize}

\section{Meaning of the signals}
Here is a description of some of the signals in the test bench produced. 
\begin{itemize}
\item \texttt{clk\_rate}: 
\item \texttt{clk\_init}: 
\item \texttt{sig\_Name}: a signal associated to \alfa{} variable \texttt{Name};
\item \texttt{comp}: the name of the instanciated component;
\item \texttt{ce}: clock enable signal;
\item \texttt{rst\_0}: initial value of the reset signal;
\item \texttt{temp\_buffer}: 
\item \texttt{temp\_buffer\_out}: 
\item \texttt{endstim}: a boolean variable that indicates the end of the stimulus file
\item \texttt{good}: a boolean variable used to check the reading of a stimulus file
\item \texttt{temp\_buffer\_Name}: a temporary buffer for \alfa{} variable \texttt{Name};
\item \texttt{stim\_file\_Name}: the declaration of the stimulus file for variable \texttt{Name}:
\item \texttt{stim\_line\_Name}: the declaration of the file line for variable \texttt{Name}:
\item \texttt{timecounter}: time counter for the design;
\end{itemize}

\section{Producing Stimuli}
\label{stimuli}
As already said, there exists a stimuli file for each input variable. 
If \texttt{inVar} is the name of an input variable of the \alfa{}�program, then its
stimuli file is named \texttt{stimInVar.txt}. This file contains one value for each instant
of time for which a value is required by the program. 

Recall that, for any \alfa{} variable, its {\em life time} is given by projecting its domain 
on the \texttt{t} index. Say this life interval is $10 \leq \texttt{t} \leq 20$, then
values will be required only for time instants between 10 and 20, therefore, the 
stimuli file contains 11 values. 

Producing automatically stimuli files can be done either manually, or by using 
the \texttt{cGen} program, as described here.

To better explain this program, assume we want to produce stimuli files for
the following program \texttt{fir}: 

\VerbatimInput{fir.alpha}

We load this file, and we schedule it:
\begin{verbatim}
load[ "file.alpha" ];
scd[]
\end{verbatim}
Then, we run the following program

\begin{verbatim}
cGen[ "file.c", { "K" -> 3, "N" -> 10 }, stimuli -> True  ]
\end{verbatim}

This produces in file \texttt{file.c} a \texttt{C} program that simulates the 
operations of the \alfa{} program. Compiling this program :
\begin{verbatim}
gcc file.c -o fir
\end{verbatim}
produces a binary file \texttt{fir} that we can run 
\begin{verbatim}
./fr
\end{verbatim}
The effect of this execution is to ask the values of the input variables and to return, 
eventually, the values of the output variables. Here is what you see at the 
console: 
\begin{verbatim}
x[0]?0
x[1]?1
x[2]?2
x[3]?3
x[4]?4
x[5]?5
x[6]?6
x[7]?7
x[8]?8
x[9]?9
x[10]?10
w[1]?1
w[2]?0
w[3]?0
y[3]=2
y[4]=3
y[5]=4
y[6]=5
y[7]=6
y[8]=7
y[9]=8
y[10]=9
\end{verbatim}
The input chosen are \texttt{x[i] = i}, and \texttt{w[0] = 1}, 
\texttt{w[1] = w[2] = 0}, which gives the result shown. 

As a side effect, this evaluation produces also 3 files: 
input files \texttt{stim\_x.txt} and \texttt{stim\_w.txt} where input values for
x and w are recorded, and an output file \texttt{stim\_y.txt}.
Here is the \texttt{stim\_x.txt} file: 
\VerbatimInput{stim_x.txt}
Here is the \texttt{stim\_w.txt} file: 
\VerbatimInput{stim_w.txt}
Here is the \texttt{stim\_y.txt} file: 
\VerbatimInput{stim_y.txt}

The \texttt{cGen} function has several options. The \texttt{noPrint} option, when set to
\texttt{True}, does not prints out the name of the inputs and outputs values (for example,
\texttt{x[1]=00000001}, but only the result. The resulting files may then be used to
be run with the test bench. 

This example is presented in the \texttt{Fir} notebook in \texttt{mmalpha/demos/NOTEBOOKS/Fir}.
\section{Bit True Simulation}
\label{bittrue}
Documentation to be done.
\newcommand{\stim}{\texttt{alpHardStim}}
\section{Generation of test files}

\label{stim}
The \texttt{VhdlTestBench} package contains now a new function, called \stim{}, which
allows stimuli files to be generated for the \vhdl{} test bench. For each input \texttt{in}, 
it produces an input file called \texttt{alpHardStimin.txt}, and similarly for each 
output. This file contains, in the order given by the time variable, the command to
be executed to generate...

\section{The \texttt{alpha2mma} Package}
\label{translator}
There exists a prototype program that allows an \alfa{} program to be 
translated into \mma{}. This package is called \texttt{alpha2mma.m}. It is currently
available in the directory
\begin{verbatim}
$MMALPHA/doc/Packages/Meta
\end{verbatim}
as it is an example of using the \texttt{Meta} package. 

The \texttt{alpha2mma.m} program is itself produced by the use of
the \texttt{meta} command, applied to a description of the \alfa{} syntax
given in the meta file \texttt{alpha2mma.meta}.
See in the notebook \texttt{meta.nb} located in the same directory how to
call the translator.

Once the \texttt{alpha2mma.m} file has been produced, one loads it by
the command 
\begin{verbatim}
<<alpha2mma.m
\end{verbatim}

Then to translate a program, say \texttt{test1.alpha} into a \mma{} program, one
does:
\begin{verbatim}
load[ "test1.alpha" ];
alpha2mma[]
\end{verbatim}
(there exist a \texttt{debug} option, as usual).

This produces in the current directory a file called \texttt{test1.simul}, which contains
definitions for the input, local and output variables. 

Then, one loads this file:
\begin{verbatim}
<<test1.simul
\end{verbatim}
and one calls the simul function :
\begin{verbatim}
simul["test1"]
\end{verbatim}

This function has the only effect of entering the context
\texttt{Alpha`Simul`test1`} in which all definitions of the variables
are available, as functions. 

To be more explicit, assume that the \texttt{test1.alpha} file looks
as follows:
\begin{verbatim}
system test1   (a :  integer; b :  integer)
returns (c :  integer);
let
  c = a + b;
tel;
\end{verbatim}

This program is translated into the following \mma{} program:
\begin{verbatim}
Begin["Alpha`Simul`test1`"];
Clear[ Alpha`Simul`test1`a, Alpha`Simul`test1`b, Alpha`Simul`test1`c];

    (* Definition of c *)
    Alpha`Simul`test1`c[f:(_Integer|_Symbol)...] := Plus[ a[f], b[f] ];
    Alpha`Simul`test1`c[ ___ ] := Message[Alpha`Simul`test1`c::params ];
End[];
\end{verbatim}

As you can see, this program enters the special context
\texttt{Alpha`Simul`test1`}, then clears the variables \texttt{a},
\texttt{b} and \texttt{c} in this context, and defines a \texttt{c} function.
The form of this function is quite special. It says basically that 
\texttt{c[ f ] := a[ f ] + b[ f ]}, where \texttt{f} is any sequence of integers or
symbols. This allows the function to be used for any sequence
of indices, a property that will be useful later on when one wants to
evaluate a structured \alfa{} program. 

When one runs \texttt{simul[ "test1" ]}, one enters the (unique) simulation context
of \texttt{test1}, where the definition of \texttt{c} are available. In this context, 
if we just evaluate \texttt{c[]}, we get the result \texttt{a[] + b[]}; in the
same manned, we can evaluate any \texttt{c[ i1, i2, ..., ik ]} expression, 
where the \texttt{ik} are either integers or symbols of \mma{}, to 
get the result 
\texttt{a[ i1, i2, ... , ik ] + b[  i1, i2, ... , ik]}.

Consider now a more involved example, where one wants to 
build a binary adder in \alfa{}. The program is given in Fig.~\ref{figadder}.
\begin{figure}[htbp]
\begin{verbatim}
system FullAdder (A :  boolean;
                  B :  boolean;
                  Cin :  boolean)
       returns   (X :  boolean;
                  Cout :  boolean);
let
  X[] = A xor B xor Cin;
  Cout[] = A and B or A and Cin or B and Cin;
tel;

system Plus :  {W | 2<=W}
               (A : {b | 0<=b<=W} of boolean;
                B : {b | 0<=b<=W} of boolean)
       returns (S : {b | 0<=b<=W} of boolean);
var
  Cin : {b | 0<=b<=W-1} of boolean;
  Cout : {b | 0<=b<=W-1} of boolean;
  X : {b | 0<=b<=W-1} of boolean;
let
  use {b | 0<=b<=W-1} FullAdder[] (A, B, Cin) returns  (X, Cout) ;
  Cin[b] =
      case
        { | b=0} : False[];
        { | 1<=b} : Cout[b-1];
      esac;
  S[b] =
      case
        { | b<=W-1} : X[b];
        { | b=W} : Cout[W-1];
      esac;
tel;
\end{verbatim}
\caption{The binary Plus program\label{figadder}}
\end{figure}

It contains two parts: a full adder (\texttt{FullAdder} program), and 
a \texttt{Plus} program, that allows one to build a binary adder of 
\texttt{W} bits by instanciating \texttt{W} times the full adder. 

Translating this program into \mma{} involves (currently) two steps. First, 
one translate the \texttt{FullAdder} program. To this end, assuming
that the program is in the file \texttt{fulladder.alpha}, one loads this program, 
one select the \texttt{FullAdder} subsystem, and one translates it :

\begin{verbatim}
load[ "fulladder.alpha" ];
getSystem[ "FullAdder" ];
alpha2mma[];
\end{verbatim}

One does the same for the main program, \texttt{Plus}
\begin{verbatim}
getSystem[ "Plus" ];
alpha2mma[];
\end{verbatim}

Then, one loads both programs :
\begin{verbatim}
<<"Plus.simul"
<<"FullAdder.simul"
\end{verbatim}
which creates definitions for the variables 

\newpage
\appendix
\section{The Test Bench for the Fir Filter}
\label{tbfile}
\VerbatimInput{fir_TB.vhd}

\newpage
\section{The Stimuli Generator for the Fir Filter}
\label{stimfile}
This program was generated with the default value of the \texttt{noPrint} option.
\VerbatimInput{fir.c}

\end{document}
