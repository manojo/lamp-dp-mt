\documentclass[11pt]{article}

\usepackage{makeidx,fancyvrb,graphicx}

\begin{document}

\newcommand{\vhdl}{\textsc{vhdl}}
\title{Description of the \vhdl{} Library}
\author{}
\date{}
\maketitle

\newcommand{\rom}{{\texttt{ROM}}}
\section{The \rom{} Component}

\subsection{Component Description}
The \rom{} component is a combinatorial read-only
memory. It returns a data, as a signed number with
a specified length, specified by an integer address.

\subsection{Generation Process}
Use \texttt{genVhdl[ "ROM" ]} function in package \texttt{Vhdl2.m}.

\subsection{Files}
\begin{description}
\item[\texttt{ROM.vhdx}:] the architecture description of the ROM (see \ref{romarchi}).
\item[\texttt{ComponentOfROM.vhdx}:] the component description of the ROM (see \ref{romcomp}).
\end{description}

\newcommand{\sizeparam}{\texttt{\$size\$}}
\newcommand{\commentparam}{\texttt{\$comment\$}}
\newcommand{\wlmunparam}{\texttt{\$wordLengthMun\$}}
\newcommand{\nameparam}{\texttt{\$name\$}}
\newcommand{\valuesparam}{\texttt{\$values\$}}

\subsection{Parameters}
\begin{description}
\item[\sizeparam{}: ] the size of the read-only memory, in words.
\item[\commentparam{}: ] a comment.
\item[\valuesparam{}: ] the values associated to the ROM. This parameter is not
mandatory, and if not supplied, 0 values are assumed.
\item[\nameparam{}: ] the name assigned to the component.
\item[\wlmunparam{}: ] the word length (minus one) of the values in the ROM. 
\end{description}

\subsection{Example}

\begin{verbatim}
--
-- Component for a combinatoric ROM
-- ROM of size 32 and of word length 2
--

COMPONENT ROM3 IS
  PORT 
  (
    address : IN INTEGER RANGE 0 TO 31;
    data : OUT SIGNED (1 DOWNTO 0)
  );
END COMPONENT;
\end{verbatim}

\newcommand{\moduloaddress}{\texttt{moduloAddress}}
\section{The \moduloaddress{} Component}

\subsection{Component Description}
The \moduloaddress{} component is a component that generates...

\subsection{Generation Process}
Use \texttt{genVhdl[ "moduloaddress" ]} function in package \texttt{Vhdl2.m}.

\subsection{Files}
\begin{description}
\item[\texttt{ModuloAddress.vhdx}:] the architecture description of the modulo address generator (see \ref{romarchi}).
\item[\texttt{ComponentOfModuloAddress.vhdx}:] the component description of the modulo address generator (see \ref{romcomp}).
\end{description}

\newcommand{\periodparam}{\texttt{\$period\$}}
\newcommand{\periodmunparam}{\texttt{\$periodMun\$}}

\subsection{Parameters}
\begin{description}
\item[\periodparam{}: ] the period of the modulo generator.
\item[\commentparam{}: ] a comment.
\item[\periodmunparam{}: ] perion minus one. This is generated automatically by the
VHDL generator.
\item[\nameparam{}: ] the name assigned to the component.
\end{description}

\newpage
\appendix

\section{The ROM}
\subsection{Architecture}
\label{romarchi}
\VerbatimInput{../../VHDL/ROM.vhdx}

\subsection{Component}
\label{romcomp}
\VerbatimInput{../../VHDL/ComponentOfROM.vhdx}

\newpage
\section{The Modulo Address Generator}
\subsection{Architecture}
\label{modumloaddressarchi}
\VerbatimInput{../../VHDL/ModuloAddress.vhdx}

\subsection{Component}
\label{modulocompcomp}
\VerbatimInput{../../VHDL/ComponentOfModuloAddress.vhdx}


\newpage
\section{The Registers}
\subsection{Architecture}
\label{romarchi}
\VerbatimInput{../../VHDL/Registers.vhdx}

\subsection{Component}
\label{romcomp}
\VerbatimInput{../../VHDL/ComponentOfRegisters.vhdx}


\newpage
\section{The FSM component}
\subsection{Architecture}
\label{romarchi}
\VerbatimInput{../../VHDL/Fsm.vhdx}

\subsection{Component}
\label{romcomp}
\VerbatimInput{../../VHDL/ComponentOfFsm.vhdx}

\subsection{Call}
\label{romarchi}
\VerbatimInput{../../VHDL/CallFsm.vhdx}


\newpage
\section{The Call reg component}
\subsection{Architecture}
\label{romarchi}
\VerbatimInput{../../VHDL/CallReg.vhdx}

\section{The Call registers component}
\subsection{Architecture}
\label{romarchi}
\VerbatimInput{../../VHDL/CallRegisters.vhdx}

\newpage
\section{The PeriodicEnable component}
\subsection{Architecture}
\label{romarchi}
\VerbatimInput{../../VHDL/PeriodicEnable.vhdx}

\subsection{Component}
\label{romcomp}
\VerbatimInput{../../VHDL/ComponentOfPeriodicEnable.vhdx}

\subsection{Call}
\label{romarchi}
\VerbatimInput{../../VHDL/CallPeriodicEnable.vhdx}







\end{document}
