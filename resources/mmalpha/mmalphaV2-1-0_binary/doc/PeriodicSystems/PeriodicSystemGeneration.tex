\documentclass[11pt]{article}
\begin{document}

\newcommand{\mmalpha}{{\sc Mmalpha}}
\newcommand{\vhdl}{{\sc vhdl}}
\newcommand{\alfa}{{\sc alpha}}

\title{Periodic System Generation in \mmalpha{}}
\author{Anne-Marie Chana \and Patrice Quinton}
\date{Version of \today}
\maketitle

\tableofcontents

\section*{To do list}
\begin{itemize}
\item Separate the types in a separate component
\item Assign values to the coefficients of the filter
\item Infer le type de ovsfaddress etc... from that of the ROM
\item Before step 1, check that the system is well-formed
\item Step 2 assumes that there are no registers in the code, so check this...
\item During Step 3, we assume that the delay of a component is given by the schedule of
its first input, which is not necessarily the case... Check this and make an extension.
\end{itemize}


\section{Introduction}
This document presents the current state of development
of an extension of \mmalpha{} to generate multi-dimensional 
periodic systems. The theory underlying this work is described in~\cite{quinton-asap04}. 

\section{The \texttt{synPeriodic} function}
This function is located in the \texttt{Synthesis.m} package. Its goal is to analyze a 
data-flow periodic systems and to find out the periods of each one of its components. 
It also generates \vhdl{} files for such a system.

Recall that such a system is a special case of \alfa{} system where all variables
are indexed with an infinite first dimension. This means that all variables of such
a system have a domain of the form \texttt{$\{ i, j, k ... | 0 \leq i , ... \}$}. The first
index, $i$, is considered as a time index.
It also uses subsystems, which have all 
the property that all their variables are also
indexed with a first infinite dimension.

Now the time dimension of each variable may be {\em stretched} or 
{\em contracted} depending on a fixed period. This means that for
some variables, the first index $i$ is in fact an affine function $a t + b$ of
an underlying, commun time clock, $t$. For another variable, this 
affine function might be of the form $c t + d$, where $c$ and $d$
are different from $a$ and $b$.

Some particular subsystems are oversamplers and undersamplers. They have
the property of changing the rate of the time dimension, i.e., the values of 
the periods $a$ or $c$ of the variables.

All other subsystems have the property of being {\em monochronous}: all
their variables have the same period. We expect these subsystems to represent
hardware components that may be run at various periods, depending on the 
way they are instanciated (we shall return to this point later on.)

In~\cite{quinton-asap04}, it is shown how one can compute the value of 
the periods of each individual signal or component of such a periodic
system, and from there, if a solution to this problem exists, to schedule in
detail the full system. 


The \texttt{synPeriodic} function analyzes the graph of the \mmalpha{} data-flow system, 
and from its dependencies, it groups signals that have the same period (returned as 
list of list). Each one of the groups has un period, say $\lambda_i$, which is calculated
by this function. It then schedules the full system, and generates the \vhdl{} components
of it. We describe hereafter how this is done.


\section{Generation of \vhdl{} Programs}
\subsection{Generalities}
The generation of \vhdl{} programs is implemented as the 
\texttt{synPeriodic} \mmalpha{} function
of the \texttt{Synthesis.m} package. 

The parameters of this function are not fully determined yet. 
\begin{verbatim}
synPeriodic[]
\end{verbatim}
generates a \vhdl{} file in the directory \texttt{\$MMALPHA/VHDL} and provides some
other results.

This function makes use of the \texttt{genVhdl} function, included in the 
\texttt{Vhdl2.m} package, and that we describe in appendix~\ref{genvhdl}.

Its operations are currently as follows:
\begin{description}
\item[Step 1:] Built the graph of periods, and solve the balance equations of this graph.
\item[Step 2:] Schedule the whole system, by injecting the value of the periods in the schedule
patterns of each component.
\item[Step 3:] Generate the delayed enable signals.
\item[Step 4:] Generate the enable signals.
\item[Step 5:] Generate the FSM that controls the periodic system.
\item[Step 6:] Generate the \vhdl{} code for the called systems.
\item[Step 7:] Generate the \vhdl{} code for the calling system.
\end{description}

We consider successively each one of these steps.

\subsection{How to Write a Periodic System}
Recall that all signals that are manipulated by periodic systems have a domain
whose first index is an infinite half-line. The other indexe can be anything 
(provided this matches the type of the subsystems). 

A periodic system is made out of either:
\begin{itemize}
\item Connexions between periodic signals.
\begin{verbatim}
A = B;
\end{verbatim}
\item Combinatorial elements.
\begin{verbatim}
A = B + (C * D);
\end{verbatim}
\item Over-samplers.
\begin{verbatim}
use overSampling[6] (A) returns (B);
\end{verbatim}
The number inside brackets is the over-sampling factor. It is a fixed integer. 
Variable \texttt{A} is the input, and variable \texttt{B} is the output. 
\item Under-samplers.
\begin{verbatim}
use underSampling[6] (A) returns (B);
\end{verbatim}
Same remark as for the over-samplers.  
\item Calls to monochronous subsystems, among which, ROM and address generators.
\begin{verbatim}
use subsystem[ pm ] (A, B, etc.) returns (X, Y, etc.)
\end{verbatim}
\end{itemize}

\subsection{Step 1: Solve the Balance Equations}
Each variable and each component are assigned a period which is computed
by solving the so-called {\em balance equations}. The rules that are applied are
the following ones:
\begin{itemize}
\item Connected signals have the same period.
\item Inputs and outputs of a subsystem have the same period (subsystems are
assumed to be monochronous).
\item The period of the output of an over-sampler is equal to \texttt{x} times
that of its input, where \texttt{x} is the sampling factor.
\item The opposite is true for an under-sampler.
\end{itemize}
The algorithm consists in finding out the connected 
components of the dependence graph (whose elements have necessarily the same
period), and solve the balance equations. This returns one or several solutions.
Currently, only the case of one single solution is handled properly.

\subsection{Step 2: Schedule the System}
We assume that each subsystem has a parameterized schedule, which is put
in the schedule library. Building a parameterized schedule is explained in 
Appendix~\ref{paramschedule}. 

Then, step 2 consists in scheduling the whole system. 

\appendix
\section{The \texttt{genVhdl} Function}
\label{genvhdl}
The \texttt{genVhdl} functions allows one to generate component parts and 
architecture parts of several elements needed for the generation of 
periodic systems. Currently, this function allows the generation of:
\begin{itemize}
\item read-only memories,
\item finite-state machines,
\item periodic enable signals.
\end{itemize}

The principle of the generation is in all cases the same: it is made from \texttt{vhdl} pattern files
available in \texttt{\$MMALPHA/VHDL}. Place-holder are filled by the parameters given to 
\texttt{genVhdl}, depending on the type of \vhdl{} program to be generated. Place-holders
are character strings of the form \texttt{\$string\$}.

\section{Read-only Memories}
\label{gen-Rom}
Although this \vhdl{} element is not specific to periodic systems, we describe here the
use of \texttt{genVhdl} to generate a read-only memory.
\begin{verbatim}
 genVhdl["ROM" , "$wordLength$" -> "5",
             "$name$" -> "kasami", "$size$" -> "7",
             "$comment$" -> "Memory for Kasami coefficients",
             "$values$" -> "{1,1,0,0,1,8,2}"];
\end{verbatim}
The first parameter, \texttt{"ROM"}, identifies the type of block generated. Other parameters
are given as string replacement rules. The second parameter, gives the number of bits
of the words of the memory (it is mandatory). The third (mandatory) parameter is the name
given to the element. The fourth (mandatory) parameter is the number of words of 
the memory. A comment place-holder allows one to add a comment to 
the produced program. 
Finally, the last parameter is the values assigned to each word of the memory.
Its number should be equal to the size. There exist another place-holder in the 
ROM, namely, to define the address size, but it is computed automatically by the generator. 

String rewriting rules can be given in any order. 

This command, when executed, returns a list of 2 strings: the first one is the 
component corresponding to the ROM, and the second one is its architecture. 

\begin{figure}[htbp]
\begin{verbatim}
COMPONENT kasami IS
  PORT 
  (
    address : IN  STD_LOGIC_VECTOR(3 DOWNTO 0);
    data : OUT STD_LOGIC_VECTOR(5 DOWNTO 0)
  );
END COMPONENT;
\end{verbatim}
\caption{Component of a ROM, as generated by the \texttt{genVhdl} 
command of~\ref{gen-Rom}.\label{figRom}}
\end{figure}

Figure~\ref{figRom} gives an example of memory generated. 

\section{Periodic enable signals}
\label{gen-enable}
In periodic systems, each component is driven by a \texttt{clock-enable} signal
that controls its execution. The \texttt{genVhdl} function allows one to produce
a \vhdl{}�component that generates a periodic signal that can be used to 
control such components. 
\begin{verbatim}
genVhdl[ "PeriodicEnable" ,  "$name$" -> "genEnable10",
     "$period$" -> "10",
     "$comment$" -> "Periodic enable generator with period 7"];
\end{verbatim}
The first component identifies the type of block produced. The second one, 
gives the name of the generated component. The third one gives the period
of the corresponding clock. Finally, the last one gives a comment. 

Fig.~\ref{enable} shows an example of component for a 10 period enable.
Notice that this component has 3 inputs: the basic clock of the system, \texttt{clk}, 
a reset signal \texttt{rst}, and a {\em global enable} signal, \texttt{ceGen}. 
This signal, when false, has the effect of {\em freezing} the generation of 
the periodic enable during at least one \texttt{clk} clock cycle. 
\begin{figure}[htbp]
\begin{verbatim}
--
-- Component for a periodic enable of period 10
-- 

COMPONENT genEnable10 IS  
  PORT(
    clk : IN STD_LOGIC;
    rst : IN STD_LOGIC;
    ceGen : IN STD_LOGIC;
    periodicGe : OUT STD_LOGIC
  );
END COMPONENT;

\end{verbatim}
\caption{Component of a periodic enable signal generator, as produced by the
\texttt{genVhdl} command in~\ref{gen-enable}.\label{enable}}
\end{figure}

\section{Finite-state machine}
\label{gen-fsm}
The third element that the \texttt{genVhdl} function allows one to generate is 
a finite-state machine. Fig.~\ref{command-fsm} displays the form of the
\texttt{genVhdl} command to produce a finite-state machine. The first three 
parameters are position parameters: the first one identifies the type of block
generated, the second one gives the list (of strings) of the outputs of the 
fsm. The type of these outputs are assumed to be \texttt{STD\_LOGIC}. The 
third parameter describes the fsm. It is a list, the element of which are made
of an integer and a string. The integer gives the time when a given action
happens, and the string presents this action, in \texttt{vhdl} code. Usually, this
action involves setting values to the outputs. 

The first element of this list should always be \texttt{\{ 0, "action" \}}, as
the initial state of the fsm is always named \texttt{state0}. 
\begin{figure}[t]
\begin{verbatim}
genVhdl[ "Fsm" ,
     {"fifo1_Out","fifo2_Out"},
     {
      {0,"BEGIN fifo1_Out <= '0'; fifo2_Out <= '1' END"},
      {4,"BEGIN fifo1_Out <= '1'; fifo2_Out <= '0' END"},
      {7,"BEGIN fifo1_Out <= '0'; fifo2_Out <= '1' END"},
      {22,"BEGIN fifo1_Out <= '1'; fifo2_Out <= '0' END"}
     },
     "$name$" -> "myFsm",
     "$comment$" -> "This is a Fsm"];
\end{verbatim}             
\caption{Command for generating a finite-state machine.\label{command-fsm}}
\end{figure}             

Fig.~\ref{fsm} displays the component generated for a finite-state machine.
Notice that this fsm is controlled by a reset signal \texttt{rst} and a general
clock-enable signal \texttt{CE\_gen}. 
\begin{figure}[htbp]
\begin{verbatim}
--                                                                                 
-- This is a Fsm                                                                   
--                                                                                 
COMPONENT myFsm IS
  PORT(
    clk : IN STD_LOGIC;
    rst : IN STD_LOGIC;
    CE_gen : IN STD_LOGIC;
    fifo1_Out : OUT STD_LOGIC;
    fifo2_Out : OUT STD_LOGIC
  );
END myFsm;
\end{verbatim}
\caption{Component of a periodic enable signal generator, as produced by the
\texttt{genVhdl} command in Fig.~\ref{command-fsm}.\label{fsm}}
\end{figure}

\section{The FIR Filter}
The filter was generated using \mmalpha{}. It consists of a controller, a module
that instanciates three types of cells. 

\subsection{The Module}
It has 6 parameters: the clock (\texttt{clk}), the clock-enable signal (\texttt{CE}), 
the reset signal (\texttt{Rst}), the coefficients (\texttt{wXMirr1In}), the x input (\texttt{xXMirr1In})
and the outputs (\texttt{Yout}). 

Note: separate the package from the remaining of the filter.
Insert a return before the components...

The Module calls the controller, \texttt{ControlfirModule}. This controller has the
same signals (clock, clock-enable and reset) as the Module, and it returns a control
signal called \texttt{pipeCw7ctl1PXInit}.

It calls three types of cells. The simplest one is \texttt{cellfirModule4}, which 
does almost nothing: initialize the value of \texttt{Y} -- bounded to \texttt{Y4(0)} in the
call -- to 0, and transmits the
value of X (in the output signal \texttt{pipeCx5}, bounded to \texttt{pipeCx54(0)} in the
call.

The second cell is \texttt{cellfirModule1}. Besides the control signals (clock, clock-enable and
reset), it has four input signals: 
\begin{itemize}
\item the w coefficient (\texttt{wXMirr1}, binded to \texttt{wXMirr1(1)}),
\item the X input (\texttt{pipeCx5Reg3Xloc}), binded to \texttt{pipeCx5Reg3Xloc(1)}),
\item the Y input (\texttt{YReg5Xloc}) binded to \texttt{YReg5Xloc(1)}),
\item and
the pipe signal (\texttt{pipeCw7Xctl1PXInitXIn} binded to \texttt{pipeCw7Xctl1PXInitXIn(1)}).
\end{itemize}
It outputs similar signals:
\begin{itemize}
\item the w coefficient (\texttt{pipeCw7Xctl1P}, binded to \texttt{pipeCw7Xctl1P1(1)}),
\item the X output (\texttt{pipeCx5}), binded to \texttt{pipeCx51(1)}),
\item the pipe output signal (\texttt{pipeIOCw9} binded to \texttt{pipeIOCw91(1)}).
\item and the Y output (\texttt{Y}) binded to \texttt{Y1(1)}),
\end{itemize}

The second cell is \texttt{cellfirModule3}. Besides the control signals (clock, clock-enable and
reset), it has four input signals: 
\begin{itemize}
\item the w coefficient (\texttt{pipeCw7Xctl1PReg2Xloc}, binded to \texttt{pipeCw7Xctl1PReg2Xloc(p)}),
\item the X input (\texttt{pipeCx5Reg3Xloc}), binded to \texttt{pipeCx5Reg3Xloc(1)}),
\item the Y input (\texttt{YReg5Xloc}) binded to \texttt{YReg5Xloc(p)}),
\item and
the pipe signal (\texttt{pipeIOCw9Reg4Xloc} binded to \texttt{pipeIOCw9Reg4Xloc(p)}).
\end{itemize}
It outputs similar signals:
\begin{itemize}
\item the w coefficient (\texttt{pipeCw7Xctl1P}, binded to \texttt{pipeCw7Xctl1P1(1)}),
\item the X output (\texttt{pipeCx5}), binded to \texttt{pipeCx53(p)}),
\item and
the pipe output signal (\texttt{pipeIOCw9} binded to \texttt{pipeIOCw93(p)}), and
\item the Y output (\texttt{Y}) binded to \texttt{Y3(p)}).
\end{itemize}

\subsection{The Controller}
At reset time (reset is active at value 0), the controller initializes a counter to value $-1$, and enters state
\texttt{initState}. It remains in this state until the counter reaches the value
34. During the first 35 cycles, the output of the pipe signal is 0. It gets the 
value 1 during cycle 36 (when the counter reaches the value 34). 
             
\end{document}
