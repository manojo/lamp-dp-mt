\documentclass[11pt]{article}
\begin{document}

\newcommand{\mma}{\texttt{Mathematica}}
\newcommand{\mmalfa}{\texttt{MMAlpha}}

\title{Note about the Distribution of \mmalfa{}\footnote{Last update of this note on \today{}}}
\author{Patrice Quinton}
\date{January 2007}
\maketitle
\section{Introduction}
\label{introduction}
There is in the main directory a file named \texttt{CopyMMA.m}, which can be used to 
create a distribution of \mmalfa{}. 

\section{How to run in}
To run this program, do the following:
\begin{enumerate}
\item Start \mma{} and start \mmalfa{}.
\item Make sure that you are in the \texttt{\$MMALPHA} directory.
\item If needed, edit the \texttt{CopyMMA.m} file in order to define
the source directory and the destination directory of the distribution.
\item Type \texttt{<<CopyMMA}.
\end{enumerate}
This should make it. 

\section{Technical Information}
The \texttt{CopyMMA.m} file is a \mma{}�program. A few comments are in order.
\begin{enumerate}
\item The source directory and the destination directory are {\em hardwired}. To change them,
change the value of variables \texttt{sourceMMA} and \texttt{destMMA} (around line 80 of the
program). 
\item The program uses two functions: \texttt{copyF}, which copies a file, and 
\texttt{copyDir}, which copies a full directory. These functions are straightforward. 
\item While building the distribution, a file named \texttt{content.txt} is created
and updated in the main directory of the destination. This file contains the description
of the files that were copied.
\item The content of the distribution is {\em hardwired}, i.e. the choice of directories
and files is in the \texttt{CopyMMA.m} program. This file is itself copied in the distribution, 
in such a way that you can check what was this program when the distribution was created.
\item If you open \texttt{CopyMMA.m}, you will see that directories are either wholly copied
-- with possibly deleting some useless files, -- or partially copied, file by file. Which method
was used depends on the number of files that had to be copied.
\end{enumerate}


\end{document}
 