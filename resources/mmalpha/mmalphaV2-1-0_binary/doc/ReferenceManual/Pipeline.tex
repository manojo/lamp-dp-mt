%
% File created on {2007, 11, 15, 17, 10, 20.427708} by makeDoc.
%
% Section header
\section{The Alpha`Pipeline` package } 
\label{label:Alpha`Pipeline`}
\alphausage{Pipeline}{Package. Contains the definition of the pipeline transformation. Containes the funcitons pipeline, pipeall, pipeIO}{Alpha/Pipeline.m}{Alpha`Pipeline`}
\index{Pipeline}

\alphausage{pipeline}{pipeline[position,pipespec] pipelines in system \$result expression given by position according to pipeline specification pipespec, and puts the result in \$result. pipeline[sys, position,pipespec] does the same to program contained in symbol sys. Parameter pipespec provides the name of the new variable, and the pipeline direction in the textual form "newname.translation" (example: "B.(i,j$ \rightarrow $i,j+1)" means that new pipeline variable is B and data movement is (0,1)). Parameter position gives the position of the expression in the program, using the conventions of the Mathematica function Position (see getPart).}{Alpha/Pipeline.m}{Alpha`Pipeline`}
\index{pipeline}

\alphausage{pipeall}{pipeall[var, exp, pipespec] pipelines in program \$result all occurrences of expression exp within equation whose left-hand side is var according to pipeline specification pipespec, and returns the result in \$result.  pipeall[sys, var, exp, pipespec] does the same to program contained in symbol sys.  pipeall[var, exp, pipespec, bd] pipelines exp of var definition  using pipespec, but extend the domain of pipeline to the boundary given by domain bd. Parameter var can be either a single variable,  a list of variables, or the empty list, in which case the expression is pipelined in all the program simultaneously. The parameters can be specified either in textual form or as  AST. For example,  pipeall["X","a.(i,j,k$ \rightarrow $i,j+1,k)","A.(i,j,k$ \rightarrow $i,j+1,k)"]  pipes all occurences of "a.(i,j,k$ \rightarrow $i,j+1,k)" in the definition of X,  using the pipeline specification "A.(i,j,k$ \rightarrow $i,j+1,k)". The pipeline specification contains the name to be used for the pipeline variable (here "A"), and the direction of pipeline given as a translation (here, vector (0,1,0)). Similarly,  pipeall["X","a.(i,j,k$ \rightarrow $i,j+1,k)","A.(i,j,k$ \rightarrow $i,j+1,k)",  "\{i,j,k$|$i$>$=0\}"]  pipes all occurences of "a.(i,j,k$ \rightarrow $i,j+1,k)" in the definition of X,  using the pipeline specification "A.(i,j,k$ \rightarrow $i,j+1,k)", withing  the boundary "\{i,j,k$|$i$>$=0\}"]. In case of error, pipeall returns Null, and leaves \$result unchanged.  For programming purposes, pipeall may  also be called with the form pipeall[sys,var,expression,pipespec],  where expression and pipeline are abstract syntax trees. Parameter expression can also be replaced by the occurrence number of the  expression. See also the function pipeInfo which gives hints about which pipeline commands to perform.}{Alpha/Pipeline.m}{Alpha`Pipeline`}
\index{pipeall}

\alphausage{pipeAll}{pipeall[var, exp, pipespec] pipelines in program \$result all occurrences of expression exp within equation whose left-hand side is var according to pipeline specification pipespec, and returns the result in \$result.  pipeall[sys, var, exp, pipespec] does the same to program contained in symbol sys.  pipeall[var, exp, pipespec, bd] pipelines exp of var definition  using pipespec, but extend the domain of pipeline to the boundary given by domain bd. Parameter var can be either a single variable,  a list of variables, or the empty list, in which case the expression is pipelined in all the program simultaneously. The parameters can be specified either in textual form or as  AST. For example,  pipeall["X","a.(i,j,k$ \rightarrow $i,j+1,k)","A.(i,j,k$ \rightarrow $i,j+1,k)"]  pipes all occurences of "a.(i,j,k$ \rightarrow $i,j+1,k)" in the definition of X,  using the pipeline specification "A.(i,j,k$ \rightarrow $i,j+1,k)". The pipeline specification contains the name to be used for the pipeline variable (here "A"), and the direction of pipeline given as a translation (here, vector (0,1,0)). Similarly,  pipeall["X","a.(i,j,k$ \rightarrow $i,j+1,k)","A.(i,j,k$ \rightarrow $i,j+1,k)",  "\{i,j,k$|$i$>$=0\}"]  pipes all occurences of "a.(i,j,k$ \rightarrow $i,j+1,k)" in the definition of X,  using the pipeline specification "A.(i,j,k$ \rightarrow $i,j+1,k)", withing  the boundary "\{i,j,k$|$i$>$=0\}"]. In case of error, pipeall returns Null, and leaves \$result unchanged.  For programming purposes, pipeall may  also be called with the form pipeall[sys,var,expression,pipespec],  where expression and pipeline are abstract syntax trees. Parameter expression can also be replaced by the occurrence number of the  expression. See also the function pipeInfo which gives hints about which pipeline commands to perform.}{Alpha/Pipeline.m}{Alpha`Pipeline`}
\index{pipeAll}

\alphausage{pipeIO}{pipeIO[var,exp,pipespec,boundary] pipelines in \$result I/O expression exp occuring in variable var to plane boundary according to pipeline specification pipespec. pipeIO[sys,var,exp,pipespec,boundary] does the same to program contained in symbol sys. Parameters var and exp are strings. Parameter pipespec is the pipeline specification string in the form "newVar.direction" (example: "X1.(t$ \rightarrow $t-1)").  Parameter boundary is given in textual form (example: " \\\{i,j $|$ 2i+3j+4$>$=0 \\\}").  For example, pipeIO["y","x","X.(t,n$ \rightarrow $t+1,n+1)"," \\\{t,n$|$n$>$=-1\\\}"] pipes all occurences of x in definition of y in \$result to plane " \\\{t,n $|$ n$>$=-1\\\}", with new name X, and pipeline direction (1,1).  If something wrong happens, pipeIO returns the system unmodified.  For programming purpose, exp, pipespec and boundary can be given as AST instead of textual form.  Also, the position of the expressions to pipeline may be given. See test file pipeIOT.m in \$MMALPHA/test.}{Alpha/Pipeline.m}{Alpha`Pipeline`}
\index{pipeIO}

