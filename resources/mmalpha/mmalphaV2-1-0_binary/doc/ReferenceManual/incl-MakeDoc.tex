\subsection{Note regarding the documentation of \MMAlfa{}}
\label{doc-makedoc}
The \MMAlfa{} software is (partially) documented in various
ways.\footnote{This chapter has been
proof-read by Patrice Quinton on January 8, 2007.}
\begin{itemize}
\item A \texttt{html} documentation is included in the distribution.
It can be accessed from the \texttt{welcome.html} file in the 
\texttt{\$MMalpha{}} directory. 
\item A {\em getting started} document is available in form of
a \texttt{pdf}
file. It is accessible from the initial \texttt{html} welcome
page and is located in:
\begin{verbatim} 
$MMAlpha/doc/QuickStart/AlphaStart.pdf
\end{verbatim} 
This document is the most up-to-date tutorial for \MMAlpha{}. It does not
describe all features of \MMAlpha{}, but normally, allows a user to
discover the main features of the software.
\item A reference manual is being developped in:
\begin{verbatim} 
$MMAlpha/doc/ReferenceManual
\end{verbatim}
and it is described hereafter.
\item Other documentation files exist, but they are not guaranteed to 
be up-to-date.
\end{itemize}

The reference manual of \MMALPHA{} is obtained through the use 
of commands in the \texttt{Makedoc} package. 
It is produced by executing the program called \texttt{genReferenceManual.m}
which is situated in the same directory.

This program contains a set of calls to the function \texttt{doDoc}
of the \texttt{Makedoc} package (see \texttt{?doDoc}). Each one of
these calls reads a package -- for example, \texttt{Alpha/Makedoc.m} --
whose name is relative to directory
\begin{verbatim} 
$MMALPHA/lib/Packages
\end{verbatim}
and translates it into a \LaTeX{} file -- for example, \texttt{Makedoc.tex}, --
in target directory \texttt{\$MMALPHA/doc/ReferenceManual}.

The reference manual itself is edited manually. It is in the 
file:
\begin{verbatim}
referenceManual.tex
\end{verbatim}
of the same directory. It contains
basically input statements for the documentation files created by 
\texttt{doDoc}. This function (which is documented hereafter), has
several options: one can either generate a standalone \LaTeX{} file, or
an input file (this helps for editing the documentation), one can 
specify a list of functions that are only to be documented, one can 
specify where the file has to be put, etc. 

The reference manual is based on the usage and note statements which are
contained in a given package. In there exists a file named 
\texttt{incl-textFile} in the target directory, then an include 
statement for this file is added in the generated \LaTeX{} file.
For example, the documentation that you are reading is in the 
file \texttt{incl-Makedoc.text} file, in directory:
\begin{verbatim}
$MMALPHA/doc/referenceManual
\end{verbatim}
Therefore, when the 
\texttt{Makedoc.m} file is processed by the corresponding 
statement in the \texttt{genReferenceManual.m} program, i.e.:
\begin{small}
\begin{verbatim}
doDoc["Alpha/MakeDoc.m", "MakeDoc.tex", 
  targetDir -> Environment["MMALPHA"] <> "/doc/ReferenceManual"];
\end{verbatim}
\end{small}
an input statement is included in the \texttt{Makedoc.tex} file 
which is generated.

The \texttt{doDoc} function tries to modify the content of the
usage and note statements so that special symbols are correctly
output in \LaTeX{}. These transformations are described in the
variable \texttt{repRules} of the package, and may be modified. 
Index statements are generated for each usage statement. 

