%
% File created on {2008, 3, 11, 15, 1, 57.629758} by makeDoc.
%
% Section header
\section{The Alpha`MakeDoc` package } 
\label{label:Alpha`MakeDoc`}
\subsection{Note regarding the documentation of \MMAlfa{}}
\label{doc-makedoc}
The \MMAlfa{} software is (partially) documented in various
ways.\footnote{This chapter has been
proof-read by Patrice Quinton on January 8, 2007.}
\begin{itemize}
\item A \texttt{html} documentation is included in the distribution.
It can be accessed from the \texttt{welcome.html} file in the 
\texttt{\$MMalpha{}} directory. 
\item A {\em getting started} document is available in form of
a \texttt{pdf}
file. It is accessible from the initial \texttt{html} welcome
page and is located in:
\begin{verbatim} 
$MMAlpha/doc/QuickStart/AlphaStart.pdf
\end{verbatim} 
This document is the most up-to-date tutorial for \MMAlpha{}. It does not
describe all features of \MMAlpha{}, but normally, allows a user to
discover the main features of the software.
\item A reference manual is being developped in:
\begin{verbatim} 
$MMAlpha/doc/ReferenceManual
\end{verbatim}
and it is described hereafter.
\item Other documentation files exist, but they are not guaranteed to 
be up-to-date.
\end{itemize}

The reference manual of \MMALPHA{} is obtained through the use 
of commands in the \texttt{Makedoc} package. 
It is produced by executing the program called \texttt{genReferenceManual.m}
which is situated in the same directory.

This program contains a set of calls to the function \texttt{doDoc}
of the \texttt{Makedoc} package (see \texttt{?doDoc}). Each one of
these calls reads a package -- for example, \texttt{Alpha/Makedoc.m} --
whose name is relative to directory
\begin{verbatim} 
$MMALPHA/lib/Packages
\end{verbatim}
and translates it into a \LaTeX{} file -- for example, \texttt{Makedoc.tex}, --
in target directory \texttt{\$MMALPHA/doc/ReferenceManual}.

The reference manual itself is edited manually. It is in the 
file:
\begin{verbatim}
referenceManual.tex
\end{verbatim}
of the same directory. It contains
basically input statements for the documentation files created by 
\texttt{doDoc}. This function (which is documented hereafter), has
several options: one can either generate a standalone \LaTeX{} file, or
an input file (this helps for editing the documentation), one can 
specify a list of functions that are only to be documented, one can 
specify where the file has to be put, etc. 

The reference manual is based on the usage and note statements which are
contained in a given package. In there exists a file named 
\texttt{incl-textFile} in the target directory, then an include 
statement for this file is added in the generated \LaTeX{} file.
For example, the documentation that you are reading is in the 
file \texttt{incl-Makedoc.text} file, in directory:
\begin{verbatim}
$MMALPHA/doc/referenceManual
\end{verbatim}
Therefore, when the 
\texttt{Makedoc.m} file is processed by the corresponding 
statement in the \texttt{genReferenceManual.m} program, i.e.:
\begin{small}
\begin{verbatim}
doDoc["Alpha/MakeDoc.m", "MakeDoc.tex", 
  targetDir -> Environment["MMALPHA"] <> "/doc/ReferenceManual"];
\end{verbatim}
\end{small}
an input statement is included in the \texttt{Makedoc.tex} file 
which is generated.

The \texttt{doDoc} function tries to modify the content of the
usage and note statements so that special symbols are correctly
output in \LaTeX{}. These transformations are described in the
variable \texttt{repRules} of the package, and may be modified. 
Index statements are generated for each usage statement. 


\alphausage{MakeDoc}{MakeDoc is a package which contains functions for the automatic creation of a reference Manual. MakeDoc contains functions doDoc and makeDoc.}{Alpha/MakeDoc.m}{Alpha`MakeDoc`}
\index{MakeDoc}

\alphanote{MakeDoc}{Functions of MakeDoc read all usage and note statements of a package, and for each one, they produce a Latex macro entry alphausage or alphanote. Before creating an entry, substitutions are performed on the text, so that special words (such as \$schedule) and special characters (such as \_) are transformed. The substitution list is in the private variable repRules in the file MakeDoc.m.} 
 
\alphausage{doDoc}{doDoc[ package, texFile ] generates in file texFile the documentation LaTex file associated to a Mathematica file package. doDoc extracts the information contained in the usage and note descriptions placed at the beginning of file package, and produces an output Latex file. doDoc[ package ] automatically writes in a file which has the name  package where the .m suffix is replaced by .tex.  doDoc[ Package, functionList ] generates the documentation only for  the functions enumerated in functionList (as Strings). doDoc has  options fullLatex and callFile which allow a complete latex file,  resp. a calling Latex file to be produced. If there exists a file incl-textFile in the target directory, then a Latex input statement is added in the documentation package.}{Alpha/MakeDoc.m}{Alpha`MakeDoc`}
\index{doDoc}

\alphausage{fullLatex}{fullLatex is an option of doDoc and makeDoc. Default value is False. If True, the latex file contains the preamble and the \\end\{document\} statement.  Otherwise, it should be used in an input file}{Alpha/MakeDoc.m}{Alpha`MakeDoc`}
\index{fullLatex}

\alphausage{targetDir}{option of doDoc. By default, it is "". Gives the name of a directory where the latex file and the calling latex file should be written.}{Alpha/MakeDoc.m}{Alpha`MakeDoc`}
\index{targetDir}

\alphausage{callFile}{callFile is an option of doDoc and makeDoc. Default valus is False. If True, a latex calling file is created.}{Alpha/MakeDoc.m}{Alpha`MakeDoc`}
\index{callFile}

\alphausage{sourceDir}{option of makeDoc. Default value is Null. If Null, the source package are those of \$MMALPHA. Their names are in the file \$MMALPHA/sources/ MaleDoc/List\_modules. Otherwise, these names are prefixed by the  value of the sourceDir option.}{Alpha/MakeDoc.m}{Alpha`MakeDoc`}
\index{sourceDir}

\alphausage{makeDoc}{This function does not work, currently. Use doDoc instead. makeDoc[] updates the Reference Manual of \MMAlpha{} by running  the file genReferenceManual.m in directory\\ \$MMALPHA/doc/ReferenceManual/\\ This produces new versions of the latex files describing the  packages. The Reference Manual can be produced using Latex on  referenceManual.tex. See also doDoc.}{Alpha/MakeDoc.m}{Alpha`MakeDoc`}
\index{makeDoc}

