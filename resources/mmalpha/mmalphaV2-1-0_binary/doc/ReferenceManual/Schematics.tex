%
% File created on {2007, 11, 15, 17, 10, 21.172837} by makeDoc.
%
% Section header
\section{The Alpha`Schematics` package } 
\label{label:Alpha`Schematics`}
\alphausage{Schematics}{This package contains functions for drawing a raw schematics of an Alpha  programs}{Alpha/Schematics.m}{Alpha`Schematics`}
\index{Schematics}

\alphausage{partShown}{partShown is an option of schematics. It defines the region of the plot that will be shown. Default is Automatic and corresponds to a region \{\{0,1\},\{0,1\}\}, which displays the full schematics. partShown $ \rightarrow $\{\{0.2,0.4\},\{0.3,1.2\}\} would display a region whose coordinates, relative to the schematics, are changed.}{Alpha/Schematics.m}{Alpha`Schematics`}
\index{partShown}

\alphausage{fSize}{fSize is an option of schematics, which changes the size of the font used to show symbols. Default value is 20 .}{Alpha/Schematics.m}{Alpha`Schematics`}
\index{fSize}

\alphausage{sPositions}{sPositions is an option of schematics. Default value is square. sPositions $ \rightarrow $ Null stacks the operators.  sPositions $ \rightarrow $ \{columns $ \rightarrow $ nb\} gives the number of columns of the diagram.  sPositions may be a list of coordinates for the operators. In this case, make sure that the number of positions is the same as the number of operators.}{Alpha/Schematics.m}{Alpha`Schematics`}
\index{sPositions}

\alphausage{square}{square is an option value for option sPosition in  the schematics function. sPosition $ \rightarrow $ square asks for a square shaped schematics layout.}{Alpha/Schematics.m}{Alpha`Schematics`}
\index{square}

\alphausage{columns}{columns is an option value for option sPosition in  the schematics function. sPositions$ \rightarrow $\{columns$ \rightarrow $nb\} asks for a nb column schematics layout.}{Alpha/Schematics.m}{Alpha`Schematics`}
\index{columns}

\alphausage{yFactor}{yFactor is the form factor of boxes used by the schematics function. Default value is 0.2 .}{Alpha/Schematics.m}{Alpha`Schematics`}
\index{yFactor}

\alphausage{offsetX}{offsetX is the X offset value for schematics. Default value is 0.5 .}{Alpha/Schematics.m}{Alpha`Schematics`}
\index{offsetX}

\alphausage{offsetY}{offsetY is the Y offset value for schematics. Default value is 0.1 .}{Alpha/Schematics.m}{Alpha`Schematics`}
\index{offsetY}

\alphausage{newName}{newName[ sys, v ] returns a new unique name built from variable v.  newName[ v ] does the same in \$result. This function cannot be used in any context. Use getNewName instead.}{Alpha/Schematics.m}{Alpha`Schematics`}
\index{newName}

\alphausage{skeleton}{skeleton[] gives the skeleton of \$result}{Alpha/Schematics.m}{Alpha`Schematics`}
\index{skeleton}

\alphausage{schematics}{schematics[] draws a schematic diagram of the  program \$result. See Options[schematics] to get the options of  schematics.}{Alpha/Schematics.m}{Alpha`Schematics`}
\index{schematics}

\alphausage{flattenEquation}{}{Alpha/Schematics.m}{Alpha`Schematics`}
\index{flattenEquation}

\alphausage{flattenSkeleton}{ddddd}{Alpha/Schematics.m}{Alpha`Schematics`}
\index{flattenSkeleton}

\alphausage{findAliases}{}{Alpha/Schematics.m}{Alpha`Schematics`}
\index{findAliases}

