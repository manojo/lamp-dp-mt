%
% File created on {2009, 4, 25, 18, 19, 40.078499} by makeDoc.
%
% Section header
\section{The Alpha` package } 
\label{label:Alpha`}
% 
\subsubsection{Introduction}
The \texttt{Alpha.m} package is the main package of Mathematica.
It is located in the directory \texttt{\$MMALPHA/lib/Packages}.
It contains the definition of the main symbols and data structures
of \MMAlpha{}. 

A special word about the \texttt{exportAlphaFunctions} function.
This function takes as a parameter the autoloadFile. This file
is located in 
\begin{verbatim}
$MMAlpha/lib/Packages/Alpha/autoload.m
\end{verbatim}
%$
and is created automatically by the \texttt{exportAlphaFunctions}.

At the end of the \texttt{Alpha.m}, a few instructions are executed
when loading this package. If the \texttt{autoloadFile} does not
exists, \texttt{exportAlphaFunctions} is called and creates it.

The function \texttt{exportAlphaFunctions} uses a local variable
\texttt{contexts} of \texttt{Alpha.m} which contains the list of 
the packages that need to be loaded when using \MMAlpha{}. This variable
has to be changed when a new package is added. For each package
of \texttt{contexts}, \texttt{exportAlphaFunctions} creates an 
entry in the autoload file that declares the corresponding package
using the \texttt{DeclarePackage} function of Mathematica. In this
way, all packages are declared, and when a symbol of the package
is evaluated for the first time, 
the corresponding package is automatically loaded.

The remaining of this chapter describes the functions of the 
Alpha package.
\alphanote{Alpha}{Documentation revised on April 25, 2008} 
 
\alphausage{Alpha}{\MMAlpha{} is a Mathematica-based implementation of the
ALPHA language.  The "Alpha`" package contains basic commands for
the \MMAlpha{} system.  The name "Alpha`" is also a root name for all
packages and symbols of \MMAlpha{}.}{Alpha.m}{Alpha`}
\index{Alpha}

\alphanote{Alpha}{In this package, there is a description of some structures (domain,
pol, zpol), of global variables (\$coneList, \$demoDirectory, \$depCone, 
\$library, \$masterNotebook, \$MMALPHA, \$myMasterNotebook, \$myNotebooks, 
\$program, \$result, \$rootDirectory, \$schedule, \$scheduleLibrary, 
\$testDirectory, \$tmpDirectory, \$showGraph, 
\$testResult, \$testDirectory, \$testIdentifier, \$testReportFile,
\$tmpDirectory, \$runTime,
\$vhdlPatternsDir), 
and finally of some functions. Among these, 
ashow, ashowLib, asave, asaveLib, getSystem, load, myStart, putSystem,
save, saveLib, show, showLib, showMat, start, and systemNames
are important for a user. Functions getPart, readAlpha, readDom, 
readExp, readDom, testFunction, tests, varTypes, writeC, astQ, 
fileName, openTemporary, getTemporary, packageLink, demoLink, docLink, 
link, exportAlphaFunctions, and setMMADir are useful for developers. 
Function writeTex, enTete, and wrap are unused or incomplete.} 
 
\alphausage{domain}{domain[dim, idx, lpol] is the structure representing 
a domain. dim is the (integer) dimension of the domain, 
idx is the list of dimension names and lpol is a list of
polyhedra or of Z-polyhedra. See also ?pol or ?zpol.}{Alpha.m}{Alpha`}
\index{domain}

\alphanote{domain}{Domains of Alpha are either polyhedra, unions of polyhedra, Z-polyhedra, 
i.e., intersection of a polyhedron and of a lattice, or unions of 
Z-polyhedra. Note also that in domains of Z-Polyhedra, the 
index part is empty.} 
 
\alphausage{pol}{pol[nbConstraints, nbRays, nbEq, nblines, cons, rays] is 
the representation of a polyhedron in Alpha.
nbConstraints (integer) is the number of
constraints (equalities and inequalities), 
nbRays (integer) is the number of rays
(including lines), nbEq (integer) is the number of equalities,
nblines (integer) is the number of lines, cons (matrix) is the
set of equalities and inequalities, and rays (matrix) is the 
set of rays and lines. Inequalities and rays are represented
by lists of integers with a leading 1, and equalities or lines
by lists of integers with a leading 0. See also ?domain and ?zpol.}{Alpha.m}{Alpha`}
\index{pol}

\alphausage{zpol}{zpol[m, lpol] is the representation of a Z-polyhedron in Alpha.
m is an Alpha matrix which represents the Z-lattice, and lpol
is a list of polyhedra. See also ?domain and ?pol.}{Alpha.m}{Alpha`}
\index{zpol}

\alphausage{\$coneList}{\$coneList contains the list of dependence cones of all 
the dependences of \$result. This variable is set by the 
initUniformization function, and it is a list of domains.}{Alpha.m}{Alpha`}
\index{\$coneList}

\alphausage{\$demoDirectory}{\$demoDirectory contains the path of the demo directory of \MMAlpha{}.}{Alpha.m}{Alpha`}
\index{\$demoDirectory}

\alphausage{\$depCone}{\$depCone is a glocal variable which contains the 
dependence cone of \$result. This variable is set by the
initUniformization function, and is in domain.}{Alpha.m}{Alpha`}
\index{\$depCone}

\alphausage{\$library}{\$library is the Mathematica symbol which holds the list of Alpha
programs (systems and subsystems) most recently loaded using
load. The system in \$result is also present in  \$library.}{Alpha.m}{Alpha`}
\index{\$library}

\alphausage{\$masterNotebook}{\$masterNotebook is the path of the master.nb notebook, which gives 
access to all demo notebooks on Alpha. To open it, use the command 
NotebookOpen[\$masterNotebook] or start[].}{Alpha.m}{Alpha`}
\index{\$masterNotebook}

\alphausage{\$MMALPHA}{\$MMALPHA contains the path of the \MMAlpha{} directory. This variable is 
set by an environment variable called MMALPHA.}{Alpha.m}{Alpha`}
\index{\$MMALPHA}

\alphausage{\$myMasterNotebook}{\$myMasterNotebook is the name of the notebook that you have placed
in the \$myNotebooks directory to access your own examples. To open
it, evaluate myStart[].}{Alpha.m}{Alpha`}
\index{\$myMasterNotebook}

\alphausage{\$myNotebooks}{\$myNotebooks contains the path of the directory containing your examples.
You can set this path in your init.m file. This file, placed in your home
directory, is loaded by Mathematica before any other Mathematica command.}{Alpha.m}{Alpha`}
\index{\$myNotebooks}

\alphausage{\$program}{\$program is a Mathematica symbol which keeps the abstract
syntax tree of the source of any transformation. It is initialized
by load, and changed when a (second) transformation is applied. The
most recent AST is kept in \$result (see ?\$result)}{Alpha.m}{Alpha`}
\index{\$program}

\alphausage{\$result}{\$result is the Mathematica symbol which holds the result of the
most recent operation (load or transformation). It is the default
source program for many transformations. When it is used by default by
a transformation, it is also modified by this transformation}{Alpha.m}{Alpha`}
\index{\$result}

\alphausage{\$rootDirectory}{\$rootDirectory contains the path of the \MMAlpha{} 
directory. It is set to \$MMALPHA}{Alpha.m}{Alpha`}
\index{\$rootDirectory}

\alphausage{\$schedule}{\$schedule is the Mathematica symbol that contains the schedule 
information of a system after the execution of the schedule or
the scd functions. The format of \$schedule is:

scheduleResult[name, 

  List[\{var, varIndices, sched[tauVect, constCoef]\}],

  objFunction].
 
where name is the name of the system, var is a variable of sys,
varIndices is the list of indexes of var, tauVect is a list
of integer corresponding to the linear part of the schedule of
var, constCoef is the affine part, and objFunction is the 
objective function that was used to find the schedule.}{Alpha.m}{Alpha`}
\index{\$schedule}

\alphausage{\$scheduleLibrary}{\$scheduleLibrary is the list of the schedules that have been computed for 
various programs since the beginning of the current session.}{Alpha.m}{Alpha`}
\index{\$scheduleLibrary}

\alphausage{\$showGraph}{\$showGraph is a global variable that is set by the report function. Use is not
garanteed.}{Alpha.m}{Alpha`}
\index{\$showGraph}

\alphausage{\$testResult}{\$testResult is a global variable that is used to store the accumulated 
result value of a test.}{Alpha.m}{Alpha`}
\index{\$testResult}

\alphausage{\$testDirectory}{\$testDirectory is the path name of the directory that contains the 
test files for all packages. Currently set to \$MMALPHA/tests/}{Alpha.m}{Alpha`}
\index{\$testDirectory}

\alphausage{\$testIdentifier}{\$testIdentifier contains the name of the test that one wants to perform.
Default value is "".}{Alpha.m}{Alpha`}
\index{\$testIdentifier}

\alphausage{\$testReportFile}{\$testReportFile is the name of the reporting file for the tests. Its 
value is "" when no report file is open, otherwise, it contains the}{Alpha.m}{Alpha`}
\index{\$testReportFile}

\alphausage{\$tmpDirectory}{\$tmpDirectory is the pathname of the directory that will be used 
for all temporary files, i.e. /tmp/ on unix, C:/tmp/ on WindowsNT 
(strongly recommended).}{Alpha.m}{Alpha`}
\index{\$tmpDirectory}

\alphausage{\$runTime}{\$runTime is a global variable that stores the running time of 
the scheduler.}{Alpha.m}{Alpha`}
\index{\$runTime}

\alphausage{ashow}{ashow[sys] pretty prints in array notation the program contained in sys. Default value of sys is \$result. See also show.}{Alpha.m}{Alpha`}
\index{ashow}

\alphausage{ashowLib}{ashowLib[lib] prints in array notation all the systems of a library lib. 
Default value of lib is \$library. Warning, currently does not work 
on notebooks, use ashow[lib] instead. See also show, ashow.}{Alpha.m}{Alpha`}
\index{ashowLib}

\alphausage{asave}{asave[sys,filename] saves the array notation pretty printed version of
\$result in file filename. Default value of sys is \$result. See also
save, load, saveLib, asaveLib.}{Alpha.m}{Alpha`}
\index{asave}

\alphausage{asaveLib}{asavelib[lib,filename] saves in array notation all systems 
contained in library lib in file filename. Default value of 
lib is \$library. See also save, asave.}{Alpha.m}{Alpha`}
\index{asaveLib}

\alphausage{fshow}{obsolete form of show.}{Alpha.m}{Alpha`}
\index{fshow}

\alphausage{getPart}{getPart[exp,position] returns the subexpression of exp which is
designated by position.  The parameter position is a list of integers
following the convention of the Mathematica function Position. For
example, getPart[exp,\{2,3\}] identifies the third subtree of the second
subtree of exp. Warning: getPart does not accept in second parameter a
list of positions such as those returned by the Mathematica function
Position.}{Alpha.m}{Alpha`}
\index{getPart}

\alphausage{getSystem}{getSystem[sys,lib] extracts the system named sys from the library lib 
and selects it as the current system. Default value of lib is 
\$library. See also putSystem.}{Alpha.m}{Alpha`}
\index{getSystem}

\alphausage{load}{load[filename] parses the ALPHA program contained in file filename 
and returns the corresponding abstract syntax tree. As a side effect, 
symbols \$program and \$result are loaded with the parsed program.
See also save, asave, asaveLib.}{Alpha.m}{Alpha`}
\index{load}

\alphausage{myStart}{myStart[] opens the \$myMasterNotebook notebook placed in the
directory \$MMALPHA/myNotebooks.}{Alpha.m}{Alpha`}
\index{myStart}

\alphausage{on}{on[] is an obsolete form of start[]. It opens the Master 
Notebook.}{Alpha.m}{Alpha`}
\index{on}

\alphausage{putSystem}{putSystem[sys,lib] puts the system sys (default \$result) into library 
lib (default \$library). If sys already exists in lib, it is replaced, 
otherwise it is appended. See also getSystem.}{Alpha.m}{Alpha`}
\index{putSystem}

\alphausage{readAlpha}{readAlpha[filename] parses the ALPHA program in file filename
and returns its abstract syntax tree. Does not modify \$program 
nor \$result.}{Alpha.m}{Alpha`}
\index{readAlpha}

\alphausage{readDom}{readDom[dom] parses the ALPHA domain dom and returns its abstact syntax
tree. To parse a parametric domain such as \{i$|$i$<$N\}, it is
mandatory to supply the parameter domain or the parameter 
names as a second argument, for example:

readDom["\{i,j $|$ ... \}", \{"N"\}]

readDom["\{i,j $|$ ... \}",domain[1,\{"N"\}, ...] ] .}{Alpha.m}{Alpha`}
\index{readDom}

\alphausage{readExp}{readExp[exp] parses the ALPHA expression exp and returns its 
abstact syntax tree. readExp[exp,paramNames] or 
readExp[exp,paramDom] are used  to parse a parametric
expression. Example: 

readExp["A.(i,j,N$ \rightarrow $N,N) + B.(i,N$ \rightarrow $i,N)"]

readExp["A.(i,j$ \rightarrow $N) + B.(i$ \rightarrow $i)", \{"N"\}]

readExp["A.(i,j$ \rightarrow $N) + B.(i$ \rightarrow $i)", domain[1,\{"N"\}, ...] ].}{Alpha.m}{Alpha`}
\index{readExp}

\alphausage{readMat}{readMat[dep] parses the ALPHA dependency dep and returns its matrix 
abstract syntax tree. To parse a parametric dependency such as (i,j$ \rightarrow $N),
it is mandatory to supply the parameter domain 
or the parameter names, e.g.:

readMat["(i,j,N$ \rightarrow $i)"]
 
readMat["(i,j$ \rightarrow $i)", \{"N"\}]

readMat["(i,j$ \rightarrow $i)",domain[1,\{"N"\}, ...] ] .}{Alpha.m}{Alpha`}
\index{readMat}

\alphausage{save}{save[sys,filename] saves the standard notation pretty printed
version of sys in file filename. Default value of sys is \$result.
See also load, asave, saveLib, asaveLib.}{Alpha.m}{Alpha`}
\index{save}

\alphausage{saveLib}{savelib[lib,filename] saves all systems contained in library lib
in file filename in standard notation.}{Alpha.m}{Alpha`}
\index{saveLib}

\alphausage{show}{show[var] pretty-prints the program, the domain, the matrix or 
the schedule contained in symbol var. Default value of var is 
\$result. show[var, p] pretty prints the domain or the matrix 
contained in symbol var taking parameter domain p into account. 
var and p should be abstract syntax trees.}{Alpha.m}{Alpha`}
\index{show}

\alphausage{showLib}{showLib[lib] prints all the systems of library lib in standard notation. 
Default value of lib is \$library. Warning, currently does not 
work in notebooks, use show[lib] instead}{Alpha.m}{Alpha`}
\index{showLib}

\alphausage{showMat}{showMat[m] pretty-prints the matrix m in Alpha format.}{Alpha.m}{Alpha`}
\index{showMat}

\alphausage{start}{start[] opens the Master Notebook.}{Alpha.m}{Alpha`}
\index{start}

\alphausage{systemNames}{systemNames[] returns the list of system names loaded in \$library.}{Alpha.m}{Alpha`}
\index{systemNames}

\alphausage{testFunction}{testFunction[expr,resOfExpr,message\_String] is used for building tests. It
evaluates expr and compare it the resOfExpr, return True if the two are 
equal, False otherwise and prints a standard message (using the information 
of message: usually a test number to identify the test). This function is
for developers.}{Alpha.m}{Alpha`}
\index{testFunction}

\alphausage{tests}{tests[PackageName] call the test procedure for the package 
PackageName. tests[] calls the test procedure 
for all \MMAlpha{} packages. Warning: a complete test takes a long time.}{Alpha.m}{Alpha`}
\index{tests}

\alphausage{varTypes}{varTypes[sys] lists the types of all variables of sys. Default value
of sys is \$result.}{Alpha.m}{Alpha`}
\index{varTypes}

\alphausage{writeC}{writeC[sys,f,opts] generates C code from the system sys in file f.  
Default value of sys is \$result and default value of f is "Alpha.c". 
opts are options that are sent to the C generator. 
Option "-p num1 num2 ..." sets Alpha parameters to value 
num1, num2, etc. The C code is correct only if all Alpha parameters 
are assigned values. The "-g" option provides a debug version, 
where all equation calls are printed out. The "-s" option is for 
the generation of C code to be interfaced with the Signal language. 
WARNING: writeC does not work properly for systems with unbounded domains.}{Alpha.m}{Alpha`}
\index{writeC}

\alphausage{writeTex}{writeTex[sys,f] generates the Latex form of the program
contained in symbol sys (default \$result) into file f 
(default "Alpha.tex") of the current directory. 
writeTex[...,"-a"] produces a program in array notation. 
Warning: writeTex overwrites an already existing output f file.}{Alpha.m}{Alpha`}
\index{writeTex}

\alphanote{writeTex}{This function is available only on Unix, and seldom used. Its 
output is not guaranteed.} 
 
\alphausage{astQ}{astQ[exp] is True if expression exp is an AST, False otherwise.}{Alpha.m}{Alpha`}
\index{astQ}

\alphausage{fileName}{fileName[\{\_\_String\}] returns the path expression corresponding to the
file according to the operating system in use. Actually, this function 
is not necessary, since Mathematica converts all pathnames from the 
Unix form.}{Alpha.m}{Alpha`}
\index{fileName}

\alphausage{openTemporary}{openTemporary[] works as the Mathematica function OpenTemporary, but
opens the file in the \$tmpDirectory and return the corresponding 
stream. The command Close[openTemporary[] returns the name of 
the temporary file (warning: there is an inconsistency in this 
function due to Mathematica between the Unix version and the WindowsNT 
version: on Unix, the full name returned is ("/tmp/...") while 
on WindowsNT, only the name of the file is returned. Please use 
getTemporaryName[] to get the name of a temporary file instead.}{Alpha.m}{Alpha`}
\index{openTemporary}

\alphausage{getTemporaryName}{getTemporaryFile[] returns the full name of a new temporary file. 
It is equivalent to Close[OpenTemporary[]] on a unix plateform and 
it has the same behaviour on other plateforms.}{Alpha.m}{Alpha`}
\index{getTemporaryName}

\alphausage{demoLink}{demoLink[demo] creates an hyperlink button aiming at the Alpha 
demo notebook "demo.nb". This button may then be cut and pasted anywhere. 
Example:

demoLink["Fir"]

creates a button aiming at the demo notebook Fir.nb 
placed in directory \$MMALPHA/demos/NOTEBOOKS/Fir.

demoLink[d,demo] creates an hyperlink button aiming at notebook 
demo.nb in directory d of \$MMALPHA/demos/NOTEBOOKS.}{Alpha.m}{Alpha`}
\index{demoLink}

\alphausage{docLink}{docLink[file] creates an hyperlink button
aiming at the Alpha documentation notebook "file.nb". This button may then be
cut-and-pasted anywhere. docLink[d:\_String, file:\_String] creates an
hyperlink button aiming at notebook file.nb in directory d of
\$MMALPHA/doc/packages.}{Alpha.m}{Alpha`}
\index{docLink}

\alphausage{link}{link[file] creates an hyperlink button aiming at the Alpha
notebook "file.nb" in directory \$myNotebooks. This button may then
be cut-and-pasted anywhere. link[d, file] creates an
hyperlink button aiming at notebook "file.nb" in directory d of
\$myNotebooks.}{Alpha.m}{Alpha`}
\index{link}

\alphausage{exportAlphaFunctions}{exportAlphaFunctions[] writes DeclarePackage commands for all functions 
of the \MMAlpha{} standard packages. This function is automatically executed
when loading Alpha.m (see after the EndPackage instruction) whenever the
autoload.m does not exists, and creates this file. The autoload.m file 
is located in the Alpha directory. Each time you want to add a new package
to \MMAlpha{}, add an entry in the local variable "contexts" of Alpha.m, then 
remove the autoload.m file: a new one will be created.}{Alpha.m}{Alpha`}
\index{exportAlphaFunctions}

\alphausage{setMMADir}{setMMADir[List[\_\_\_String]] does the SetDirectory[] function work, but 
the path is specified as a list of strings and hence is valid on 
different plateforms (Windows, Unix).}{Alpha.m}{Alpha`}
\index{setMMADir}

\alphanote{setMMADir}{Actually this function is not
needed as Mathematica changes on the fly the path names depending on 
the platform.} 
 
\alphausage{wrap}{wrap[exp,contextFile,result] evaluates the expression exp in the
context given in the file contextFile, and returns the result in 
the file result. This function is not currently finished.}{Alpha.m}{Alpha`}
\index{wrap}

\alphanote{wrap}{wrap is used for running examples in separate sessions
of Mathematica, for example, using a remote \MMAlpha{} installation. The
idea would be to control the execution, and to use only one licence
to run these examples. The price to pay would be the start time of 
Mathematica.} 
 
\alphausage{enTete}{enTete[symb1\_] prints out the standard skeleton for
programming a new function named symb1 in \MMAlpha{} (for developper use
only).}{Alpha.m}{Alpha`}
\index{enTete}

\alphausage{mute}{mute is an option (Boolean). If True, function prints absolutely no information.}{Alpha.m}{Alpha`}
\index{mute}

\alphausage{test1}{test1[] executes part 1 of tests. See documentation in doc/Tests.}{Alpha.m}{Alpha`}
\index{test1}

\alphausage{test2}{test2[] executes part 2 of tests. See documentation in doc/Tests.}{Alpha.m}{Alpha`}
\index{test2}

\alphausage{test3}{test3[] executes part 3 of tests. See documentation in doc/Tests.}{Alpha.m}{Alpha`}
\index{test3}

\alphausage{test4}{test4[] executes part 4 of tests. See documentation in doc/Texts.}{Alpha.m}{Alpha`}
\index{test4}

