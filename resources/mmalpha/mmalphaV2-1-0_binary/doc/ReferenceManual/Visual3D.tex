%
% File created on {2008, 3, 11, 16, 16, 25.845024} by makeDoc.
%
% Section header
\section{The Alpha`Visual3D` package } 
\label{label:Alpha`Visual3D`}
\alphanote{Visual3D}{Documentation revised on August 10, 2004} 
 
\alphausage{Visual3D}{The Alpha`Visual3D` package contains a few functions to visualize and animate 3D domains. The main function is vshow, other functions  are facets, and listPlanes. Some functions are exported only for debugging purposes.}{Alpha/Visual3D.m}{Alpha`Visual3D`}
\index{Visual3D}

\alphausage{facets}{facets[dom] computes the list of polygons corresponding to the domain dom.}{Alpha/Visual3D.m}{Alpha`Visual3D`}
\index{facets}

\alphausage{listPlanes}{listPlanes[l,p] returns the pair \{l,lp\} where lp is the lists of planes in p which contain point l.}{Alpha/Visual3D.m}{Alpha`Visual3D`}
\index{listPlanes}

\alphausage{maxR}{maxR is an option of vshow. It allows the viewpoint of the final picture to  be set. See stepR for more details.}{Alpha/Visual3D.m}{Alpha`Visual3D`}
\index{maxR}

\alphausage{minR}{minR is an option of vshow. It allows the r parameter of the viewpoint  to be changed. By default, minR is 1. See also stepR and maxR, and ViewPoint of Mathematica.}{Alpha/Visual3D.m}{Alpha`Visual3D`}
\index{minR}

\alphausage{orderPolygon}{only here for debugging purposes.}{Alpha/Visual3D.m}{Alpha`Visual3D`}
\index{orderPolygon}

\alphausage{stepR}{stepR is an option of vshow. It allows the step of the r parameter of the  viewpoint to be changed, when one wants the domains to be animated.  By default, stepR is 1. Combined with minR and maxR, it allows a sequence of pictures with viewpoint between minR and maxR, by steps stepR to be  drawn.}{Alpha/Visual3D.m}{Alpha`Visual3D`}
\index{stepR}

\alphausage{threeDDomainQ}{only here for debugging purposes.}{Alpha/Visual3D.m}{Alpha`Visual3D`}
\index{threeDDomainQ}

\alphausage{twoDDomainQ}{only here for debugging purposes.}{Alpha/Visual3D.m}{Alpha`Visual3D`}
\index{twoDDomainQ}

\alphausage{vp1}{vp1 is an option of vshow, allowing the first parameter of ViewPoint to be changed. Default value is 3.}{Alpha/Visual3D.m}{Alpha`Visual3D`}
\index{vp1}

\alphausage{vp3}{vp3 is an option of vshow, allowing the third parameter of ViewPoint to  be changed. Default value is 1.}{Alpha/Visual3D.m}{Alpha`Visual3D`}
\index{vp3}

\alphausage{vshow}{vshow[d] shows a 2D or 3D graphic picture of the 3D domain d.  vshow[var] shows the domain of variable var in \$result (see Options[vshow] for more details).}{Alpha/Visual3D.m}{Alpha`Visual3D`}
\index{vshow}

\alphausage{units}{only here for debugging purposes.}{Alpha/Visual3D.m}{Alpha`Visual3D`}
\index{units}

