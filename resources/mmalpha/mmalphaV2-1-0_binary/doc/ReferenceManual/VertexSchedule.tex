%
% File created on {2009, 4, 25, 18, 19, 40.128578} by makeDoc.
%
% Section header
\section{The Alpha`VertexSchedule` package } 
\label{label:Alpha`VertexSchedule`}
% 
\subsection{Recent modifications (as of Dec. 2004)}
An additional note regarding extensions of 
\texttt{VertexSchedule.m} to handle multi-rate dataflow systems is
available in file
\begin{verbatim}
$MMALPHA/doc/referenceManual/dataflow.tex
\end{verbatim}

\alphanote{VertexSchedule}{Documentation revised on August 1st, 2004.} 
 
\alphausage{VertexSchedule}{The Alpha`VertexSchedule` package is one of the two scheduling packages
of \MMAlpha{}. Functions of the Alpha`Schedule` package should be preferred,
but in some cases, VertexSchedule offers some interesting functionalities.
Symbols defined in this package are \$dependencyConstraints, 
\$optimalityConstraints, \$scheduleDepTable, \$scheduleLibrary, 
\$scheduleMDSol. Functions are checkSchedOptions, constOf, depComponents,
depCycles, depGraph, DomDimension, DomSingletonQ, DomTrueDimension,
eliminateVar, encodeSchedule, equations, ishow, isIndentityOnDim,
getLinPart, getUseSchedule, loadScheduleLibrary, matrixTransPart, 
polDomainQ, timeMinSchedConstraints, saveScheduleLibrary, scd, slackOf,
sortEquations, zpolDomainQ
}{Alpha/VertexSchedule.m}{Alpha`VertexSchedule`}
\index{VertexSchedule}

\alphausage{\$dependencyConstraints}{\$dependencyConstraints contains the  dependency constraints used for
solving the scheduling. It is a list whose entries correspond directly
to the entries of the deptable. The last element is the set of
dependency variables. This global variable is set when calling the
scheduleConstraints  function}{Alpha/VertexSchedule.m}{Alpha`VertexSchedule`}
\index{\$dependencyConstraints}

\alphausage{\$optimalityConstraints}{\$optimalityConstraints contains the  optimality (total time, etc)
constraints used for solving the scheduling. It is a list, whose
entries are pairs \{variables, constraints\}. This global variable is set
by the timeMinSchedConstraints function}{Alpha/VertexSchedule.m}{Alpha`VertexSchedule`}
\index{\$optimalityConstraints}

\alphausage{\$scheduleDepTable}{when set, \$scheduleDepTable containts the dependency table of \$result. It is
computed by a call to scd.}{Alpha/VertexSchedule.m}{Alpha`VertexSchedule`}
\index{\$scheduleDepTable}

\alphausage{\$scheduleLibrary}{\$scheduleLibrary containts the schedules found by the scheduler when
called on the elements of the current library. These schedules are used
when the scheduler is called with the structured option}{Alpha/VertexSchedule.m}{Alpha`VertexSchedule`}
\index{\$scheduleLibrary}

\alphausage{\$scheduleMDSol}{\$scheduleMDSol is an array which gathers information during a call to 
the scd function. \$scheduleMDSol[i] contains a list \{o,c,v,sol\}
consisting  of the objective function, constraints, variables, and
solution to the linear problem solved when dealing with dimension i of
the schedule. For a unidimensional schedule, only level 1 is set. For a
dataflow schedule, level 1 corresponds to  the dataflow leve, and other
levels to the multidimensional schedule of the  inside indexes.}{Alpha/VertexSchedule.m}{Alpha`VertexSchedule`}
\index{\$scheduleMDSol}

\alphausage{adaptUses}{adaptUses[] modifies all use statements, by adding extra input 
equations}{Alpha/VertexSchedule.m}{Alpha`VertexSchedule`}
\index{adaptUses}

\alphausage{listOfAdaptedSignals}{listOfAdaptedSignals is an option of adaptUses, False by default. If
True, adaptUses returns a list formed of the equivalence between the
adapted signals and the old signals}{Alpha/VertexSchedule.m}{Alpha`VertexSchedule`}
\index{listOfAdaptedSignals}

\alphausage{addSchedule}{addSchedule is an option of loadScheduleLibrary. Default value is 
True. If True, the schedule contained in file sys.scdlib is added
to \$scheduleLibrary, where it replaces a previous schedule for the
current system.}{Alpha/VertexSchedule.m}{Alpha`VertexSchedule`}
\index{addSchedule}

\alphausage{affineDepConsts}{affineDepConsts[ dp, v, dur, taus, alphas, opts ] computes the constraints
due to dependency do, on variable v, with duration dur, coefficients taus and
alpha, and options opts.}{Alpha/VertexSchedule.m}{Alpha`VertexSchedule`}
\index{affineDepConsts}

\alphausage{checkSchedOptions}{checkSchedOptions[options] checks the options against the 
options of scd}{Alpha/VertexSchedule.m}{Alpha`VertexSchedule`}
\index{checkSchedOptions}

\alphausage{constOf}{constOf[symbol] gives the value of symbol}{Alpha/VertexSchedule.m}{Alpha`VertexSchedule`}
\index{constOf}

\alphausage{depComponents}{depComponents[] computes the strongly connected components of 
the dependence graph}{Alpha/VertexSchedule.m}{Alpha`VertexSchedule`}
\index{depComponents}

\alphausage{durationsNonZero}{durationsNonZero is an option of scd. It gives the rank in the dependence
table of the dependences the duration of which is non zero. Those durations
are assumed to be 1. A negative rank is counted from backward from the end
of the dependence table.}{Alpha/VertexSchedule.m}{Alpha`VertexSchedule`}
\index{durationsNonZero}

\alphausage{depCycles}{depCycles[] computes the elementary cycles in the dependence graph. 
depCycles[ selfDep$ \rightarrow $False ] computes the elementary cycles, 
without the self dependences}{Alpha/VertexSchedule.m}{Alpha`VertexSchedule`}
\index{depCycles}

\alphausage{depGraph}{depGraph[] computes and displays the dependence graph of \$result in
the format of the Combinatorica package. This function assumes that
the dependence graph has already been computed. See also Options[
depGraph ]. Inputs are green, outputs are blue and selfvertices are
red.}{Alpha/VertexSchedule.m}{Alpha`VertexSchedule`}
\index{depGraph}

\alphausage{depGraphViz}{depGraphViz[] computes and displays the dependence graph of \$result in
the format of the GraphViz package. This function assumes that
the dependence graph has already been computed. See also Options[
depGraph ]. Inputs are green, outputs are blue and selfvertices are
red. Option displaySchedule allows the schedule to be displayed.}{Alpha/VertexSchedule.m}{Alpha`VertexSchedule`}
\index{depGraphViz}

\alphausage{displaySchedule}{displaySchedule is an option of depGraphViz that adds the schedule
of all variables}{Alpha/VertexSchedule.m}{Alpha`VertexSchedule`}
\index{displaySchedule}

\alphausage{showViz}{showViz[] draws a dot file for the alpha program}{Alpha/VertexSchedule.m}{Alpha`VertexSchedule`}
\index{showViz}

\alphausage{variables}{variables is an option of several functions, for example depGraph or
appSched. Default value is all, meaning all variables of a system}{Alpha/VertexSchedule.m}{Alpha`VertexSchedule`}
\index{variables}

\alphausage{extraEdges}{Option of depGraph. Seems to be unused...}{Alpha/VertexSchedule.m}{Alpha`VertexSchedule`}
\index{extraEdges}

\alphausage{variables}{Option of depGraph. Default value is all. Allows one to speficy the list
of variables to which the graph is restricted.}{Alpha/VertexSchedule.m}{Alpha`VertexSchedule`}
\index{variables}

\alphausage{factor}{Option of depGraph. Default value is 1. Gives a size factor for arrow heads
and points of the dependence graph.}{Alpha/VertexSchedule.m}{Alpha`VertexSchedule`}
\index{factor}

\alphausage{labelSize}{Option of depGraph. Default is 0.1. Size of the labels.}{Alpha/VertexSchedule.m}{Alpha`VertexSchedule`}
\index{labelSize}

\alphausage{labelFactor}{Option of depGraph. Default is 1.2. Gives the distances of the labels 
from the center of the graph.}{Alpha/VertexSchedule.m}{Alpha`VertexSchedule`}
\index{labelFactor}

\alphausage{DomDimension}{DomDimension[d] gives the dimension of domain d, including the 
parameter dimension (see DomTrueDimension to get the dimension 
with de parameter dimension)}{Alpha/VertexSchedule.m}{Alpha`VertexSchedule`}
\index{DomDimension}

\alphausage{DomSingletonQ}{DomSingletonQ[sys, d] is True if the true dimension of d is 0 and 
if d is the universal domain. DomSingletonQ[d] is equivalent to 
DomSingletonQ[\$result,d]}{Alpha/VertexSchedule.m}{Alpha`VertexSchedule`}
\index{DomSingletonQ}

\alphausage{DomTrueDimension}{DomTrueDimension[sys,d] gives the dimension of domain d less the 
dimension of the parameter space of sys. DomTrueDimension[d] is 
equivalent to DomDimension[\$result,d]}{Alpha/VertexSchedule.m}{Alpha`VertexSchedule`}
\index{DomTrueDimension}

\alphausage{eliminateVar}{eliminateVar[v] removes variable v, after checking that the variable
is useless}{Alpha/VertexSchedule.m}{Alpha`VertexSchedule`}
\index{eliminateVar}

\alphausage{encodeSchedule}{encodeSchedule[sys, params, sol, opts] returns the encoding of the
schedule sol of system sys, with parameters params, using options opts.
Among the options of encodeSchedule, one option allows one to choose
between the Farkas format and the Vertex format}{Alpha/VertexSchedule.m}{Alpha`VertexSchedule`}
\index{encodeSchedule}

\alphausage{equations}{equations is an option value of the option variables of depGraph. If 
True, the variables of are ordered in the order given by the equations, 
with input first.}{Alpha/VertexSchedule.m}{Alpha`VertexSchedule`}
\index{equations}

\alphausage{ishow}{ishow[] pretty-prints \$result, without calling the external C 
program. It supports some features of the imperative form (such as
aliases) which ashow does not support.}{Alpha/VertexSchedule.m}{Alpha`VertexSchedule`}
\index{ishow}

\alphausage{isIdentityOnDim}{isIdentityOnDim[mat,\{1,3\}] (for instance) 
checks that matrix mat1 is identity on dimension 1 and 3}{Alpha/VertexSchedule.m}{Alpha`VertexSchedule`}
\index{isIdentityOnDim}

\alphausage{getLinPart}{getLinPart[m] computes the mma form of the linear part of the Alpha
matrix m, considered as an affine function. Note: use getLinearPart
instead.}{Alpha/VertexSchedule.m}{Alpha`VertexSchedule`}
\index{getLinPart}

\alphausage{getUseSchedule}{getUseSchedule[ x ] obtains the schedule of the use entry of a dependency table x,
by looking in the global variable \$scheduleLibrary.}{Alpha/VertexSchedule.m}{Alpha`VertexSchedule`}
\index{getUseSchedule}

\alphausage{loadScheduleLibrary}{loadScheduleLibrary[sys] loads in global variable \$scheduleLibrary 
the content of the file sys.scdlib}{Alpha/VertexSchedule.m}{Alpha`VertexSchedule`}
\index{loadScheduleLibrary}

\alphausage{matrixTransPart}{matrixTransPart[m] computes the mma form of the translation part of
the Alpha matrix m, considered as an affine function.}{Alpha/VertexSchedule.m}{Alpha`VertexSchedule`}
\index{matrixTransPart}

\alphausage{parameterRules}{parameterRules is an option of loadScheduleLibrary. Default value
is \{\}. It contains replacement rules for the parameters of the 
schedule which is loaded. For example, if option parameterRules $ \rightarrow $ \{"K"$ \rightarrow $2\} 
is set, then parameter "K" will be replaced by value 2 while reading
the schedule.}{Alpha/VertexSchedule.m}{Alpha`VertexSchedule`}
\index{parameterRules}

\alphausage{periodicFactor}{periodicFactor is an option of getUseSchedule. Its default value is 1.The first parameter of the schedule of a subsystem is multiplied byfactor}{Alpha/VertexSchedule.m}{Alpha`VertexSchedule`}
\index{periodicFactor}

\alphausage{saveScheduleLibrary}{saveScheduleLibrary[sys] saves the content of global variable \$scheduleLibrary 
in the file sys.scdlib. saveScheduleLibrary[] saves the content of \$scheduleLibrary in
the file name.scdlib, where name is the name of the system currently contained
in \$result. To save only the schedule of the system in \$result, use 
saveScheduleLibrary[ onlyMainSystem $ \rightarrow $ True ].}{Alpha/VertexSchedule.m}{Alpha`VertexSchedule`}
\index{saveScheduleLibrary}

\alphausage{onlyMainSystem}{onlyMainSystem is an option of saveScheduleLibrary. If True, only 
the schedule of the system in \$result is saved, otherwise (default)
all schedules in \$scheduleLibrary are saved.}{Alpha/VertexSchedule.m}{Alpha`VertexSchedule`}
\index{onlyMainSystem}

\alphausage{timeMinSchedConstraints}{Timeminschedconstraints[] computes the constraints needed to minimize
the total execution time of
\$result. timeMinSchedConstraints[var\_String] computes only the
dependence constraints for variable var. Options of
timeMinSchedConstraints are sameLinearPart (default False),
addConstraints (\{\}) and verbose (False).}{Alpha/VertexSchedule.m}{Alpha`VertexSchedule`}
\index{timeMinSchedConstraints}

\alphausage{saveSchedule}{saveSchedule is an option of scd. If True, \$schedule is saved in 
file sys.scd, where sys is the name of the scheduled system.}{Alpha/VertexSchedule.m}{Alpha`VertexSchedule`}
\index{saveSchedule}

\alphausage{scd}{***scd[] finds out the schedule that minimizes the total computation time
for a bounded Alpha program. Fails if the system is not bounded. scd can 
also be used to find a multidimensional schedule, or a dataflow schedule.
An option allows a structured schedule to be found.}{Alpha/VertexSchedule.m}{Alpha`VertexSchedule`}
\index{scd}

\alphausage{periods}{periods is a parameter of scd. It is the list of periods associated with
the subsystems, whenever these subsystems contain the parameters "\$P".}{Alpha/VertexSchedule.m}{Alpha`VertexSchedule`}
\index{periods}

\alphausage{onlyUseDep}{onlyUseDep is an option of scd that specifies the use dependencies that are
to be kept. Default value is all, otherwise, it is a list of integers.}{Alpha/VertexSchedule.m}{Alpha`VertexSchedule`}
\index{onlyUseDep}

\alphausage{simplex}{simplex[o,c,v] finds the minimum of function o on contraints c (list) 
for list of variables v. Variables are not assumed to be linear}{Alpha/VertexSchedule.m}{Alpha`VertexSchedule`}
\index{simplex}

\alphausage{slackOf}{slackOf[listOfConstraints] gives the difference between the lhs and 
the rhs in the constraints, using the solution contained in \$scheduleMDSol[1].
This function allows somehow to explore a given schedule.}{Alpha/VertexSchedule.m}{Alpha`VertexSchedule`}
\index{slackOf}

\alphausage{slg}{slg[g] shows graph g. slg[g,Directed] shows g as a directed graph.
This function is a slight modification of ShowLabeledGraph.}{Alpha/VertexSchedule.m}{Alpha`VertexSchedule`}
\index{slg}

\alphausage{pal}{}{Alpha/VertexSchedule.m}{Alpha`VertexSchedule`}
\index{pal}

\alphausage{sortEquations}{sortEquations[sys:\_system] returns a the same system where the equation 
have been reordered in such a way that all computation executed 
simultaneously (with duration 0) can be executed in the textual order,
it currently assumes that the time is one dimensionnal and the first indice.}{Alpha/VertexSchedule.m}{Alpha`VertexSchedule`}
\index{sortEquations}

\alphausage{zpolDomainQ}{zpolDomainQ[d] is true if d is a domain with z-polyhedra}{Alpha/VertexSchedule.m}{Alpha`VertexSchedule`}
\index{zpolDomainQ}

\alphausage{matrix2mma}{matrix2mma[m] computes the mma form of the Alpha matrix m.}{Alpha/VertexSchedule.m}{Alpha`VertexSchedule`}
\index{matrix2mma}

\alphausage{eqsDomain}{}{Alpha/VertexSchedule.m}{Alpha`VertexSchedule`}
\index{eqsDomain}

\alphausage{statScheduleConstraints}{statScheduleConstraints[] does as scheduleConstraints, but provides 
a few statistics}{Alpha/VertexSchedule.m}{Alpha`VertexSchedule`}
\index{statScheduleConstraints}

\alphausage{summaryScheduleConstraints}{summaryScheduleConstraints[] computes the constraints needed to minimize the 
total time, and projects these constraints on the Input and Output variables.}{Alpha/VertexSchedule.m}{Alpha`VertexSchedule`}
\index{summaryScheduleConstraints}

\alphausage{variablesOf}{variablesOf[ exp ] gives the list of variables in the 
list of expression exp}{Alpha/VertexSchedule.m}{Alpha`VertexSchedule`}
\index{variablesOf}

