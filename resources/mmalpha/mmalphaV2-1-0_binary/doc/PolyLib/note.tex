\documentclass[12pt]{article}
\begin{document}

\newcommand{\polylib}{{\sc PolyLib}}
\title{A note regarding the installation of \polylib{}}
\author{Patrice Quinton}
\date{October 23, 2007}
\maketitle
This short note contains a few remarks regarding the installation of \polylib{}.

%\section{Installing the 5.02 version}
%The version I run was 5.02, which I inherited from Tanguy Risset. The source should be
%on 
%\begin{verbatim}
%http://icps.u-strasbg.fr/~loechner/polylib
%\end{verbatim}
%but I did not check whether this version is the most current one.

%After untaring the tar file, I got a directory where the \texttt{doc/INSTALL} file
%contains some instructions, but this does not work. 

%To obtain an executable, I did what Tanguy said:
%\begin{verbatim}
%./autogen.sh
%./configure --prefix="path"
%make
%make install
%\end{verbatim}

%The first command runs the \texttt{autogen.sh} file. However, for an unknown reason, 
%on the MacIntosh, the \texttt{libtoolize} tool is named \texttt{glibtoolize}... So the second
%line of the \texttt{autogen.sh} file has to be corrected.

\section{Installing the version of Strasburg}
I got the most recent available version of \polylib{} on site 
\begin{verbatim}
http://icps.u-strasbg.fr/~loechner/polylib
\end{verbatim}
and I tried to install it. In this version, the installation documentation is correct, as
\texttt{autoload.sh} does not need to be run prior to \texttt{configure}. By running
only 
\begin{verbatim}
./configure --prefix="$MMALPHA/sources/Poly"
make
make install
\end{verbatim}
The second command runs the configure tool that creates \texttt{makefiles} and prepares
libraries. The \texttt{--prefix} option specifies where the \polylib{} is to be created.

The third commands compiles everything, and the \texttt{make install} command 
creates the library. It adds binary files in the directory \texttt{path/bin}, libraries in directory \texttt{path/lib}, and 
finally, include files in directory \texttt{path/include}.

I got a runnable version, that I checked using \texttt{make longtest}. The 
\texttt{make test} command indicated in the install documentation does not work.

In this package, the html documentation is available, but the latex source files 
are not.

The documentation is also available.
\section{Tests}
Running \texttt{make check} runs the tests which are in directory \texttt{Tests}.

\section{Additional remarks}
Depending on whether you are on PPC or Intel processor, you have to recompile 
the Polylib. But this is not sufficient to generate a correct Domlib: you should recompile
Domlib, and make sure that {\em Read\_Alpha and other software are also recompiled}, 
as currently, this is how the lib.darwin directory is updated. If you do not do so, then
compiling Domlib may fail. 

\section{Exploring web sites}
There are now several libraries for doing polyhedral computations. A survey of what these
libraries do (or cannot do) would be useful.


\end{document}
 