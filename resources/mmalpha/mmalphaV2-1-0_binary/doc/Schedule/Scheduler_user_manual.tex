\documentclass[11pt,fullpage]{article}

\usepackage{makeidx}
\newcommand{\aalpha}{{\sc Alpha}}
\newcommand{\ammalpha}{{\sc MmAlpha}}
\newcommand{\err}{\top}
\newcommand{\indef}{\bot}
\newcommand{\mma}{{\sc Mathematica}}
\makeindex

\title{User manual of  the Alpha {\bf Scheduler}\\
Version of \today{}}
\author{Tanguy Risset and Patrice Quinton}
\date{\today}

\begin{document}
%\bibliographystyle{plain}

\maketitle
\tableofcontents
\newpage
This documentation\footnote{It improves and replaces the previous 
scheduler manual that was  called \texttt{docSched.tex}. Unfortunately, this documentation 
in far from being complete.} provides some information regarding the {\aalpha} 
schedulers. It is available in the file:\\
{\tt \$MMALPHA/doc/Schedule/Scheduler\_user\_manual.pdf}

Section~\ref{Introduction} describes the principles of the Farkas and
the vertex schedulers. Section~\ref{sec2} presents the main utilization modes
of the \texttt{schedule} function, 
while section~\ref{sec3} describes the options of this function. 
In section~\ref{Result}, the output format of 
the scheduler is detailed. 
Section~\ref{settings} describes briefly the \texttt{Schedule.m}
package. A short example is shown in section~\ref{Examples}. 
Troubles are described in section~\ref{troubles}.

\section{Introduction}
\label{Introduction}
The scheduler is a package of {\ammalpha} that attempts 
to find a schedule for a given {\aalpha} program. An {\aalpha} 
program~\cite{Wilde93,Mauras89} has no sequential ordering: 
any sequential or parallel ordering of the computations is semantically 
valid provided it respects the data dependencies of the program.


Scheduling an {\aalpha} program
consists in giving a computation date for each variable of the
program. The time is considered as a discrete
single clock. The basic goal of the scheduler is to find a valid execution order
that is {\em good} with respect to a particular criterion. The
theoretical basis for the scheduling process is inherited from 
research on
systolic array synthesis and on automatic
parallelization. Currently two techniques are implemented for
the computation of a scheduling function, we will call them the {\em
Farkas method}~\cite{Feautrier92aa} (method used by default), and the
{\em vertex method}~\cite{MQRS90}. 



In order to compute a schedule, the scheduler gathers all the constraints that
the schedule must meet and builds a linear program (LP) which is then
solved with a particular software.  
The two implemented methods
proceed differently but both give a {\em
mono-dimensional affine by variable} schedule. They can also provide
a {\em multi-dimensional affine by variable} schedule by changing the
options of the schedule (see section~\ref{sec2}).

A mono-dimensional affine by variable schedule assigns to each \aalpha{} equation of
the form
{\tt
A[i,j] = \ldots }, an execution date $T_A(i,j)$  which 
is given by an affine function of the indices
($i,j$) and of the system parameters ($N$):
$$T_{A}(i,j) = \tau^i_{A}
i + \tau^j_{A} j+ \tau^N_{A} N + \alpha_{A} \quad .$$ 
The vector $(\tau^i_{A} ,
\tau^j_{A},\tau^N_{A})$ is called the {\em scheduling vector}.  For
instance: $T_A(i,j) = 2i + j +3$ is a mono-dimensional affine schedule
for variable $A$; Its corresponding scheduling vector is $(2,1)$.

One important subset of mono-dimensional affine schedules are {\em  affine with
constant linear part} schedules, where the value of the scheduling vector
is  the same for each local variable of the program;
In this case, all variables must have
the same number of indices.

Multidimensional schedules are used for programs that do not admit
a linear execution time (for instance, if the execution time is
$N^2$). In this case, the schedule function is a vector of linear
functions and the order that is defined by the schedule 
is the lexicographic order on the components
(see~\cite{Feautrier92bb} for details).


\section{The \ammalpha{} \texttt{schedule} function }
\label{sec2} 
The  function to be called  in order to schedule a program is
 {\tt schedule[]}. Its effects is to schedule the {\aalpha} program contained in 
 the 
 {\tt \$result} \ammalpha{} variable,  and to put the result in 
 the global variable {\tt \$schedule} (see section~\ref{sec5} for the structure of the
resulting schedule). Options allow several parameters to be changed (see section~\ref{sec3}).

The possible forms of the use of the function {\tt schedule} are:
\begin{itemize}
\item {\tt schedule[]}: finds an affine by variable schedule for {\tt
\$result} which minimizes the global execution time of the system and assigns this schedule to
{\tt \$schedule}.
\item {\tt schedule[sys]}:
finds an affine by variable schedule for the {\aalpha} system {\tt sys}; the schedule minimizes 
the global execution time. The resulting schedule is assigned to
{\tt \$schedule}
\item {\tt
schedule[option\_1->value\_1,\ldots,option\_n->value\_n]}:
finds a schedule for {\tt \$result} which respects the chosen options and
assign it to {\tt \$schedule}
\item {\tt
schedule[sys,option\_1->value\_1,\ldots,option\_n->value\_n]}\\
find a schedule for the {\aalpha} system {\tt sys} which respects the chosen options and
assigns it to {\tt \$schedule}.
\end{itemize}

\section{Options of the \texttt{schedule} function}
\label{sec3}
Options have default values indicated hereafter. To change these
values, put one of the corresponding rules as a parameter to the {\tt
schedule} function. 
The Farkas method and the vertex method have
completely different implementations, hence their options are sometimes
different. Choosing between the Farkas method
and the vertex method is done with the {\tt schedMethod} option. 
\subsection{Main options}
\subsubsection*{schedMethod }
\label{schedMethod}
This option\footnote{As of \today{}, this option does not work. To use the 
vertex method, run the \texttt{scd[]} command.} allows one to switch from the Farkas method to the vertex
method. Its type is symbolic. Changing the value of this option greatly
influences the other options (for instance, it changes default values
of some options: {\tt durationByEq},...), be sure to check the others
options you use. The possible values are:
\begin{itemize}
\item {\tt schedMethod -> farkas} (default) selects the Farkas method.
\item {\tt schedMethod -> vertex} selects the vertex method.   
\end{itemize}
One of the main differences between the two methods is that the Farkas
method uses the \texttt{Pip} software to solve the LP while the vertex method
uses the \mma{} linear solver.

\subsubsection*{scheduleType}
\label{schedType}\index{scheduleType}
This option gives the type of schedule that \texttt{schedule} looks for.
Its type is symbolic, and its possible values are:
\begin{itemize}
\item {\tt scheduleType -> affineByVar} (default): affine by variable schedule.
\item {\tt scheduleType -> sameLinearPart}:   affine by variable scheduling with constant linear part. This option is often used for systolic designs. 
\item {\tt scheduleType -> sameLinearPartExceptParam}:  affine by variable schedule with constant linear part except for the parameters.
\end{itemize}
The default option allows one to get any kind of affine schedule. Option \texttt{sameLinearPart}
constraints the schedule to have all a common linear part. As a consequence, all variables
must have the same dimension (not checked).

\subsubsection*{multidimensional}
\label{multiDimensional}\index{multidimensional}
By default, the scheduler looks for a mono-dimensional schedule. If the
\texttt{multidimensional}�is set, the scheduler also looks for multi-dimensional 
schedules. 
This option is boolean, hence
its possible values are:
\begin{itemize}
\item {\tt multidimensional -> False} (default): the schedule will be mono-dimensional.
\item {\tt multiDimensional -> True}: the schedule will be
multidimensional. If you set this options, be aware that it greatly
influences the value of other options: {\tt addConstraints}, {\tt durations} 
must be of type list of list (one list by schedule dimension, this means that you have to know the number of dimensions of the resulting schedule). Options {\tt optimizationTime} is automatically set to multi.
\end{itemize}

\subsubsection*{optimizationType}
\label{optimizationType}\index{optimizationType}
This option sets the objective function chosen. Its type is symbolic, and
 its possible values are:
\begin{itemize}
\item {\tt optimizationType -> time} (default): the total latency is
minimized (In the Farkas method, this minimization always correspond 
to the lexicographic minimization of the coefficients of the global
execution time which is an affine function of the parameters. In the
vertex method the way this minimization is performed can be changed by
the option {\tt objFunction}).
\item {\tt optimizationType -> Null }: no objective function (the
coefficients of the scheduling vectors are minimized in a
lexicographic order)
\item {\tt optimizationType -> delay}: tries to minimize the delays on
the dependencies (not implemented currently).
\item {\tt optimizationType -> multi} is for internal use, when a
multidimensional scheduling is computed, just be aware that the three above values for this option are not compatible with 
multi-dimensional scheduling.
\end{itemize}

%\subsubsection*{ratOrInt}
%This option indicates whether the resolution of the LP by mppip is
% done in integer mode (resulting scheudling vectors with integral
% coefficients) or in rationnal mode (resulting scheduling vectors may contains
% rationnal coefficients).  Its type is integer, the possible values
% are:
%\begin{itemize}
%\item {\tt \{ratOrInt -> 0\} } : rationnal solution, in this case, the
%scheduling function obtained by taking the floor of the scheduling
%functions is valid (\cite{Feautrier92aa}).
%\item {\tt \{ratOrInt ->  1\} } : (default) integer solution
%\end{itemize}

\subsubsection*{addConstraints}
\label{addConstraints}\index{addConstraints}
This option allows one to add some constraints to the generated LP.
Adding constraints is very important if you want to control the
resulting schedule.  The type of this option is a list of strings, each one of which
represent a constraint. The type of constraints authorized are
affine constraints on scheduling vectors.
Constraints can have two forms: one can force the value of 
a scheduling
vector or one can force coefficients of schedule to meet some constraints:
\begin{itemize}
\item Forcing a  variable $A[i,j]$ to be scheduled at time $i+2j+2$ can be done 
by the adding {\tt "TA[i,j]=i+2j+2"} to the constraints list.
\item For more precise constraints,  one can directly access each component 
of the schedule functions of each variable. For instance {\tt TAD2}
represents coefficient $\tau_A^j$ of the schedule of $A$ and {\tt CA} represents
the constant coefficient $\alpha_A$). With these names, one can add linear constraints 
on these coefficients using operators {\tt ==} or {\tt >=}. For instance,\\
{\tt \{"TAD1 == 1","TAD2 == 2", "CA >= 2" \} }\\is the same constraint
as above except that the constant is allowed to be greater than two.
\end{itemize}  
These options can also be used during the search of a multi-dimensional schedule.
In that case, its value is a list of lists of string, each
list of strings corresponding to constraints imposed on a dimension of the schedule.
Example of
use for mono-dimensional scheduling:
\begin{verbatim}
schedule[addConstraints-> {"TA[i,j,N]=i+2j-2",
    "T1D2==2","T1D1+2T1D3>=1"}]}
\end{verbatim}

\subsection*{durations}
\label{durations}\index{durations}
This option 
allows the user to specify precisely the execution time for each equation
or for each dependence. By default, each equation is assumed to need exactly
one cycle to be evaluated. Sometimes, one want to be more precise, 
especially when designing a circuit. For example, an equation of 
the form \texttt{A = B} imposes that \texttt{A} be scheduled one cycle
after \texttt{B}. 

The type of this option is a
list of integers. It may be affected by the value of the {\tt durationByEq}
option.

For the Farkas method, 
the possible values of this option are:
\begin{itemize}
\item {\tt durations -> \{\}} (default): each equation 
%(if durationByEq is True) 
lasts one cycle.
%(resp. each dependence is 1 if durationByEq is False),
% whatever complex is the computation (default value).
\item {\tt duration -> \{0,0,1,1,0,...0,3,1\}} (list of integers):
\begin{itemize}
\item If the \texttt{durationByEq} option is set to \texttt{True} (default in the Farkas
method): integer number $i$ indicates the duration of the equation
defining variable number $i$ of the Alpha program. The 
variables are numbered in their order of declaration in the 
\aalpha{} system:  input variables,
then output variables, and finally local variables.
\item If the \texttt{durationByEq} option is set to \texttt{False} (default in the vertex method),
 integer number $i$ indicating the duration of dependence number $i$.  The 
 dependencies are numbered in the order given by the \texttt{dep} function: to
 get the list of dependences run the function \texttt{show[ dep[] ]}.
\end{itemize}
\end{itemize}
\paragraph*{Note:} as of \today, I think that the option \texttt{durationByEq -> True} 
is only valid for the Farkas method and the option \texttt{durationByEq -> False} is only valid
for the vertex method. 


\subsubsection*{outputForm}
\label{outputForm}\index{outputForm}
This options allows a non standard output to be obtained. The
standard schedule output is described in section~\ref{sec5}. 
One can
also obtain as a result the linear program to be solved in various formats, or the
schedule polytope, i.e. the polytope which contains all the valid
solutions in the \aalpha{} format. The possible values of \texttt{outputForm} are:
\begin{itemize}
\item {\tt outputForm -> scheduleResult} (default): standard schedule
output form (Alpha`ScheduleResult structure, see section~\ref{sec5}).
\item {\tt outputForm -> lpSolve}: outputs in the format of the
lp\_solve software,
the linear programming
problem to solve in order to find the schedule.
\item {\tt outputForm -> lpMMA}: outputs the linear programming problem
to solve in order top find the schedule in the format of the linear
solver of Mathematica. Warning; as of \today{}, this option is not
implemented.
\item {\tt outputForm -> domain}: outputs the schedule polytope,
i.e. the \aalpha{} domain which is composed of all the constraints of the LP
to solve. WARNING, this works only for SMALL programs (~3 instructions), otherwise the domain is too big to be handled by polylib.
\end{itemize}

\subsubsection*{debug}
\index{debug}\label{debug}
Prints additional information and does not destroy the temporary files build
for the interface with PIP. The type of this option is boolean, possible value are:
\begin{itemize}
\item  {\tt  debug -> False} (default):   debug mode not set.
\item  {\tt  debug -> True }: debug mode set.
\end{itemize}

\subsubsection*{verbose}
\index{verbose}\label{verbose}
 If set, the scheduler returns additional information, otherwise, 
 it returns only the result. 
 This option is boolean, its possible values are:
\begin{itemize}
\item  {\tt  verbose -> True} (default): normal printing.
\item  {\tt  verbose -> False}: nothing is printed out.
\end{itemize}

\subsection{Advanced options}
These options are here for an advanced use of the schedule function.
\subsubsection*{resolutionSoft}
This function indicates which software will be used to solve the LP. Its type is symbolic, the possible values are: 
\begin{itemize}
\item {\tt resolutionSoft -> pip}: uses P. Feautrier's PIP software (only
available for the Farkas method).
\item {\tt resolutionSoft -> mma}: uses the \texttt{ConstrainedMin} function of
\mma{} (only implemented in the vertex method).
\item {\tt resolutionSoft -> lpSolve}: uses the {\tt lp\_solve} software(not implemented yet).
\end{itemize}

\subsubsection*{objFunction}
\index{objFunction}
This option is used to indicate how the minimization of the objective function 
(which is usually a function of the parameters) is performed. 
Its type is symbolic, the possible values are: 
\begin{itemize}
\item {\tt objFunction -> lexicographic}: minimizes   lexicographically 
the coefficient  of this function in the order of
the declaration of the corresponding parameters in the Alpha program (default in Fakas implementation).
\item {\tt objFunction -> lexicographic["N","P","M"]}: (not implemented anywhere) 
\item {\tt objFunction -> 2"N" + "P"} minimize 2 time the coefficient of parameter "N" plus one time the coefficient of parameter "P" (only implemented in the vertex resolution).
\end{itemize}

\subsubsection*{onlyVar}
\index{onlyVar}
Indicates which variables to schedule (useful if you have a very long program and which to schedule only part of it. Its type is a list of strings, the possible values are: 
\begin{itemize}
\item {\tt onlyVar -> all} (default value): schedules all the variables.
\item {\tt onlyVar -> \{"a","B","c"\}} (list of strings): schedules only the specified variables,   Warning, if  some variables are
   needed for the computation of the variables listed in the option,  the function will try to
   find their execution dates in \$schedule (This feature is only implemented in the Farkas method).
\end{itemize}

\subsubsection*{onlyDep}
\index{onlyDep}
Indicates which dependences to schedule (used for multidimensional scheduling).
Its type is a list of
integers and its possible values are:
\begin{itemize}
\item {\tt onlyDep -> all} (default value): schedules all the dependencies.
\item {\tt onlyDep -> \{1,4,5\}} (list of integers): schedules only the specified dependencies, the number correspond to their position in the list of dependencies which is returned by  the dep[] function (only implemented in the Farkas method).
\end{itemize}

\subsubsection*{subSystems}
\index{subSystems}
Indicates
whether or not the schedule takes into account calls to other subsystems.
\begin{itemize}
\item {\tt subSystems -> False}: (default value).
\item {\tt  subSystems -> True}.
\end{itemize}
In the Farkas method, the default value is not to consider subsystems, and if there
are some calls, the method fails. Therefore, to schedule a structured system, one
must run:
\begin{verbatim}
schedule[ subSystem -> True ]
\end{verbatim}
or equivalently:
\begin{verbatim}
structSched[]
\end{verbatim}


\subsubsection*{subSystemsSchedule}
\index{subSystemsSchedule}
Indicates where are the schedules of the subsystems used in the 
system we schedule. Its type is a list of schedules (see the format in section~\ref{sec5}.
\begin{itemize}
\item {\tt subSystemsSchedule -> \$scheduleLibrary} (default value).
\item {\tt  subSystemsSchedule -> List[schedule..]}.
\end{itemize}
Note that the scheduler automatically appends to \texttt{\$scheduleLibrary} 
the schedules of all new systems that it schedules. 

\section{Format of \texttt{schedule}'s results}
\label{Result}
\label{sec5}
The result given by the function {\tt schedule} has a special form.
Two new head names are
introduced: {\tt Alpha`scheduleResult} and {\tt Alpha`sched}.  

The
outermost structure is  a structure starting with the head {\tt
Alpha`ScheduleResult}, where the first argument is the name of the system
(string)  and where the
second argument is the schedule itself. 

The Last argument of {\tt Alpha`scheduleResult} is for internal use.

The syntax of this structure is
described here.\\
\begin{verbatim}
<schedResult> = 
  Alpha`ScheduleResult[ 
    scheduleType_Integer, <sched3List>, objFunc_]
    
<sched3List> = { nameVar_String, indices_List, 
  Alpha`sched[ tauVector_List, constCoef_Integer ] }
\end{verbatim}
Example: 
\begin{verbatim}
scheduleResult[ "test1",
   {{"a", {"i", "k"}, sched[ {2, 2}, -4]}, 
    {"b", {"j"}, sched[ {0}, 0]}, {"c", {"i"}, sched[{0}, 15 ]}, 
    {"A", {"i", "k"}, sched[ {2, 2}, -3 ]}, 
    {"C", {"i", "k"}, sched[ {2, 2}, -2 ]}}, {15} ]
\end{verbatim}

Other examples are given in section~\ref{Examples}.


\section{Technical settings}
\label{settings}
The packages is called {\tt Schedule.m}. 
It uses packages \\{\tt FarkasSchedule},
{\tt VertexSchedule} and {\tt scheduleTools.m}. 

The Farkas implementation
uses the {\tt pip} software (version D.1~\cite{Feautrier88b}) which must be present in the 
binary directory\footnote{Explain where...}.

The communication between \mma{} and {\tt pip} is done by files. These
files are written in the directory indicated by the \mma{} variable \texttt{\$tmpDirectory}.  
Currently, all methods should work on Solaris, Windows and MacOS X plateforms.

\section{Examples}
\label{Examples} 
Consider the program of Fig.~\ref{figureEx}. In the following, we
give the result of the scheduler with different options.
\begin{figure}[htbp]
\begin{verbatim}
system matmult : {M |M>1} 
  (a,b : {i,j | 1<=i,j<=M} of real)
returns
  (c : {i,j | 1<=i,j<=M } of real);
var
  C : {i,j,k | 1<=i,j<=M; 0<=k<=M} of real; 
let
  c[i,j] = C[i,j,M];
  C[i,j,k] = case
        {|k=0} : 0[];
        {|1<=k<=M} : C[i,j,k-1]+a[i,k]*b[k,j];
  esac;
tel;
\end{verbatim}
\caption{Matrix multiplication}
\label{figureEx}
\end{figure}

\subsubsection*{Default use}
If you type the following command (after loading the program of Fig.~\ref{figureEx}):\\
{\tt schedule[]}\\
the output written on the screen session should look like:
\begin{small}
\begin{verbatim}
In[65]:=
schedule[]
Checking options...
Dependence analysis...
Building LP...
LP: 92 variables, 80 Constraints
Writing file for PIP....
Solving the LP...
Total execution Time: 1+M
T_a{{i,j,M} = 0
T_b{i,j,M} = 0
T_c{i,j,M} = 1+M
T_C{i,j,k,M} = k

Out[65]=
scheduleResult[matmult,{{a,{i,j,M},sched[{0,0,
  0},0]},{b,{i,j,M},sched[{0,0,0},0]},{c,{i,j,M},sched[{0,0,1},1]},{
      C,{i,j,k,M},sched[{0,0,1,0},0]}},{1,1}]
\end{verbatim}
\end{small}
The first lines indicates which computations are performed. 
% the lines starting by {\tt Version} are output by
%Pip. 
Then the result is printed on the screen. Here, for instance
the schedule is affine by variable. {\tt c[i,j,k]} is computed at time
{\tt 1+M}. The result (after {\tt Out[65]}) is the corresponding
\mma{} structure assigned to {\tt \$schedule}.


\subsubsection*{Adding a constraint}
Suppose that we want to impose that variable {\tt C[i,j,k]} be
schedule at time $2i+j+k+7$ and that $\tau_a^i=\tau_b^i=\tau_c^i$. 
We have to add the two constraints: {\tt "TC[i,j,k]=2i+j+k+7"},
{\tt "TaD1==TbD2"} and {\tt "TaD1==TcD2"}, hence the command is: 
\begin{verbatim}
In[15]:= schedule[addConstraints->{"TA[i,j,k]=2i+j+k+7",
  "TaD1==TbD2", "TaD1==TcD1}]
\end{verbatim}
The result is 
\begin{verbatim}
T_a{i, j, N} = i
T_b{i, j, N} = i
T_c{i, j, N} = 8 + i + j + 2 M
T_C{i, j, k, N} = 7 + 2i + j +  k
\end{verbatim}
Notice that in this schedule, \texttt{c} and \texttt{C} do not
have the same $\tau^i$ coefficient. We may, if needed, fix this by adding the
constraint \texttt{TcD1==TCD1}. 

\section{Troubles}
\label{troubles}
\subsection*{No schedule is found}
If no schedule is found, a message tells the user
and there may be several reasons. 
\begin{itemize}
\item No schedule exists (no way of solving this problem). You should be aware that
finding a schedule for a system of recurrence equations is in the general case undecidable.
Finding out an affine schedule is decidable, but an absence of affine schedule does
not guarantee that there does not exist a schedule. Most often, however, the scheduler
fails because there is an error in the \aalpha{} program which contains a dependence
cycle: the only solution is to check the program (use the \texttt{analyze} command, then 
check your program "by hand"). 
\item No schedule of the chosen type exists: try a multi-dimensional schedule, or
try another type of schedule.
%In this case, you can try
% the old scheduler which was more powerful (ask a \aalpha
% developer).
\item There exists a schedule but the time is not bounded. 
In this case try with the option {\tt objFunction} set to 1): this prevents the scheduler
to try optimizing the total time of the algorithm. 
\item  The program is not semantically correct: try the 
\texttt{analyze} function.
\end{itemize}

\subsection*{An awful error message}
In general, when something goes wrong during the scheduling, the error is captured correctly.
Sometimes,
the error may come with the following message:
\begin{verbatim}
General::aofil: /tmp/mat.tmp already open as /tmp/mat.tmp.
OpenWrite::noopen: Cannot open /tmp/mat.tmp.
General::stream: \$Failed is not a string, InputStream[ ], 
or OutputStream[ ].
\end{verbatim}
This may come from the fact that you have interrupted the previous
execution of the scheduler: \mma{} tries then to open a file which was
not closed. You can close these files (here {\tt /tmp/mat.tmp}) by
typing: 
\begin{verbatim}
Close["/tmp/mat.tmp"]
\end{verbatim}
In general it is not recommended to interrupt the evaluation of the schedule function.
\bibliographystyle{alpha}
\bibliography{biblio}
\printindex

\end{document}
