 \documentclass[11pt]{article}
\begin{document}

\newcommand{\mmalfa}{{\sc mmalpha}}
\newcommand{\dar}{{\sc darwin}}
\newcommand{\mma}{{\sc mathematica}}


\title{About the Darwin version of \mmalfa{}\\
Version of \today{}}
\author{Patrice Quinton}
\date{}
\maketitle
This short note presents the current status of the new \dar{} version of 
\mmalfa{}. 

\section{How to run it}
\begin{enumerate}
\item Set the shell to \texttt{tcsh} (I do not know how to use \texttt{bash}). To do so, 
start the Terminal application, and look in the preference: you should set the
initial command to \texttt{/bin/tcsh}.
\item Create a \texttt{.cshrc} file in your home directory with the definition 
of the \texttt{MMAlpha} environment variable (where MMAlpha is), the command
to extend the search path. For example:
\begin{verbatim}
setenv MMALPHA /Users/quinton/mmalpha
alias mma '/Applications/Mathematica\ 5.2.app/Contents/MacOS/Mathematica'
set path=($MMALPHA/bin.$OSTYPE $path)
\end{verbatim}
First command defines the \texttt{MMAlpha} environment variable, the second one says where
Mathematica is, and the third one adds to the \texttt{path} variable the 
location of the binary files of \mma{}.
\item Create an \texttt{init.m} file in your home directory. An example of such a file is
in appendix~\ref{init}.
\item Add an alias with the location of Mathematica (see the \texttt{.cshrc} above.
\item Open a terminal window. This should run the \texttt{.cshrc} file 
\item Type \texttt{mma}.
\end{enumerate}

That should make it! Actually, it does not : the \texttt{init.m} file does not start...

\section{Compiling for \dar{}}

\section{Troubleshouting}
\texttt{init.m} does not start...

\appendix{}
\label{init}
\section{Typical \texttt{init.m} file}
There is no difference between this \texttt{init.m} file and 
another for Windows or Linux.
\begin{verbatim}
(* This is a simple init.m file *)
Module[ {env},
  Catch[
    env = Environment["MMALPHA"];
    If[ env === $Failed, 
          mmalpha::noenv = "The MMALPHA environment variable is not set";
          Throw[ Message[ mmalpha::noenv ] ]
    ];
    Get[ env<>"/config/init.m" ];
    Alpha`$myNotebooks = "/Users/quinton/myNotebooks";
    Alpha`$myMasterNotebook = "myMaster.nb";
    Print[ Alpha`$myNotebooks ];
  ]
]
\end{verbatim}

\end{document}
