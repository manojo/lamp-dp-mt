\documentclass[11pt]{article}
\begin{document}

\newcommand{\mma}{\texttt{Mathematica}}
\newcommand{\mmalfa}{\texttt{MMAlpha}}

\title{Technical manual of \mmalfa{}\footnote{Version of \today{}}}
\author{Patrice Quinton}
\date{May 2005}
\maketitle
\section{Introduction}
\label{introduction}
This document presents the organisation of the LipForge distribution of 
\mmalfa{}. This new archive is being built by Tanguy Risset and Patrice Quinton.

\newcommand{\bincyg}{\texttt{bin.cygwin}}
\newcommand{\bindarw}{\texttt{bin.darwin}}
\newcommand{\binlinux}{\texttt{bin.linux}}
\newcommand{\config}{\texttt{config}}
\newcommand{\demos}{\texttt{demos}}
\newcommand{\sources}{\texttt{sources}}
\newcommand{\tests}{\texttt{tests}}
\newcommand{\doc}{\texttt{doc}}
\newcommand{\lib}{\texttt{lib}}
\newcommand{\libcyg}{\texttt{lib.cygwin}}
\newcommand{\libdarw}{\texttt{lib.darwin}}

\newcommand{\readme}{\texttt{readme}}
\newcommand{\cgen}{\texttt{Code\_gen}}
\newcommand{\makefile}{\texttt{Makefile}}
\newcommand{\bin}{\texttt{bin}}

\section{Organization}
\label{organization}
The distribution contains the following directories:
\begin{enumerate}
\item \bincyg{}, \bindarw{}, \binlinux{}, etc. are the directories where the binary 
files for various systems are put. These binary files are the result of 
compiling source files.
\item \config{} contains various files needed to configure the installation.
\item \demos{} contains notebooks for various demonstrations of \mmalfa{}.
\item \doc{} contains the documentation.
\item \lib{} contains the mma{} packages.
\item \libcyg{}, \libdarw{}, etc. contain some library files obtained as
byproducts of compiling source files.
\item \sources{} contains the source files.
\item \tests{} contains test files.
\end{enumerate}
In addition, the main directory contains a \readme{} file, and several files
for the \texttt{html} documentation. 

The main directory also contains a file \texttt{CopyMMA.m}. This is a 
\mma{} program that can be used to create a distribution. The documentation 
of this program is in directory \texttt{doc/Distribution}. 

\section{The \sources{} directory}
\label{sources}
\subsection{Content}
The \sources{} directory contains one directory for each 
\texttt{C} program used by \mmalfa{}. As any other directory, it contains
a \readme{} file, \texttt{html} files. 

\newcommand{\domlib}{\texttt{Domlib}}
\newcommand{\makeinclude}{\texttt{Makeinclude}}
\newcommand{\mathlink}{\texttt{Mathlink}}
\newcommand{\pip}{\texttt{Pip}}
\newcommand{\polylib}{\texttt{Polylib}}
\newcommand{\pretty}{\texttt{Pretty}}
\newcommand{\readalfa}{\texttt{Read\_Alpha}}
\newcommand{\writealfa}{\texttt{Write\_Alpha}}
The directories are:
\begin{enumerate}
\item \cgen{},
\item \domlib{},
\item \makeinclude{},
\item \mathlink{} (obsolete),
\item \pip{},
\item \polylib{},
\item \pretty{}
\item \readalfa{},
\item \writealfa{}.
\end{enumerate} 

It contains also a sed directory, which is obsolete. The \mathlink{} directory is
also obsolete.

It contains a make file, named \makefile{}. 

\subsection{Compiling source files}
It is possible to recompile the source files. The 
method is as follows. 

\newcommand{\envostype}{\texttt{OSTYPE}}
\newcommand{\envmmalpha}{\texttt{MMALPHA}}
\newcommand{\vardir}{\texttt{DIR}}


In the \sources{} directory, type
\begin{verbatim}
make all
\end{verbatim}
The  \makefile{} is then executed. If the \envostype{} variable is set, 
the corresponding binary files are produced and put in the 
corresponding \bin{} directory. In case of problem, check the
value of \envostype{} and of \envmmalpha{}. 

\subsection{How it works}
The \makefile{} in \sources{} contains a few variable definitions and 
a few rules. The \vardir{} variable defines the list of directories
where files should be compiled. 
The \texttt{all} rule, defines what should be done for all element
of \vardir{}: namely, go in the corresponding directory, and run make. 
The \texttt{all} rule depends on any file contained in the bin or lib
directories. 

The \texttt{clean} rule defines what should be done to clean directories.
Again, it consists in going in the corresponding directory, and running
\begin{verbatim}
make -f Makefile clean
\end{verbatim}

Finally, two rules help create the lib and bin directories, if they
are missing. 

Once a directory is entered, say \pretty{}, this directory contains
a make file named \makefile{}, some source files, some Object
directories, a \readme{} file and \texttt{html} files. 

The organization of this \makefile{} depends on the considered 
source directory. 

\subsubsection{The \pretty{} directory}
For the \pretty{} directory, it is very simple, and is explained in 
the corresponding \makefile{}. This directory contains 
some source files which are needed for \readalfa{} and 
\writealfa{}. 

\subsubsection{The \cgen{} directory}
For the \cgen{} directory, the structure of the make file
is much more complicated. 

First, the \makefile{} calls the file \texttt{Makefile.config} which is
situated in the \makeinclude{} directory, where some variables 
will be set. This make file also calls system dependent make file, 
say \texttt{Makefile.darwin} for the \texttt{darwin} system, where
some system dependent definitions are set. 

\newcommand{\varcppflags}{\texttt{CPPFLAGS}}
\newcommand{\varldflags}{\texttt{LDFLAGS}}
\newcommand{\vardummy}{\texttt{DUMMY}}
\newcommand{\varloadlibes}{\texttt{LOADLIBES}}
\newcommand{\varname}{\texttt{NAME}}
\newcommand{\varobjs}{\texttt{OBJS}}
\newcommand{\varlibs}{\texttt{LIBS}}
\newcommand{\rules}{\texttt{rules}}
\newcommand{\varcc}{\texttt{CC}}
\newcommand{\varpolyinclude}{\texttt{POLYINCLUDE}}
\newcommand{\varbindir}{\texttt{BINDIR}}
\newcommand{\varobjdir}{\texttt{OBJDIR}}
\newcommand{\varextraflags}{\texttt{EXTRA\_FLAGS}}

Then, back to the local make file, some other variables are defined:
\begin{itemize}
\item \varcppflags{}: include paths for the compiler,
\item \varldflags{}: library paths for the loader (where libraries should be looked for),
\item \varloadlibes{}: libraries for the loader,
\item \vardummy{}: list of files to be removed when cleaning
\item \varname{}: the name of the binary file to be created
\item \varobjs{}: the list of all object files
\item \varlibs{}: the list of libraries needed. This variable is used for completing 
make dependencies.
\end{itemize}

After these definition, one additional rule (which I do not understand). 

And finally, call to a generic \makefile{}.\rules{} (which is barely understandable...).

\subsubsection{The \readalfa{} and \writealfa{} directories}
These directories are defined in much the same way as the \cgen{} 
directory. 

\subsubsection{The \domlib{} directory}
\label{domlib}
The \domlib{} directory is different, after a deep modification that was
recently done, in order to become less dependent on the \polylib{}
which is distributed within \mmalfa{}.
\begin{itemize}
\item First, the main \makefile{} file contains a switch depending on the
system type. If the OS is Cygwin, the Visual C++ environment is called, 
otherwise, an appropriate make file is called. 
\item I only define here what a make file for a Unix-like system looks like.
It contains the definition of a few variables.
\begin{itemize}
\item \varcc{}: which compiler to use (could be removed)
\item \varpolyinclude{}: the directory where the include files for the \polylib{}
are. In the current version, this directory is where the \polylib{} installation 
puts these files. (A companion document explains how to install the \polylib{}, 
from the Web distribution. In the installation procedure, one must 
run a command
\begin{verbatim}
./configure.in --prefix="Path"
make
make install
\end{verbatim}
where \texttt{Path} is the directory where the \texttt{include} and \texttt{lib}
directories of \polylib{} are put. Say for example, you install 
\polylib{} with 
\begin{verbatim}
./configure.in --prefix="$MMALPHA/sources/Poly/$OSTYPE"
make
make install
\end{verbatim}
then (provided \envmmalpha{} and \envostype{} are properly set), 
you will find in \sources{}\texttt{/Poly/}\envostype{} the include
and lib files.
\item \varbindir{}: where the binary files will be put, enventually,
\item \varobjdir{}: where the object files will be put, during compilation,
\item \varextraflags{}: additional flags for the compiler. Uses some
variables that are set by \texttt{MakeIncludes/Makefile.\$OSTYPE},
\end{itemize}

\end{itemize}


\end{document}
 