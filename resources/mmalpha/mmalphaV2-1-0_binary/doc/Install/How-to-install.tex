\section{How to install \MMAlfa{}}
%Initial version : Tanguy Risset
%Last Modification : Feb 2004 by Patrice Quinton

This section contains MMAlpha installation instructions for Unix users,
Windows NT or XP users, MacOsX and Linux users.\footnote{The source of this document
is in the \texttt{doc/Install} directory. It appears also as an appendix of the AlphaStart document.}

The \MMAlfa{} sofware can be downloaded at url \texttt{http://www.irisa.fr/cosi/ALPHA/}.
In case of problems, send a mail to \texttt{patrice.quinton@irisa.fr}~.

\subsection{Installing \MMAlfa{} on Unix-like systems} 
By Unix-like, we mean Unix, Mac OS X, or Linux. The following 
procedure has been checked for Unix and Mac OS X, not for Linux.
\begin{enumerate}
\item 
\label{stepOneMac}
This first action must be done only if 
\MMAlfa{} has not yet been installed on the server 
or the computer you
are using. If \MMAlfa{} is already installed, go to step~\ref{stepTwoMac}.

The current distribution, as obtained from the 
\MMAlfa{} web site, is a file named \texttt{mmalphaV2-1-0.zip}.
Double-click on this file to expand this directory.
You may also run the following command
\begin{verbatim}
unzip mmalphaV2-1-0.zip
\end{verbatim}
from a shell window.
%De-archive the complete distribution in some directory, say
%\texttt{/home/group/name/} (a directory called alpha is created)
%by executing the \texttt{install\_tar} script available at 
%\begin{verbatim}
%http://www.irisa.fr/cosi/ALPHA/DOWNLOAD
%\end{verbatim}
\item
\label{stepTwoMac}
Create a \texttt{MMALPHA} environment variable containing the
path of the directory where \mmalfa{} has been installed.
If you use a \texttt{csh} shell, you have to type:
\begin{verbatim}
setenv MMALPHA path-of-mmalpha
\end{verbatim}
and for a \texttt{bash} shell:
\begin{verbatim}
export MMALPHA=path-of-mmalpha
\end{verbatim}
These commands may alternatively be added to your \texttt{.cshrc}
or to your \texttt{.bashrc} file (or to your \texttt{.login} file) in order to be
executed automatically when starting a shell. 
\item
\label{stepThreeMac}
You have then to set 
the \texttt{PATH} environment variable to contain the directory
where the binary files of \mmalfa{} are. On Mac OS X, this directory
is \texttt{path-of-mmalpha/bin.darwin}. 
For a \texttt{csh} shell:
\begin{verbatim}
setenv PATH ${MMALPHA}/bin.darwin:${PATH}
\end{verbatim}
and for a \texttt{bash} shell:
\begin{verbatim}
export PATH=${MMALPHA}/bin.darwin:${PATH}
\end{verbatim}

Again, you may add this command to your \texttt{.cshrc} or
\texttt{.bashrc} file.
%\item 
%Execute the initialization script 
%that is now in \texttt{\$MMAlpha/init\_user.csh}
%(or in \texttt{\$MMAlpha/init\_user.bash} if you use the 
%\texttt{bash} shell) by the command:
%\begin{verbatim}
%source $MMALPHA/config/init_user.csh
%\end{verbatim}
%(or 
%\begin{verbatim}
%source $MMALPHA/config/init_user.csh
%\end{verbatim}
%for the \texttt{bash} shell.)
%The only files that should be edited by you are the
%files in directory \texttt{\$MMALPHA/config/}.
\item
\label{stepThree}
Copy  the  file \texttt{\$MMALPHA/config/init-for-you.m} under the 
name \texttt{init.m} to 
your \mma{} base directory\footnote{The \mma{} base directory
used to be the user home directory in \mma{} versions 
prior to version 5. Under \mma{}, the home directory
is given by evaluating the \texttt{\$HomeDirectory} \mma{} variable.
Since version 5 (see documentation of \mma{}), 
the \mma{} base directory is given by the
\texttt{\$UserBaseDirectory} variable. For example, in 
MacOS X, 
it is located in \texttt{\~/Library/Mathematica/Kernel} directory:
this is where you should put the \texttt{init.m} file.}
(or append it to your base \texttt{init.m} file if it already 
exists). 
For users unfamiliar with Mathematica, recall
that this \texttt{init.m} file is executed whenever Mathematica's kernel
is launched.  
\item
\label{stepFour}
It's ready! Run \mma{} by typing in a shell window the command:
\begin{verbatim}
mathematica
\end{verbatim}
and once a notebook is started, type and evaluate in it the command
\begin{verbatim}
start[]
\end{verbatim}
to launch the master notebook of \MMAlfa{}. From this master notebook, examples
and explanations are available.
\item To check that everything is OK, evaluate successively the following commands
\begin{verbatim}
test1[]
test2[]
test3[]
test4[]
\end{verbatim}
Each one of these command starts a set of tests and shoul return the value
\texttt{True}. If this is not the case, something in your installation is wrong. 
In case of problems, see Section~\ref{troubles}. 

Notice that executing these
commands generates a lot of error messages, and even, unexpected messages
in the shell windows: as long as the final result of the test is \texttt{True}, this is
not a problem. 
\end{enumerate}

\paragraph*{Remarks}
\begin{itemize}
%\item The \texttt{PATH} location may change for the now
%\texttt{intel-pc} processors. I have not yet decided where it should be.

\item On MacOS X, 
\mma{} should be started from the Terminal application, as
otherwise, the \texttt{\$MMALPHA} environment variable 
cannot be set (at least, I do not know how to set this variable...). 
Moreover, after installing \mma{}, it is needed to add an alias
to start \mma{}. Usually, \mma{} is installed in directory \texttt{/Applications}, for
example : 
\begin{verbatim}
/Applications/Mathematica\ 5.1.app/Contents/MacOS/Mathematica
\end{verbatim}
So, you may add an alias in your \texttt{.cshrc} file. 
\end{itemize}

\subsection{Installing MMALPHA on Windows NT or Windows XP}
\paragraph*{Warning:} version V2 has not been tested on Windows NT, since I do not have
currently access to such a configuration. PQ, Jan. 2, 2009.
\begin{enumerate}
\item 
\label{stepOneWin}
This first action must be done only if 
\MMAlfa{} has not yet been installed in the server you
are using. If \MMAlfa{} is already installed, go to step~\ref{stepTwoWin}.
The current distribution is a file \texttt{mmalphaV2-1-0.zip} file. Double-click on this file to expand this directory.
\item
\label{stepTwoWin}
You have to set two environment variables: \texttt{MMALPHA} and 
\texttt{Path}. To do so:
\begin{enumerate}
\item
Open the system configuration panel (start \texttt{->} configuration panel)
\item Double click on the "System" icon.
\item Choose the "Advance" panel.
\item Click on "Environment variables".
\item Consider the user variable panel (top part).
\item Create a \texttt{MMALPHA} variable whose value is the path of the directory where
\MMAlfa{} is installed. Typically, the value of this variable should be 
\begin{verbatim}
C:\...\mmalphaV2-1-0
\end{verbatim}
To create such a variable, click on \texttt{New}, and fill 
the name and value. 
This step allows \mma{} to know
where \MMAlfa{} is located.
\item Append
\begin{verbatim}
;%MMALPHA%\bin.cygwin32
\end{verbatim}
to the user Path environment 
variable. To do so, select the \texttt{Path} variable, click \texttt{Modify}, 
place the cursor at the end of the string and type 
\begin{verbatim}
;%MMALPHA%\bin.cygwin32
\end{verbatim}
\footnote{Or possibly, cygwin.}This step allows the Domlib library
to be launched.
\end{enumerate}
\item		
\label{stepThree}
Copy  the  file \texttt{\$MMALPHA/init.m} to 
your \mma{} base directory\footnote{The \mma{} base directory
used to be the user home directory in \mma{} versions 
prior to version 5. Under \mma{}, the home directory
is given by evaluating the \texttt{\$HomeDirectory} \mma{} variable.
Since version 5 (see documentation of \mma{}), 
the \mma{} base directory is given by the
\texttt{\$UserBaseDirectory} variable. For example, in 
Windows,
it is probably located in \texttt{\~/Library/Mathematica/Kernel} directory:
this is where you should put the \texttt{init.m} file.}
(or append it to your base \texttt{init.m} file if it already 
exists). 
For users unfamiliar with Mathematica, recall
that this \texttt{init.m} file is executed whenever Mathematica's kernel
is launched.  

\item
It's ready! Start \mma{}. The first evaluation in 
the initial notebook should start \MMAlfa{}. Normally, the
Messages window of \mma{} opens and contains a few
lines indicating that \mma{} was started successfully.
Evaluate start[] in any notebook to launch the master 
notebook of \mma{}.
From this master notebook, examples
and explanations are available.
\item To check that everything is OK, evaluate successively the following commands
\begin{verbatim}
test1[]
test2[]
test3[]
test4[]
\end{verbatim}
Each one of these command starts a set of tests and shoul return the value
\texttt{True}. If this is not the case, something in your installation is wrong. 
In case of problems, see Section~\ref{troubles}. 
Notice that executing these
commands generates a lot of error messages, and even, unexpected messages
in the shell windows: as long as the final result of the test is \texttt{True}, this is
not a problem. 
\end{enumerate}

\subsection{In Case of Problems}
\label{troubles}
Before reading this section, make sure that you have tried to evaluate
the \texttt{test1[]} through \texttt{test4[]} commands. 

There is a \mma{}�program called \texttt{diagnose.m}, located in 
\texttt{\$MMAlpha/config/diagnose.m}, that may help locate problems. 
To run it, start a fresh version of \MMAlfa{} (from the console window), 
then evaluate 
\begin{verbatim}
SetDirectory[ "/ your mmalpha dir/config"]
<<diagnose.m
\end{verbatim}
This will check various potential problem, report them in the \mma{} window, 
and write also a file \texttt{diagnosis.txt} in your home directory. 

Here are a few difficulties you may encounter, and some ways to 
overcome them. Additional information is given in the test
documentation, in file
\begin{verbatim}
$MMALPHA/doc/Tests/testing-MMALPHA.pdf
\end{verbatim}

\subsection{The \texttt{MMALPHA} environment variable is not set}
To check this, evaluate
\begin{verbatim}
Environment[ "MMALPHA" ]
\end{verbatim}
in your notebook. If it does not answer the proper value, there
is a problem.

First, remember that you must start \mmalfa{} from a shell, otherwise, 
\mma{} does not inherit from your environment variables (this is true 
on Unix-like systems, not on Windows -like).

Second, it may be that your \texttt{init.m} file was not loaded: see next
section.

\subsubsection{The initialization file is not correctly installed}
Most of the problems come with the fact that
the \texttt{init.m} file may not be installed in the correct
directory. If this is the case, evaluating \texttt{test1[]} for
example will just return \texttt{test1[]} unevaluated, as
\mma{} was not able to load \mmalfa{}: indeed, loading
is done through the \texttt{init.m} file.

After launching \mma{}, make sure that the Kernel is also launched
by evaluating any \mma{} expression (for example, \texttt{2+2}). 
Before this evaluation is done, the \mma{}'s Kernel should 
evaluate the content of the \texttt{init.m} file, which results in 
a message in the \texttt{Messages} window of \mma{}. This
message ends with:
\begin{verbatim}
Alpha V2.0 Initialization
The Documentation can be found in ...
Current version in ...
Current directory is ...
If you use the notebook interface, you can open the master notebook: 
In[1]:= start[];
\end{verbatim}
If the Messages window is not opened, the \texttt{init.m} file was
not called and may not be in the proper directory. To check this, 
evaluate the expression \texttt{\$UserBaseDirectory} in \mma{}, 
and then check that the \texttt{init.m} file is in the 
\texttt{\$UserBaseDirectory/Kernel}.

\subsubsection{The Domlib does not start}
This happen if in the Messages window, the following 
message appears:
\begin{verbatim}
Warning: could not install domlib
\end{verbatim}

Also, if you run the \texttt{test1[]} command, 
you do have a \texttt{True} result. 

Several reasons may be the cause of this problem.
\begin{enumerate}
\item Variable \texttt{\$PATH} may not contain the directory where
the Domlib is. The Domlib is in the directory \texttt{\$MMALPHA/bin.ostype},
where \texttt{ostype} is \texttt{darwin} for MacOS X, 
\texttt{linux} for Linux, and \texttt{cygwin32} for Windows. 
On Unix-like systems, check the value of this variable by
the command \texttt{echo \$PATH}. On Windows, check the value of
the \texttt{PATH} environment variable.
\item The binary file for the domlib may not be in the proper
directory. This means that \MMAlfa{} has not been installed 
properly, or altered. Again, check that the directory 
\texttt{\$MMALPHA/bin.ostype} contains \texttt{domlib}. If this is
not the case, unzip the \mmalfa{} distribution to get it (if you
do not find it in the distribution... just send me a mail !)
\item The \texttt{domlib} binary does not fit with your configuration. 
You then have to recompile it. I have not yet been able to configure
\mmalfa{} so that recompiling is easy, but I try. Please, refer
to the test documentation first, then to the documentation about
domlib.
\end{enumerate}

\subsubsection{Other problems}
You may encounter other problems, for example, some tests are corrects, 
some other are not. After the execution of the \texttt{test1[]} through
\texttt{test4[]} commands, you find a test report file in the 
\texttt{\$MMALPHA/tests} directory: send me this file, and I'll try
to find out what is wrong in order to help you.