\documentclass[a4paper,11pt]{article}
\usepackage{psfig}
\usepackage{epic}
\usepackage{eepic}

\title{Theoretical basis of the Pipeline and PipeControl package}
\author{Tanguy Risset \and Patrice Quinton\footnote{Last modification: 22 nov. 2009}}
\date{Version of \today}
\begin{document}

\maketitle
\section*{Introduction}
This document is intended to provide the precise semantics of 
the functions of the  {\tt Pipeline.m} package, together with 
their implementation details and examples of use.
These functions are: {\tt  pipeline, pipeall, pipeInput, pipeOutput,
pipeIO}.

\section{ToAlpha0v2}
\subsection{Space time separation}
See \texttt{ToAlpha0v2.m}.

Call to \texttt{spaceTimeDecomposition}, with time position and space position.
An option (exceptions) allows some variables to be ignored. 

Compute the list of variables to consider : all variables, except output of 
use expressions. Then, call function \texttt{spaceTimeCase} for each
variable. 

This function check that the variable is defined, and that the definition is
a case expression. If it is not the case, it returns the system unchanged.

Otherwise, it calls \texttt{makeSTCase}. This function will return a new
expression that will replace the old one. 

Two cases to be considered: one or more than one branches. 

If the expression has only one branch. We compute the domain
of the RHS expression (using \texttt{expDomain}). We compute the
intersection of this domain and of the LHS domain (actually, the
declaration domain of the variable), and we
signal if the expression domain is not included in the LHS domain. 
Then, we return this intersection domain.

If there are more than one branch, we proceed recursively. Let $d_0$
the domain of the first branch. Assume that the $n-1$ remaining
branches are in ST form, and let $d_i: exp_i$ be the
other branches. 

\begin{itemize}
\Item compute the intersection $d_p$ of the space projection
of $d_i$ and the space projection of $d_i$\,;
\item if $d_p$ is empty, we add $d_i : exp_i$ to the list
of restrictions, and we keep $d_0$ unchanged for the next
round.
\item if not, we compute $d = d_0 \cap d_i$\,; and $dinter = d_i \ d$
\end{itemize}

\section{PipeControl}
The \texttt{pipeControl} function aims to optimize a 
control definition equation. Such an equation has the form
\begin{verbatim}
Vctrl[ t, p1, ... ] = 
case
  domProcs: 
	case
 	 domTime1: exp1;
 	 domTime2; exp2;
	esac;
esac;
\end{verbatim}
(It is called by the \texttt{pipeAllControl} function, which finds out all
variables that are supposed to be control equations. So, if called directly, 
the property is not necessarily checked, and the equation may not fit
this format.)

The idea is as follows. First, we try to find out a separating hyperplane
for both expressions. The idea would be to find out either a
processor domain, so that the value of the control signal is 
related to the processor position, or a time plane, so that 
the value of the control could be defined by a controller and
broadcasted to all processors. 

Once such an hyperplane is found, it gives a direction
along which to propagate the \texttt{True} and \texttt{False}
values to the boundaries of the domain. 

We first compute the direction vector of the hyperplane, and
then, we compute its null space vectors. We select these vectors
in such a way that their first component (related to time) is not
zero, otherwise, that would mean that we cannot pipeline, but only
delocalize the control. 



\subsection{Checking}
The \mma{} function checks the following conditions:
\begin{enumerate}
\item The variable to pipeline exists (message \texttt{pipeControl::unknownvar}).
\item The dimension of the variable is less than or equal to 4.
\item The first dimension is the time.
\item 
\end{enumerate}

\end{verbatim}
\section{Pipeline}
\subsection{Explanation of the transformation}
The pipeline transformation is used to transform a program with 
a definition of a variable $var_1$ which contains an
expression $expr$ implying a non uniform dependence $f$ ($Ker(f) \neq
\emptyset$): $var_1[z] = F(expr[f(z)])$,  into an 
equivalent\footnote{equivalence to be proved} program which contains a uniform
dependence $d$\footnote{or several uniform dependencies} for the
definition of the same variable $var_1$.


The principle of the transformation is the following, 
the user gives : 
\begin{itemize}
\item the name of a new variable to be created 
 (pipeline variable: $pipeExpr$);
\item  the exact instance of 
the expression to be pipeline 
($expr[f(z)]$ in  $var_1$);
\item  and a pipeline vector
$d$. 
\end{itemize}
In the transformed program, 
the variable  $pipeExpr$ is defined on the whole domain $D$ where 
the expression $expr[f(z)]$ is used.
 $pipeExpr[x]$ is initialized to  $expr[f(x)]$ on the 
{\em border\footnote{at the points where we cannot retrieve one more pipeline vector without getting out of $D$} of $D$
 along $d$} (we call this border $B$).
 This value is  duplicated (we say {\em pipelined})
along $d$ on the rest of the domain $D$
($pipeExpr[z] = pipeExpr[z-d]$). Then, the corresponding part 
of the definition of $var_1$ is replaced by: 
$var_1[z] = pipeExpr[z]$ (see figure~\ref{fig0}).

This transformation cannot be done for every pipeline vector.
Note that  the transformation imply that the values
$pipeExpr[z+kd]$ $ \forall z \in B,\ k \in N$ will correspond to the
instance $expr[f(z)]$, thus we must have $expr[f(z)]=expr[f(z+kz)] \
\forall z \in B, \ k \in N$. This is verified if $d \in Ker(f)$. This 
imply that $f$ must be non invertible. Note that the pipeline vector 
may be a not primitive vector, but this may induce warning during 
analysis of the resulting program because it correspond to 
the assumption that the domain of use of the expression is not flat.


\subsection{Implementation}

\begin{figure}
\input{ex1.eepic}
\caption{Example of pipeline Transformation}
\label{fig0}
\end{figure}

The {\tt pipeline} function should be used by programmers only, 
the users should use {\tt pipeall}. In the parameter of 
the function, the name of the variable and the pipeline vector are catched in the 
same expression :{\tt "Name.function"} where the function is the translation
in the direction of the pipeline vector.

% The three forms possible for the pipeline function arguments are: 
%\begin{verbatim}
%pipeline[occur:List[_Integer..],pipeSpec_String]\n 
%pipeline[sys_Alpha`system,occur:List[_Integer..],pipeSpec_String]\n
%pipeline[sys:Alpha`system,occur:List[_Integer..],newVarName_String,
%  pipelineVector_Alpha`matrix]
%\end{verbatim}


At the moment, the domain of the new variable build is the
intersection of the context domain of the expression to pipeline
({\tt getContextDomain}) with the domain of the expression to
pipeline ({\tt expDomain}). If the pipeline vector is not in the
kernel of the dependence function, the transformation is
aborted.

\subsection{pipeall}

{\tt pipeall} is like a pipeline more convivial. If an expression is
present in several places in the definition of a variable, or even in several
definition of different variables, {\tt
pipeall} will pipeline all occurrences of this expression.  In that
case, the domain on which is defined the variable is the union of all
the domains which would result of each individual pipelines.
The main difference with pipeline are the way the arguments are asked 
(see {\tt ?pipeall}). Be careful, you cannot expressed the expression 
to be pipelined using the array notation, you must specify 
{\tt expr.(x->f(x))} instead of {\tt expr[f(x)]}.

\section{Pipeline of Inputs and Outputs}
the {\tt pipeline} transformation pipelines a dependance which was
  originally a broadcast (dependence non invertible).  We may need to
  pipeline an expression even if it is not broadcasted, this
  is particularly useful when the expression to pipeline must be
  input (resp. output) in an architecture, in which they must go
  through some cells before being used in a computation (array with a
  loading phase).  This can be done with the {\tt pipeIO} function, which 
  should be used through the functions {\tt pipeInput} and {\tt pipeOutput}.
  {\tt pipeIO} perform a {\em routing}. It takes some data at some 
  place of the iteration space and bring it into another place of the 
  iteration space.

\subsection{PipeInput}
\begin{figure}[ht]
\input{ex3.eepic}
\caption{Example of input pipe with {\tt pipeInput}}
\label{fig01}
\end{figure}

The user gives:
\begin{itemize}
\item the name of a new variable to be created 
 (pipeline variable: $pipeExpr$);
\item  the exact instance of 
the expression to be pipeline 
($expr[f(z)]$ in  $var_1[f(z)]$, note that $f$ may be non-singular);
\item  a pipeline vector
$d$;
\item and the half space  in which 
the pipeline is to be performed (bounded by an hyperplane $H$).
\end{itemize}

Pipelining an expression as
an input consists in the following transformation: we suppose the expression 
$expr[f(x)]$ is used 
an expression somewhere $x$  ($Var[x]=expr[f(x)]$ on domain $A$, 
see figure~\ref{fig01}).  After the transformation, 
the expression $expr$ is used somewhere else $y$ (along the bounding 
hyperplan $H$) and the value is propagated to location $x$ by the
pipeline variable ($Var[x]=pipeExpr[x]$ on the original domain $A$,
$pipeExpr[z]=pipeExpr[z-d]$ inside the pipeline Domain $D$,
 and $pipeExpr[y]=epxr[f[x]]$ on the border $B$).
This transformation is very close to the usual pipeline transformation

This transformation is illustrated on  figure~\ref{fig1}
and~\ref{fig2}. figure~\ref{fig1} represents the original program
and figure~\ref{fig2} represents the program after the execution of 
the command:  {\tt pipeInput["C", "b.(i,j->i)","B1.(i,j->i+1,j+1)",
" \{i,j | i >= 0 \} "]} and normalization.
 The example also corresponds to the illustration 
of figure~\ref{fig01}. $Var1$ is {\tt C}, $expr[f(x)]$ is {\tt b.(i,j->i)},
$d$ is {\tt (1,1)} (represented by {\tt (i,j->i+1,j+1)}),  $pipeVar$
is {\tt B1}, and $H$ is  {\tt \{i,j | i >= 0\}}. 


\begin{figure}[ht]
\begin{verbatim}
system silly: {N | N>1}
                (a : {i,j|1 <= i,j <= N} of boolean; 
                 b : {i|1 <= i <= N} of boolean)
     returns (c : {i|1 <= i <= N} of boolean);
  var 
        C :  {i,j|1 <= i <= N; 0<= j <=N} of boolean;
  let
C[i,j] = case
       {|j=0} : b[i];
       {| j>=1} : C[i,j-1] + a[i,j];
         esac;
      c[i]=C[i,N];
  tel;   
\end{verbatim}
\caption{simple program before the use of pipeInput}
\label{fig1}
\end{figure}

\begin{figure}[ht]
\begin{verbatim}
......
var
  B1 : {i,j | (j+1,0)<=i<=j+N; j<=0; 2<=N} of integer;
  C : {i,j | 1<=i<=N; 0<=j<=N} of boolean;
let
  B1[i,j] = 
      case
        {| i=0; -N<=j<=-1; 2<=N} : b[i-j];
        {| 1<=i<=j+N; j<=0; 2<=N} : B1[i-1,j-1];
      esac;
  C[i,j] = 
      case
        {| 1<=i<=N; j=0; 2<=N} : B1;
        {| 1<=j} : C[i,j-1] + a[i,j];
      esac;
.......
\end{verbatim}
\caption{Program of figure~\ref{fig1}, after use of {\tt pipeInput}:
{\tt pipeInput["C", "b.(i,j->i)","B1.(i,j->i+1,j+1)",
" \{i,j | i >= 0 \} "]}}
\label{fig2}
\end{figure}

\subsection{PipeOutput}
\begin{figure}
\input{ex2.eepic}
\caption{Example of output pipe with {\tt pipeOutput}}
\label{fig02}
\end{figure}

The user gives (as for {\tt pipeInput}:
\begin{itemize}
\item the name of a new variable to be created 
 (pipeline variable: $pipeExpr$);
\item  the exact instance of 
the expression to be pipeline 
($expr[f(z)]$ in  $var_1[f(z)]$, note that $f$ may be non-singular);
\item  a pipeline vector
$d$;
\item and the half space  in which 
the pipeline is to be performed (bounded by an hyperplane $H$).
\end{itemize}

 Pipelining an expression as
an output consists in the following transformation:
 we suppose we use at some place $x$ 
an expression which  was produce at some place $y$
 ($Var[x]=expr[y]$ on domain $A$). In the transformed 
program, we use this expression in another variable  
at  place $y$ ($VarPipe[y] = expr[y]$) and the value 
in pipelined in $VarPipe$ until another place $f(x)$ where it is consumed 
by $Var$ ($Var[x] = VarPipe[f(x)]$ on domain $B$, see figure~\ref{fig02} for 
example of output pipe).


This transformation is illustrated on  figure~\ref{fig3}.
 figure~\ref{fig1} represents the original program
and figure~\ref{fig2} represents the program after the execution of 
the command:  {\tt pipeOutput["c", "C","C1.(i,j->i+1,j+1)",
"\{i,j| i <= N\}"]}. The transformation performed is illustrated on 
figure~\ref{fig02} where $Var$ is {\tt c}, $expr[f(x)]$ is {\tt C.(i,j->i,j)},
$d$ is {\tt (1,1)} (represented by {\tt (i,j->i+1,j+1)}),  $varPipe$
is {\tt C1} and $H$ is {\tt. \{i,j| i <= N\}}.




\begin{figure}
\begin{verbatim}
var 
	
  C1 : {i,j | j-N+1<=i<=N; N<=j; 2<=N} of integer;
.........
  C1[i,j] = 
       case
         {| 1<=i<=N; j=N} : C;
          {| j-N+1<=i<=N; N+1<=j} :  C1[i-1,j-1];
       esac;
  c[i] = C1[N,-i+2N];
tel;
\end{verbatim}
\caption{program of figure~\ref{fig1}, after use of pipeOuput:
 {\tt pipeIO["c", "C","C1.(i,j->i+1,j+1)","\{i,j| i <= N\}"]}}
\label{fig3}
\end{figure}

\subsection{Implementation}
Both {\tt pipeOutput} and  {\tt pipeInput} are implemented
by the same function : {\tt pipeIO}, we will briefly described the 
implementation of this function here.
 The fact that the pipeline is an input pipe or an output pipe is
 determined by the scalar dot of the pipeline vector and 
the normal to the hyperplane $H$ bounding the half space.
If the pipeline vector goes {\em towards} $H$ then, this must be an output 
pipe, if it {\em comes from} $H$, this must be an input Pipe

Three domains are distinguished:
\begin{itemize}
 \item the domain of the original 
expression $expr[f(z)]$ (that we will call $domExpr$), 
which is the domain where
the pipeline is initialized in the case of an output pipe and the 
domain where the pipeline ends in the case of an input pipe 
(domain $A$ on figure~\ref{fig01} and~\ref{fig02}). 
\item the pipeline domain (that we will call $realDomPipe$) 
which is the domain  on which the value is pipelined 
(domain $D$ on figure~\ref{fig01} and~\ref{fig02}). 
\item the domains where the pipeline ends (that we will call $pipeEndDom$), 
which is in fact the domain where the pipeline is initialized in
 the case of an input pipe
(domain $B$ on figure~\ref{fig01} and~\ref{fig02}). 
\end{itemize}
$domExpr$ is computed as the intersection of the context domain of the
expression to pipeline ({\tt getContextDomain}) and the domain of the
expression itself ({\tt expDomain}). $realDomPipe$ is computed by
adding to $domExpr$ a ray (which is the pipeline vector in the case of
an output pipe and its opposite in the case of an input pipe) and then
by intersection the resulting domain with the half space bounded by 
 $H$. $pipeEndDom$ is computed by shifting $realPipeDomain$ by
the pipeline vector in case of input pipe (resp. its opposite in case
of output pipe) and retrieving the $realDomPipe$.

there remains the problem of founding, given on index point in
$pipeEndDom$, what is the corresponding point in $domExpr$ (resp. the
other way around in the case of input pipe). This is done by
observing the following fact. If we find a function of the indices
whose value is constant on the pipeline path and which is perfectly
determined by the pipeline path (unique for each pipeline path),
then giving this value will determine a unique antecedent point
$z_1$ in $exprDom$ and a unique antecedent point $z_2$ in
$pipeEndDom$. Hence, we will be able to use the {\tt
inverseInContext} function is order to find for a point in
$pipeEndDom$ the corresponding point in $exprDom$. the function to
build must have a square matrix thus we take a square $n\times n$
matrix of rank $n-1$ for which the kernel is generated by the
pipeline vector. And it works...  The use of {\tt inverseInContext}
impose that the domains of the expressions to pipeline must be flat
(dimension $n-1$).



\end{document}


