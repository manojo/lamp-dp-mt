\newcommand{\aalpha}{{\alfa}}
\newcommand{\lp}{{\sc lp}}
\chapter{\index{Scheduling}Scheduling {\alfa} programs}
\label{chapscheduling}

{\sc Version of \today}

This chapter explains how to use the scheduler for finding valid
execution order of computations of {\alfa} programs.
 
\section{Introduction}
An {\aalpha} program does not convey any sequential ordering: an
execution order is semantically valid provided it respects the data
dependencies of the program.  Consider for instance the program of
figure~\ref{figSched1} which represents the computation of the product
of a $M\times N$ matrix \texttt{a} by a $M$ vector \texttt{b}. To execute
such a program on a sequential machine, we must chose a
computation
order which respects data dependencies. Such an
order is shown in figure~\ref{figSched2}-(a). Another possible
order would be the demand-driven execution scheme used in the
simulation, as shown in chapter~\ref{chapcodegen}\,: to evaluate a
variable, evaluate recursively the expressions that this variable
depends on.  Parallel executions are also possible. The program of
figure~\ref{figSched2}-(b) shows a possible parallel order for
computing the matrix-vector multiplication.  

\begin{figure}[htbp]
%\htmlimage{scale=\myscale}
\verbatiminput{alpha_progs/MatVect.alpha}
\caption{{\aalpha} program for the matrix-vector product, 
no execution order is specified.}
\label{figSched1}
\end{figure}

\begin{figure}[htbp]
%\htmlimage{scale=\myscale}
\begin{minipage}{8cm}
\begin{verbatim}
For i=1 to N
   C[i,0]=0
EndFor
For i=1 to N
   For j=1 to M
       C[i,j]=C[i,j-1]+
                a[i,j] * b[j]
   EndFor
EndFor
For i=1 to N
   c[i]=C[i,M]
EndFor

           (a) 
\end{verbatim}
\end{minipage}
\begin{minipage}{8cm}
\begin{verbatim}
ForAll i=1 to N
   C[i,0]=0
EndFor
For j=1 to M
   ForAll i=1 to N
       C[i,j]=C[i,j-1]+
                 a[i,j] * b[j]
   EndForAll
EndFor
ForAll i=1 to N
   c[i]=C[i,M]
EndForAll

           (b)
\end{verbatim}
\end{minipage}
\caption{
Two possible order of execution for the program of figure~\ref{figSched1}. The 
left one is parallel. (\texttt{ForAll} represents a parallel loop).}
\label{figSched2}
\end{figure}

The basic goal of the scheduler is to find out valid execution orders, 
called {\em schedules} in what follows. 
A secondary goal is to find out {\em good} schedules. 
However, there is no best schedule: the quality of
a schedule depends on a particular criterion. 

In our model, the time is considered as a discrete single clock and we
look for parallel ordering of the computations.  The theoretical basis
for the scheduling process is inherited from the research on systolic
array and automatic parallelization. The scheduler implements two
different procedures: one, called the Farkas method, is defined
in~\cite{Feautrier92aa}; the other one, called the vertex method, is
presented in~\cite{citationdespaa}.

The scheduler of \alfa{} is an analysis tool: it does not modify the
current program in \texttt{\$result}.  The result is the description of a
possible parallel schedule. For instance, for the program of
figure~\ref{figSched1}, if we assume that the inputs are available at
time 0 and that each assignement takes 1 clock tick, we
will obtain the information that, for all \texttt{i}, value {\tt
C[i,j]} can be computed at step \texttt{j}.  Thus, for all \texttt{i}, the
values \texttt{c[i]} can be computed at step \texttt{M+1}. This schedule
can be summarized by:
\begin{equation}
\label{Schedeqn1}
\begin{array}{l} T_a[i,j] = 0 \\ T_b[i] = 0 \\ 
T_C[i,j] = j \\ T_c[i] = M+1 \end{array} 
\end{equation}
This particular schedule corresponds exactly
to the parallel execution order of the program of figure~\ref{figSched2}-(b).

In the present chapter, we are interested in {\em affine by variable}
schedules, i.e schedules in which, for each variable, the execution
date is an affine function of the indices (and possibly of the
structural parameters). The idea of the scheduling process is to
gather all the constraints that the schedule must verify in a linear
programming problem ({\sc lp}) and to solve this \lp{} using a {\sc lp}-solver.

\section{Basic Farkas scheduler}

The scheduling process is quite complex, since many parameters
can influence the result. In this section, 
we present the basics of the scheduler.

\subsection{How to use the \texttt{schedule} function}

The function to be called in order to schedule a program is {\tt
schedule}. Its effects is to find out a schedule for the {\aalpha} program
contained in \texttt{\$result}, and to put the schedule in the
global \mma{} variable \texttt{\$schedule}. Options allow one to modify
parameters, hence the possible forms of a call to the function {\tt
schedule} listed here: 
\begin{itemize}
\item \texttt{schedule[]}\\ finds an affine-by-variable schedule for {\tt
\$result} and assigns it to
\texttt{\$schedule} (this schedule attempts to  minimize
 the global execution time).
\item \texttt{schedule[sys]}\\ 
finds an affine-by-variable schedule for the {\aalpha} system \texttt{sys}
 and assigns it to
\texttt{\$schedule}
\item {\tt
schedule[option\_1->value\_1,\ldots,option\_n->value\_n]}\\
finds a schedule for \texttt{\$result} which respects the chosen options and
assigns it to \texttt{\$schedule}
\item {\tt
schedule[sys,option\_1->value\_1,\ldots,option\_n->value\_n]}\\
finds a schedule for the {\aalpha} system \texttt{sys} which
 respects the chosen options and
assigns it to \texttt{\$schedule}
\end{itemize}

For example, calling the scheduler on the matrix-vector multiplication
of figure~\ref{figSched1} prints out the following informations:
\begin{verbatim}
In[30]:= schedule[];
Dependence analysis...
Duration : 1 for each equation 
Building LP...
LP: 82 Constraints
Writing file for PIP....
Solving the LP...
Version C.3 MultiPrecision (mpip, rev. 1.3.0)
cross : 407646, alloc : 1
Max cross-product result: 4 (1 digits, base 10)
Max numerator term: 8 (2 digits, base 10)
Max denominator: 2 (1 digits, base 10)
n 1 u 130''' s 11'''
T_a{i, j, N, M} = 0
T_b{i, N, M} = 0
T_c{i, N, M} = 1 + M
T_C{i, j, N, M} = j

Out[30]= scheduleResult[1, {{a, {i, j, N, M}, sched[{0, 0, 0, 0}, 0]}, 
   {b, {i, N, M}, sched[{0, 0, 0}, 0]}, {c, {i, N, M}, sched[{0, 0, 1}, 1]}, 
   {C, {i, j, N, M}, sched[{0, 1, 0, 0}, 0]}}]
\end{verbatim}
 
The first lines indicate which computations are performed.  The Farkas
scheduler first built a \lp{}, then writes this \lp{} in a file,
then calls a \lp{}-solver. Writing this file takes some time, thus
informations are displayed for the user not to become anxious.

The lines starting by \texttt{Version} are output by the \lp{}-solver
\pip{}. The result of calling Pip is then shown. Here, for
instance the schedule says that \texttt{B[i,j,k]} is computed at time
\texttt{i}. The result (after \texttt{Out[30]}) is the corresponding
\mma{} expression representing \texttt{\$schedule}.

For such a little program, only a few seconds are necessary to find
out a schedule. But it may take quite a long time for larger {\aalpha}
programs.

\subsection{Format of the output of \texttt{schedule}}

The output of the \texttt{schedule} function has a special form. 
It is enclosed in a structure whose head is
\texttt{Alpha`ScheduleResult}. The first argument of this
structure is the type of 
the schedule (integer, coded as the option scheduleType, 
see section~\ref{schedType}) and its 
second argument is the schedule itself. 
The syntax of this structure is:\\
\texttt{
<schedResult> ::= Alpha`scheduleResult[scheduleType\_Integer,\{sched...\}]\\
$\begin{array}{l} \mbox{ <sched>} ::= \\ ~\\  ~ \end{array}\begin{array}{l} \mbox{\{ nameVar\_String,}\\
          \mbox{ indices\_List,}\\
           \mbox{ Alpha`sched[tauVector\_List,constCoef\_Integer] \}}
\end{array}$
}\\
The effect of the \texttt{schedule} function is to assign this form 
to te global variable \texttt{\$schedule}.

Another example is given in section~\ref{ExampleSched}. 
A schedule can be displayed using the \texttt{show} command:
\begin{verbatim}
In[31]:=show[$schedule]

T_a{i, j, N, M} = 0
T_b{i, N, M} = 0
T_c{i, N, M} = 1 + M
T_C{i, j, N, M} = j

Out[31]:=
\end{verbatim}

\subsection{Using the result of the scheduler}

We may use the result of the scheduler as an information on the program, 
but we can also tranform the {\aalpha} program such that this information
become explicit in the program.  For instance, on the matrix-vector example, 
if we perform the \index{change of basis}change of basis
\texttt{(i,j -> j,i)} on variable \texttt{C}
and if we rename (for instance) 
the first index \texttt{t} and the second one \texttt{j},
we obtain the program of figure~\ref{figSched3}.

\begin{figure}[htbp]
%\htmlimage{scale=\myscale}
\verbatiminput{alpha_progs/MatVectSched.alpha}
\caption{Scheduled version of  program of figure~\ref{figSched1}}
\label{figSched3}
\end{figure}

This program is functionnaly equivalent to the one of figure~\ref{figSched1}
but it shows exactly at which clock tick each  value of the local variable
\texttt{C} is computed: \texttt{C[t,i]} is computed at step \texttt{t}. We call
a program which has this property a \index{scheduled program} {\em
scheduled} {\aalpha} program.  Of course, in a scheduled {\aalpha}
program, the first index of the local variables defines a partial
order which is compatible with data dependencies.

The scheduled program of figure~\ref{figSched3} has been obtained by
evaluating the expression
\index{applySchedule}\texttt{applySchedule[]}.  This command applies a
change of basis to each local variable of the program contained in
\texttt{\$result} in such a way that the first index in the new basis
represent the time. The change of basis is not performed on the inputs
and outputs, as otherwise the new program would not be equivalent to the
previous one. Another example of the use of \texttt{applySchedule[]}
is shown in section~\ref{ExampleSched}.

\section{Advanced scheduling}
In this section we introduce some of the parameters of the 
\texttt{schedule} command which allow more precise results to be
obtained.  A detailed description of this function can be found in
the documentation 
file \texttt{\$MMALPHA/doc/user/Scheduler\_user\_manual.ps}.

\subsection{Options of schedule}
The \texttt{schedule} function has many options. These options
have default values indicated hereafter. To change these
values, put one of the corresponding replacement
rules as a parameter to the
\texttt{schedule} function (see the example in section~\ref{ExampleSched}).

\paragraph{Option \texttt{scheduleType}}
\label{schedType}
This option gives the type of schedule.
Its possible values are:
\begin{itemize}
\item \texttt{scheduleType -> affineByVar} (default) 
affine by variable scheduling;
\item \texttt{scheduleType -> sameLinearPart} affine by variable scheduling with constant linear part;
\item \texttt{scheduleType -> ?}  affine by variable scheduling with constant linear part except for the parameters.
\end{itemize}

\paragraph{objFunction }
This option gives the objective function chosen. Its type is integer,
the possible values are:
\begin{itemize}
\item \texttt{objFunction -> 0 } (default) the total latency is minimized. 
\item \texttt{objFunction -> 1 } no objective function (the
coefficients of the scheduling vectors are minimized in a
lexicographic order).

It is mandatory not to minimize the total execution time when this 
time is not bounded (otherwise, the schedule will fail).
\end{itemize}

\paragraph{ratOrInt}
The resolution of the linear programming problem generated is done 
by the software {\sc mp}Pip which is an infinite precision version
 of the sofware {\sc pip}~\cite{Feautrier88a}. The resolution can be done with
integer programming or rational linear programming.
  This option indicates whether the resolution of the {\sc lp} by {\sc mp}Pip
 is
 done in integer mode (resulting scheduling vectors with integral
 coefficients) or in rationnal mode (resulting scheduling vectors may contains
 rationnal coefficients).  Its type is integer, the possible values
 are:
\begin{itemize}
\item \texttt{ratOrInt -> 0 } : rationnal solution.
\item \texttt{ratOrInt ->  1 } : (default) integer solution.
\end{itemize}

\paragraph{addConstraints}
This option allows to add some constraints to the {\sc lp} generated.
This option is very usefull to guide the scheduling process.
 Its type is a list of string, each string representing a constraint. The
 constraints authorized are affine constraints on scheduling
vectors.
There are two types of  constraint added, one can force a scheduling
vector value or simply set linear constraints on its components. 
\begin{itemize}
\item forcing a  variable $A$ to be schedule at time i+2j+2 can be done 
with the constraints: \texttt{"TA[i,j]=i+2j+2" }. 
\item for more precise constraint,  one can directly access to each components 
of the schedule functions of each variable. For instance \texttt{TAD2}
 will represent the variable coefficient of the second indice
 in the schedule of variable \texttt{A} (and \texttt{CA} will represent 
the constant part in the schedule of variable \texttt{A}).
 With these names, one can set linear constraints 
on these coefficients using operators \texttt{==} or \texttt{>=}. For instance,
\texttt{\{"TAD1 == 1","TAD2 == 2", "CA >= 2" \} } is the same constraint
as above except thar the constant is allowed to be greater than two.
\end{itemize}  
Example of use:\\
\texttt{schedule[addConstraints->\{"TA[i,j,N]=i+2j-2",
"TBD2==2","TBD1+2TCD3>=1"\}]}



\paragraph{duration} indicates how to count the duration of each computation. The possible value for the duration option are: 
\begin{itemize}
\item \texttt{duration -> \{\} }: (default) each equation takes 1 step
	 (whatever complex the computation is).
\item \texttt{duration -> \{Integer...\} }: 

The duration of each equation is given by the user. The list must
contain as many integer as there are variables (do not forget the
input variables). For instance on the program of figure~\ref{figSched1}, the
command \texttt{schedule[]} is equivalent to the command:
\texttt{schedule[duration->\{1,1,1,1\}]}.
\end{itemize}


\section{Another example}
\label{ExampleSched} 
Consider the program of figure~\ref{figSched4}, which represent a
uniform program for the \index{multiplication of matrices}
multiplication of matrices. in this section, we give the result of the
schedule with different options.
\begin{figure}[htbp]
%\htmlimage{scale=\myscale}
\verbatiminput{alpha_progs/MMunif.alpha}
\caption{Uniform matrix matrix product}
\label{figSched4}
\end{figure}

\paragraph{Default use}
If you type the following command (after having loaded the program of figure~\ref{figSched4}):\\
\texttt{schedule[]}\\
the result should be: 
\begin{verbatim}
T_a{i, j, N} = 0
T_b{i, j, N} = 0
T_c{i, j, N} = 1 + 2 N
T_B{i, j, k, N} = i
T_A{i, j, k, N} = j
T_C{i, j, k, N} = k + N

Out[14]= scheduleResult[1, {{a, {i, j, N}, sched[{0, 0, 0}, 0]}, 
   {b, {i, j, N}, sched[{0, 0, 0}, 0]}, {c, {i, j, N}, sched[{0, 0, 2}, 1]}, 
   {B, {i, j, k, N}, sched[{1, 0, 0, 0}, 0]}, 
   {A, {i, j, k, N}, sched[{0, 1, 0, 0}, 0]}, 
   {C, {i, j, k, N}, sched[{0, 0, 1, 1}, 0]}}]
\end{verbatim}
The result (after \texttt{Out[14]}) is the
Mathematica structure assigned to \texttt{\$schedule}.

\paragraph{Adding a constraint}
Suppose that the inputs \texttt{a[i,j]} and \texttt{b[i,j]} arrive
respectively at time \texttt{i+j} and \texttt{i+j+3}. Moreover, we
want a schedule of unique linear part (because this allow a locally
connected array to be obtained from a uniform program).  We have to
add the two constraints: \texttt{"Ta[i,j]=i+j"} and
\texttt{"Tb[i,j]=i+j+3"}. Also, we need to set the option scheduleType
to 2, hence the command is:
\begin{verbatim}
In[15]:= schedule[addConstraints->{"Ta[i,j]=i+j","Tb[i,j]=i+j+3"},
                   scheduleType->2]
\end{verbatim}
The result is 
\begin{verbatim}
T_a{i, j, N} = i + j
T_b{i, j, N} = 3 + i + j
T_c{i, j, N} = 5 + i + j + N
T_B{i, j, k, N} = 3 + i + j + k
T_A{i, j, k, N} = i + j + k
T_C{i, j, k, N} = 4 + i + j + k
\end{verbatim}

Note that the unique linear part option is applied on local variables
only (indeed, it cannot be enforced on inputs and outputs which do not have
the same number of indices.) After scheduling, one can obtain a
scheduled program by typing the command: \texttt{applySchedule[]}. The
result is the program of figure~\ref{figSched5}.

Warning: the \texttt{applySchedule} function finds the change of basis 
by unimodular completion of the scheduling vector. The unimodular 
completion is not always possible (for instance, if the scheduling 
vector is null), in that case, applySchedule will perform 
a generalized change of basis and add a new dimension to the domain of 
the corresponding variable. A warning is printed out during this operation.

\begin{figure}[htbp]
%\htmlimage{scale=\myscale}
\verbatiminput{alpha_progs/MMunifSched.alpha}
\caption{Scheduled  matrix matrix product}
\label{figSched5}
\end{figure}

\subsection{What if no schedule can be found?}
Due to the restriction to linear schedule, there may 
happen that the schedule function only answers: 
\texttt{No schedule was found, sorry...}
There maybe several reasons for that:
\begin{itemize}
\item The program is not semantically correct, try \texttt{analyze[]}.
\item There exists a schedule but the time is not bounded. 
In this case try with the option \texttt{objFunction} set to 1.
\item No affine one dimensional schedule exists. You can try to find
a multidimensional schedule with the \texttt{multiSched}.
\item No schedule exists (there is no way of solving this problem).
\end{itemize}

\section{To come soon}
\begin{itemize}
\item Multi-dimensional schedule.
\item Structured scheduling.
\item Data-flow scheduling. 
\end{itemize}


