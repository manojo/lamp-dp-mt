\chapter{Modelling synchronous architectures}
\label{chap2}
In this chapter, we present the {\alfa} language through the 
description of very simple -- and somehow artificial -- examples
of synchronous circuits. Our goal is to show the basic constructors
of the language: their syntax, their meaning, and how they can be
used in practice. 
% 
\section{Pointwise operators\index{pointwise operators}}
\begin{figure}[htbp]
\htmlimage{scale=1}
\colorbox{red}{
\centerline{\input{/udd/quinton/documents/lncs_574/seminaire_litp/figures/exemple_1}}}
\caption{An adder}\label{adder}
\end{figure}
Consider the architecture\index{architecture} shown in figure~\ref{adder}: there is a 
simple adder, which takes to input flows, respectively {\tt x} and 
{\tt y}, and returns {\tt S}. 

This architecture is described by the {\alfa} program of figure \ref{adder-alpha}.
\begin{figure}[htbp]
\htmlimage{scale=1}
%\input{adder.alpha}
\begin{verbatim}
system exemple1 (x,y: {t|1<=t} of integer)     -- 1
returns (z: {t|1<=t} of integer);              -- 2
var S: {t|1<=t} of integer;                    -- 3
let                                            -- 4
  S = x + y;                                   -- 5
  z = S;                                       -- 6
tel;                                           -- 7
\end{verbatim}
\caption{An adder in {\alfa}}\label{adder-alpha}
\end{figure}
This program is called a {\em system}. Line 1 is the head of the system, 
with name {\tt exemple1}. This system has two input arguments named
{\tt x} and {\tt y}. These arguments, called {\em variables}, are
defined on the set {\tt \{t|1<=t\}}. Line 3 defines a local variable
{\tt S}. Between the {\tt let} and {\tt tel} keywords of lines 4 and
7, we have two equations: line 5 says that {\tt S} is the pointwise sum of 
{\tt x} and {\tt y}. In other words, $\forall t, S_t = x_t + y_t$. 
Line 6 says that {\tt z} and {\tt S} are identical.
\subsection*{Processing this example in {\mmalfa}}
A few words on {\mma}. A more complete introduction is given in 
appendix~\ref{mma-introduction}
\begin{verbatim}
load["adder.alpha"]    (* loads the alpha file *)
ashow[];               (* shows its content *)
\end{verbatim}

%
\section{Substitution}
\begin{figure}[htbp]
\htmlimage{scale=1}
%\input{adderMultiplexerSubs.alpha}
\begin{verbatim}
system adderMultiplexerSubs (x,y : {t|1<=t} of integer)
       returns (z : {t|5<=t} of integer);
  var
    S : {t|1<=t} of integer;
  let
    S = case
        {t|t=1} : 0.(t->);
        {t|t>1} : x.(t->t-1) + y.(t->t-1);
        esac;
    z = {t|t>=5} : case
                   {t|t=1} : 0.(t->);
                   {t|t>1} : x.(t->t-1) + y.(t->t-1);
                   esac;
  tel;
\end{verbatim}
\caption{Substitution}\label{adderMultiplexerSubs-alpha}
\end{figure}
\subsection*{Exercices in {\mma}}

%
\section{Normalization}
\begin{figure}[htbp]
\htmlimage{scale=1}
\centerline{\input{/udd/quinton/documents/lncs_574/seminaire_litp/figures/exemple_7}}
\caption{Normalizing}\label{adderMultiplexerNorm}
\end{figure}
The corresponding program is shown in figure \ref{adderMultiplexerNorm-alpha}.
\begin{figure}[htbp]
\htmlimage{scale=1}
%\input{adderMultiplexerNorm.alpha}
\begin{verbatim}
system adderMultiplexerNorm (x,y : {t|t>=1} of integer)
       returns (z : {t|t>=5} of integer);
  var
    S : {t|t>=1} of integer;
  let
    S = case
        {t|1=t} : 0.(t->);
        {t|t>1} : x.(t->t-1) + y.(t->t-1);
        esac;
    z = {t|t>=5} : x.(t->t-1) + y.(t->t-1);
  tel;
\end{verbatim}
\caption{Normalization}\label{adderMultiplexerNorm-alpha}
\end{figure}
\subsection*{Exercices in {\mma}}

