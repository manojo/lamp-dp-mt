\chapter{Deriving \index{systolic}systolic architectures using {\alfa}}
\label{chapderiving}

{\sc Version of \today}

In this chapter, we describe a methodology for deriving
systolic architectures using {\mmalfa}. This \index{design
methodology}methodology is illustrated by means of an {\alfa}
program for the matrix-vector multiplication. A companion
notebook called \texttt{matvect.nb} allows the reader to
execute interactively this demonstration. This notebook is
situated in directory:\\
\texttt{\$MMALPHA/demos/NOTEBOOKS/Matvect}\\
A link to this notebook is available in the 
master notebook (see~\ref{master}).

\section{Introduction}

The methodology presented here is inherited from the research of the
``systolic community''~\cite{Meg1,KungHT,QuintonRo89} and from its
adaptation to {\alfa}~\cite{SieWilde94,Dinechin96c,Patricia96}.

The successive steps of the derivation are the following ones:
\begin{enumerate}
\item High level specification in {\alfa}.
\item Uniformization.
\item Scheduling and mapping.
\item Control signal generation.
\item \label{step5} Generation of an {\alphard} specification. 	\\
~\\These steps may be followed by three additional ones: 
\item Translation to {\vhdl}.
\item Logic synthesis.
\item Layout synthesis.
\end{enumerate}

The five first transformations deal with producing a so-called
\alphard{} program.  {\alphard} is a subset of the {\alfa} language
used for architectural description at the register transfer level.
The description of {\alphard} format can be found
in~\cite{Patricia96}, and is presented in chapter~\ref{chapvhdl}.  In
the present chapter, we only describe how to perform the first four above
steps using {\mmalfa}. The result will be a single {\alfa}
program which can be interpreted as a systolic array for executing the
original {\alfa} specification. The main difference between this
format -- called {\alphaz} -- and {\alphard} is their structuring: an
{\alphard} program is structured into several sub-systems while an
{\alphaz} program contains only one {\alfa} system.

\section{Example of the matrix-vector product\index{matrix-vector product}}
We start by an \alfa{} description of the algorithm, 
as shown in figure~\ref{figder1}. 

\begin{figure}[htbp]
%\htmlimage{scale=\myscale}
\verbatiminput{alpha_progs/MatVect.alpha}
\caption{{\alfa} program for the matrix-vector}
\label{figder1}
\end{figure}

The dependence graph of this program 
is shown in the left hand side of figure~\ref{figder2}, 
for $N=M=4$. In this
figure, it can be seen that each component of the {\tt b} vectors is
broadcasted to several computations. For instance, $b_1$ is used in
the computation of $ a_{1,1}, a_{1,2}, a_{1,3}$ and $a_{1,4}$. The
first step of the synthesis consists of ``pipelining'' these values
in order to obtain a uniform dependence graph such as 
that shown in the right hand side of
figure~\ref{figder2}. This step is often called {\em uniformization}
or {\em localization}
in the litterature.

\subsection{Uniformization\index{Uniformization}}

Methodologies for automatic uniformization were
proposed by several researchers
(see~\cite{QuintonVa88} for example). However, performing the
uniformization process completely automatically is difficult as there
are many possibilities for uniformizing a given program. Hence, 
this transformation requires some help from the user.

Before proceeding, we first perform a pre-processing 
transformation, which consists of adding a local 
variable \texttt{A} to replace the occurrence of
\texttt{a} in the definition of \texttt{C}. The purpose of
this modification is to anticipate a change of basis 
that will be done later on, and which cannot be applied on
an input variable without altering the semantics of
the program. The \mmalfa{} function which does this
is:\\
\texttt{addLocal["A = a"];}\\
%We first perform a single pre-processing step which consists of adding all
%the constraints present in the parameter domain declaration (present
%at the top of the system) inside every domain appearing in the
%program. Then we add, for each input variable, a local variable
%(remember that we cannot perform transformations involving inputs and
%output variables). Next, we have to \index{pipeline}pipeline the
%broadcasted expressions.

We are now ready to uniformize the occurence \texttt{b[i]}
in the definition of \texttt{C}. 
The pipeline vector will be the direction of
the $\vec{\imath}$ axis in our example, as shown in 
figure~\ref{figder2}.

\begin{figure}[htbp]
%\htmlimage{scale=\myscale}
\centerline{\includegraphics{figures/tr1}}
\caption{Dependence graph for the program of figure~\ref{figder1}, before 
and after uniformization.}
\label{figder2}
\end{figure}

In our example, the {\mmalfa} function for uniformizing is: 
\begin{verbatim}
pipeall["C","b.(i,j->j)","B1.(i,j->i+1,j)"];
\end{verbatim}
The parameters given to 
{\tt pipeall} can be read as follows: 
\begin{quote}
In the definition of {\tt C},
pipeline the expression
{\tt  b.(i,j->j)}. The pipeline vector is
$(1,0)$, the variable which will pipeline the value will be called
{\tt B1}.
\end{quote}
Two remarks: 
\begin{itemize}
\item The array notation {\tt b[i]} cannot be used
here, as it would be ambiguous. 
\item The pipeline vector is indicated as a translation in
{\alfa} form.  
\end{itemize}
The result of applying these commands is shown in
figure~\ref{figder3} where only the equation of the resulting {\alfa}
program are shown.  Note that the domains of the branches
of \texttt{B1} definition have been computed by {\tt
pipeall}. Note also that the dependence graph of this program
-- represented in the right hand side of figure~\ref{figder2} -- is
now uniform. For more information on the pipelining process refer
to the
documentation~\cite{docpipeline}.

\begin{figure}[htbp]
%\htmlimage{scale=\myscale}
\begin{verbatim}
B1[i,j] = 
      case
        {| i=1; 1<=j<=M; 2<=N; 2<=M} : b[j];
        {| 2<=i<=N; 1<=j<=M; 2<=M} : B1[i-1,j];
      esac;
A[i,j] = a;
C[i,j] = 
      case
        {| j=0} : 0[];
        {| 1<=j} : C[i,j-1] + A[i,j] * B1;
      esac;
\end{verbatim}
\caption{Definitions of \texttt{B}, \texttt{A}, and \texttt{C}
of program of figure~\ref{figder1} after uniformization}
\label{figder3}
\end{figure}

\subsection{\index{Scheduling}Scheduling and \index{mapping}mapping}
\label{schedAlpha}

A schedule of this program is found by the {\tt schedule} function.  
(see chapter~\ref{chapscheduling} for more details.) 
Here we call this function with the option
{\tt scheduleType} in order to find out
a schedule where 
all variables share the same linear part: this
preserve the uniformity of the program. 
The {\mmalfa} command which does this is: 
\begin{verbatim}
schedule[scheduleType->sameLinearPart];
\end{verbatim}
The result yields $T_C[i,j] = i+j$ and is illustrated in
figure~\ref{figder4}. The mapping must be given by the
user. Following the standard systolic methodology, the 
\index{allocation}allocation
function $a$ is a projection of the $n$-dimensional index space onto a
$n-1$-dimensional space. Here, we assume that we want to  project on the
$\vec{\jmath}$ axis ({\tt (i,j -> j)} (see figure~\ref{figder4}). This results
in a \index{linear array}linear array of $N$ processors.
 
\begin{figure}[htbp]
%\htmlimage{scale=\myscale}
\centerline{\includegraphics{figures/tr3}}
\caption{Scheduling and mapping of the uniform program}
\label{figder4}
\end{figure}


\subsection{Applying a spatio-temporal reindexing}
We now have all the spatio-temporal information -- timing, and spatial
mapping -- for obtaining a systolic array.  We can make this
information explicit in the {\alfa} program by performing a change of basis: 
in this way, in the new domains, the first index will represent the
execution date and the second index will represent the processor
number. 

%In our case, this corresponds to the \index{change of
%basis}change of basis by the function \texttt{f}: {\tt (i,j->i+j,j)}. 
%The following sequence of \mma{} expressions does
%the job: 
%\begin{verbatim}
%f = readMat["(i,j,N,M->i+j,i,N,M)"];
%id2 = readMat["(t,p,N,M->t,p,N,M)"];
%g = composeAffines[f,id2]
%changeOfBasis["C",g]
%changeOfBasis["A",g]
%changeOfBasis["B1",g]
%\end{verbatim}
%Recall that \texttt{id = expression} denotes an assignment in \mma{}.
%The first line assigns to symbol $f$ the expression obtained by
%parsing the
%\alfa{} matrix expression \texttt{"(i,j,N,M->i+j,i,N,M)"}
%using the {\tt readMat} function. 
%(Notice that this expression contains
%the parameters \texttt{N} and 
%\texttt{M} of the {\tt MatVect} program.)
%The {\tt id2}
%symbol is assigned a function to rename the indices 
%($(t,p)$ instead of $(i,j)$) for readability. 
%Then the change of basis 
%is performed on all local variables using the
%\texttt{changeOfBasis} functio. 

The function which does this transformation is:\\
\begin{verbatim}
applySchedule[];
renameIndices[{"t","p"}];
\end{verbatim}
\index{\texttt{applySchedule}}
(The command \texttt{rename} allows one to change the 
names of the indices.)
The resulting program is shown in 
figure~\ref{figder5}.  This program is scheduled and mapped: for
example, {\tt C[t,p]} will be computed at step {\tt t} on processor
{\tt p}.

Note that \texttt{applySchedule} chooses the 
mapping of the program. It is possible to modify this
mapping by applying another change of basis.\footnote{There is an 
incoherence here between the notebook and this chapter...}

\begin{figure}[htbp]
%\htmlimage{scale=\myscale}
\verbatiminput{alpha_progs/MVsched.alpha}
\caption{{\alfa} program after change of basis}
\label{figder5}
\end{figure}

\subsection{Generation of an {\alphaz} program}
\label{alphaZ}
The generation of {\alphaz} code consists of \index{Control signal}control
signal generation, and simplification of the code. 
The control signal
generation is necessary when the computation of a variable 
changes over execution time. For instance,
consider variable {\tt C} of the program of
figure~\ref{figder5}. 
When evaluating {\tt C[t,p]}, a processor {\tt p}
must know whether {\tt t=p} of {\tt t>p}, and this can only be done
with the help of a control signal. In addition, this control signal
should be inialized by a controller. 

The situation is different for variable \texttt{B1}.
There are two different definitions
of {\tt B1[t,p]}
depending on whether {\tt p=1} or {\tt p>1}: 
{\tt b[t-p]} and {\tt B1[t-1,p-1]}.
But these conditions
do not involve the time {\tt t}. As a consequence, 
no control signal needs to be generated, as these
conditions will be directly reflected in the structure of 
cell number 1, and cells number \texttt{p}, \texttt{p>1}.


\begin{figure}[htbp]
\begin{verbatim}
C[t,p] = 
    case
      {| p<=t<=p+M; 1<=p<=N; 2<=N; 2<=M} : 
         case
           {| t=p; 1<=p<=N; 2<=N; 2<=M} : (if (loadC) then 0[] else True[]);
           {| p+1<=t<=p+M; 1<=p<=N; 2<=N; 2<=M} : 
              (if (not loadC) then C[t-1,p] + A[t,p] * B1[t,p] else False[]);
         esac;
    esac;
loadC[t,p] = 
    case
      {| p<=t<=p+M; 1<=p<=N; 2<=N; 2<=M} : 
         case
           {| t=p; 1<=p<=N; 2<=N; 2<=M} : True[];
           {| p+1<=t<=p+M; 1<=p<=N; 2<=N; 2<=M} : False[];
         esac;
    esac;
\end{verbatim}
\caption{Definition of {\tt C} after space-time case separation, and 
control signal generation.}
\label{figder6}
\end{figure}

Control signal generation is performed by a single call to the \mma{}
function {\tt toAlpha0v2}:
\begin{verbatim}
toAlpha0v2[];
\end{verbatim}
The \alphaz{} program is shown in figure~\ref{figder7}. It can directly 
be interpreted as the array of figure~\ref{figder8}. 

\begin{figure}[htbp]
%\htmlimage{scale=\myscale}
\verbatiminput{alpha_progs/MVfinal.alpha}
\caption{ {\alphaz} program before translation in {\alphard}, this program 
represent the array of figure~\ref{figder8}}
\label{figder7}
\end{figure}

%Before generating control
%signals, we perform a preliminary transformation called 
%{\em space-time case
%separation}, which consists of replacing the usual one level case with
%a two level case, the first level involving conditions involving
%only processor numbers.  Figure~\ref{figder6} shows how 
%the definition of {\tt C} has been modified
%after the space-time case separation. Note
%that the first case level contains only one branch as there are no
%condition involving only processors indices.
%
%In the program of figure~\ref{figder5}, only {\tt C} needs a
%control signal. The variable needing a control signal 
%can be automatically detected (after space-time case separation)
%with the command {\tt needsMuxQ}. For instance,  
%{\tt needsMuxQ[getDefinition["C"]]} returns {\tt True}.
%The control signal generation 
%is made automatically with the function {\tt makeMuxControl}, for
%these actions, the commands are: 
%
%\begin{verbatim}
%spaceTimeCase["C", {1}, {2}];
%spaceTimeCase["A", {1}, {2}];
%spaceTimeCase["B1", {1}, {2}]; 
%makeMuxControl["C", "loadC", {1}, {2}];
%\end{verbatim}
%
%
%At this point, current implementation state imposes to fix the values
%of the parameters.  After the signal has been generated, it is
%necessary to pipeline it to the border of the array, this should be
%done with the {\tt pipeControl} command (or with the {\tt pipeline}
%command if you want a very specific pipeline). The last two command
%call two low-level transformations: split dependencies. Hence the
%{\mmalfa} commands are:
%
%\begin{verbatim}
%pipeControl["loadC",{1,0}];
%assignParameterValue["N",4];
%assignParameterValue["M",4];
%EquationSpatTemp[]
%makeSimpleExpr[]
%\end{verbatim}




\subsection{Obtaining an {\alphard} specification}

{\alphard} is a subset of {\Alpha} which represents hardware in a
structured manner. {\alphard} will be described precisely in
section~\ref{alphard}.  From the {\alphaz} specification, {\alphard}
code is obtained automatically by the command {\tt alpha0ToAlphard}

\begin{verbatim}
alpha0ToAlphard[controlVars[]];
\end{verbatim}

\begin{figure}[htbp]
%\htmlimage{scale=\myscale}
\centerline{\includegraphics{figures/tr6}}
\caption{\index{array}Array obtained after high level synthesis (here for $N=3$)}
\label{figder8}
\end{figure}

\section{Appendix: list of commands}
\begin{verbatim}
addLocal["A = a"];
pipeall["C","b.(i,j->j)","B1.(i,j->i+1,j)"];
schedule[scheduleType->sameLinearPart];
applySchedule[];
renameIndices[{"t","p"}];
toAlpha0v2[];
alpha0ToAlphard[controlVars[]];
\end{verbatim}
