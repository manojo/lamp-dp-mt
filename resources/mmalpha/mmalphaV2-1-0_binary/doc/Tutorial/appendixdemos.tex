\chapter{The demos of {\mmalfa}}
\label{demos}

{\sc Version of \today{}}

This chapter presents a few demos of
{\mmalfa}.\index{demos,demonstrations} 
There two types of demos: some demos
have a notebook interface, and the others
should be executed in kernel mode. There are also various
documentations available with the \alfa{} archive.

\section{Notebook demos}
All these demonstrations can be accessed directly by 
hyperlinks of the 
so-called {\em master} notebook, which is started by
evaluating the command \texttt{start[]}.

\begin{description}
\item [Introduction.] Introduction to {\Alpha}

\item [Domlib.] Elementary computations on domains with Domlib.

\item [Fir: Design of a FIR filter.]  The FIR demo shows the typical
steps for the synthesis of a linear architecture for a Finite Impulse
Response filter.  The demo comprises the following steps: loading the
initial program, pipelining, scheduling, assign parameter values,
control signal generation, {\alphaz} generation, {\alphard} generation.

\item [matvect: architecture for matrix vector
multiplication.] 

The MATVECT demo shows the typical steps for the
synthesis of a linear architecture for a Matrix Vector Multiplication.
The demo comprises the following steps: loading the initial program,
pipelining, scheduling, assign parameter values, control signal
generation,  {\alphaz} generation, {\alphard} generation.

\item[Tutorial\_intro: ]
very introductory tutorial. Corresponds to document~\cite{first-tutorial}.

\item[mma-intro: ]
quick introduction to \mma{}. Explains how Alpha ASTs are represented
in \mma{}. Warning: the abstract syntax presentation may not be as reliable as
the appendix of this manual (see chapter~\ref{appendixalpha}).
\item[Kalman: ]
this example describes an application of \mmalfa{} to a Kalman
filter. This application, quite complex, shows how to use 
structured scheduling. Reference~\cite{parelec} describes the 
background of this application. 
\item[Kalman-Sqrt: ]
this example describes an application of \mmalfa{} to a Kalman
filter, using a so-called square root covariance method. 
This application, quite complex, shows how to use 
structured scheduling. Reference~\cite{cce} describes the 
background of this application. 

\item[Neural-Network: ]
this notebook is not ready yet, but will be soon.
\item[Fuzzy-Logic: ]
an application of fuzzy logic to channel equalization. 
\end{description}

Another serie of notebook allows one to become familiar with
different packages of \mmalfa{}: 
\begin{description}
\item[Alpha: ] the Alpha package.
\item[Control: ] the Control package.
\item[Matrix: ] the Matrix package.
\item[ChangeOfBasis: ] the ChangeOfBasis package.
\item[Normalization: ] the Normalization package.
\item[Cut: ] the Cut package.
\item[Decomposition: ] the Decomposition package.
\item[Static: ] the Static package.
\item[PipeControl: ] the PipeControl package. 
\item[Vhdl: ] the Vhdl package.
\item[CheckAlpHard: ] the AlphHard package.
\item[Meta: ] the Meta package.
\item[Schedule: ] the various schedule packages.
\item[dataFlowSchedule: ] 

\item[matlib: ]

\item[Visual: ] the Visual package. 

\end{description}



\section{kernel demos}
These demos are avaiblable at the following address:
\begin{verbatim}
/udd/alpha/alpha\_beta/Mathematica/demos/demos.html
\end{verbatim}

\begin{description}
 \item [INIT: An initiation to {\mmalfa}.] The INIT demo shows a few
  basic functions of MMAlpha. The demo comprises the following steps:
  loading and showing a program, normalizing, showing domains etc...
  \index{INIT demo}

 \item [FIR: Design of a FIR filter.]  The FIR demo shows the typical
steps for the synthesis of a linear architecture for a Finite Impulse
Response filter.  The demo comprises the following steps: loading the
initial program, pipelining, scheduling, assign parameter values,
control signal generation, {\alphaz} generation, {\alphard} generation.\index{FIR demo, Finite impulse
response filter}

 \item [MATVECT: architecture for matrix vector
multiplication.] 
The MATVECT demo shows the typical steps for the
synthesis of a linear architecture for a Matrix Vector Multiplication.
The demo comprises the following steps: loading the initial program,
pipelining, scheduling, assign parameter values, control signal
generation,  {\alphaz} generation, {\alphard} generation.\index{MATVECT demo,
Matrix vector architecture}

% \item [VERIF: Verification of the Matrix vector architecture.]
%The VERIF demo shows the verification of a linear architecture for a
%Matrix Vector Multiplication. The demo shows the use of subsystems,
%the inlining of subsystems, the use of libraries, C code generation
%and execution.\index{VERIF demo}


 \item [SVD: analysis of a Singular Value Decomposition
algorithm.]  The SVD demo shows the analysis of a larger Alpha
program. It shows the use of subsystems, the nlining subsystems, the
CNF code generation, the C code generation and execution.

% \item [SVDUPDATE: another demo on Singular Value Decomposition.]
%The SVD demo shows the analysis of another version of the SVD algorithm. 

 \item [BINMULT: Synthesis of a multiplier.]  This demonstration
shows the synthesis of a bit-serial multiplier, from an initial
specification structured into several parameterized subsystems.  The
demo script contains the following steps: loading and displaying a
multi-system program, conversion into C code for simulation,
pipelining of the diffusions in the initial program, flattenning all
the systems for synthesis, scheduling before space/time
transformation, space/time transformation, pipelining of the inputs
(so that all the data enters the first processor in bit-serial
manner), simulation of the resulting array.

\item [ESTIM: synthesis of motion estimation] a two dimensionnal example, with 
simulation, on a real image, derivation of {\alphaz} code (warning, long demo).
\end{description}