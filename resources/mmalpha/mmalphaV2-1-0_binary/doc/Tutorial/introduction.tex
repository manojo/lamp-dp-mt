\chapter*{Foreword}
\mmalfa{} is a free software, available under the 
\gpl{}. It can be downloaded from 
the site:\\
\sitemmalfa{}\\
If you have any problem 
in understanding this document or trying the \mmalfa{} software, please 
send an e-mail to the following address:
{\tt alpha@irisa.fr}. 

Although we cannot guarantee a strong support, we shall try
our best to help solving your problem. 

Many people contributed to \mmalfa{}. 
\begin{itemize}
\item Pierrick Gachet designed an early version of \alfa{}.
\item Christophe Mauras designed the first real version of \alfa{}, and
implemented the ancestor of \mmalfa{}, called {\sc Alpha du Centaur}.
\item Herv� Le Verge continued the work of Christophe Mauras, and 
worked on reduction operators. He also implemented the first version
of the polyhedral library.
\item Doran Wilde designed the first version of the parser, of the
pretty-printers, of \texttt{WriteC}, and of many early packages
of \mmalfa{}. He developped a new, more efficient version of the 
polyhedral library. 
\item Zbignew Chamsky contributed to the design of the first
\mmalfa{} packages. 
\item Florent Dupont de Dinechin implemented the structuration of 
\alfa{}, and deeply improved the \mmalfa{} environment. 
\item Tanguy Risset is the conductor of the \mmalfa{} 
entreprise. He designed the scheduler, and many other packages.
\item Patrice Quinton was the inspirer of \alfa{}, and implemented
several minor packages of \alfa{}. He also designed numerous
notebook. 
\item Sanjay Rajopadhye insuffled many ideas and designed the 
reduction expansion package. 
\item Patricia Le Moenner implemented the Vhdl generator.
\item Fabien Quiller� is currently implementing a new C generator.
\end{itemize}



We also thank the EEC, who sponsored the development of \alfa{}
through several grants. The late C$^3$ research cooperation 
network also helped initial research on \alfa{}.



\chapter{Introduction}

\label{chap1}
In this document, we introduce the basics of {\alfa} and we explain
how to use {\mmalfa}.



%What is Alpha
\section{What is \Alpha?}
{\alfa} is a language for the specification and the derivation of
systolic algorithms and architectures, and more generally, for the
specification and design of parallel algorithms and architectures based on
the formalism called the {\em polyhedral model}. 



\label{Whatis}
\nocite{Alpha97a,Alpha97b}

% History
{\Alpha} is a functional data parallel language invented in the {\api}
research group in the {\irisa} laboratory in Rennes. The project was
headed by Patrice Quinton and the first definition of the language was
proposed by Mauras in~\cite{Mauras89} in 1989. The original motivation
was to provide a language for expressing algorithms in an extended
version of the formalism of recurrence equations proposed by Karp,
Miller and Winograd~\cite{KaMiWi67}. The goal of this language was to
provide a high level tool for the synthesis of parallel regular
{\vlsi} architectures.



Although {\Alpha} stands for the language itself, it is often also
associated with the environment in which it is currently developed:
{\bf {\mmalfa{}}}. {\mmalfa{}} is an interface based on the {\mma{}}
software from which one can manipulate {\Alpha} programs.



%Motivations of {\alfa}

\section{What is {\Alpha} for?}
\label{whatfor}
{\Alpha} is currently a research tool which provides research problems
in different computer science areas fields like: functional language semantics,
parallelization, code generation, optimization, polyhedral theory,
{\vlsi} synthesis, systolic arrays, etc.



One important long term goal is to promote the use of high level
functional languages for the synthesis of parallel {\vlsi}
architectures. 



%What can we do with it?

\section{What can you expect from \alfa{}?}
From the short term point of view, \alfa{} can be
useful for:
\begin{itemize}
\item Providing a correct recurrence equation specification for a particular 
algorithm.

\item Simulating such a specification.

\item Transforming and simplifying a recurrence equation specification.

\item Computing on convex polyhedra.

\item Scheduling and detecting parallelism.

\end{itemize}

If you are ready to invest a little more time on {\mmalfa{}}, you will 
probably be able to 
\begin{itemize}
\item Describe a systolic architecture with {\AlpHard} (see
section~\ref{alphard}).
\item Generate {\vhdl} from this description.
\item Provide a path from high level functional specification of
an algorithm to the layout description of a {\vlsi} algorithm
realizing it.

\end{itemize}

If you becomed hooked to \alfa{}, you will be able to take advantage
of a {\em completely open framework} in which you will be 
able to design your own tools, or interface your own programs: 
the source code of all functions described in this manual is
available. 



% Principles

\section{How does \mmalfa{}{} work?}

{\mmalfa{}} is written in C (for a small part) and in
\mma{} (for a larger part). 



The user should only see the \mma{} interface.
\mma{} provides an interpreted language with high level built-in
function for symbolic computations:  {\mmalfa{}} uses these facilities for
transforming {\Alpha} programs. \mma{} has also a nice programming
language in which one can do general computation as in any other
programming language.



The basic principle of the {\mmalfa{}} environment is the following one:
\mma{} stores an internal representation of an {\Alpha} program
(called Abstract Syntax Tree or AST) and perform computations on this
internal representation via user's commands. These commands can be for
example: viewing the {\Alpha} program, checking its correctness,
adding a variable, generating C code to simulate it, generate a
systolic array, generate \vhdl{} code, etc. All the computations
happen on the internal AST which is stored in the \mma{} environment. 



Specific modules in C are used for two purposes: various parsers and
unparsers and computations on polyhedra. All the C functions are
accessed via \mma{}, but they can also be called from a C program:
this is particulary useful for the polyhedral library.  The polyhedral
library, written by H. Le Verge and D. Wilde (see
appendix~\ref{polylib}) is the computationnal kernel of the
{\mmalfa{}} environment. Its use provides many transformations that
cannot be done easily by hand.





%Describe the document

\section{What is in this document}
The organization of this document is as follows. 
Chapter~\ref{chapstart} is a brief overview of the 
\mmalfa{} environment. 



In chapters~\ref{chapalphascal} and \ref{chapalphaarray}, we introduce
{\alfa} following an architecture-oriented approach. In other words,
we describe {\alfa} programs whose interpretation is always an
architecture.



Chapter~\ref{chapderiving} illustrates the
derivation of systolic architectures from a high-level description.



In chapter~\ref{chapfunctional}, we introduce {\alfa} as a functional 
data parallel programming language. This point of view is complementary
to that of the first two chapters, in that it show how to use {\alfa}
as an algorithmic language.



Chapter~\ref{chapssubsystems} deals with structured and parameterised
{\alfa}, which gives the full power of the language, both for the 
description of hardware and the specification of algorithms.



Chapters~\ref{chapanalyze} and~\ref{chapcodegen} are more ``practical''
and tool-oriented.  In chapter~\ref{chapanalyze}, we describe the
static analyzer of {\mmalfa}\,: it is a useful tool to check some
properties of an {\alfa} program.  Chapter~\ref{chapcodegen} explaines
how to generate code and simulate {\alfa} programs.



Chapter~\ref{chapscheduling} is devoted to scheduling {\alfa}
programs. Scheduling is one of the essential steps of the derivation
of architectures as well as compiling {\alfa} to sequential and
parallel machines. 



Chapter~\ref{chapalfhard} (still under heavy work...) is
intended to explain how \AlpHard{} is defined
and how to 
generate \vhdl{} code frome \alfa{}.



%A roadmap for the tutorial.

\section{How to read this tutorial?}
Depending on your interest in \mmalfa{}, you will be interested
in particular chapters of this tutorial. 



First, read chapter~\ref{chapstart}: it does not hurt... and 
does not take too much time. By the way, there is a separate, 
slightly extended version of this chapter available in document
~\cite{gettingstarted}.



Chapter~\ref{chapderiving} will be worth reading, if you are
interested in systolic array design.



You may skip chapters~\ref{chapalphascal} and \ref{chapalphaarray}
if you are not immediately interested in deeply understanding 
the mechanisms of \alfa{}. 



If you are interested in writing a "real" \alfa{} program, 
you will have to read chapters~\ref{chapssubsystems}, 
\ref{chapanalyze}, \ref{chapcodegen}, and~\ref{chapscheduling}. 



Finally, if you want to produce "real hardware", read chapter
~\ref{chapalfhard}.



\section{Other documents}
This tutorial is not the only information available on 
\alfa{} and \mmalfa{}. There are papers, for the academic 
oriented people. There are also many demonstration notebooks. 
To access them, start \mma{}, type the command \texttt{start[]}, 
and enjoy\footnote{or dislike...}.



\section{What we plan in the (near?) future}
The current tutorial is not yet complete. Several aspects of 
{\alfa} and {\mmalfa} (already implemented) need to be described
in detail. Among them, the generation of {\sc Vhdl}, the subset
of {\alfa} called {\alfhard}, and the functions in {\mmalfa} 
which allow one to manipulate convex domains and {\alfa} 
programs. Knowing how long it takes to write and check a 
documentation, we have chosen to release a \mmalfa{} version 
without a complete documentation, for those who are interested
in working with us. 



