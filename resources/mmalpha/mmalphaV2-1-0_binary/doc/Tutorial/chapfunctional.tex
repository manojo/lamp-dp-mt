\chapter{The Alpha Language and System -- Another point of view}
\label{chapfunctional}
\section{Introduction}

\begin{tobedone}
The last figure is garbage. The psfig inside has been commented out. 
See reference document in old pdf turorial.
\end{tobedone}
Alpha is a functional language, based on recurrence
equations\index{recurrence equation}.  All variables are type-declared
at the beginning of a program, and represent multidimensional arrays,
whose shapes can be arbitrary polyhedra.  For example, the declaration
\verb!A: {i,j| 0<j<i<=N} of real!, specifies a (strictly) lower
triangular, real matrix\index{lower triangular linear system}.  To
introduce the main features of the language, consider the problem of
solving a system of linear inequalities, $Ax=b$, where $A$ is a lower
triangular, $n\times n$ matrix with unit diagonal.  It is well known
that this can be solved using forward substitution\index{forward
substitution}, as given by the following formula.  \[\mbox{for } i =
1\ldots n,~~x_i = b_i - \sum_{j=1}^{i-1} A_{i,j} x_j\]

\begin{figure}[tb]
%\htmlimage{scale=\myscale}
\setlength{\unitlength}{10mm}
\begin{center}
%\begin{picture}(16,2.5)
%\thicklines
%\put(-0.1,-0.7){\framebox(16.2,3.7){}}
%\begin{boxedminipage}[b]{16cm}
%{\footnotesize
\begin{verbatim}
system ForwardSubstitution : { N | N>1 }       -- comments are like this
                 ( A : { i,j | 0<j<i<=N } of real;    -- a 2D input variable
                   B : { i   | 0<i<=N } of real)      -- a 1D input variable
         returns ( X : { i   | 0<i<=N } of real );    -- a 1D output variable
  let
    X[i] = B[i] - reduce(+, (i,j -> i), A * X.(i,j->j))[i]
  tel;
\end{verbatim}
%\end{boxedminipage}
%\end{picture}
\end{center}
%\vspace*{-0.8cm}
\caption{Alpha program for the forward substitution algorithm}
\label{fig-fs}
\end{figure}

  The corresponding Alpha program (Fig.~\ref{fig-fs}) is identical, except for
syntactic sugar.  The first line names the system and declares that it has a
positive integer parameter, {\tt N}.  The next three lines are declarations of
the input and output variables of the system.  Note how each variable is
defined over a certain range of indices, specified by a set of linear
inequalities.  This defines a polyhedron, called its {\em domain}.  For
example, the domain of {\tt A} is triangular, while {\tt B} and {\tt X} are
one dimensional vectors.  Variables declared before the ``{\tt returns}''
keyword are input variables, and those after it are outputs.  Although our
example does not illustrate it, a system may also have local variables, which
are declared after the system header, using the keyword {\tt var}.  The body
of the program is a number of equations delineated by the {\tt let} and {\tt
tel} keywords.  Here, we have a single equation corresponding to the above
formula, which specifies that we compute {\tt X[i]} by performing a reduction
(any associative and commutative binary operator may be used; here we have
{\tt +}).  The {\tt (i,j->i)} specifies a projection of a two dimensional
index space to one dimension.  Intuitively, it states that within the body of
the {\tt reduce}, we have two index variables, {\tt i} and {\tt j}, but the
latter is not visible outside the scope of the {\tt reduce}.  The body of the
{\tt reduce} is the expression, {\tt A * X.(i,j->j)}, where {\tt (i,j->j)}, is
a {\em dependency}. It denotes the fact that to compute the body at index
point, {\tt [i,j]}, we need the value of {\tt X} at index point {\tt j} (the
dependency of {\tt A} is not explicitly written---it is the identity).
Dependencies are an important aspect of Alpha.  They describe ``indexing
functions'' and have the syntax,
 \verb! (idx, idx, ...  -> i-expr, i-expr, ...)!, where each {\tt idx} is an
index name, and {\tt i-expr} is an {\em affine} expression of the system
parameters and the {\tt idx}'s.  The Alpha system uses this syntax for
specifying a multidimensional affine function in many different contexts.

\section{Polyhedra and affine functions}
  Alpha imposes the restriction that all variable domains are polyhedral, and
all dependencies are {\em affine} functions.  Although this may constrain our
expressive power, it is a conscious choice that was made for many reasons.  A
large number of linear algebra and DSP and image processing algorithms fall
within these limitations.  It is also known 
%\cite{feautrier91}, 
that any
nested loop program (in a conventional language) that satisfies certain
constraints (loop bounds are affine functions of the parameters and outer
indices, and the body contains only assignments to array variables which are
accessed using only affine functions of the loop indices) can be modeled by a
formalism that is similar to an Alpha program.  Such assumptions are made in
many techniques used in parallelizing compilers.  By concentrating explicitly
on this subset of programs, we can make use of powerful static analysis and
optimization tools based on polyhedra and linear programming (implemented in
our {\bf polyhedral library}.
%\cite{wilde93-poly}).  
Finally, it is precisely
these restrictions that interact coherently and form the foundation of our
transformational system.  They also ensure that all Alpha programs can be put
into a ``normal'' form, and in fact, all examples that we use here have been
so normalized.

\section{Transformations in Alpha}

  We now introduce some of the Alpha transformations used in converting our
program into a single assignment C program.  
\subsection{Serialization}
The first one is used to {\bf
serialize} the reduction operation, and requires two parameters: the
``direction'' in which we accumulate the partial sums, and the name of the
temporary variable.  For our example, if we choose to accumulate in the {\em
increasing} direction of {\tt j} (specified as ``{\tt (i,j->i,j+1)}'') and
name the new variable, {\tt f}, we obtain the Alpha program shown in
Fig.~\ref{fig-serial}.  We also see two new syntactic constructs, the {\em
case} (which has the usual meaning), and the {\em restrict}, which has the
syntax,
 \verb!<polyhedron>: <expr>!, and denotes the expression, {\tt <expr>} but
restricted to the subset of index points in {\tt <polyhedron>}.

  Observe that the domain of {\tt f} is \verb!{i,j| 0<=j<i; 2<=i<=N}! (a
``nearly triangular'' trapezium defined by the points {\tt [2,0]} {\tt [2,1]}
{\tt [N,N-1]} and {\tt [N,0]}).  Also note that the equation for {\tt X} now
has the expression {\tt f[i,i-1]} replacing the reduction, and the new
equation for {\tt f} consists of an initialization at the boundary, and an
accumulation using the {\em body of the original reduction}.  All this is
automatically determined by the system, using the polyhedral library.

\begin{figure}[tb]
%\htmlimage{scale=\myscale}
\setlength{\unitlength}{10mm}
\begin{center}
\begin{picture}(16,5.7)
\thicklines
\put(-0.1,-0.7){\framebox(16.2,6.7){}}
\put(11,0.5){\framebox{The domain of {\tt f}.}}
\put(11,1.3){\begin{picture}(6,3)
              \put(0,0){\includegraphics{figures/domain-f.pdf}}
              \put(-0.3,3.2){$i$}
              \put(0.5,3.6){$j$}
              \put(-1,2.7){\small\tt [2,0]}
              \put(1,2.7){\small\tt [2,1]}
              \put(-1,0){\small\tt [N,0]}
              \put(3.4,0){\small\tt [N,N-1]}
              \end{picture}}
\begin{boxedminipage}[b]{16cm}
\hspace*{2cm}$\vdots$\small
\begin{verbatim}
var
  f : {i,j | (2,j+1)<=i<=N; 0<=j} of real;
let
  f[i,j] = case
        {| j=0} : 0[];
        {| 1<=j} : f[i,j-1] + A * X[j];
           esac;
  X[i] = case
        {| 2<=i} : B[i] - f[i,i-1];
        {| i=1} : B[i];
         esac;
tel;
\end{verbatim}
\end{boxedminipage}
\end{picture}
\end{center}
\vspace*{-0.8cm}
\caption{The forward substitution program after serialization}
\label{fig-serial}
\end{figure}

\subsection{Change of basis}
  Perhaps the most important transformation in the Alpha system is the {\bf
change of basis}.  The intuition behind it is as follows.  Since an Alpha
variable can be viewed as a multidimensional array defined over a polyhedral
domain, we should be able to ``move'' (or otherwise change the ``shape'' of)
its domain and construct an equivalent program.  For example, suppose we want
to rotate the domain of {\tt f} in Fig.~\ref{fig-serial} counterclockwise
by $45^\circ$, i.e., we desire to map the point {\tt [1,0]} to {\tt [1,1]},
and {\tt [1,1]} to {\tt [0,1]} (this makes a vertical line diagonal, and a
diagonal one horizontal).  It is easy to verify that the mapping
 \verb!(i,j -> i-j,i)! achieves this; the resulting program and transformed
domain are shown in Fig.~\ref{fig-transf} (the changes are highlighted).  In
general, when variable {\tt V} is transformed, the system must determine: (i)
its new domain, (ii) the new case structure of its equation, (iii) the new
dependencies for the uses of {\em all} variables in the equation for {\tt V},
and (iv) the new dependencies for {\em uses} of {\tt V} (in all other
equations).  All of this is done automatically using the polyhedral library,
and relies on the fact that the language allows a coherent interaction among
the domains, the dependencies and the (normal form) structure of the programs.

\begin{figure}[tb]
%\htmlimage{scale=\myscale}
\setlength{\unitlength}{10mm}
\begin{center}
\begin{picture}(16,5)
\thicklines
\put(-0.1,-0.7){\framebox(16.2,6.3){}}
%\put(3.4,4.5){\oval(4.7,0.5)}
%\put(2.5,3){\oval(2.2,0.9)}
%\put(6.9,2.8){\oval(5.9,0.5)}
%\put(5.5,1.55){\oval(1.4,0.5)}
\put(11,1){\begin{picture}(6,3)
             \put(0,0){\includegraphics{figures/rot-f}}
             \put(-0.3,3){$i$}
             \put(0.5,3.6){$j$}
             \put(0.3,2.3){\small\tt [2,2]}
             \put(0.8,3.5){\small\tt [1,2]}
             \put(3,3.5){\small\tt [0,N]}
             \put(2.5,-0.3){\small\tt [N,N]}
              \end{picture}}
\begin{minipage}[b]{16cm}
\hspace*{2cm}$\vdots$\small
\begin{verbatim}
  f : {i,j | 1<=i<=j; 2<=j<=N} of real;
let
  f[i,j] = case
        {| i=j}    : 0[];
        {| i<=j-1} : f[i+1,j] + A[j,-i+j] * X[-i+j];
           esac;
  X[i] = case
        {| 2<=i} : B[i] - f[1,i];
        {| i=1} : B[i] - 0[];
         esac;
tel;
\end{verbatim}
\end{minipage}
\end{picture}
\end{center}
\vspace*{-0.8cm}
\caption{The transformed program and the new domain of {\tt f}}
\label{fig-transf}
\end{figure}

The syntax for the affine mapping in a change of basis
transformation is the same as for dependencies, but there is a subtle
difference.  We should be able to transform the variable in any
(affine) manner, but we must always ensure that the {\em number} of
index points remains invariant.  For this, the affine change of basis
must admit an integral inverse (i.e., be {\em unimodular}).  We have
also developed a {\em generalized change of basis}, which allows
``non-square'' mappings (non-unimodular transformations are encoded as
a special case of this).


