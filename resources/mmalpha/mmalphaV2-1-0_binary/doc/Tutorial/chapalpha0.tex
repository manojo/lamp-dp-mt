\chapter{The Alpha0 format}
\label{chapalfha0}

{\sc Version of \today}

In this chapter, we describe the {\Alpha0} format which is basically a non-structured version of the {\AlpHard} language. We also 
 describe the methodology to go from an {\Alpha} program to an {\Alpha0} program and then how to translate it into {\AlpHard}. 

\subsection{Definition of {\AlphaZ}}
{\AlphaZ} was introduced by C. Dezan in her Phd thesis
(\cite{dezan}). It was conceived as the lowest level of {\Alpha} or
equivalently as a subset of {\Alpha} which can describe circuits at the 
register transfer level. the main weakness of this subset came from the fact
that {\Alpha} programs were not structured.  Since, the structuring of
{\Alpha} was provided by F. Dupont~\cite{DupontQuRi95} and the
structured version of {\AlphaZ} was studied by P. Le
Mo\"enner~\cite{Patricia96}.  {\AlpHard} (described in
chapter~\ref{chapalphard}) is now intended to be this subset of
{\Alpha} with structural interpretation, from which we can translate
{\Alpha} into {\vhdl} or other {\sc rtl} description
languages. However, during the transformation path from {\Alpha} to
{\AlpHard}, the automatic structuring is a complicated process, hence
it appeared that the {\AlphaZ} format is still necessary as an
intermediate format with all the register transfer level informations
but in an unstructured form. We describe hereafter, the second version
 the {\AlphaZ} format (Alpha0v2) which is slightly different from the version described in~\cite{dezan}


% As a platform to {\AlpHard}, we need to remove the second point of
% the previous definition, namely the fact that all temporal domains are
% of the form $\{t~|~ t>=\}$. The other basic principles remains the same
% with slight differences. spatial cases contain no dependencies,
% connections are represented using a simple space dependency (no
% case). Temporal cases are always nested in a spatial case.
% Control initialization  equations are called {\em control} equation,
% the {\em Mirror} equations are the equations affecting  input, output 
% and control signal to local variables (hence, control signal 
% become data signals). All these descriptions are summarized by the following 
% definition:

An {\Alpha} program is in {\AlphaZ} form if it contains no {\tt use} statement
and if it meets the following conditions:

\begin{enumerate}
\item there exits, for all declared domains (except for the domains of 
inputs and outputs variables) an interpretation function for the indices
(see~\cite{dezan} for precise definition of {\em interpretation function}).
 Basically, each index is either 
a temporal index (indicating living dates of the signal) or a spatial index
(indicating in which processor the signal lives).


\item The equations of the system define the output and local
 variables.  Each operator involved in the equations has a structural
 interpretation. these equations are of four types: data equations,
 connection equations, control equations and mirror equations.
\begin{itemize} 
\item data equations define the different signals of the program
which are necessarily local variables.
They are composed of the following operators:  
\begin{itemize} 
  \item pointwise operators represent the corresponding combinatorial 
operators. 
  \item restriction are used to restrict the spatial interpretation of 
indices.
\item  dependencies are temporal dependencies representing registers 
or identity dependencies representing connections between two signal inside one cell.
\item case are spatial case allowing to gather several signals.
\item the {\tt if} operator (with nested {\tt case} in each branches)
is interpreted as a multiplexer.
\end{itemize}
\item connection equations are limited to the following form: a single
spatial dependency between two signals ({\tt A=B.(t,p->t,p-1)} for
instance).
\item control equations allow initialize control signals. they define signals
which are only temporal (no spatial indices).
\item Mirror equations: equation between input variables of the system and 
local variable of the system (only for interface). equation limited to an 
affine function applied to a variable ({\tt A[t,p]=a[f(t,p)]} for input mirror
equation and {\tt b[i,j]=B[f(i,j)]} for output mirror equations)..
\end{itemize}
\end{enumerate}

Examples of {\AlphaZ} programs are present in the directory 
{\tt \$MMALPHA/examples/Alpha0}, a small example is presented hereafter
 (figure~\ref{fig04}).

\section{From {\Alpha} to {\AlphaZ}}
The first attempt to go from High level specification in {\Alpha}
to {\sc rtl} specification in {\Alpha0} has been done be Sie and 
Wilde~\cite{SieWilde94}. This methodology has been 
extended and we discribe it briefly here, more informations
can be obtained by executing the Notebooks demos: {\tt Fir} and {Matvect}.

Consider the very simple program of figure~\ref{fig01}

{\small
\begin{figure}[h]
\begin{verbatim}
system NtimeNot :{N | 2<=N}
                (a : {i | 1<=i<=N} of boolean)
       returns  (b : {i | 1<=i<=N} of boolean);
var
  Acc : {i,j | 1<=i<=N; 1<=j<=N} of boolean;
let
  Acc[i,j] = 
      case
        {| j=1} : a[i];
        {| 2<=j} : not Acc[i,j-1];
      esac;
  b[i] = Acc[i,N];
tel;
\end{verbatim}
\caption{An {\Alpha} program computing {\tt N} times {\tt Not} 
on an array {\tt a}}
\label{fig01}
\end{figure}
}

Translating the program in {\AlphaZ} consists in the following steps: 
\begin{enumerate}
 \item {\bf Uniformize the program}. Here the program is already
uniform, in general this procedure may be complicated {\mmalpha}
provides tools for the automatic or designer-guided uniformization
(see the documentation on the chapter~\ref{chapuniformize}).
\item {\bf Schedule the program}. It consists in finding an execution
date (and a place also) for each computation. Usually, once
uniformized, the program should be scheduled with a unique linear part
(see the chapter~\ref{chapscheduling} for detail on scheduling). The
function {\tt applySchedule} choose an allocation function. In our example, 
you can execute  the following commands:
\begin{verbatim}
schedule[]
applySchedule[]
renameIndices[{"t","p"}]
\end{verbatim}
We obtain a linear array of processors,
the resulting program is shown in figure~\ref{fig02}
{\small
\begin{figure}[h]
\begin{center}
\begin{verbatim}
system NtimeNot :{N | 2<=N}
                (a : {i | 1<=i<=N} of boolean)
       returns  (b : {i | 1<=i<=N} of boolean);
var
  Acc : {t,p | 1<=t<=N; 1<=p<=N} of boolean;
let
  Acc[t,p] = 
      case
        {| t=1} : a[p];
        {| 2<=t} : not Acc[t-1,p];
      esac;
  b[i] = Acc[N,i];
tel;
\end{verbatim}
\caption{Program of figure~\ref{fig01} after applying schedule}
\label{fig02}
\end{center}
\end{figure}
}

\item {\bf Pipeline Inputs and Output}. You can see in figure~\ref{fig02}
that the  {\tt a} is input in every processors. If we want only the first processor to input data from the host, we could have pipelined the input with the following command:\\
\verb/pipeIO["Acc","a.(t,p->p)","aIn.(t,p->t+1,p+1)","{t,p| p>=1}"]/. We can do the same treatment for the output. Here we chose not to pipeline the I/O for sake of simplicity.

\item {\bf Go Down to Alpha0}. When the previous steps have been correctly 
executed, this step should be automatic with {\tt toAlpha0v2[]} command. In 
Our case, if we apply this function to the program of figure~\ref{fig02},
the printing on the screen are the following:
\begin{verbatim}
In[__]:= toAlpha0v2[];
Time index: {1}  space indices: {2}
Calling spaceTimeDecomposition[];
Calling makeAllMuxControl[];
  Equation of Acc...
   is in ST form
    Adding multiplexer.
  Equation of b...
Calling pipeAllControl[];
  Pipelining control for: Acc_ctl1
     From dimension 2 To dimension 1
       Warning, no pipe vector was found for control signal Acc_ctl1
       --> assuming broadcasted signal
     Control generated in cell: {p | 1<=p<=N; 2<=N}
Calling decomposeSTdeps[];
  In equation of Acc, adding a local variable: Acc_reg1
 Decomposing the space/time dependencies
Calling makeInputMirrorEqus[];
  Adding mirror equation for input a
Out[__]:=
\end{verbatim}

The important treatment here is the generation and pipeline of a
control signal. Indeed, the condition {\tt t=1} of the program of
figure~\ref{fig02} has to be realized in hardware.  Hence a control
signal was generated and a multiplexer added in each processors. Then
{\mmalpha} tries to pipeline the control signal (it often happens in
systolic architectures that the control can be pipelined). In our
particular example, all processors start their computation
simultaneously, hence the control signal is not pipelined but
broadcasted. Other treatments are just syntactic rewriting. The result
{\AlphaZ} program is show on figure~\ref{fig03}. note the precise information
that appears on the life time of each signal (in this simple program,
each signal lives from step 1 to step {\tt N}, but in general it is much 
more complicated) and on the input date of each data (mirror equation 
contains the information of {\em how to use} the hardware -- i.e. interface 
specification --).
\end{enumerate}
{\small
\begin{figure}[h]
\begin{verbatim}
system NtimeNot :{N | 2<=N}
                (a : {i | 1<=i<=N} of boolean)
       returns  (b : {i | 1<=i<=N} of boolean);
var
  a_mirr1 : {t,p | t=1; 1<=p<=N} of boolean;
  Acc_reg1 : {t,p | 2<=t<=N; 1<=p<=N} of boolean;
  Acc_ctl1_In : {t,p | 1<=t<=N; 1<=p<=N} of boolean;
  Acc : {t,p | 1<=t<=N; 1<=p<=N} of boolean;
  Acc_ctl1 : {t | 1<=t<=N} of boolean;
let
  a_mirr1[t,p] = a[t+p-1];
  Acc_reg1[t,p] = Acc[t-1,p];
  Acc_ctl1_In[t,p] = Acc_ctl1[t];
  Acc_ctl1[t] = 
      case
        case
          {| t=1} : True[];
          {| 2<=t} : False[];
        esac;
      esac;
  Acc[t,p] = 
      case
        {| 1<=t} : if (Acc_ctl1_In) then 
               case
                 {| t=1} : a_mirr1;
                 {| 2<=t} : False[];
               esac else 
               case
                 {| t=1} : False[];
                 {| 2<=t} : not Acc_reg1;
               esac;
      esac;
  b[i] = Acc[N,i];
tel;
\end{verbatim}
\caption{Program {\AlphaZ} derived from the program of
figure~\ref{fig01}.  This {\AlphaZ} program describes a linear array
of {\tt N} processors.  The equation defining {\tt Acc} represent a
multiplexer controlled by {\tt Acc\_ctl1} selecting {\tt a\_mirr1} or
{\tt Acc\_reg1}. The equation defining {\tt Acc\_ctl1\_In} is a control
equation. The equations defining {\tt a\_mirr1} and {\tt b} are mirror
equations. The equation defining {\tt Acc\_reg1} is interpreted as a
register. The equation defining {\tt Acc\_ctl1\_In} is interpreted as 
a connection equation: broadcast of the control  signal {\tt Acc\_ctl1}
to all processors.}
\label{fig03}
\end{figure}
}
\section{From {\AlphaZ} to {\AlpHard}}
The translation to {\AlpHard} is automatic. the command to use is {\tt
alpha0toAlphard[]}. {\mmalpha} automatically detects all the spatial
regions on which the computations are identical (the whole spatial
region in our case: {\tt $\{ p \| 1 \leq p \leq N\}$} and structures
the program with one cell (i.e. one subsystem) for each region, one
controller for initialization of control signal, one module for the
complete circuit and one interface to keep the same I/O as the
original program of figure~\ref{fig01}. The interface system obtained in our 
example is shown on figure~\ref{fig04}.
{\small
\begin{figure}[h]
\begin{verbatim}
system NtimeNot :{N | 2<=N}
                (a : {i | 1<=i<=N} of boolean)
       returns  (b : {i | 1<=i<=N} of boolean);
var
  a_mirr1 : {t,p | t=1; 1<=p<=N} of boolean;
  Acc : {t,p | 1<=t<=N; 1<=p<=N} of boolean;
let
  a_mirr1[t,p] = a[t+p-1];
  b[i] = Acc[N,i];
  use  NtimeNotModule[N] (a_mirr1) returns  (Acc) ;
tel;
\end{verbatim}
\caption{interface of the {\AlpHard} Program obtain from the program of figure~\ref{fig03} by the {\tt alpha0ToAlphard[]} command}
\label{fig04}
\end{figure}
}
