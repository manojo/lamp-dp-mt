\chapter{The {\alfhard} language}
\label{chapalfhard}

{\sc Version of \today}

In this chapter, we describe the {\alfhard} language. We also 
 describe the {\vhdl} generator. 


\section{\AlpHard}
\label{alphard}
The main goal of \alfa{} is to allow someone to produce 
a hardware implementation for an \alfa{} specification. To this
end, a subset of \alfa{}, called \AlpHard{}, is defined. 
We quickly introduce this language. 

\subsection{Basic concepts}

{\AlpHard} is a subset of the {\Alpha} language that enables a
structural definition of a regular architecture (systolic arrays,
etc.) to be given. 

{\AlpHard} is intended to meet two important goals.  First, {\AlpHard}
programs are obtained by automatic transformation of {\Alpha}
programs, and hence, \AlpHard{} it must be a coherent subset of
{\Alpha}. Second, the architectural description given by {\AlpHard}
must:
 \begin{itemize}
 \item Provide {\em structuring} because complex design process must be
   hierarchical.
 \item Provide {\em genericity} in order to allow component reuse.
 \item Allow {\em regularity} to be described, in order to reuse hardware
   descriptions and simplify the design process.
\end{itemize}

Thus, {\AlpHard} is hierarchical.  At the lowest level of description
  one has {\em cells} consisting of combinational circuits, registers,
  multiplexors, etc.  A cell may also contain other cells.  At the
  same level in the hierarchy, we may also have {\em controllers}
  responsible for initialization\footnote{In systolic arrays, the
  control signals are themselves distributed in a regular fashion, and
  their data-path can also be described in terms of cells.}.  At this
  point we have a description of a piece of circuit that has no
  ``spatial dimension'': it represents a single processing element
  which may later be instantiated at multiple spatial locations (like
  the full adder presented in section~\ref{sectstruct}).

The next level of the hierarchy is the {\em module} which allows the
user to specify how different cells are assembled regularly in one or
more spatial dimensions. This is achieved by {\em instantiating}
previously declared cells. Typically, controllers are instantiated
only once with no spatial replication. The separation of temporal and
spatial aspects is also reflected by the fact that the equations
describing the behavior of cells have only one local index variable
(time), and in the equations for modules, the {\em dependencies} have
only spatial indices, indicating pure (delay-less) interconnections
(thus all registers are viewed as part of the cells).



\subsection{A small example}

These ideas are illustrated in figure~\ref{figstupid}
and~\ref{figstupid2}, which show a three-cell module, where each cell
is a simple register-inverter. The {\AlpHard} code corresponding to
the {\em cell} definition is in figure~\ref{figstupid2}. 
The 
system {\tt RegInvCell} has two parameters (lines 1) corresponding
to its start time and its duration.  Lines 3-5 are the declaration of
input and output signals of the cell, the {\em domain} between the
braces represents the lifetime of signals {\tt A1} and {\tt B1}.  The
equation defining {\tt B1} (line 7) consists of an
inverter combined with a delay {\tt A1[t-1]} which 
represents a register.  Notice that the only index appearing in the
equations is the time, {\tt t}.
{\small
\begin{figure}[h]
\begin{verbatim}
system                                                          -- 1  
RegInvCell:{Tinit,Duration| Tinit,Duration >= 0}                -- 2  
           (A1 : {t | Tinit<=t<=Tinit+Duration} of boolean)     -- 3  
    returns                                                     -- 4  
           (B1 : {t | Tinit+1<=t<=Tinit+Duration+1} of boolean);-- 5  
let                                                             -- 6  
        B1[t] = not A1[t-1];                                    -- 7 
tel;                                                            -- 8 

\end{verbatim}

\begin{verbatim}
system
RegInvModule:{Tinit,Duration,Size| Tinit,Duration, Size > 0}     -- 9  
	(a: {t|  Tinit<=t<=Tinit+Duration} of boolean)           -- 10
   returns                                                       -- 11  
	(b: {t|  Tinit+Size<=t<=Tinit+Duration+Size} of boolean);-- 12
var                                                             
A : {t,p | Tinit+p-1<=t<=Tinit+Duration+p-1; 1<= p <= Size}  of boolean;  
B : {t,p |  Tinit+p<=t<=Tinit+Duration+p; 1<= p <= Size}  of boolean;
let
use {p | 1<= p <= Size} RegInvCell[Tinit+p-1,Duration]           -- 17
                        (A) returns (B);                         -- 18
A[t,p] = case                                                    -- 19   
      {| p=1}: a[t];                                             -- 20   
      {| p>1}: B[t,p-1];                                         -- 21   
      esac;                                                      -- 22   
b[t]=B[t,Size];                                                  -- 23
tel;                                       
\end{verbatim}
\caption{An {\AlpHard} program describing a simple cell, ({\tt
    RegInvCell}), and instantiation of {\tt Size} copies of it in a module
({\tt regInvModule}).
  Note how {\tt p} is used to specify the value {\tt Tinit} for each instance
  (the {\tt p}-th cell starts {\tt p-1} cycles after the first one).}
\label{figstupid}
\end{figure}
}

The {\tt RegInvModule} system is a {\em module} which uses the cell
 {\tt RegInvCell} .  Multiple copies are instantiated by means of the
construct,
\verb.use {p|1<=p<=Size}..  Such a multiple instantiation also means that the
variables {\tt A} and {\tt B} will have two indices, {\tt t}
(inherited from the {\tt RegInvCell} definition) and {\tt p} (from the
multiple use in line 17).  We also see that the value of the {\tt Tinit}
parameter is instantiated as a function of {\tt p} so that the cells start at
different instants of time. Lines 19-22 specify the connections between the
input/output of the cells in the module.  Notice that the constraints in
the {\tt case} involve only spatial indices (if it had involved temporal 
indices --i.e. {\tt t}--, we would need some additional control hardware not described here).  Also note that the time index,
{\tt t}, is the same on both sides of the equation (no register outside the cells). 


\begin{figure}[ht]
$$\includegraphics{figures/stupid1}$$
\caption{A simple architecture consisting of three identical cells}
\label{figstupid2}
\end{figure}


For more information on {\AlpHard} see~\cite{Patricia96}
\cite{Lemoenne} and there are 
example of Alphard programs in the the directory 
{\tt \$MMALPHA/examples/Alphard}

To have information on
the {\vhdl} generator, see the notebooks: 
\begin{itemize}
\item matvect demonstration,
\item Fir demonstration,
\item Vhdl.
\end{itemize}
They are all available as hyperlinks in the \texttt{master.nb}
that you can start using the command \texttt{on[]} or
\texttt{start[]}.

\section{generating {\vhdl} from {\AlpHard}}
The {\AlpHard} format was conceived by P. Le Mo�nner together
with the translator to {\vhdl}. The targeted {\vhdl} is 
{\em synthetizable \vhdl}, it has been tested on commercial tools like
 Compass, Synopsis,\ldots. Currently, the {\vhdl} translator allows
to generate {\vhdl} code from {\AlpHard} programs which represent
 linear systolic arrays. The reason why we cannot generate code for 
2-dimensional arrays is that the {\em synthesizable  \vhdl 
subset}~\cite{VHDLSynth} does not allow nested {\tt generate} instructions\footnote{The {\vhdl}  {\tt generate} instruction is used for expressing the 
repetitive use of a cell in an array}. 

\subsection{Setting up the translation}
the only operation  that have to be done before translating 
{\AlpHard} code is to affect the parameters value in the module and the controler
(we cannot
described arrays with parametric size in {\vhdl}). As the current {\AlpHard}
program is a library, this process may be a little complicated (you have 
to ensure that the parameter instantiations between the caller and the 
called program is respected). Currently this can be done with
 {\tt assignParameterValue} function. this function assign the parameter value 
in one system ({\tt \$result}), if you want to recursively assign the parameter 
value to the systems called by {\tt \$result}, you can use the
 {\tt fixParameter} function which assigns a value to a given parameter for 
all systems in {\tt \$library}\footnote{You have to ensure that this parameter 
has indeed exactly the same value in all systems of the library, this is not the case for {\tt Tinit} in our example}. In the example of figure~\ref{figstupid}, 
you can use the following commands:\\
\begin{verbatim}
 assignParameterValue["Size",10];
assignParameterValue["Tinit",0];
assignParameterValue["Duration",100]
\end{verbatim}
The resulting program is shown in figure~\ref{figstupid3}. The command
{\tt alphardToVHDL[]} will generate one file per system of the library.
Here, one file for the cell (named
{\tt regInvCell.vhd}) and one file for the module (named
{\tt regInvModule.vhd}). The {\vhdl} code for the
cell {\tt RegInvCell}  and the module {\tt RegInvModule}  are
 shown in figure~\ref{figVH1} and~\ref{figVHD2} 



{\small
\begin{figure}[h]
\begin{verbatim}
system RegInvCell :{Tinit,Duration | 0<=Tinit; 0<=Duration}
                  (A1 : {t | Tinit<=t<=Tinit+Duration} of boolean)
       returns    (B1 : {t | Tinit+1<=t<=Tinit+Duration+1} of boolean);
var
  A2 : {t | Tinit+1<=t<=Tinit+Duration+1; 0<=Tinit} of boolean;
let
  A2[t] = A1[t-1];
  B1[t] = not A2;
tel;
\end{verbatim}
\begin{verbatim}
system RegInvModule (a : {t | 0<=t<=100} of boolean)
       returns      (b : {t | 10<=t<=110} of boolean);
var
  A : {t,p | p-1<=t<=p+99; 1<=p<=10} of boolean;
  B : {t,p | p<=t<=p+100; 1<=p<=10} of boolean;
let
  use {p | 1<=p<=10} RegInvCell[p-1,100] (A) returns  (B) ;
  A[t,p] = 
      case
        {| p=1} : a[t];
        {| 2<=p} : B[t,p-1];
      esac;
  b[t] = B[t,10];
tel;
\end{verbatim}
\caption{The  {\AlpHard} program of figure~\ref{figstupid} with particular 
value for the parameters of the module. Note that the cell is still parameterized.}
\label{figstupid3}
\end{figure}
}
 

{\small
\begin{figure}[h]
\begin{verbatim}
-- VHDL Model Created for "system RegInvCell" 
 --  20/5/1999 11:27:36

library IEEE;
use IEEE.std_logic_1164.all;
library COMPASS_LIB;
 use COMPASS_LIB.STDCOMP.all;
library COMPASS_LIB; 
 use COMPASS_LIB.COMPASS.all;
  
entity RegInvCell is
      Port ( Ck : In std_logic;
      A1 : In std_logic;
      B1 : Out std_logic );
end RegInvCell;


architecture Behavioral of RegInvCell is

  signal A2 : std_logic;

begin


  process(ck)
  begin
    if (ck='1' AND ck'event) then
        A2 <= A1;
    end if;
  end process;

  B1 <= ( not A2);


end Behavioral;
\end{verbatim}
\caption{{\vhdl} code generated from  the {\tt RegInvCell} cell of figure~\ref{figstupid3}}
\label{figVHD1}
\end{figure}
} 


{\small
\begin{figure}[h]
\begin{verbatim}
todo
\end{verbatim}
\caption{{\vhdl} code generated from  the {\tt RegInvModule} cell of 
figure~\ref{figstupid3}}
\label{figVHD2}
\end{figure}
} 

\subsection{Description of the generated \vhdl}
{\tt todo:} describe the translation of operators, registers, multiplexer,
controler, cells, modules.
