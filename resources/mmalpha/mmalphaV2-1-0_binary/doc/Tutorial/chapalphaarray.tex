\chapter{Arrays of synchronous operators}
\label{chapalphaarray}
In this chapter, we show how arrays of hardware operators can be modelled in
{\alfa}\index{arrays}. In chapter~\ref{chapalphascal}, the basic
operators of the language were described. As we shall see here, a 
simple extension of these operators allows two-dimensional -- and
higher-dimensional -- variables and expressions to be used. 

\section{Describing array of elements}
Figure~\ref{adder-array} shows an array of adders. 
\section{An array of adders}
\begin{figure}[htbp]
%\htmlimage{scale=\myscale}
%\colorbox{white}{
\centerline{\includegraphics{figures/exemple_9}}
%}
\caption{An array of adders}\label{adder-array}
\end{figure}

The corresponding {\alfa} code is shown on figure~\ref{adder-array-alpha}.
\begin{figure}[htbp]
%\htmlimage{scale=\myscale}
\verbatiminput{alpha_progs/adder-array.alpha}
\caption{An array of adders in {\alfa}}\label{adder-array-alpha}
\end{figure}
Blabla
\section{Change of basis}
Any {\alfa} program can be seen as a black box which takes input
variables, and produces output variables. Therefore, intermediate
variables are unimportant, and, in particular, one can reindex them
using one-to-one index mappings. Obviously, 
\begin{verbatim}
A[i] = B[i];
B[i] = C[i];
\end{verbatim}
is identical to 
\begin{verbatim}
A[i] = B[i-1];
B[i] = C[i+1];
\end{verbatim}
 The change of basis operation does just this reindexing. In the
context of {\alfa}, the reindexing functions that we consider need to
be restricted to affine {\em unimodular functions},\index{affine
functions}\index{unimodular functions} that is to say, affine mappings
which admit an integral inverse.

Consider variable {\tt Z} in program~\ref{adder-array-alpha}. Consider
the reindexing function $B$ from $\Z^2$ to itself, defined by $B(t,p)
= (t+p,p)$. One can show that $B$ is unimodular. Let us denote
$B$ using the {\alfa} syntax, as {\tt (t,p -> t+p,p)}. This function 
is one-to-one, and its inverse is $B^{-1}$, denoted {\tt (t,p -> t-p,p)}.

To apply the change of basis defined by $B$ on variable {\tt Z}, 
we need to do the following operations:
\begin{enumerate}
\item Replace the domain of definition of {\tt Z} by its image under
the $B$ mapping. As the domain of {\tt Z} is 
{\tt \{t,p|t>=1;4>=p;p>=0\}}, its image by $B$ is 
{\tt \{t,p | p+1<=t; 0<=p<=4\}}.
\item Replace any occurrence of {\tt Z} in a right-hand side of 
an equation by {\tt Z.(t,p -> t+p,p)}.
\item Replace the entire equation {\tt Z = expr} defining {\tt Z}
by {\tt Z = expr.(t,p -> t-p,p)}.
\end{enumerate}

The new program is shown in figure~\ref{adder-array-bn-alpha}. We can normalize
it, which leads to program~\ref{adder-array-cb-alpha}.
\begin{figure}[htbp]
%\htmlimage{scale=\myscale}
\verbatiminput{alpha_progs/adder-array-bn.alpha}
\caption{Program adderarray, after change of basis {\tt (t,p->t+p,p)}
performed on {\tt Z}}\label{adder-array-bn-alpha}
\end{figure}

This new program can be interpreted as the architecture shown in 
figure~\ref{adder-array-cb-tex}.
\begin{figure}[htbp]
%\htmlimage{scale=\myscale}
\verbatiminput{alpha_progs/adder-array-cb.alpha}
\caption{Program adderarray, after change of basis 
and normalization}\label{adder-array-cb-alpha}
\end{figure}

\begin{figure}[htbp]
%\htmlimage{scale=\myscale}
%\colorbox{white}{
\centerline{\includegraphics{figures/exemple_10}}
%}
\caption{An array of adders}\label{adder-array-cb-tex}
\end{figure}

% Reduction
%\centerline{\input{/udd/quinton/documents/lncs_574/seminaire_litp/figures/exemple_8}}
%\input{/udd/quinton/documents/lncs_574/seminaire_litp/exemples_alpha/exemple_8}
%
%\transparent{0cm}{{Convolution}}{0cm}
%{\small
%\input{/udd/quinton/cours/cours_nouveau/convgen}
%}
%\transparent{0cm}{{Convolution apr\`{e}s s\'{e}rialisation}}{0cm}
%{\small
%\input{/udd/quinton/cours/cours_nouveau/conv}
%}
%\transparent{0cm}{{Convolution, apr\`{e}s changement de base}}{0cm}
%{\small
%\input{/udd/quinton/cours/cours_nouveau/conv1}
%}
%\begin{slide}{}
%\titre{Conception typique (Mathematica Alpha)}
%{\small
%\begin{enumerate}
%\item Analyse syntaxique, v\'{e}rification
%\item Placement des variables (manuel)
%\item Uniformisation (automatique)
%\item Ordonnancement (automatique)
%\item Changement  de base (automatique)
%\item Separation espace et temps
%\item Remplacement des {\bf case} par des signaux de contr\^{o}le (automatique)
%\item Generation circuit (MadMacs ou VHDL) 
%\end{enumerate}
%}
%\end{slide}




\subsection*{Processing this example in {\mmalfa}}
Load the array of adders:
\begin{verbatim}
load["adder-array.alpha"];   (* loads the alpha file *)
ashow[];                     (* shows its content *)
\end{verbatim}
Apply a change of basis:
\begin{verbatim}
changeOfBasis["B.(t,p->t+p,p)"];  (* apply a change of basis *)
ashow[];                     (* shows the result *)
\end{verbatim}
Notice that the result of a change of basis is automatically normalized. 

\section{Conclusion}
In this chapter, we have seen how arrays of elements can be 
described using {\alfa}. In order to present hardware examples, we
have limited ourself to programs where the indexes can be
interpreted as the time and the space (i.e., {\tt t} and {\tt p}).
Actually, {\alfa} programs are not restricted to such indexes. 
{\em Au contraire}, hardware-like {\alfa} programs represent the 
ultimate goal of a synthesis process which starts from a 
behavioural specification. They are in fact a proper subset of all 
{\alfa} programs.

In Chapter~\ref{chapfunctional}, we shall consider {\alfa} from the
data parallel perspective. 