\chapter{Code generation and simulation of {\alfa} programs}
\label{chapcodegen}
In this chapter, we explain how to generate C code from an
{\alfa} program. As seen in Chapter~\ref{chapfunctional}, 
{\alfa} programs are functional, and therefore, do not convey any
execution ordering. However, the validation of {\alfa} programs, 
or even, the generation of code for a general-purpose or
dedicated processor, requires the possibility to simulate the 
evaluation of {\alfa} expressions. 

This can be done using the {\tt writeC} {\mmalfa} command, as seen here.

\section*{An example}
Consider the following {\alfa} program:
\begin{figure}[htbp]
%\htmlimage{scale=\myscale}
\verbatiminput{alpha_progs/adder-WriteC.alpha}
\caption{A very simple example\label{adderwritec}
}
\end{figure}
To simulate this program, we first load this program, 
then generate a C program in file ``adder.c'', using 
\begin{verbatim}
load["adder-WriteC.alpha"];  (* load file *)
writeC["adder.c"];
\end{verbatim}
Then we can compile this program: 
\begin{verbatim}
cc -o adder adder.c
\end{verbatim}
The execution of this program is shown here:
\begin{verbatim}
adder
Input x =2
x = 2
Input y =3
y = 3
z = 5
\end{verbatim}
The C code prompts for the value of the input variables {\tt x} and
{\tt y}.
\section{Program with parameters}
Programs with parameters cannot be simulated without giving 
a value to the parameters. This is obtained through an option of
{\tt writeC}. 
\begin{figure}[htbp]
%\htmlimage{scale=\myscale}
\verbatiminput{alpha_progs/adder-WriteC1.alpha}
\caption{A very simple example\label{adderwritec1}
}
\end{figure}
\begin{verbatim}
load["adder-WriteC1.alpha"];  (* load program *)
writeC["adder1.c","-p 3"];
cc -o adder1 adder1.c
adder1Input x[1] =2
x[1]= 2
Input x[2] =3
x[2]= 3
Input x[3] =4
x[3]= 4
Input y[1] =-2
y[1]= -2
Input y[2] =-3
y[2]= -3
Input y[3] =-4
y[3]= -4
z[1]= 0
z[2]= 0
z[3]= 0
\end{verbatim}
\subsection{Additional examples}
\begin{verbatim}
SetDirectory["/home/frodon/d01/api/quinton/alpha/sobel"];
load["sobel.a"];ashow[];
analyze[];
writeC["-p 5 35 3"];
schedule[{objFunction -> 1,addConstraints -> {"Tpixel[i,N,M,p]=i"},duration -> 2}]
\end{verbatim}