\section{Translating to \vhdl{}}



\alfa{} programs cannot be translated directly to 

\vhdl{} (that would be to easy!). 

Only a subset of \alfa{}, called \alphard{}, can be

translated. 



The \texttt{alphaToVHDL} command, or the \texttt{av} command which is

similar, take an \alphard{} program, and translate it into \vhdl{}.



Normally, \alphard{} programs should be produced using the synthesis

process, which ends up using the \texttt{AlphaToHard} function: 

if this is the case, the synthesized program should be correct by

construction, and the \texttt{alphaToVHDL} function provides a

correct \vhdl{} model too. However, it is possible to 

write directly \alphard{} programs, and one can check such

programs, using various \texttt{check} functions (see~\ref{check}).



However, it is interesting to look in detail into the form of the

\alphard{} language. This it what we do here, by examining how various

pieces of hardware are represented in \alphard{}, and translated in

\vhdl{}.

*****************





\texttt{alphaToVHDL[]} generates \vhdl{} files from the library 

\texttt{$library}. Initial 

activity time of the corresponding architecture is 0. 

\texttt{alphaToVHDL[tinit]}

generates \vhdl{} files from \texttt{$library} 

with initial activity time \texttt{tinit}. 



\texttt{alphaToVHDL[lib, tinit]} generates \vdhl{} 

files from the library (list of 

systems) or the single system \texttt{lib}.



Actually, it is equivalent to use the 

\texttt{av} command, which is shorter. 



Note the following. First, both commands translate all programs loaded in 

\texttt{$library}. This means that, if you modify the current program in

\texttt{$result}, \texttt{$library}

will be changed only if you put

\texttt{$result} back in \texttt{$library}

(see the \texttt{putSystem} command.) 



Second, the \vhdl{} files which are generated have the name

\texttt{pg.vhd}

where

\texttt{pg}

is the actual system name of the loaded program (not to be confused with 

the name of the file containing it.)



\subsection{Options}

\texttt{alphaToVHDL} and \texttt{av} have options.



\begin{itemize}

\item \texttt{debug->True}

provide a trace of execution of the translater. It may be helpful to 

detect which part of the program was not accepted by the translater. 

\item \texttt{verbose->True}

gives a little bit more information. 

\item \texttt{tempFile -> False}

puts the \vhdl{} files in the /tmp/ directory\footnote{I do not 

know where this temporary directory is in Win NT.}

\end{itemize}



\section{How \alphard{} program are written}



Combinatorial logic and register



The following program is a cell containing an adder. As all cells, it is 

parameterized by the itinial time, named 

\texttt{Tinit}

and the duration of its use, named

\texttt{Duration}. 



The domain of these parameters is not important for combinatorial logic, 

but uncorrect values may cause the 

\texttt{analyze} 

command to fail on such a program.

Here is the program: 

\begin{verbatim}

system addreg: {Tinit, Duration | Tinit>0; Duration>1} 

(x,y : {t|Tinit<= t<= Tinit+Duration-1} of integer)

returns (z: {t|Tinit+1<= t<= Tinit+Duration-1} of integer);

var X, Y : {t|Tinit+1<= t<= Tinit+Duration-1} of integer;

let

	X[t] = x[t-1]; 

	Y[t] = y[t-1];

	z[t] = X + Y;

tel;

\end{verbatim}

We now call the translator:

\begin{verbatim}

av[verbose -> True, tempFile -> True]

\end{verbatim}

and here is the generated \vhdl{} code: 

\begin{verbatim}

-- VHDL Model Created for "system addreg" 

 --  27/4/1999 18:11:54

library IEEE;

use IEEE.std_logic_1164.all;

library COMPASS_LIB;

 use COMPASS_LIB.STDCOMP.all;

library COMPASS_LIB; 

 use COMPASS_LIB.COMPASS.all;

  

entity addreg is

      Port ( Ck : In std_logic;

      x : In Integer range 0 to 255;

      y : In Integer range 0 to 255;

      z : Out Integer range 0 to 255 );

end addreg;



architecture Behavioral of addreg is

  signal X : Integer range 0 to 255;

  signal Y : Integer range 0 to 255;

begin



  process(ck)

  begin

    if (ck='1' AND ck'event) then

        X <= x;

    end if;

  end process;



  process(ck)

  begin

    if (ck='1' AND ck'event) then

        Y <= y;

    end if;

  end process;



  z <= (X + Y);



end Behavioral;

\end{verbatim}



\section{Cell 1 of the FIR}



This program is a cell of the FIR. It consists of combinational logic

and registers. 

\begin{verbatim}

-- Example for VHDL translator -- September 1998

-- This example is a cell extracted from the FIR module

system cellFirModule3 :{p,N,M | 2<=p<=N; N<=M-1}

  (Y_reg7loc : {t | p+1<=t<=p-N+M+1; 2<=p<=N; N<=M-1} of integer; 

   xPipe_reg5loc : {t | p<=t<=p-N+M; 2<=p<=N; N<=M-1} of integer; 

   HPipeES_reg4loc : {t | 2p-N-2<=t<=p-1; 2<=p<=N; N<=M-1} of integer;

   xPipeES_reg3loc : {t | 2p-N-1<=t<=p; 2<=p<=N; N<=M-1} of integer;

   xPipe_ctl2P_reg2loc : {t | p<=t<=p-N+M; 2<=p<=N; N<=M-1} of boolean;

   HPipe_ctl1P_reg1loc : {t | p-1<=t<=p-N+M-1; 2<=p<=N; N<=M-1} of boolean

  )

returns

  (Y : {t | p+2<=t<=p-N+M+2; 2<=p<=N; N<=M-1} of integer; 

   xPipe : {t | p+1<=t<=p-N+M+1; 2<=p<=N; N<=M-1} of integer;

   HPipeES : {t | 2p-N<=t<=p; 2<=p; N<=M-1} of integer; 

   xPipeES : {t | 2p-N+1<=t<=p+1; 2<=p; N<=M-1} of integer;

   xPipe_ctl2P : {t | p+1<=t<=p-N+M+1; 2<=p<=N; N<=M-1} of boolean;

   HPipe_ctl1P : {t | p<=t<=p-N+M; 2<=p<=N; N<=M-1} of boolean

  );

var

  HPipe_ctl1Ploc6 : {t | p<=t<=p-N+M; 2<=p<=N; N<=M-1} of boolean;

  xPipe_ctl2Ploc5 : {t | p+1<=t<=p-N+M+1; 2<=p<=N; N<=M-1} of boolean;

  xPipeESloc4 : {t | 2p-N+1<=t<=p+1; 2<=p; N<=M-1} of integer;

  HPipeESloc3 : {t | 2p-N<=t<=p; 2<=p; N<=M-1} of integer;

  xPipeloc2 : {t | p+1<=t<=p-N+M+1; 2<=p<=N; N<=M-1} of integer;

  HPipe : {t | p<=t<=p-N+M; 2<=p<=N; N<=M-1} of integer;

  HPipe_ctl1 : {t | p<=t<=p-N+M; 2<=p<=N; N<=M-1} of boolean;

  HPipe_ctl1P_reg1 : {t | p<=t<=p-N+M; 2<=p<=N; N<=M-1} of boolean;

  HPipeES_reg4 : {t | 2p-N<=t<=p; 2<=p; N<=M-1} of integer;

  HPipe_reg6 : {t | p+1<=t<=p-N+M; 2<=p<=N} of integer;

  HPipe_reg8 : {t | p+2<=t<=p-N+M+2; 2<=p<=N; N<=M-1} of integer;

  xPipe_ctl2 : {t | p+1<=t<=p-N+M+1; 2<=p<=N; N<=M-1} of boolean;

  xPipe_ctl2P_reg2 : {t | p+1<=t<=p-N+M+1; 2<=p<=N; N<=M-1} of boolean;

  xPipeES_reg3 : {t | 2p-N+1<=t<=p+1; 2<=p; N<=M-1} of integer;

  xPipe_reg5 : {t | p+2<=t<=p-N+M+1; 2<=p<=N} of integer;

  xPipe_reg9 : {t | p+2<=t<=p-N+M+2; 2<=p<=N; N<=M-1} of integer;

  Y_reg7 : {t | p+2<=t<=p-N+M+2; 2<=p<=N; N<=M-1} of integer;

let

  HPipe_ctl1P[t] = HPipe_ctl1Ploc6[t];

  xPipe_ctl2P[t] = xPipe_ctl2Ploc5[t];

  xPipeES[t] = xPipeESloc4[t];

  HPipeES[t] = HPipeESloc3[t];

  xPipe[t] = xPipeloc2[t];

  Y_reg7[t] = Y_reg7loc[t-1];

  xPipe_reg5[t] = xPipe_reg5loc[t-2];

  HPipeES_reg4[t] = HPipeES_reg4loc[t-2];

  xPipeES_reg3[t] = xPipeES_reg3loc[t-2];

  xPipe_ctl2P_reg2[t] = xPipe_ctl2P_reg2loc[t-1];

  HPipe_ctl1P_reg1[t] = HPipe_ctl1P_reg1loc[t-1];

  xPipe_reg9[t] = xPipeloc2[t-1];

  HPipe_reg8[t] = HPipe[t-2];

  HPipe_reg6[t] = HPipe[t-1];

  xPipeESloc4[t] = xPipeES_reg3[t];

  HPipeESloc3[t] = HPipeES_reg4[t];

  xPipe_ctl2Ploc5[t] = xPipe_ctl2P_reg2[t];

  xPipe_ctl2[t] = xPipe_ctl2Ploc5[t];

  xPipeloc2[t] = 

    case

      {| t=p+1; 2<=p<=N; N<=M-1} : 

         if (xPipe_ctl2[t]) then xPipeESloc4[t] else 0[];

      {| p+2<=t<=p-N+M+1; 2<=p<=N} : 

         if (xPipe_ctl2[t]) then 0[] else xPipe_reg5[t];

    esac;

  HPipe_ctl1Ploc6[t] = HPipe_ctl1P_reg1[t];

  HPipe_ctl1[t] = HPipe_ctl1Ploc6[t];

  HPipe[t] = 

    case

      {| t=p; 2<=p<=N; N<=M-1} : 

         if (HPipe_ctl1[t]) then HPipeESloc3[t] else 0[];

      {| p+1<=t<=p-N+M; 2<=p<=N} : 

         if (HPipe_ctl1[t]) then 0[] else HPipe_reg6[t];

      esac;

  Y[t] = Y_reg7[t] + HPipe_reg8[t] * xPipe_reg9[t];

tel;

\end{verbatim}

And here is the generated \vhdl{} program:

\begin{verbatim}

-- VHDL Model Created for "system cellFirModule3" 

 --  27/4/1999 19:14:50



library IEEE;

use IEEE.std_logic_1164.all;

library COMPASS_LIB;

 use COMPASS_LIB.STDCOMP.all;

library COMPASS_LIB; 

 use COMPASS_LIB.COMPASS.all;

  

entity cellFirModule3 is

      Port ( Ck : In std_logic;

      Y_reg7loc : In Integer range 0 to 255;

      xPipe_reg5loc : In Integer range 0 to 255;

      HPipeES_reg4loc : In Integer range 0 to 255;

      xPipeES_reg3loc : In Integer range 0 to 255;

      xPipe_ctl2P_reg2loc : In std_logic;

      HPipe_ctl1P_reg1loc : In std_logic;

      Y : Out Integer range 0 to 255;

      xPipe : Out Integer range 0 to 255;

      HPipeES : Out Integer range 0 to 255;

      xPipeES : Out Integer range 0 to 255;

      xPipe_ctl2P : Out std_logic;

      HPipe_ctl1P : Out std_logic );

end cellFirModule3;





architecture Behavioral of cellFirModule3 is



  signal HPipe_ctl1Ploc6 : std_logic;

  signal xPipe_ctl2Ploc5 : std_logic;

  signal xPipeESloc4 : Integer range 0 to 255;

  signal HPipeESloc3 : Integer range 0 to 255;

  signal xPipeloc2 : Integer range 0 to 255;

  signal HPipe : Integer range 0 to 255;

  signal HPipe_ctl1 : std_logic;

  signal HPipe_ctl1P_reg1 : std_logic;

  signal HPipeES_reg4 : Integer range 0 to 255;

  signal HPipe_reg6 : Integer range 0 to 255;

  signal HPipe_reg8 : Integer range 0 to 255;

  signal xPipe_ctl2 : std_logic;

  signal xPipe_ctl2P_reg2 : std_logic;

  signal xPipeES_reg3 : Integer range 0 to 255;

  signal xPipe_reg5 : Integer range 0 to 255;

  signal xPipe_reg9 : Integer range 0 to 255;

  signal Y_reg7 : Integer range 0 to 255;

  signal xPipe_reg5_S1 : Integer range 0 to 255;

  signal HPipeES_reg4_S1 : Integer range 0 to 255;

  signal xPipeES_reg3_S1 : Integer range 0 to 255;

  signal HPipe_reg8_S1 : Integer range 0 to 255;



begin



  HPipe_ctl1P <= HPipe_ctl1Ploc6;



  xPipe_ctl2P <= xPipe_ctl2Ploc5;



  xPipeES <= xPipeESloc4;



  HPipeES <= HPipeESloc3;



  xPipe <= xPipeloc2;



  process(ck)

  begin

    if (ck='1' AND ck'event) then

        Y_reg7 <= Y_reg7loc;

    end if;

  end process;



  process(ck)

  begin

    if (ck='1' AND ck'event) then

        xPipe_reg5_S1 <= xPipe_reg5loc;

    end if;

  end process;



  process(ck)

  begin

    if (ck='1' AND ck'event) then

        xPipe_reg5 <= xPipe_reg5_S1;

    end if;

  end process;



  process(ck)

  begin

    if (ck='1' AND ck'event) then

        HPipeES_reg4_S1 <= HPipeES_reg4loc;

    end if;

  end process;



  process(ck)

  begin

    if (ck='1' AND ck'event) then

        HPipeES_reg4 <= HPipeES_reg4_S1;

    end if;

  end process;



  process(ck)

  begin

    if (ck='1' AND ck'event) then

        xPipeES_reg3_S1 <= xPipeES_reg3loc;

    end if;

  end process;



  process(ck)

  begin

    if (ck='1' AND ck'event) then

        xPipeES_reg3 <= xPipeES_reg3_S1;

    end if;

  end process;



  process(ck)

  begin

    if (ck='1' AND ck'event) then

        xPipe_ctl2P_reg2 <= xPipe_ctl2P_reg2loc;

    end if;

  end process;



  process(ck)

  begin

    if (ck='1' AND ck'event) then

        HPipe_ctl1P_reg1 <= HPipe_ctl1P_reg1loc;

    end if;

  end process;



  process(ck)

  begin

    if (ck='1' AND ck'event) then

        xPipe_reg9 <= xPipeloc2;

    end if;

  end process;



  process(ck)

  begin

    if (ck='1' AND ck'event) then

        HPipe_reg8_S1 <= HPipe;

    end if;

  end process;



  process(ck)

  begin

    if (ck='1' AND ck'event) then

        HPipe_reg8 <= HPipe_reg8_S1;

    end if;

  end process;



  process(ck)

  begin

    if (ck='1' AND ck'event) then

        HPipe_reg6 <= HPipe;

    end if;

  end process;



  xPipeESloc4 <= xPipeES_reg3;



  HPipeESloc3 <= HPipeES_reg4;



  xPipe_ctl2Ploc5 <= xPipe_ctl2P_reg2;



  xPipe_ctl2 <= xPipe_ctl2Ploc5;



  xPipeloc2 <= 

    xPipeESloc4 when xPipe_ctl2 = '1'  else 

    0 when xPipe_ctl2 = '1'  else xPipe_reg5;



  HPipe_ctl1Ploc6 <= HPipe_ctl1P_reg1;



  HPipe_ctl1 <= HPipe_ctl1Ploc6;



  HPipe <= 

    HPipeESloc3 when HPipe_ctl1 = '1'  else 

    0 when HPipe_ctl1 = '1'  else HPipe_reg6;



  Y <= (Y_reg7 + (HPipe_reg8 * xPipe_reg9));



end Behavioral;

\end{verbatim}



\section{Control cell of the FIR}

Here is the controller for the FIR program. 

\begin{verbatim}

system controlFirModule :{N,M | 2<=N<=M-1}

  ( )

returns

  (HPipe_ctl1P_Init : {t | t=1; 2<=N<=M-1} | 

                      {t | 2<=t<=-N+M+1; 2<=N} of boolean; 

   xPipe_ctl2P_Init : {t | t=2; 2<=N<=M-1} | 

                      {t | 3<=t<=-N+M+2; 2<=N} of boolean);

let

  HPipe_ctl1P_Init[t] = 

      case

        {| t=1; 2<=N<=M-1} : True[];

        {| 2<=t<=-N+M+1; 2<=N} : False[];

      esac;

  xPipe_ctl2P_Init[t] = 

      case

        {| t=2; 2<=N<=M-1} : True[];

        {| 3<=t<=-N+M+2; 2<=N} : False[];

      esac;

tel;

\end{verbatim}

This program is parameterized, and therefore, a \vhdl{}

program cannot be generated without assigning values to 

the parameters. This is done by means of the following

commands: 

\begin{verbatim}

load["controlFirModule.alpha"];

assignParameterValue["N", 5]; 

assignParameterValue["M", 10]; 

av[tempFile -> True]

\end{verbatim}

The \vhdl{} program for this controller is presented

in appendix~\ref{vdhl:appendix}.



\section*{Appendix: \vhdl{} code of the Fir controller}

\label{vhdl:appendix}

\begin{verbatim}

-- VHDL Model Created for "system controlFirModule" 

 --  27/4/1999 18:52:17



library IEEE;

use IEEE.std_logic_1164.all;

  

entity controlFirModule is

      Port ( Ck : In std_logic;

      Rst : In std_logic;

      HPipe_ctl1P_Init : Out std_logic;

      xPipe_ctl2P_Init : Out std_logic );

end controlFirModule;





architecture state_machine of controlFirModule is

  signal cpt: Integer;



  type HPipe_ctl1P_InitStates is (E0,E0bis,E1,E2);

  signal currentHPipe_ctl1P_InitState, nextHPipe_ctl1P_InitState :HPipe_ctl1P_InitStates;



  type xPipe_ctl2P_InitStates is (E0,E0bis,E1,E2);

  signal currentxPipe_ctl2P_InitState, nextxPipe_ctl2P_InitState :xPipe_ctl2P_InitStates;

BEGIN 



reset_sm : PROCESS

 begin

-- compass stateMachine adj currentHPipe_ctl1P_InitState

-- compass stateMachine adj currentxPipe_ctl2P_InitState



  WAIT UNTIL (Ck = '1' AND Ck'event);



  IF Rst ='1' THEN 

   cpt <= -1;

   currentHPipe_ctl1P_InitState <= E0;

   currentxPipe_ctl2P_InitState <= E0;

  ELSE

   cpt <= cpt + 1;

   currentHPipe_ctl1P_InitState <= nextHPipe_ctl1P_InitState;

   currentxPipe_ctl2P_InitState <= nextxPipe_ctl2P_InitState;

  END IF;

 END PROCESS;



evolution_HPipe_ctl1P_Init : PROCESS(cpt, currentHPipe_ctl1P_InitState)

 begin 

 CASE currentHPipe_ctl1P_InitState IS

  WHEN E0 => IF (cpt < 0) THEN nextHPipe_ctl1P_InitState <= E0;

    ELSIF (cpt = 0) THEN nextHPipe_ctl1P_InitState <= E1; END IF;

  WHEN E1 =>  IF (cpt = 1) THEN nextHPipe_ctl1P_InitState <= E2; END IF;

  WHEN E2 => IF ((cpt >= 2) AND (cpt < 6 )) THEN nextHPipe_ctl1P_InitState <= E2;  END IF;

   IF (cpt = 6) THEN nextHPipe_ctl1P_InitState <= E0bis; END IF;-- remise a zero de la SM

  WHEN OTHERS => -- erreurs et hors service

   nextHPipe_ctl1P_InitState <= E0bis ;

 END CASE;

END PROCESS;



output_HPipe_ctl1P_Init : PROCESS(currentHPipe_ctl1P_InitState)

 begin

 CASE currentHPipe_ctl1P_InitState IS

  WHEN E1=>HPipe_ctl1P_Init <= '1'; 

  WHEN E2=>HPipe_ctl1P_Init <= '0'; 

  WHEN OTHERS =>

   HPipe_ctl1P_Init <= 'X';

 END CASE;

END PROCESS;

evolution_xPipe_ctl2P_Init : PROCESS(cpt, currentxPipe_ctl2P_InitState)

 begin 

 CASE currentxPipe_ctl2P_InitState IS

  WHEN E0 => IF (cpt < 1) THEN nextxPipe_ctl2P_InitState <= E0;

    ELSIF (cpt = 1) THEN nextxPipe_ctl2P_InitState <= E1; END IF;

  WHEN E1 =>  IF (cpt = 2) THEN nextxPipe_ctl2P_InitState <= E2; END IF;

  WHEN E2 => IF ((cpt >= 3) AND (cpt < 7 )) THEN nextxPipe_ctl2P_InitState <= E2;  END IF;

   IF (cpt = 7) THEN nextxPipe_ctl2P_InitState <= E0bis; END IF;-- remise a zero de la SM

  WHEN OTHERS => -- erreurs et hors service

   nextxPipe_ctl2P_InitState <= E0bis ;

 END CASE;

END PROCESS;



output_xPipe_ctl2P_Init : PROCESS(currentxPipe_ctl2P_InitState)

 begin

 CASE currentxPipe_ctl2P_InitState IS

  WHEN E1=>xPipe_ctl2P_Init <= '1'; 

  WHEN E2=>xPipe_ctl2P_Init <= '0'; 

  WHEN OTHERS =>

   xPipe_ctl2P_Init <= 'X';

 END CASE;

END PROCESS;



END state_machine;

\end{verbatim}

