\documentclass[12pt]{article}
\begin{document}

\newcommand{\domlib}{{\sc domlib}}
\newcommand{\ML}{{\sc Mathlink}}
\newcommand{\mprep}{{\sc mprep}}
\newcommand{\polylib}{{\sc polylib}}
\newcommand{\mma}{{\sc Mathematica}}
\newcommand{\mkf}{{\texttt Makefile}}
\newcommand{\gcc}{\texttt{gcc}}
\newcommand{\doml}{\texttt{domlib.c}}
\newcommand{\tm}{\texttt{domlibtm.c}}
\newcommand{\mmAlpha}{\textsc{MMAlpha}}
\newcommand{\MMAlpha}{\textsc{MMAlpha}}
\newcommand{\MMA}{\textsc{Mathematica}}
\newcommand{\true}{\texttt{True}}
\newcommand{\false}{\texttt{False}}

\title{Writing a test file for \mmAlpha{}}
\author{Patrice Quinton}
\date{Version of \today}
\maketitle

\section{Introduction}
Testing programs is recommended by everybody... and done only
by a few people. \mmAlpha{} is no exception...

The purpose of this document is to explain what policy
is to be followed when one wants to develop a new package for
\mmAlpha{}.

A companion document, located in \texttt{\$MMALPHA/doc/Tests}
explains how to test the \mmAlpha{}. 

In a few words, here are the recommendations:
\begin{enumerate}
\item All packages must be provided with a \MMA{} test file. 
Let \texttt{myPack.m} by a brand new \MMAlpha{} package (the
same policy applies to C program). Then, a \MMAlpha{} program, 
\texttt{TestmyPack.m} must be written, and placed in the 
directory \texttt{\$MMAlpha/tests} in a 
new directory \texttt{TestmyPack}. If you follow these
recommendations, your test program will be called when evaluating
the \MMAlpha{} expression \texttt{tests["myPack"]}. 
\item Your test program, \texttt{TextmyPack.m} must 
return a value \true{} if all tests
have been passed, \false{} otherwise. 
\item Test your test program! A good policy is to incorporate
all examples you have used when developing your code. 
Remember that your tests are going to be run among 
several thousands, so make sure that your tests are 
correct.
\end{enumerate}
In what follows, we present quickly facilities provided
for testing by \MMAlpha{}, and we give an example. 

\section{Functions for testing}
Two functions are used: \texttt{tests}, and \texttt{testFunction}. These
functions were improved in the V2 version of \MMAlpha{}. 

\subsection{The \texttt{tests} function}
This function is part of the \texttt{Alpha.m} package. 
Evaluating \texttt{tests[]} runs {\em all test files which
are in the test directory}. This may take a few hours... We
use it only to check a new version\footnote{This may change...}.

Evaluating \texttt{tests["myPack"]} runs the test program
for package \texttt{myPack}. You can therefore run tests 
for various \MMAlpha{} packages, for example, \texttt{tests["Alpha"]}
(try it!). 
Notice that \MMA{} error messages may happen during the test; this
does not mean that the test has failed. Indeed, some test programs
test that \MMAlpha{} detects abnormal situations. 

\subsection{The \texttt{testFunction}}
A function, called, \texttt{testFunction}, is provided in 
the \texttt{Alpha.m} package, as a means to organize tests
in a systematic way. Using this function is recommended. Its operation
is as follows. 
\begin{verbatim}
testFunction[exp, result, testID]
\end{verbatim}
compares the evaluation of \texttt{exp} to \texttt{result}.
It returns 
\texttt{True} if they are equal, 
\texttt{False} and emits a message if they are different.
\texttt{testID} is a string which identifies the test.

This function can be used in many different ways. For example, 
here is a test of the \texttt{Alpha.m} package, which checks 
the \MMAlpha{} parser: 
\begin{verbatim}
testFunction[load["Test1.alpha"];$result,<<"RTest1","Alpha 1"]
\end{verbatim}
The evaluation of this expression loads the file \texttt{Test1.alpha}
which is supposed to be in the test directory, and compares the 
value of \texttt{\$result}, the \MMAlpha{} variable which contains
the current AST, with the content of the file \texttt{RTest1}, 
in the same directory. This test is identified as \texttt{"Alpha 1"}.

\section{An example}
Fig.~\ref{ex} gives an example of test file. 
\begin{figure}[htbp]
\begin{verbatim}
Module[{dir, res1, testresult},

  dir = Directory[];  (* Save current directory *)

  (* Go in test directory *)
  SetDirectory[Environment["MMALPHA"]<>"/tests/TestAlpha/"];
  Print["Test for Alpha.m"];

  (* Build a list of boolean values *)
  res1= 
  {
    testFunction[load["Test1.alpha"];$result,<<"RTest1","Alpha 1"],
    testFunction[load[22],Null,"Alpha 2"],
    testFunction[Check[show[],$Failed],Null,"Alpha 3"],
    testFunction[show[22],Null,"Alpha 4"],
    (* ... *)
    ReadList["MV1.c",Record,RecordSeparators->{}]===
      ReadList["MV1.c",Record,RecordSeparators->{}],True,"writeC 1"],
    DeleteFile["MV1.c"];True
  }

  (* And the list *)
  testResult = Apply[And,res1];

  (* Back to initial dir *)
  SetDirectory[dir];

  (* Diagnostic *)
  If[ testResult,
    Print["**** Test OK for Alpha.m "],
    Print["**** Something was wrong for Alpha.m"]];

  (* Do not forget to return the result *)
  testResult
] (* End module *)
\end{verbatim}
\caption{Example of test, adapted from the file 
\texttt{Alpha.m}}\label{ex}
\end{figure}

\end{document}

