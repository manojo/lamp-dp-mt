\documentclass[a4paper,11pt]{article}
%\usepackage{psfig}
\usepackage{epic}
\usepackage{eepic}


\newcommand{\mma}{{{\sc mma}lpha}}
\newcommand{\Alpha}{{{Alpha}}}
\newcommand{\AlphaZ}{{{\Alpha}0}}
\newcommand{\Alphard}{{Alphard}}
\newcommand{\vhdl}{{\sc vhdl} }

\newtheorem{definition}{Definition}[section]

\begin{document}
\title{Syntax and semantics of the {\AlphaZ} subset of Alpha (Version 2)}
\author{}
\date{\today}

 \maketitle


\section{Introduction}
{\AlphaZ} was introduced by C. Dezan in her Phd thesis (\cite{dezan}). It 
was conceived as the lowest level of {\Alpha} or equivalently as a subset of 
{\Alpha} with a structural interpretation. The main weakness of 
this subset came from the fact that {\Alpha} programs were not structured.
Since then, the structuring of 
{\Alpha} was provided by F. Dupont~\cite{DupontQuRi95} and the structured 
version of {\AlphaZ} was studied by P. Le Mo\"enner~\cite{Patricia96}.
{\Alphard} is now intended to be this subset of {\Alpha} with structural 
interpretation, from which we can translate {\Alpha} into {\vhdl} or other 
{\sc rtl} description languages. However, during the 
transformation path from {\Alpha} to {\Alphard}, the automatic structuring 
is a complicated process, hence it appear that the {\AlphaZ} format is still 
necessary as an intermediate format with all the structural information 
but in an unstructured form\footnote{il faudra unitliser deux mots different 
pour structural et structuring}. The current document describes the second 
version of the {\AlphaZ} format, 

\section{Definition of {\AlphaZ}}
In this section we recall the format {\AlphaZ} 
introduced in~\cite{dezan}. 
Then we introduce the additionnal notions of version 2.
\subsection{Original {\AlphaZ}}
An {\Alpha} program is in {\AlphaZ} form if it meets the three following conditions:
\begin{enumerate}
\item there exits for all declared domains (except for the domains of 
inputs and outputs variables) an interpretation function for the indices
(see~\cite{dezan} for precise definition). Each index is either 
a temporal index or a spatial index.

\item each variable is a {\em signal} that has, for each spatial
interpretation, the temporal interpretation $\{t ~|~ t>=0\}$

\item The equations of the system define the outputs and local variables.
Each operator involved in the equations has a structural interpretation. 
These equations are of two types: data equations and control equations.
\begin{itemize} 
\item Data equations define the different signals of the program.
They are composed of the following operators:  
\begin{itemize} 
  \item pointwise operators represent the corresponding combinatorial 
operators. 
  \item restriction are used to restrict the spatial interpretation of 
indices.
\item  dependencies are temporal dependencies representing registers or
 spatial dependencies representing connections.
\item case expressions are spatial case allowing to gather several connections
\end{itemize}
\item control equations allow signals to be constructed from input variables
and the behavior to be controled. 
Such equations use the following constructions:
\begin{itemize}
\item input dependencies correspond to relations between formal 
input and variables of the programs indicating where and when 
input are sent.
\item output dependencies indicate where and when output results are valid.
\item case control (temporal case?)
 define control signals from boolean constant
(in a {\em finite automate} fashion). 
\end{itemize}
\end{itemize}
\end{enumerate}

\subsection{{\AlphaZ} version 2}
As a platform to {\Alphard}, we need to remove the second point of
the previous definition, namely the fact that all temporal domains are
of the form $\{t~|~ t>=\}$. The other basic principles remain the same
with slight differences. Spatial cases contain no dependencies,
connections are represented using a simple space dependency (no
case). Temporal cases are always nested in a spatial case.
Control initialization equations are called {\em control} equation,
the {\em Mirror} equations are the equations affecting inputs, output 
and control signal to local variables (hence, control signal 
become data signals). All these descriptions are summarized by the following 
definition:

\begin{enumerate}
\item There exits for all declared domains (except for the domains of 
inputs and outputs variables) an interpretation function for the indices
(see~\cite{dezan} for precise definition). Each index is either 
a temporal index or a spatial index.

\item The equations of the system define the output and local
variables.  Each operator involved in the equations has a structural
interpretation. These equations are of four types: data equations,
connection equations, control equations and mirror equations.
\begin{itemize} 
\item Data equations define the different signals of the program
which are necessarily local variables.
They are composed of the following operators:  
\begin{itemize} 
\item Pointwise operators represent the corresponding combinatorial 
operators. 
\item Restriction are used to restrict the spatial interpretation of 
indices.
\item Dependencies are temporal dependencies representing registers 
or identity dependencies representing connections between two signal inside one cell.
\item Cases are spatial case allowing several signals to be gathered.
\item The {\tt if} operator (with nested {\tt case} in each branches)
is interpreted as a multiplexer.
\end{itemize}
\item Connection equations are limited to the following form: a single
spatial dependency between two signals ({\tt A=B.(t,p->t,p-1)} for
instance).
\item Control equations allow control signals to be initialized. 
They define signals
which are only temporal (no spatial indices).
\item Mirror equations: equation between input variables of the system and 
local variable of the system (only for interface). equation limited to an 
affine function applied to a variable ({\tt A[t,p]=a[f(t,p)]} for input mirror
equation and {\tt b[i,j]=B[f(i,j)]} for output mirror equations)..
\end{itemize}
\end{enumerate}
\bibliography{/udd/risset/TEX/ref/newbib}
\bibliographystyle{unsrt}

\end{document}

