\documentclass[12pt]{article}
\usepackage{fullpage,epic,eepic,epsfig,makeidx,color,moreverb,boxedminipage,version}

\sloppy

\excludeversion{tobedone}

\newcommand{\C}{\texttt{C}}
\newcommand{\gpl}{Gnu Public License}
\newcommand{\pip}{PiP}
\newcommand{\lpsolve}{LP-Solve}

\newcommand{\sitemmalfa}
{\texttt{\htmladdnormallink{http://www.irisa.fr/cosi/ALPHA}{http://www.irisa.fr/api/ALPHA}}}
\newcommand{\emailmmalfa}{\texttt{alpha@irisa.fr}}

\newcommand{\Alpha}{{\sc Alpha}}
\newcommand{\AlpHard}{{\sc AlpHard}}
\newcommand{\MMA}{{\sc MMAlpha}}
\newcommand{\irisa}{ Irisa}
\newcommand{\vlsi}{{\sc vlsi}}
\newcommand{\vhdl}{{\sc vhdl}}
\newcommand{\alfa}{\Alpha}
\newcommand{\mmalfa}{\MMA}
\newcommand{\MMAlfa}{\MMA}
\newcommand{\mmalpha}{\MMA}
\newcommand{\mma}{{Mathematica}}
\newcommand{\polylib}{{\sc polylib}}
\newcommand{\domlib}{{\sc domlib}}
\newcommand{\Opt}[1]{{\rm\sl [} #1 {\rm\sl ]}}
\newcommand{\Group}[1]{{\rm\sl (} #1 {\rm\sl )}}
\newcommand{\Alt}{$\mid$}

\newcommand {\setn}   {\hbox{\it I\hskip -2pt N}}     % ensemble N
\newcommand {\setz}   {\hbox{\it Z\hskip -4pt Z}}     %ensemble Z
\newcommand {\setq}   {\hbox{\it Q\hskip -6pt {\it l}\hskip 4pt}}     % ensemble Q

\newcommand{\ostype}{\texttt{\$OSTYPE}}
\newcommand{\envmmalpha}{\texttt{\$MMALPHA}}
\newcommand{\cygwin}{\texttt{cygwin}}
\newcommand{\cygwintrde}{\texttt{cygwin32}}
\newcommand{\linux}{\texttt{linux}}
\newcommand{\unix}{\texttt{unix}}
\newcommand{\darwin}{\texttt{darwin}}
\newcommand{\solaris}{\texttt{solaris}}
\newcommand{\sunqu}{\texttt{sun4}}
\newcommand{\sunos}{\texttt{sun OS}}

\newcommand{\varcc}{\texttt{CC}}
\newcommand{\varld}{\texttt{LD}}
\newcommand{\varpolylib}{\texttt{POLYLIB}}
\newcommand{\varcflags}{\texttt{CFLAGS}}
\newcommand{\varlflags}{\texttt{LFLAGS}}
\newcommand{\varbin}{\texttt{BIN}}
\newcommand{\varobj}{\texttt{OBJ}}
\newcommand{\vardepobj}{\texttt{DEPOBJ}}
\newcommand{\varmathprep}{\texttt{MATHPREP}}
\newcommand{\varmathprepcor}{\texttt{MATHPREPCOR}}
\newcommand{\varcppflags}{\texttt{CPPFLAGS}}
\newcommand{\varldflags}{\texttt{CLDFLAGS}}
\newcommand{\varloadlibes}{\texttt{LOADLIBS}}
\newcommand{\vardummy}{\texttt{DUMMY}}
\newcommand{\varname}{\texttt{NAME}}
\newcommand{\varobjs}{\texttt{OBJS}}
\newcommand{\varlibs}{\texttt{LIBS}}
\newcommand{\varpolydir}{\texttt{POLYDIR}}
\newcommand{\varmathinclude}{\texttt{MATHINCLUDE}}
\newcommand{\varlibdir}{\texttt{LIBDIR}}


%\newcommand{\domlib}{{Domlib}}
\newcommand{\zdomlib}{{ZDomlib}}
%\newcommand{\envmmalpha}{\texttt{\$MMALPHA}}
%\newcommand{\envostype}{\texttt{\$OSTYPE}}



\newcounter{Nex}

\newtheorem{ex}{Exemple}[section]

\makeindex


\newcommand{\ML}{{\sc Mathlink}}
\newcommand{\mprep}{{\sc mprep}}
%\newcommand{\mma}{{\sc Mathematica}}
\newcommand{\mkf}{{\texttt Makefile}}
\newcommand{\gcc}{\texttt{gcc}}
\newcommand{\doml}{\texttt{domlib.c}}
\newcommand{\tm}{\texttt{domlibtm.c}}
%\newcommand{\mmalfa}{\textsc{MMAlpha}}
\begin{document}

\title{Compiling \domlib{} and other C programs}
\author{Api, then Cosi, then R2D2 and Compsys\thanks{
Api, Cosi and R2D2 are the names of the 
research groups that successively hosted
research related on \alfa{} and \mmalpha{} at Irisa, Rennes, France. 
Since 2001, Compsys in ENS Lyon also participates in 
the development of \mmalpha{}}.\\Revision of \today{}}
\date{January 2009}
\maketitle

\begin{abstract}
This document explains how the C files for \domlib{}, \polylib{} etc.
are organized and compiled in the \mmalfa{} distribution.
\end{abstract}

%\title{Compiling C programs for \mmalfa{}}
%\author{Patrice Quinton}
%\date{September 2007}
%\maketitle


\section*{To do List}
\begin{description}
\item[2/1/2009:] seems that POLYLIB-LIB is useless. Check it...
\item[2008/10/04:] Cannot compile stuff on Linux, on Irisa machine, on \mma{} Version 4.0.1, since
I cannot run mprep... Why?
\item[2008/10/04:] Cannot compile Domlib properly for version \texttt{5.2}, but it works for version 6.0.
\item[X]  Clean the mess in this file
\item[?]  Add a DBG and a if in Makefile.darwin in domlib
\item[X]  Add varlist option to check the value of all variables
\end{description}

\section{Introduction}
\label{intro}
This document\footnote{The source of this document is in:\\
\texttt{\$MMALPHA/Mathematica/doc/sources/Compiling-Domlib/Compiling-Domlib.tex}\\
and this file should be in\\
\texttt{\$MMALPHA/Mathematica/doc/developer/Compiling-Domlib.pdf}\\
} is intended for \mmalfa{} developers who need to recompile or
modify the \domlib.

\section{About PolyLib}
Almost all C programs of \mma{} depend on the polyhedral library \polylib{}, that was developped initally
at IRISA and is now further developped and maintained in Strasburg. For convenience, 
the \mma{} distribution provides a version of \polylib{}. Before compiling
any program, you should make sure that it is properly installed. 

To be more explicit, the \texttt{POLYLIB} variable of file \texttt{Makefile.config}
should contain the path to the directory where the \polylib{} is. 

\section{Where are the executable files?}
They are located in \envmmalpha{}\verb\/bin.\\ostype{}, where \ostype{}
is the Operating System type. Currently, this type can be 
\cygwin{}, \cygwintrde{}, \linux{}.

\newcommand{\domlibexe}{\texttt{domlib}}
\newcommand{\pipexe}{\texttt{pip}}
\newcommand{\codegenexe}{\texttt{code\_gen}}
\newcommand{\readalphaexe}{\texttt{read\_alpha}}
\newcommand{\writealphaexe}{\texttt{write\_alpha}}
\newcommand{\writecexe}{\texttt{write\_c}}
\newcommand{\mathlink}{\texttt{mathlink}}

Typically, one finds the following executables:
\domlibexe{}, \pipexe{}, \codegenexe{}, \readalphaexe{}, 
\writealphaexe{} and \writecexe{}.

\footnote{In \texttt{bin.linux}, it seems that these executable files have
a different name... This information has to be checked.}


\section{Environment variables}
Two environment variables have to be set: 
\envmmalpha{}, \ostype{}.\footnote{Explain how to set these variables...} 

\envmmalpha{} is the directory where \mmalfa{} is installed. It should have been defined 
for your installation to run (see the installation procedure).

\ostype{} is the type of your operating system, which should be set when starting 
a shell. (I do not know what it should be for \cygwin{}.)

\section{Where are the source files?}
The source files are in \envmmalpha{}\verb\/sources\. There is one directory for each 
executable.

\section{Compiling C files for MMAlpha}

\subsection{To Compile a New Version}
Compiling a new version is a somewhat tricky (and risky) operation. Tricky, since you
have to make sure that every variable, option etc. is set. Risky as you may erase some
stuff that works pretty well in an old version. 

So first, make sure that you have a copy of the binaries and of the libraries which work...

Then, good luck!

First, a description of the various directories. These depend on several parameters:
your OS type (\darwin{}, \cygwin{}, \linux{}, \unix{}, \solaris{}, etc.) and the Mathematica version.
Your OS type should be set in an environment variable called OSTYPE, usually
defined by the system (this is true on Unix-like systems, not necessarily on 
Windows...). 

\paragraph*{Compiling all programs}
The reference directory is \texttt{\$MMALPHA/sources}. It contains \texttt{Makefile} which
is used to compile whathever program you need. Before you compile run:
{\begin{verbatim}
make checkvars >>file
\end{verbatim}
This will check a couple of things and report the results in the \texttt{file} file. 
Examine \texttt{file} for possible errors.

To compile, run
{\begin{verbatim}
make all
\end{verbatim}

\paragraph*{Compiling one program}
To compile one particular program \texttt{pg}, run
{\begin{verbatim}
make pg
\end{verbatim}

You may even make several programs at once (\texttt{make pg1 pg2}). 

\paragraph*{Cleaning}
To clean before compiling, run 
{\begin{verbatim}
make clean
\end{verbatim}
Seems however, not to clean everything.

\paragraph*{Checking variables}
To check the variables, run
{\begin{verbatim}
make checkvars
\end{verbatim}

\paragraph*{Testing stuff}
To test (no guarantee about what this tests), run
{\begin{verbatim}
make tests
\end{verbatim}

\subsection{Source directories}
The second important directory is \texttt?\$MMALPHA/sources/program} where your
program is (say for example Domlib). There, you will find source files for Domlib,
a make file (\texttt{Makefile}) that will be called by the main make file, and other 
special make files depending on your OS type: \texttt{Makefile.darwin} or
\texttt{Makefile.linux} etc... 

\subsection{Objects directories}
There is also a directory \texttt{Obj.ostype} for each system type, which should be
created by \texttt{Makefile} (not sure it is), and where the objects file are put as well as 
the final binary. 

Finally, the binaries are all copied in the directory \texttt{\$MMALPHA/bin.ostype}.

\section{Versions...}
The compilation process depends both on your operating system type and of your Mathematica version.
I try to provide as many versions as possible for \mma{} version 6, 5 and 4\footnote{I may soon give up for version 4, since
version 7 is already here... PQ. Jan 2009.}. However, prefered operating systems are Mac OSX and Linux.

Currently, the {\em mathematica version} is wired in the file \texttt{Makefile.config} of
the directory \texttt{\$MMALPHA/sources/MakeIncludes}. 
A variable called \texttt{MMAVERSION}
containts its number (for example, \texttt{6.0}, or \texttt{4.0.1}). The value of this variable
is very important. Indeed, the version number and the OS type define the suffixes of 
some make files. For example, if you want to compile for \mma{} version 6.0 on Linux,
then \texttt{Makefile.config} will call \texttt{Makefile.linux}, which in turn will call
\texttt{Makefile.linux.6.0} in the \texttt{MakeIncludes} directory. This allows some variables
to be set properly depending on the OS and the \mma{} version. To understand that it
is tricky, keep in mind that: 
\begin{itemize}
\item The location of the preparation program (\texttt{mprep}) and the compiling add-ons
of \mma{} necessary to prepare and compile \texttt{Domlib} depends strongly on the version.
\item After version 5, \texttt{Domlib} had to be changed, as the API of the 
\texttt{Mathlink} library were changed...
\item Some \texttt{lex} or \texttt{yacc} files had to be changed also for new versions of some OS...
\end{itemize}

Therefore, I suggest the following procedure.

\begin{enumerate}
\item Check that \envmmalpha{} and \ostype{} are properly set.
\item Look for your \mma{} version. To be sure, start \mma{} and evaluate \texttt{\$VersionNumber}.
\item In file \texttt{\envmmalpha{}/sources/MakeIncludes/Makefile.config}, check that the variable
\texttt{MMAVERSION} is set to your version number.
\item Check that there exist a file \texttt{Makefile.\ostype{}} in the directory
\texttt{\envmmalpha{}/sources/MakeIncludes}. If it does not exist, create it, on the
model of \texttt{Makefile.darwin} for example. You may have to modify the last line
of this file:
\begin{verbatim}
include ../MakeIncludes/Makefile.darwin.$(MMAVERSION)
\end{verbatim}
where you may replace \texttt{darwin} by the name of your operating system\footnote{I could have replaced
here the \texttt{darwin} name by variable \texttt{\ostype{}}, but... depending on your configuration, this
environment variable may be defined in various ways.}.
\item Check that there exist a file \texttt{Makefile.\ostype{}.MMAVERSION} in the directory
\texttt{\envmmalpha{}/sources/MakeIncludes}. If it does not exist, create it, for example, by
a modification of \texttt{Makefile.darwin.6.0}. In this file, you have to set the \texttt{MATHLINK}
variable in the second line (the remaining should be left unchanged.) This variable
says where the compiler additions of your \mma{} distribution are located. This depends on 
your configuration and on your \mma{} version. On my own machine, for version 6, they are 
{\small
\begin{verbatim}
MATHLINK = /Applications/Mathematica.app/SystemFiles/Links/MathLink/\
    DeveloperKit/CompilerAdditions
\end{verbatim}
}
and for version 5.2:
{\small
\begin{verbatim}
MATHLINK = /Applications/Mathematica\ 5.2.app/AddOns/MathLink/\
     DeveloperKit/MacOSX-x86/CompilerAdditions
\end{verbatim}
}
\end{enumerate}

If you have a problem, run a checkvars, and ... check the values of you variables.

\section{How things are}
The \texttt{sources} directory contains one directory per C software: \texttt{Code\_Gen}, 
\texttt{Domlib}, 
\texttt{Pip}, \texttt{Poly}, \texttt{Poly-darwin}, \texttt{Polylib}, \texttt{Pretty}, 
\texttt{Read\_Alpha}, \texttt{Write\_Alpha}.

The \texttt{sources} directory contains one directory per C software: \texttt{Code\_Gen}, 
\texttt{Domlib}, 
Pip, Poly, Poly-darwin, Polylib, Pretty, Read\_Alpha, Write\_Alpha.

\subsection{The \texttt{sources/Makefile}}

There is one Makefile in this directory. It contains a variable called DIR which contains
the list of programs to compile.
There is one \texttt{Makefile} in the source directory. It contains a variable called \texttt{DIR} which contains
programs to compile. For each program \texttt{pg} to compile, \texttt{Makefile}
calls 
\texttt{pg/Makefile}. For example, to compile \texttt{Domlib}, it goes in directory
\texttt{Domlib} and calls \texttt{make}. 

\subsection{Where the binaries are}
All binaries produced by compilation are in file \texttt{../bin.OSTYPE}, 
where \texttt{OSTYPE} is defined as an
environment variable (this may cause a problem, as the \texttt{OSTYPE} is probably not
sufficient to decide which binary to use). Another flag, \texttt{MACHTYPE} is probably also needed...
at least for \texttt{MaCOS}.
\subsection{Where the libraries are}
The libraries produced by compilation are in file \texttt{../lib.OSTYPE}.

In each one of these directories, there is a makefile, which starts with an include
of another file \texttt{../MakeIncludes/Makefile.config}.

\subsection{The directory \texttt{sources/MakeIncludes}}
In this directory, there are a set of make files that are common to most of the 
programs to be compiled.
These files are :
\begin{itemize}
\item \texttt{Makefile.config}. A configuration file.
\item \texttt{Makefile.OSTYPE}, for all OS types. In this file, which are called by \texttt{Makefile.config},
there are configurations commands which depend on the OS type.
\item \texttt{Makefile.OSTYPE.MMAVERSION}. These files contain commands that are 
specific to a \mma{} version, for a particular OS type. They are called by 
\texttt{Makefile.OSTYPE}.
\item A file \texttt{Makefile.checkvars} which is called when the option \texttt{checkvars}
is set. This option does not work totally properly.
\end{itemize}


Thus we have the following chain of makes:
\begin{enumerate}
\item \texttt{Makefile} (in \texttt{sources})
\item Calls a make file \texttt{Makefile} located in some directory of DIR (say \texttt{read\_alpha}).
\begin{enumerate}
\item This \texttt{Makefile} includes \texttt{Makefile.config}, which is common to all programs, and
is in directory \texttt{sources/MakeIncludes}. 
\item \texttt{Makefile.config} includes itself a special configuration file customized for 
a system, named \texttt{Makefile.OSTYPE} and located in the
same directory, which may redefine some variables that were already defined in 
\texttt{Makefile.config}.
\item \texttt{Makefile.OSTYPE} includes another file, named
\begin{verbatim}
Makefile.OSTYPE.version
\end{verbatim}
located in the same directory, where 
\texttt{version} is the version number of Mathematica you are using (say 5.0, 5.2, 6.0). In this file, 
the set-up of variables related to Mathlink is done, as these may depend on the version 
of Mathematica.
\end{enumerate}
\item After including, \texttt{Makefile} includes also \texttt{Makefile.rules}
also located in \texttt{sources/MakeIncludes}, which contains
rules.  \texttt{Makefile.OSTYPE} in directory \texttt{read\_alpha}
where there are special rules for compiling this program on operating system \texttt{OSTYPE}. 
\end{enumerate}

\section{The configuration files, details}

\subsection{\texttt{Makefile.config}}
\label{config}
Defines \texttt{SHELL}, 
\texttt{MMAVERSION},
\texttt{LDLIB},
\texttt{CC},
\texttt{CFLAGS},
\texttt{CXX},
\texttt{YACC},
\texttt{LEX},
\texttt{LFLAGS},
\texttt{STRIP},
\texttt{MAKEFILES},
\texttt{BINDIR},
\texttt{LIBDIR},
\texttt{OBJDIR},
\texttt{NEEDBINDIR},
\texttt{NEEDLIBDIR},
\texttt{NEEDOBJDIR},
\texttt{SOURCES},
\texttt{POLYDIR},
\texttt{POLYINCLUDE},
\texttt{POLYOBJS},
\texttt{CODEGENDIR},
\texttt{CODEGENOBJS},
\texttt{PRETTYDIR},
\texttt{PRETTYOBJS},
\texttt{LPSOLVEDIR},
\texttt{OMEGALIB},
\texttt{OMEGABASIC}.

\subsection{The file Makefile.cygwin}
This is a configuration file where special variables are set. 
Defines SHELL, MMALPHA, MMAVERSION, SUFFIX, DEFINES, LD, PRECISIONFLAG, LDFLAGS, 
MATHLINK, OMEGADIR, MATHPREP, MATHPREPCOR, MATHINCLUDE, MATHLIB, 
MATHLOADLIB, MATHEXTRALIBS and MATHEXTRAOBJS. 

All variables related to Mathlink are set in a file \texttt{Makefile.darwin.mmaversion}
where \texttt{mmaversion} is the number of the Mathematica version for which you
compile. There exists only a file for version 4.2 of Mathematica.

\paragraph*{Notes:} the Mathlink stuff is actually copied in the subdirectory
\$MMALPHA/sources/Mathlink. 

\subsection{The file Makefile.linux}
Defines DEFINES, LD, PRECISIONFLAG, LDFLAGS, 
MATHLINK, OMEGADIR, MATHPREP, MATHPREPCOR, MATHINCLUDE, MATHLIB, 
MATHLOADLIB, MATHEXTRALIBS and MATHEXTRAOBJS. There is a file Makefile.linux-new. I
do not know the differences.

In the \texttt{linux} file, there is the definition of
\texttt{DEFINES}, 
\texttt{LD} (\texttt{gcc},
\texttt{PRECISIONFLAGS},
\texttt{LDFLAGS}. 
\texttt{OMEGADIR}.
Then this file calls in the same directory
another configutation file related to \mma{} (see \ref{mmafile}).

All variables related to Mathlink are set in a file \texttt{Makefile.linux.mmaversion}
where \texttt{mmaversion} is the number of the Mathematica version for which you
compile. 

MATHLINK, OMEGADIR, MATHPREP, MATHPREPCOR, MATHINCLUDE, MATHLIB, 
MATHLOADLIB, MATHEXTRALIBS and MATHEXTRAOBJS. 

\paragraph*{Notes:} the MMAVERSION variable is currently wired in the file
\texttt{makefile.linux} as 4.2, thus making a call to \texttt{makefile.linux.4.2}. 
There exists another such file, for version 3.0.1, referring to the location 
of Mathematica at Irisa, but I think that it is obsolete.


In the \texttt{cygwin} file, there are special definitions
(this file is not up-to-date).

In the \texttt{darwin} file, there is the definition of 
\texttt{DEFINES}, 
\texttt{LD} (\texttt{gcc},
\texttt{LDFLAGS}. Then this file calls in the same directory
another configutation file related to \mma{} (see \ref{mmafile}).

\subsection{The file Makefile.darwin}
Defines DEFINES, LD, PRECISIONFLAG, LDFLAGS, OMEGADIR.

All variables related to Mathlink are set in a file \texttt{Makefile.darwin.mmaversion}
where \texttt{mmaversion} is the number of the Mathematica version for which you
compile. 

MATHLINK, OMEGADIR, MATHPREP, MATHPREPCOR, MATHINCLUDE, MATHLIB, 
MATHLOADLIB, MATHEXTRALIBS and MATHEXTRAOBJS. 



\subsection{The Makefile.rules file}
This file contains general rules to compile programs. These rules are explained in the
file itself. 

\subsubsection{List of variables, and where they are defined}

In Makefile.config, there is an inclusion of the definitions in MakeIncludes/Makefile.\$(OSTYPE)
which are peculiar to each OS type.

\begin{description}
\item[BINDIR]: in Makefile.config. \$(MMALPHA)/bin.\$(OSTYPE). Where the binaries will be
put. 
\item[CC]: in Makefile.config. I guess it is the compiler...
\item[CFLAGS]: in Makefile.config. -O3 -g -I\$(POLYINCLUDE) \$(PRECISIONFLAG). I guess that
these are the flags of the compiler.
\item[CODEGENDIR]: in Makefile.config. \$(SOURCES)/Code\_Gen. Where Code\_Gen is
\item[CODEGENOBJS]: in Makefile.config. \$(OBJDIR)/gen.o and nodeprocs.o. The objects of 
Code\_gen.
\item[CXX]: in Makefile.config g++
\item[DIR]: in Makefile. The list of directories where there are files to be compiled.
\item[LDLIB]: in Makefile.config. ld -r. It is the command for the linker.
\item[LEX]: in Makefile.config. flex
\item[LFLAGS]: in Makegile.config. Flags for flex
\item[LIBDIR]: in Makefile.config. \$MMALPHA/lib.\$(OSTYPE). Where the libraries are put.
\item[LPSOLVEDIR]: in Makefile.config. Where LP solve is, unused currently.
\item[OMEGALIB]: in Makefile.config. Where the OMEGA library is. unused currently.
\item[MMALPHA]: environment variable. May also be wired by a definition in Makefile.config
or in a system config file.
\item[MMAVERSION]: environment variable. May also be wired by a definition in Makefile.config
or in a Makefile.ostype.version file. Defines the version number of Mathematica. 
\item[MAKEFILES]: in Makefile.config. Contains the list of makefiles, i.e. Makefile itself (in the 
sources directory), Makefile.rules, Makefile.config, and Makefile.\$(OSTYPE) located in 
the directory MakeIncludes.
\item[NEEBINDIR]: in Makefile.config. \$(BINDIR)/.make. This variable is checked by a rule in
Makefile.rules. 
\item[NEEDLIBDIR]: in Makefile.config. \$(LIBDIR)/.make. Use?
\item[NEEDOBJDIR]: in Makefile.config. \$(OBJDIR)/.make. Use?
\item[OBJDIR]: in Makefile.config. Obj.\$(OSTYPE). Defines where the object files will be put.
This is relative to the directory of the application.
\item[OSTYPE]: environment variable
\item[POLYDIR]: in Makefile.config. \$(SOURCES)/Polylib. Where the source files of the polylib are.
\item[POLYOBJS]: in Makefile.config. The list of object files of the polylib. These files are all in 
\$(OBJDIR). I changed this line in July 2007, I do not know why... Maybe to add object files that were
needed...
\item[POLYXOBJ]: in Makefile.config. Another set of object files
\item[POLYMSG]: in Makefile.config. \$(OBJDIR)/errormsg.o Defines an object file where the 
error messages are defines for Polylib.
\item[PRETTYOBJS]
\item[PRETTYDIR]: in Makefile.config. \$(SOURCES)/Pretty. Where Pretty is.
\item[PRETTYOBJS]: in Makefile.config. $(PRETTYDIR)/$(OBJDIR)//itemprocs.o and writeitem.o. 
The objects of Pretty.
\item[SHELL]: in Makefile.config. /bin/sh
\item[SOURCES]: in Makefile.config. \$(MMALPHA)/sources. The source files of MMAlpha.
\item[STRIP]: in Makegile.config. strips. Meaning?
\item[YACC]: in Makefile.config. bison
\item[YFLAGS] in Makefile.config. Flags of Yacc
\end{description}

\subsubsection{ReadAlpha}

\begin{description}
\item First line includes Makefile.config
\item[CPPFLAGS] = -I\$(POLYDIR) -g (flags for cc)
\item[LOADLIBES] = empt
\item[DUMMY] = list of files to be destroyed when cleaning
\item[NAME] = the name of the program, read\_alpha
\item[OBJS] = the list of object files, \$(OBJDIR)/read\_alpha.o, node.o, \$(POLYOBJS)
\item[LIBS] = libraries to link again
\end{description}

Then we have a command of the form
\$(NAME) : \$(BINDIR)/\$(NAME)
	@\# bla 

Then special rules or dependencies for .o files
Seems that the object files and the temporary binary are in Obj.darwin.

\subsubsection{WriteAlpha}

\begin{description}
\item First line includes Makefile.config
\item[CPPFLAGS] = -I\$(POLYDIR)
\item[LDFLAGS] = -L\$(LIBDIR) \# -mwindows
\item[LOADLIBES] = empty
\item[DUMMY] = list of files to be destroyed when cleaning
\item[NAME] = the name of the program, read\_alpha
\item[OBJS] = the list of object files, \$(OBJDIR)/yacc.o, nodeprocs.o, node2item.o, \$(POLYOBJS),
\$(POLYMSG) \$(PRETTYOBJS)
\item[LIBS] = libraries to link again, i.e, \$(LIBDIR)/libpoly.a, libpretty.a and libpolymsg.a
\end{description}

Then we have a command of the form
\$(NAME) : \$(BINDIR)/\$(NAME)
	@\# bla 

Then special rules or dependencies for .o files
Seems that the object files and the temporary binary are in Obj.darwin.


\subsection{The files \texttt{sources/MakeIncludes/Makefiles.OSTYPE.Number}}
\label{mmafile}
There is one such file for each OS and each version of \mma{} which 
is supported. Currently: 
\begin{itemize}
\item For \texttt{darwin}: versions 5.2 and 6.0
\item For \texttt{linux}: versions 3.0.1 and 4.2. 
\end{itemize}
They define: 
\texttt{MATHLINK},
\texttt{MATHPREP},
\texttt{MATHINCLUDE},
\texttt{MATHLIB},
\texttt{MATHLOADLIB},
\texttt{MATHEXTRALIBS},
\texttt{MATHEXTRAOBJS}.

The \texttt{linux} version is set for IRISA and should be checked.

\paragraph*{Notes:} the MMAVERSION variable is currently wired in the file
\texttt{makefile.linux} as 4.2, thus making a call to \texttt{makefile.linux.4.2}. 
There exists another such file, for version 3.0.1, referring to the location 
of Mathematica at Irisa, but I think that it is obsolete.

\subsection{The \texttt{MakeIncludes/Makefile.rules} file}
\label{rules}
This file is called somewhere (I do not remember where) and sets common
dependence rules. There are rules for:
\begin{itemize}
\item 
\texttt{BINDIR},
\texttt{OBJDIR} (binaries, objects), 
\texttt{NEEDBINDIR} (to create BINDIR, create the directory etc.)
\texttt{NEEDLIBDIR},
\texttt{NEEDOBJDIR},
\texttt{LIBDIR} (for each library).
\end{itemize}


\section{Organization for \texttt{Domlib}}
The \texttt{Domlib/Makefile} calls another 
\texttt{Makefile.OSTYPE}, except for cygwin. 

For each OS, there is a \texttt{Makefile.OSTYPE}. 
This Makefile calls first \texttt{MakeIncludes/Makefile.config} (see \ref{config}).

It calls then \texttt{MakeIncludes/Makefile.OSTYPE.MMAVERSION} (see \ref{mmafile})
(check that this file is not already called by the config file).
It defines
\texttt{POLYLIB-LIB},
\texttt{DBG},
\texttt{EXTRA\_CFLAGS}.
It contains dependence rules for 
\texttt{BINDIR},
\texttt{OBJDIR}. Seems that there is no call to Makefile.rules.

\paragraph*{Notes:} the Mathlink stuff is actually copied in the subdirectory
\$MMALPHA/sources/Mathlink. 

This directory conaints a 
\texttt{MATHLINK} directory where all the stuff needed for the \mma{} interface
is put. The 
\texttt{Obj.OSTYPE} directory is where the object files will be put.


\section{\texttt{ReadAlpha}}

\begin{description}
\item First line includes Makefile.config
\item[\texttt{CPPFLAGS}] = -I\$(POLYDIR) -g (flags for cc)
\item[LOADLIBES] = empty
\item[DUMMY] = list of files to be destroyed when cleaning
\item[NAME] = the name of the program, read\_alpha
\item[OBJS] = the list of object files, \$(OBJDIR)/read\_alpha.o, node.o, \$(POLYOBJS)
\item[LIBS] = libraries to link again
\end{description}

These definitions are followed by special dependence rules, and 
also include the general rules in MakeIncludes/Makefile.rules. Finally, 
there is a file
\texttt{Makefile.checkvars} which contains the rules for 
checking variables. All variables are not checked. 

\subsubsection{WriteAlpha}

\begin{description}
\item First line includes Makefile.config
\item[\texttt{CPPFLAGS}] = -I\$(POLYDIR)
\item[\texttt{LDFLAGS}] = -L\$(LIBDIR) \# -mwindows
\item[\texttt{LOADLIBES}] = empty
\item[\texttt{DUMMY}] = list of files to be destroyed when cleaning
\item[\texttt{NAME}] = the name of the program, read\_alpha
\item[\texttt{OBJS}] = the list of object files, \$(OBJDIR)/yacc.o, nodeprocs.o, node2item.o, \$(POLYOBJS),
\$(POLYMSG) \$(PRETTYOBJS)
\item[\texttt{LIBS}] = libraries to link again, i.e, \$(LIBDIR)/libpoly.a, libpretty.a and libpolymsg.a
\end{description}

These definitions are followed by special dependence rules, and 
also include the general rules in MakeIncludes/Makefile.rules. Finally, 
there is a file
\texttt{Makefile.checkvars} which contains the rules for 
checking variables. All variables are not checked. 

\subsubsection{WriteC}

\begin{description}
\item First line includes Makefile.config
\item[\texttt{CPPFLAGS}] = -I\$(POLYDIR)
\item[\texttt{LDFLAGS}] = -L\$(LIBDIR) \# -mwindows
\item[\texttt{LOADLIBES}] = empty
\item[\texttt{DUMMY}] = list of files to be destroyed when cleaning
\item[\texttt{NAME}] = the name of the program, read\_alpha
\item[\texttt{OBJS}] = the list of object files, \$(OBJDIR)/yacc.o, nodeprocs.o, node2item.o, \$(POLYOBJS),
\$(POLYMSG) \$(PRETTYOBJS)
\item[\texttt{LIBS}] = libraries to link again, i.e, \$(LIBDIR)/libpoly.a, libpretty.a and libpolymsg.a
\end{description}

These definitions are followed by special dependence rules, and 
also include the general rules in MakeIncludes/Makefile.rules. Finally, 
there is a file
\texttt{Makefile.checkvars} which contains the rules for 
checking variables. All variables are not checked. 




\section{List of variables, and where they are defined}

In Makefile.config, there is an inclusion of the definitions in MakeIncludes/Makefile.\$(OSTYPE)
which are peculiar to each OS type.

\begin{description}
\item[\texttt{BINDIR}]: in Makefile.config. \$(MMALPHA)/bin.\$(OSTYPE). Where the binaries will be
put. 
\item[\texttt{CC}]: in Makefile.config. I guess it is the compiler...
\item[\texttt{CFLAGS}]: in Makefile.config. -O3 -g -I\$(POLYINCLUDE) \$(PRECISIONFLAG). I guess that
these are the flags of the compiler.
\item[\texttt{CODEGENDIR}]: in Makefile.config. \$(SOURCES)/Code\_Gen. Where Code\_Gen is
\item[\texttt{CODEGENOBJS}]: in Makefile.config. \$(OBJDIR)/gen.o and nodeprocs.o. The objects of 
Code\_gen.
\item[\texttt{CXX}]: in Makefile.config g++
\item[\texttt{DEFINES}]: in Makefile.OSTYPE. 
\item[\texttt{DIR}]: in Makefile. The list of directories where there are files to be compiled. 
\item[\texttt{LDLIB}]: in Makefile.config. ld -r. It is the command for the linker.
\item[\texttt{LEX}]: in Makefile.config. flex
\item[\texttt{LFLAGS}]: in Makegile.config. Flags for flex
\item[\texttt{LDFLAGS}]: in Makegile.OSTYPE. Flags for the loader.
\item[\texttt{LIBDIR}]: in Makefile.config. \$MMALPHA/lib.\$(OSTYPE). Where the libraries are put.
\item[\texttt{LPSOLVEDIR}]: in Makefile.config. Where LP solve is, unused currently.
\item[\texttt{OMEGALIB}]: in Makefile.config. Where the OMEGA library is. unused currently.
\item[\texttt{MMALPHA}]: environment variable. May also be wired by a definition in Makefile.config
or in a system config file.
\item[\texttt{MMAVERSION}]: environment variable. May also be wired by a definition in Makefile.config
or in a Makefile.ostype.version file. Defines the version number of Mathematica. 
\item[\texttt{MAKEFILES}]: in Makefile.config. Contains the list of makefiles, i.e. Makefile itself (in the 
sources directory), Makefile.rules, Makefile.config, and Makefile.\$(OSTYPE) located in 
the directory MakeIncludes.
\item[\texttt{NEEBINDIR}]: in Makefile.config. \$(BINDIR)/.make. This variable is checked by a rule in
Makefile.rules. 
\item[\texttt{NEEDLIBDIR}]: in Makefile.config. \$(LIBDIR)/.make. Use?
\item[\texttt{NEEDOBJDIR}]: in Makefile.config. \$(OBJDIR)/.make. Use?
\item[\texttt{OBJDIR}]: in Makefile.config. Obj.\$(OSTYPE). Defines where the object files will be put.
This is relative to the directory of the application.
\item[\texttt{OMEGADIR}]: in Makefile.linux. Directory of the OMEGA software, not used
currently.
\item[\texttt{OSTYPE}]: environment variable
\item[\texttt{POLYDIR}]: in Makefile.config. \$(SOURCES)/Polylib. Where the source files of the polylib are.
\item[\texttt{POLYOBJS}]: in Makefile.config. The list of object files of the polylib. These files are all in 
\$(OBJDIR). I changed this line in July 2007, I do not know why... Maybe to add object files that were
needed...
\item[\texttt{POLYXOBJ}]: in Makefile.config. Another set of object files
\item[\texttt{POLYMSG}]: in Makefile.config. \$(OBJDIR)/errormsg.o Defines an object file where the 
error messages are defines for Polylib.
\item[\texttt{PRECISIONFLAGS}:] in Makefile.linux. Check this. 
\item[\texttt{PRETTYDIR}]: in Makefile.config. \$(SOURCES)/Pretty. Where Pretty is.
\item[\texttt{PRETTYOBJS}]: in Makefile.config. $(PRETTYDIR)/$(OBJDIR)//itemprocs.o and writeitem.o. 
The objects of Pretty.
\item[\texttt{SHELL}]: in Makefile.config. /bin/sh
\item[\texttt{SOURCES}]: in Makefile.config. \$(MMALPHA)/sources. The source files of MMAlpha.
\item[\texttt{STRIP}]: in Makegile.config. strips. Meaning?
\item[\texttt{YACC}]: in Makefile.config. bison
\item[\texttt{YFLAGS}]: in Makefile.config. Flags of Yacc
\item[\texttt{MATHLINK}]: in MakeIncludes/Makefile.OSTYPE.number.
\item[\texttt{MATHPREP}]: in MakeIncludes/Makefile.OSTYPE.number.
\item[\texttt{MATHINCLUDE}]: in MakeIncludes/Makefile.OSTYPE.number.
\item[\texttt{MATHLIB}]: in MakeIncludes/Makefile.OSTYPE.number.
\item[\texttt{MATHLOADLIB}]: in MakeIncludes/Makefile.OSTYPE.number.
\item[\texttt{MATHEXTRALIBS}]: in MakeIncludes/Makefile.OSTYPE.number.
\item[\texttt{MATHEXTRAOBJS}]: in MakeIncludes/Makefile.OSTYPE.number.
\item[\texttt{EXTRA\_CFLAGS}]: in Domlib/Makefiles.darwin.
\item[\texttt{CPPFLAGS}]: in ReadAlpha, WriteAlpha, WriteC
\item[\texttt{LOADLIBES}]: in ReadAlpha, WriteAlpha, WriteC
\item[\texttt{DUMMY}]: in ReadAlpha, WriteAlpha, WriteC
\item[\texttt{NAME}]: in ReadAlpha, WriteAlpha, WriteC
\item[\texttt{OBJS}]: in ReadAlpha, WriteAlpha, WriteC
\item[\texttt{LIBS}]: in ReadAlpha, WriteAlpha, WriteC
\end{description}

\appendix

\section{Installing \polylib{}}
Look in MMALPHA/doc/Polylib/note.pdf for more details. I describe here the full 
procedure, but in the current distribution, \polylib{} version 5.22.3 is already
present in the distribution: you may then skip steps 1 and 2. 

\begin{enumerate}
\item Get polylib from Strasburg site (http://icps.u-strasbg.fr/polylib/)
\item Copy it in \texttt{\$MMALPHA/sources}
\item Gunzip it and untar it (double click, or gunzip)
\item The name of this polylib should be something like \texttt{polylib-5.22.3}. Go in this
directory (in a shell).
\item Run the following commands:\label{trois}
\begin{verbatim}
./configure --prefix="$MMALPHA/sources/Polylib"
make
make install
\end{verbatim}
\item Run \texttt{make test} to check that everything is OK
\end{enumerate}

During step~\ref{trois}, the value of the environment variable
\texttt{\$MMALPHA} should be the path of your \mma{} installation. 

When \polylib{}�is installed, make sure that the value
of \texttt{POLYDIR} in file \texttt{Makefile.config} is that 
of the path to the \polylib{} (as the source files of \polylib{} are needed to
compile the C programs of \mma{}), and that 

\section{Introduction}
This document explains how the \domlib{} library is installed, in this directory. It was
written while porting the \domlib{} on Mac OS X. 

\section{To be checked}
Is your Polylib up-to-date? Look in sources, where you must find both Polylib and
Poly (where the files produced by the make command of Polylib are).


\section{Content of the directory}
The directory
contains the source, \texttt{domlib.c}, a make file, \texttt{Makefile}, the \ML{} template file 
\texttt{domlib.tm}. The make creates an object file \texttt{domlib.o}, a binary 
file \texttt{domlib}. On the fly, it translates the template file into a source file
\texttt{domlibtm.c}.

The file \mkf{} calls another make file, located in:
\begin{verbatim}
$MMALPHA/sources/MakeIncludes/Makefile.$OSTYPE
\end{verbatim}
In this make file, the location of various directories or programs are set. In particular, 
that of MLINKDIR.

To run properly, the Polylib must have been compiled previously. See documentation
on Polylib, but in a few words, just go in PolyLib and compile it by specifying
that the directory is Poly. This location is defined in Makefile.darwin, and it may be
changed. The commands to be done, once you have copied the gziped filed 
obtained from strasburg are:
\begin{verbatim}
./configure --prefix="$MMALPHA/sources/Poly"
make all
make install
\end{verbatim}

\section{The rational of the installation}
To be compiled, \domlib{} needs a \ML{} preprocessor called \mprep{}, and the
\polylib{} library. \ML{}�compiler additions are also needed. 

On this machine, \mma{} is located in\footnote{This depends on the current version of \mma{} that you have.}
\begin{verbatim}
/Applications/Mathematica 5.2.app/
\end{verbatim}
The variable \texttt{MLINKDIR} contains the location
\begin{verbatim}
/Applications/Mathematica 5.0.app/AddOns/Mathlink/DeveloperKit/MacOSX-x86
\end{verbatim}
On other systems, this location may change. If you do not have an Intel Processor, you
may have to switch \texttt{MLINKDIR} to either 
\begin{verbatim}
/Applications/Mathematica 5.0.app/AddOns/Mathlink/DeveloperKit/Darwin
\end{verbatim}
or
\begin{verbatim}
/Applications/Mathematica 5.0.app/AddOns/Mathlink/DeveloperKit/Darwin-PowerPC64
\end{verbatim}

In the \mkf{} , variables contain the system name, the place of the \polylib{} header files, 
the place of the \polylib{} library. The compiled additions are then located. 
Variable \texttt{POLYLIB\_INCLUDE} contains the directory where the include
files of the \polylib{} are. Similarly variable \texttt{POLYLIB\_LIB} 
contains the directory where the libraries of the \polylib{} are.

Variable \texttt{CADDSDIR} gives the place where the compiler additions of 
\mma{} are. 

Variable \texttt{EXTRA\_CFLAGS} is set to 
\begin{verbatim}
-DLINEAR_VALUE_IS_LONGLONG -DUNIX
\end{verbatim}

Variable \texttt{INCDIR} is set to {CADDSDIR}.
Variable \texttt{INCPOLYDIR} is set to {POLYLIB\_INCLUDE}.
Same for \texttt{LIBDIR} and \texttt{LIBPOLYLIB}.

Variable \texttt{MPREP} is set to where \texttt{mprep}�is.

The \texttt{domlibtm.c}
file is obtained by running \mprep{} (provided by \mma{}) on \texttt{domlib.tm}. 

The \texttt{domlib} binary file is obtained by first running \gcc{} on 
both \doml{} and \tm{} then 
linking the object files with the appropriate \polylib{} library. This library is 
itself installed {\em separately}.\footnote{See companion \polylib{} documentation.} 
In the current version, we used the 64 bit version, but other versions are 
available. 

The form of the gcc command is 
\begin{verbatim}
gcc -DLINEAR_VALUE_IS_LONGLONG -DUNIX domlibtm.o domlib.o 
  -L${LIBDIR} -L${CADDSDIR}
  -L${POLYLIB} -lpolylib64 -lML -o $@ -v
\end{verbatim}

\section{Running the \domlib{} from \mma{}}
To run this library from \mma{}, the following conditions must be satisfied.
\begin{enumerate}
\item The \domlib{} binary file must be located in a directory whose
address appears on the \texttt{\$Path} variable of \mma{}. 
\item To load the library, run 
\begin{verbatim}
Install[ "domlib" ]
\end{verbatim}
in \mma{}. The list of available functions in the library are obtained by evaluating
\begin{verbatim}
?Dom*
\end{verbatim}
\end{enumerate}
In fact, this library is usually run within the \mmalfa{} software, using the 
\texttt{Domlib} package. 
\section{Perspectives}
A better way to organize the library would be to make sure that it is portable
accross platforms. To do so, we plan to build a version using the \texttt{autoconf}
and \texttt{automake} software, and more generally, the \texttt{gnu} building tools.

\section{Additional remarks}
In this directory, the installation is done using a make file. To compile for
Mac OS X, the UNIX macro should be defined, and it is by means of the 
EXTRA\_CFLAGS option -DUNIX.

\section{Mathematica at Irisa}
At Irisa, Mathematica is located in directory
\begin{verbatim}
/soft/Mathematica
\end{verbatim}
Several versions are available. The most recent one is \verb\V4.0.2\. It is located
in 
\begin{verbatim}
/soft/Mathematica/V4.0.2
\end{verbatim}
Three version are available : for \texttt{Linux}, \texttt{SGI} and \texttt{Solaris}. 
Inside the directory, there are subdirectories \texttt{CompilerAdditions} where one can find
the \texttt{mprep} pre-processor. 

The transit.irisa.fr machine is a Linux machine. I could not compile on this architecture, 
for unknown reasons. (mprep does not work). 

The europe machine is a solaris machine. I was able to compile. I had to use in Makefile.solaris the 
same as in Linux. In particular, the MATHEXTRALIB has to be -lnsl -l socket
I had to add the \texttt{\$Path} definition, which is not standard (init file?). 

\section{The Make file for the MMAlpha distribution}
In sources, one find Makefile. 
This file defines LDLIB as ld -r, CC as gcc, CFLAGS as -03 -g -I\$(POLYINCLUDE) \$(PRECISIONFLAG).
It defines some variables for Yacc and Lex. 

It includes the makefile.system which is in the same directory MakeIncludes, here
Makefile.darwin.
In Makefile.darwin, LD is set to gcc, PRECISIONFLAGS, LDFLAGS, MATHLINK, MATHPREP,
MATHINCLUDE, MATHLIB, MATHLOADLIB (-lML -lm) and MATHEXTRALIBS is set
to empty.

It then defines MAKEFILES, 
BINDIR, LIBDIR, OBJDIR, NEEDBINDIR, NEEDLIBDIR, NEEDOBJDIR, 
SOURCES, 


For all binary files in DIR, goes in directory, then runs make. 

In the Domlib directory, one finds the Makefile.darwin file. It first calls the Makefile.config
in the directory MakeIncludes. 


\section{A new version of \domlib{} (May 6, 2008)}
It appears that the MathLink functions were modified between \mma{} version 4
and version 5. I thus had to modify Domlib accordingly. The current version of 
Domlib, compiled with the tools fo version 6, is correct for version 5 and 6. I did
not try yet to recompile it for version 5, to see it there are some differences. Actually, 
I guess that there are such differences. 

The Mathlink documentation of version 6.0 is quite useful. Actually, the main differences
is in the functions \texttt{MLGetFunction} and \texttt{MLCheckFunction}, which do not
support the same arguments... Too bad ! 

I added 3 functions to \domlib{}: \texttt{mirrorDomain}, \texttt{mirrorMatrix} 
and \texttt{mirrorZDomain}. These function may be useful to see if there is something
wrong. 

The old version of domlib is kept in the directory \texttt{sources/Domlib-Version4}. The source
files are those which were compiled for Powermac on version 5. 


\section{Compiling \domlib{}}
In the \zdomlib{} file, one finds the \texttt{domlib.c}
file, which is the source file of \domlib{}, the 
\texttt{domlib.tm} file which is the patterns of 
the \domlib{} functions for \mathlink{}, and finally, 
several make files.

I guess that compiling \domlib{} is done by simply typing
\texttt{make} in the \zdomlib{} directory. But the best approach
is to run the make file that is contained in \texttt{\$MMALPHA/sources}.

A \texttt{README} file gives some explanations regarding \domlib{}.
Beware that only the first few lines seem to be uptodate. 

\subsubsection{The \texttt{Makefile} file}
The file \texttt{Makefile} contains a switch to make files that 
depend on the OS type (this one should be, I guess, in the 
environment variable \ostype{}). For example, for 
\cygwin{}, the corresponding make file is \texttt{Makefile.Visual}.

\subsubsection{The \texttt{Makefile.Visual} file}
The first thing that this make file does is to include the 
configuration make file, the name of which is \texttt{Makefile.config} and
which is located in the \envmmalpha{}\verb\MakeIncludes\ directory. 

Therefore, you will have to look at the documentation of this
make file given in~\ref{doc-config-make} if you want to 
understand everyhing. 

In addition, the configuration make file calls other make files 
situated in the same directory...


Coming back to \texttt{Makefile.Visual}, it defines 
the following variables:
\begin{itemize}
\item \varcc{}: the compiler, here \texttt{cl};
\item \varld{}: the linker, \texttt{link};
\item \varpolylib{}: where the \polylib{} is (it seems to me that
this variable is already set);
\item \varcflags{}: redefines the compilation flags;
\item \varlflags{}: refedines the link flags;
\item \varbin{}: gives the name of the binary file for domlib;
\item \varobj{}: where are the object codes. Here, they are in 
a directory \texttt{Obj.cygwin} located in the same directory.
\item \vardepobj{}: a list of object files which are going to be
generated by the compiling.
\end{itemize}
Then one finds some rules to create \domlib{}. The \texttt{all} rule
is for creating binary files. I do not understand totally the other 
rules...

One particular rule allows the \texttt{mlink.c} file to 
be created using \mmalpha{}'s tools from \texttt{domlib.tm}.

\subsubsection{The \texttt{Makefile.linux} file}
I am not sure that this file works properly. It defines 
\varcppflags{}, \varldflags{}, \varloadlibes{}, \vardummy{},
\varname{}, \varobjs{}, \varlibs{}. It uses \varpolydir{}, 
\varmathinclude{}, \varlibdir{}

\section{Conclusion}

\printindex

\newpage
\tableofcontents

\end{document}
