\documentclass[12pt]{article}
\usepackage{fullpage,epic,eepic,epsfig,makeidx,color,graphicx,moreverb,boxedminipage,version}
\sloppy


%\usepackage{fullpage,verbatim,vrbinput,vrbtabs,psfig,version,epic,eepic}
\usepackage{makeidx}
%\usepackage{french}
%\usepackage{mathematica}

\newcommand{\mmalpha}{{\sc mmalpha}}
\excludeversion{corrige}
\excludeversion{maversion}
\makeindex
\begin{document}


\title{The Meta Package}
\author{Patrice Quinton}
\date{Version of \today}
\maketitle
\abstract{This document presents the \texttt{Meta} package of \mmalpha{}. This
package allows one to define quickly translators for Mathematica expressions, in 
particular, \mmalpha{} Abstract Syntax Trees. Translators are specified by a set of rules
which are translated as a Mathematica package. When loaded, together with a 
separate semantic file, one gets a translator. The \texttt{meta} package is used in \mmalpha{}
to produce VHDL code. }

\section{Introduction}
The 
\texttt{meta}
 package allows one to define a set of meta rules for the translation of
 a Mathematica expression, and to translate these rules as a Mathematica
 package.
 The functions of this package can then be applied to an expression, in
 order to translate this expression into something useful.
 In the following, we explain how this meta translator operates, and we
 illustrate this by means of a few examples.
This package is used in various parts of \mmalpha{}. 

Briefly speaking, creating a translator consists in:
\begin{enumerate}
\item writing a set of meta-rules in a \texttt{.meta} file,
\item writing a semantic file in a \texttt{.sem} file (not mandatory, but often convenient),
\item evaluating the \texttt{.meta} file with the \texttt{meta} function. This creates
a \texttt{.m} Mathematica package which contains the translator,
\item load the translator which then becomes available.
\end{enumerate}

The organization of this document is as follows. Section~\ref{define} presents
the syntax of meta-rules. In Section~\ref{using}, the use of the \texttt{meta} package is described. 
Options of the \texttt{meta} function are explained in Section~\ref{options}. Section~\ref{translation}
explains how meta-rules are translated. An example of translator is shown in 
Section~\ref{example}. The meta code and the semantic file of this translator are 
shown in Appendix~\ref{checkCellmeta} and Appendix~\ref{checkCellSem} respectively.

Directory \texttt{\$MMALPHA/doc/Packages/Meta} contains a notebook with 
examples. In particular, there is a nice translator of (a part of) Alpha to Mathematica.

You may access this notebook from the master notebook of 
\mmalpha{}\footnote{
Recall that the master notebook may be loaded by means of the command 
\texttt{on[]} from the Mathematica front-end.}
or by evaluating the expression
\begin{verbatim}
docLink["Meta"]
\end{verbatim}
in any notebook. This evaluation produces a button which opens the 
notebook.


\section{How to define meta rules}
\label{define}
The input file contains a list of meta-rules.
 Let 
\texttt{myTranslator.meta}
 be this file.
 A meta rule has the following syntax (in BNF form):
\verbatiminput{bnfMeta.tex}

A rule therefore describe a so-called 
{\em abstract node}, or a 
{\em switch rule}.
 
\subsection{Abstract nodes}
Abstract nodes represent expressions which correspond to an internal node
 of the abstract syntax tree.
 For example, the first level of an Alpha program has the form 
\texttt{system[n,p,id,od,ld,e]}, where 
\texttt{system} is the head, 
\texttt{n} is a string representing the name of the system, 
\texttt{p} is the parameter, 
\texttt{id} is the list of input variable declarations, 
\texttt{od} is the list of output variable declaration, 
\texttt{ld} the local declarations, and 
\texttt{e} the list of equations.
Such a node is described by the following meta-rule: 
\begin{verbatim}
SYSTEM ::= 
  system[ n:_String, p:DOMAIN, id:{DECLARATION}, od:{DECLARATION}, 
  ld:{DECLARATION}, e:{EQUATION} ] :> semantics...
\end{verbatim}
The left-hand-side of the rule, 
\texttt{system}, is the name of the rule.
 Following the 
\texttt{::=}, is the pattern which describes the form of the abstract node, followed,
 after the symbol 
\texttt{:>} by the semantics associated to this node.

The first argument of this pattern, 
\texttt{n:\_String}, has the form 
\texttt{<symbol>:<pattern>}.
 This pattern will be used to check that the corresponding terminal element
 in the AST is a string.
 If not, an error message will be issued.
 The 
\texttt{n} symbol names the argument which matches this position, and can be used
 in the semantics of the rule as explained later.

The second argument, 
\texttt{p:DOMAIN}, has the form 
\texttt{<symbol>:<rule name>}.
 It generates recursively a call to the meta-rule 
\texttt{DOMAIN.}
The subexpression which matches this second position is named 
\texttt{p}.
 The next argument, 
\texttt{id:{DECLARATION}}, is treated in a similar way, except that one expect a list of expressions
 matching the 
\texttt{DECLARATION} meta-rule.
 
\subsection{Switch rules}
\label{switchrules}
Switch rules allow alternate abstract rules to be called, depending on a
 pattern.
 For example, the following meta-rule is used to select the various forms
 of Alpha expressions:
\begin{verbatim}
EXPRESSION ::= 
  {_binop -> BINOP, 
   _var -> VAR, 
   _affine -> AFFINE, 
   _restrict -> RESTRICT 
  }
\end{verbatim}

The arguments of this rule are particular cases of Mathematica patterns,
 namely, 
\texttt{<typed\_blanks>}.
 What we call here a typed\_blank is a Mathematica expression of the form
 \texttt{\_type}, which matches any Mathematica expression whose head is 
\texttt{type}.
 
This rule can be read as follows: if the head of the current expression
 is 
\texttt{binop}, then apply meta-rule 
\texttt{BINOP}, if it is 
\texttt{var}, then apply meta-rule 
\texttt{VAR}, etc.

\subsection{Arguments}
\subsubsection{Arguments of abstract nodes}

The form of arguments depends on the type of subtree which is expected in
 the AST.
 If the subtree simply consists of a terminal element, a simple pattern
 is used, for example, 
\texttt{name:\_String}.

For non terminal subtrees, arguments of the form 
\texttt{<symbol>:<symbol>} or 
\texttt{<symbol>:{<symbol>}} are used.
 The latter form allows a list of expressions to be described.
 Finally, an argument of the form 
\texttt{<symbol>:{List}} generates a call to a particular function which expects a (Mathematica)
 list of objects, and returns this list unchanged.
 
\subsubsection{Switch arguments}

A switch argument has the form 
\texttt{\_Symbol -> <rule name>} or 
\texttt{\_Symbol -> {<rule name>}} .
 The 
\texttt{<rule name>} is the name of a rule which is selected by the head of the node, as specified
 by the pattern.
 As an example: 
\texttt{\_var -> varexpression} describes an alternative where the 
\texttt{varexpression} rule is called, when the head of the node is 
\texttt{var}.
 
\subsection{Semantic expressions}

The semantic part of a meta-rule is a Mathematica expression, involving
 symbols which name either arguments in the right-hand side part of the
 meta-rule, or results of the translation of these arguments by the semantic
 functions associated to the meta-rules.
 
Let us explain this on an example.
Consider the meta-rule:
\begin{verbatim}
EQUATION::= equation[c:_String, e:EXPRESSION] :> 
    semanticFunction[ equation, {c, tre} ] 
\end{verbatim}

The above rule describes the abstract node 
\texttt{EQUATION}.
 We expect this expression to contain a string, then an 
\texttt{EXPRESSION}, the latter being described by the 
\texttt{EXPRESSION} meta-rule.
 The string, when found, will be named 
\texttt{c}.
 The equation is named 
\texttt{e}.
 By convention, the result of the call to function 
\texttt{translateEXPRESSION} applied to 
\texttt{e} is named 
\texttt{tre} and can be used in the semantic part.
 The semantic expression, 
\texttt{semanticFunction[ equation, {c, tre} ]}, is a call to a function defined in a separate file, called the semantic
 file.
 There are absolutely no constraints in the way the semantic expression
 is defined.
 In this example, I choosed to name the functions I use 
\texttt{semanticFunction}, whose first parameter is the head of the expression, and the second one
 is the list formed by 
\texttt{c} and by the translation of the 
\texttt{EXPRESSION}.
 

\subsubsection*{Warning}
Although one can put any Mathematica expression as the semantic part of
 a meta-rule, calling a semantic function is a good way to avoid trouble.
 Indeed, when the meta translator reads the meta-rules, it uses the Mathematica
 interpreter which evaluates as many expressions as possible.
 Therefore, the result of the semantic expression might be very surprising.
 For example, lets assume that you write the semantic part
\texttt{c === "blabla"}
expecting the semantic of your rule to be 
\texttt{True} if 
\texttt{c} is equal to the string 
\texttt{"blabla"}, and 
\texttt{False} otherwise.
When the Mathematica parser reads this expression, it evaluates it, and
 the result of this evaluation is 
\texttt{False}, as the symbol 
\texttt{c} is different from the string 
\texttt{"blabla"}. The problem comes from the fact that we expected this
evaluation to happen later...
There are probably ways to avoid this kind of problem using complex Mathematica tricks (avoiding
Mathematica to evaluate an expression is sometimes hard...)

\section{Using Meta}
\label{using}
Assume you want to write a translator called 
\texttt{myTranslator}.
\begin{enumerate}
\item Write the meta-rules in a file named 
\texttt{myTranslator.meta}.
\item Write semantic functions (if needed) in a separate package named 
\texttt{myTranslator.sem} .
 This package should follow some rules which are described in appendix
\ref{checkCellSem}.
\item Load the \texttt{Meta.m} package (if it is not already loaded).
\item Evaluate the expression
\newline 
\texttt{meta[ "myTranslator", debug->True ]}
\item 
Load the file 
\texttt{myTranslator.m} which was created by 
\texttt{meta}.
\item To translate an expression \texttt{expr} of type 
\texttt{EXPRESSION}, evaluate
\newline 
\texttt{myTranslatorTranslateEXPRESSION[expr]}. 
Another better way is to introduce in the semantic package the function
 which you want to call to activate your translator on a tree. 
 See for example the semantic file given as an example in section \ref{checkCellSem}: 
 the semantic file contains the definition of a \texttt{checkCell} function.
\end{enumerate}

\section{Options of \texttt{meta}}
\label{options}
\begin{itemize}
\item \texttt{verbose}: if \texttt{True}, gives some information. Default value is 
\texttt{False}.
\item \texttt{debug}: if \texttt{True}, gives a lot of information. Default value is 
\texttt{False}.
\item \texttt{check}: if \texttt{True}, calls to functions are generated inside
a \texttt{Check} statement. This may help debugging the translator. Default value is 
\texttt{False}.
\item \texttt{directory}: gives the directory where the semantic file is to be
found and read. If \texttt{Null}, this directory is \texttt{\$MMALPHA/lib/Packages/Alpha}.
Default value is \texttt{Null}. 
\end{itemize}

\section{How meta-rules are translated}
\label{translation}
A meta rule named 
\texttt{RULE} is translated as a function 
\texttt{myTranslatorTranslateRULE}.
 This function is put in a package called 
\texttt{Alpha`myTranslator`}.
 In such a way, it is not possible to create conflicting names, provided
 the name of the translator is different from the names of the MMAlpha packages.

The package is equipped with a symbol called 
\texttt{myTranslatorDebug}, whose initial value is 
\texttt{False}.
 This symbol can be used as a debug flag by the semantic functions, in order
 to debug the translator.

The package 
\texttt{myTranslator.m} loads automatically the semantic file 
\texttt{myTranslator.sem} when it is itself loaded.

\section{An example}
\label{example}
Appendix \ref{checkCellmeta}
 shows the meta-rules for a program, named 
\texttt{checkCell}, which analyzes an Alpha 
program which can be interpreted as a 
\texttt{AlphHard} 
"cell", and reports errors if not.
A so-called cell is basically an Alpha program which describes a Register-Trans
fer-Level description of a piece of hardware (see
\cite{patricia}
 for details).

The meta-rules read an AST and check that this AST represents a cell.
 For example, such a program cannot contain use statements, etc.
 etc.

Appendix \ref{checkCellsem}
 presents the semantic file CheckCell.sem which is associated with the previous
 file.

\appendix


%\chapter{Appendix}


\section{Meta-rules for the checkCell translator\label{checkCellmeta}}

\verbatiminput{CheckCell.meta}

\section{Semantic file for the checkCell translator (part of)
\label{checkCellSem}}

\verbatiminput{CheckCell.sem}

\end{document}
