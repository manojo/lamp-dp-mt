\documentclass[12pt]{article}
\begin{document}

\newcommand{\domlib}{{\sc domlib}}
\newcommand{\ML}{{\sc Mathlink}}
\newcommand{\mprep}{{\sc mprep}}
\newcommand{\polylib}{{\sc polylib}}
\newcommand{\mma}{{\sc Mathematica}}
\newcommand{\mkf}{{\texttt Makefile}}
\newcommand{\gcc}{\texttt{gcc}}
\newcommand{\doml}{\texttt{domlib.c}}
\newcommand{\tm}{\texttt{domlibtm.c}}
\newcommand{\mmAlpha}{\textsc{MMAlpha}}
\newcommand{\MMAlpha}{\textsc{MMAlpha}}
\newcommand{\MMA}{\textsc{Mathematica}}
\newcommand{\true}{\texttt{True}}
\newcommand{\false}{\texttt{False}}

\title{Testing \mmAlpha{}}
\author{Patrice Quinton}
\date{Version of \today}
\maketitle

\begin{abstract}
This document gives information on testing \mmAlpha{}. Some other information 
is also available in \texttt{doc/Install}. Information regarding compiling \texttt{domlib}
is given in \texttt{doc/Developers/Compiling}.
\end{abstract}
\section{Right after installing \mmAlpha{}}
\mmAlpha{} provides testing facilities which are explained in this document. 
After installation, run \texttt{test1[]}. This checks various things, in particular, that 
the \texttt{domlib} is properly installed. 
The report of this test is written in file \texttt{\$MMALPHA/tests/testReportFile.txt}. 
If this test does not return the value \texttt{True}, please send a mail together
with this file to 
\begin{verbatim}
patrice.quinton@irisa.fr
\end{verbatim}

\paragraph{Warning:} Notice that executing \texttt{test1[]} erases the previous version of the test 
report file.

\paragraph{Do it yourself:} You may also fix by yourself some common errors: report to Section~\label{troubleshooting}.

\paragraph{Tested configurations}
The current distribution runs properly on a MacBook Pro, whith \mma{} version 5 and 6. 
Version 7.0 of \mma{} is under test.

\section{More tests}
Other tests are done using the commands 
\texttt{test2[]}, 
\texttt{test3[]}, 
and \texttt{test4[]}. All these tests append information in the 
file \texttt{\$MMALPHA/tests/testReportFile.txt}. 

If one of these tests fails (i.e., it does not return the value \texttt{True}),
please send a mail together
with the test report file to 
\begin{verbatim}
patrice.quinton@irisa.fr
\end{verbatim}

If all these tests are successful, your installation should be correct.

\section{Exploring tests}
Commands \texttt{test1}, \texttt{test2}, etc. make use of another test program, 
which allows each individual package to be tested. For example, 
\begin{verbatim}
test[ "Domlib" ]
\end{verbatim}
runs a test for the domlib and returns \texttt{True} if the test was successful. This 
command writes information in the test report file. If one test fails, the name
of this test is written in the test report file. Say test \texttt{"domlib 23"} fails, then 
you can replay it by evaluating : 
\begin{verbatim}
test[ "Domlib" , "domlib 23"]
\end{verbatim}
The nature of this test will then be displayed. 
Notice however, that running an individual test may not always provide the
same result as when it is run in the context of a full package test. The reason is that
some of the package test were written before this facility was offered. I am trying\footnote{As of December 2008.}
to fix this, but test programs are numerous and long...

Test files are in directory \texttt{\$MMALPHA/Tests}. Commands \texttt{test1} to
\texttt{test4} are part of \texttt{\$MMALPHA/lib/Packages/Alpha.m}.

\section{Common Problems at Installation and How to Fix Them}
(See also \texttt{doc/Install}).

Check this in order. 
\begin{enumerate}
\item Is environment variable \texttt{\$MMALPHA} set? If this is not the case, \mma{} does not
start \MMAlpha{}. 
To check this, start \MMAlpha{} and evaluate
\begin{verbatim}
Environment[ "MMALPHA" ]
\end{verbatim}
If this variable was not set, then the result is \texttt{\$Failed}, otherwise, it should be the path
of the directory where you installed \MMAlpha{}. To fix this problem, report to the installation 
manual. Remember on \texttt{MacOSX} that you must start \mma{} from a shell window, and not
directly by double-clicking on the \mma{} icon: in this case, \mma{} does not import the 
environment variables.
\item Does the environment variable \texttt{\$PATH} contain the directory where the binaries
of \MMAlpha{} are ? If this is not the case, \texttt{domlib} cannot be loaded. 
Check this by evaluating
\begin{verbatim}
$Path
\end{verbatim}
and checking that there appears a path of the form \texttt"{... /mmalpha/bin.ostype"} where 
\texttt{ostype} depends on your operating system. For \texttt{MacOSX}, is should be
\texttt{bin.darwin}, for \texttt{Linux}, it should be \texttt{bin.linux}, etc. To fix this, report to
the installation manual and set the environment variable. You can get the value of your
OS type by evaluating \texttt{Environment[ "OSTYPE"]} in \mma{}. 
\item Does the directory of the binaries exist? To check this, get the value of environment 
variable \texttt{\$OSTYPE}, and check that there exist a directory \texttt{bin.ostype} in 
your \MMAlpha{} repository. If such a directory does not exist, the problem has to be
reported to the author of this note... 
\item Even if the \texttt{bin.ostype} directory exists, it may not contain the proper 
binary files: \texttt{domlib}, etc. 
\end{enumerate}

\end{document}

